%alg\documentclass[12pt,a4paper]{amsart}
\documentclass[10pt,a4paper]{article}
%\ifx\pdfpageheight\undefined\PassOptionsToPackage{dvips}{graphicx}\else%
%\PassOptionsToPackage{pdftex}{graphicx}
%\PassOptionsToPackage{pdftex}{color}
%\fi

%\usepackage{diagrams}

\usepackage{color}
\newcommand\coloremph[2][red]{\textcolor{#1}{\emph{#2}}}

\newcommand\greenemph[2][green]{\textcolor{#1}{\emph{#2}}}
\newcommand{\EMP}[1]{\emph{\textcolor{red}{#1}}}

%\usepackage[all]{xy}
\usepackage{url}
\usepackage{verbatim}
\usepackage{latexsym}
\usepackage{amssymb,amstext,amsmath,amsthm}
\usepackage{epsf}
\usepackage{epsfig}
% \usepackage{isolatin1}
\usepackage{a4wide}
\usepackage{verbatim}
\usepackage{proof}
\usepackage{latexsym}
%\usepackage{mytheorems}
\newtheorem{theorem}{Theorem}[section]
\newtheorem{corollary}{Corollary}[section]
\newtheorem{lemma}{Lemma}[section]
\newtheorem{proposition}{Proposition}[section]
\newcommand{\ras}{\twoheadrightarrow}

\usepackage{float}
\floatstyle{boxed}
\restylefloat{figure}


%%%%%%%%%
\def\oge{\leavevmode\raise
.3ex\hbox{$\scriptscriptstyle\langle\!\langle\,$}}
\def\feg{\leavevmode\raise
.3ex\hbox{$\scriptscriptstyle\,\rangle\!\rangle$}}

%%%%%%%%%


\newcommand\myfrac[2]{
 \begin{array}{c}
 #1 \\
 \hline \hline 
 #2
\end{array}}


\newcommand{\nats}{\mathbb{N}}

\newcommand{\Fin}[1]{T(#1)}

\newcommand{\ODisc}{\mathsf{ODisc}}
\newcommand{\Stone}{\mathsf{Stone}}
\newcommand{\CHaus}{\mathsf{CHaus}}
\newcommand{\Open}{\mathsf{Open}}
\newcommand{\Closed}{\mathsf{Closed}}
\newcommand{\AbG}{\mathsf{Ab}}
\newcommand{\OAbG}{\mathsf{Ab_{ODisc}}}
\newcommand{\refl}{\mathsf{refl}}
\newcommand{\ra}{\rightarrow}
\newcommand{\Noo}{\nats_{\infty}}
\newcommand\norm[1]{\left\lVert #1 \right\rVert}
\newcommand\cHH{\check{H}}%\newcommand\cHH{\check{\mathrm{H}}}
\newcommand\HH{\mathrm{H}}


\begin{document}

\title{Cohomology of Stone spaces}

\author{}
\date{}
\maketitle

%\rightfooter{}

\section*{Introduction}

We define $\ODisc$ to be the type of sets that are sequential colimit of finite sets.

The goal of this note is to show a main Lemma for showing that $H^1(S,M) = 0$ if
$S:\Stone$ and $M:S\ra \OAbG$, where $\OAbG$ is the type of abelian groups in $\ODisc$.

 By finite set, we mean a finite set with a given strict linear ordering.

\section{Lemma about Stone spaces}

Using the characterisation of Stone spaces as compact Hausdorff spaces that are totally disconnected one shows
that $\Sigma_SF:\Stone$ if $S:\Stone$ and $F:S\ra\Stone$, i.e. Stone spaces are closed by sum types.

From this follows the following result.

\begin{proposition}\label{sumStone}
  If we write $S = \varprojlim S_n$, with $\pi_n^m : S_m\ra S_n$ for $n\leqslant m$
  and $\pi_n:S\ra S_n$, with $S_n$ finite,
  then we can find $F_n(u),~u:S_n$ family of finite sets with $\pi_n^m:F_m(u)\ra F_n(\pi_n^m(u))$
  and $\pi_n:F(x)\ra F_n(\pi_n(x))$,
  such that $F(x) = \varprojlim F_n(\pi_n(x))$ for $x:S$.
\end{proposition}

\begin{corollary}\label{corStone}
  We can find an inverse system $G_n(x),~x:S$ of finite sets, with $\pi_n^m(x) : G_m(x)\ra G_n(x)$
  with $F(x) = \varprojlim G_n(x)$ for $x:S$. If we have $\Pi_S\norm{F}$ we have $\Pi_SG_n$ for all $n$.
\end{corollary}

\begin{proof}
  We take $G_n(x) = F_n(\pi_n(x))$.
\end{proof}

\section{\v{C}ech Cohomology}

Let $X$ be a type and $F_x$ a family of type over $X$ and $M:X\rightarrow \AbG$. We consider the cochain
complex
$$
C(X,F,M) ~=~ \Pi_XM^F\ra \Pi_XM^{F\times F}\ra \Pi_XM^{F\times F\times F}\ra\dots
$$
with $dv_x(a_0,\dots,a_n) = \Sigma (-1)^iv_x(a_0,\dots,\hat{a_i},\dots,a_n)$.
We define $\cHH^n(X,F,M)$ to be the nth cohomology group of this cochain complex. The two following Lemmas
have a direct proof.

\begin{lemma}\label{cech1}
  If we have an element in $\Pi_XF$ then $\cHH^n(X,F,M) = 0$ for $n>0$.
\end{lemma}

\begin{lemma}\label{cech2}
  We have $\cHH^n(X,F,M^F) = 0$ for $n>0$.
\end{lemma}

\section{Product and direct sequential limit}

Let us define $j^n:\Pi_SE_n\ra \Pi_SE$ by $j^n~u~x = i^n~(u~x)$ and
$j^n_m :\Pi_SE_n\ra \Pi_SE_m$ by $j^n_m~u~x = i^n_m~(u~x)$ for $m\geqslant n$.

\begin{proposition}
   $\Pi_SE = \varinjlim \Pi_S(E_n)$. 
\end{proposition}

\begin{proof}
The image $I_n~x$ of $i_n~x$ is an open subset of $E~x$. If $s:\Pi_SE$, we also have
$$
\forall_{x:S}\exists_n I_n~x~(s~x)
$$
and hence, since $S$ is Stone, we have $n$ such that $\forall_{x:S} I_n~x~(s~x)$.

By local choice, there is a surjective map $p:S'\ras S$ with $S'$ Stone and $t:\Pi_{y:S'}E_n~(p~y)$
such that $s~(p~y) = i^n~(p~y)~(t~y)$. We then have
$$\forall_{y_0~y_1}p~y_0=p~y_1\ra \exists_{m\geqslant n} i^n_m~(p~y_0)~(t~y_0) = i^n_m~(p~y_1)~(t~y_1)$$
and so we can find $m\geqslant n$ such that
$$\forall_{y_0~y_1}p~y_0=p~y_1\ra i^n_m~(p~y_0)~(t~y_0) = i^n_m~(p~y_1)~(t~y_1)$$
We can then define $u:\Pi_{x:S}E_m~x$ by $u~x = i^n_m~x~(t~y)$ for $p~y = x$ and we have $s~x = i^m~(u~x)$
for all $x:S$.

Let us define $j^n:\Pi_SE_n\ra \Pi_SE$ by $j^n~u~x = i^n~(u~x)$ and
$j^n_m :\Pi_SE_n\ra \Pi_SE_m$ by $j^n_m~u~x = i^n_m~(u~x)$ for $m\geqslant n$.
We have just proved that for all
$s:\Pi_SE$ there exists $m$ and $u:\Pi_SE_m$ such that $j^m~u = s$.
Furthermore, if we have $j^n~u_0 = j^n~u_1$ then we have, for all $x:S$
some $m\geqslant n$ such that $i^n_m~(u_0~x) = i^n_m~(u_1~x)$. Since this equality is an open proposition
there exists $m\geqslant n$ such that  $i^n_m~(u_0~x) = i^n_m~(u_1~x)$ for all $x:S$ that is
$j^n_m~u_0 = j^n_m~u_1$.
\end{proof}

\begin{corollary}\label{cor1}
  If $S:\Stone$ with $S = \varprojlim S_n$ and $S_n$ finite, and $E:\ODisc$
  then  $E^S = \varinjlim E^{S_n}$.
\end{corollary}

\begin{lemma}\label{exlim}
  If we have $C = \varinjlim C_n$, an inverse limit of cochain complex
  and each $C_n$ is exact then $C$ is exact.
  %% If we have 3 direct systems $A_n,i^n_m$ and $B_n,j^n_m$ and $C_n,k^n_m$ of abelian groups
  %% of direct limits $A,i^n$ and $B,j^n$ and $C,k^n$ and $u:A\ra B$ and $v:B\ra C$
  %% and $u_n:A_n\ra B_n$ and $v_n:B_n\ra C_n$ with $u_m\circ i^n_m = j^n_m\circ u_n$
  %% and  $v_m\circ j^n_m = k^n_m\circ v_n$, and each $A_n\ra B_n\ra C_n$ is exact
  %% then $A\ra B\ra C$ is exact.
\end{lemma}

Let $\OAbG$ be the type of abelian group in $\ODisc$.

\begin{theorem}\label{main}
If $S:\Stone$ and $F:S\ra\Stone$ such that $\Pi_S\norm{F}$
and $M:S\ra\ODisc$ then the complex $C(S,F,M)$ is exact.
\end{theorem}

\begin{proof}
 We write $S = \varprojlim S_n$ and $\pi_n:S \ra S_n$.  
 Using Corollary \ref{corStone}, we can write $F(x) = \varprojlim G_n(x)$ with $G_n(x)$ finite
 set and $\Pi_SG_n$. Using Corollary \ref{cor1} and Lemma \ref{exlim}, it is enough
 to show that each $C(S,G_n,M)$ is exact.
 We have an element in $\Pi_SG_n$ and the exactness of the sequence results from Lemma \ref{cech1}.
\end{proof}

\section{Application to cohomology over a Stone space}

\begin{theorem}\label{cohom1}
  $H^1(S,M) = 0$ if $S:\Stone$ and $M:S\ra\OAbG$.
\end{theorem}

\begin{proof}
  Let $\chi$ be an element of $\Pi_SK(M,n)$. We show that $\chi$ is merely $*$.

  By local choice, we have
  a surjective map $\Sigma_SF\ras S$ with $F:S\ra\Stone$ such that we merely have
  $\Pi_{x:S}(T_x\ra \chi(x) = *)$.

%  Let $\chi$ be an element of $\Pi_SK(M,1)$, we want to show $\norm{\chi = *}$. We have $\Pi_{x:S}\norm{\chi(x) = *}$.
%By local choice, we have $p:S'\ras S$ such that $\norm{\chi\circ p = *}$.

Since we want to prove a proposition, we can assume that we have $\alpha$ in   $\Pi_{x:S}(T_x\ra \chi(x) = *)$.
For $y_0,y_1:F_x$ we have $\alpha_x(y_0)$ and $\alpha_x(y_1)$ in $\chi(x)=*$.
Hence we have
$$v_x(y_0,y_1) = \alpha(y_0)^{-1}\cdot \alpha(y_1):* = *$$
and $*=*$ is $M_x$.
The map $v:\Pi_{S}M^{F\times F}$ is a cocycle, and it 
follows from Theorem \ref{main} that it is a coboundary. 
Hence, we can find $u:\Pi_{S}M^{F}$ such that $v_x(y_0,y_1) = u_x(y_1)-u_x(y_0)$.
From this, we show $\chi = *$: we define $\beta(x) = \alpha_x(y)\cdot u_x(y)$ for $y$ in $F_x$ and we
have $\beta(x):\chi(x) = *$ for $x:S$.
\end{proof}

Another way to formulate this is that the pointed groupoid $\Pi_SK(M,1)$ is connected, and yet another formulation
is that the canonical map $K(\Pi_SM,1)\ra\Pi_SK(M,1)$ is an equivalence.

\begin{theorem}\label{cohom2}
  For $n>0$, the canonical map $K(\Pi_SM,n)\ra\Pi_SK(M,n)$ is an equivalence.
  In particular we have $H^n(S,M) = 0$ for $n>0$
\end{theorem}

\begin{proof}
  We have proved this for $n=1$ and we show this by induction, assuming $n>1$ and that this holds for $n-1$.

  By induction, the canonical map  $K(\Pi_SM,n)\ra\Pi_SK(M,n)$ is an embedding, and it is enough to show
  that $\Pi_SK(M,n)$ is connected.

  Let $\chi$ be an element of $\Pi_SK(M,n)$. We show that $\chi$ is merely $*$.

  By local choice, we have
  a surjective map $\Sigma_SF\ras S$ with $F:S\ra\Stone$ such that we merely have
  $\Pi_{x:S}(T_x\ra \chi(x) = *)$. This means that the image of $\chi$
  under the diagonal map
  $$
  \Pi_SK(M,n)\ra\Pi_SK(M,n)^F
  $$
  merely is $*$. Since $K(M_x^{F_x},n)\ra K(M_x,n)^{F_x}$ is an embedding by induction,
  we get that the image of $\chi$ by the map
  $$
  \Pi_SK(M,n)\ra\Pi_SK(M^F,n)
  $$
  merely is $*$.

  If we now consider the exact sequence in $\OAbG$
  $$0\ra M_x\ra M_x^{T_x}\ra L_x\ra 0$$
  we have the associated fibration sequence
  $$\Pi_SK(L,n-1)\ra \Pi_SK(M,n)\ra \Pi_SK(M^F,n)$$
  and we can conclude since, by induction, $\Pi_SK(L,n-1)$ is connected.
\end{proof}

\section{Cohomology of a compact Hausdorff space}

Let $X$ be in $\CHaus$. We can then find $F:X\rightarrow\Stone$ such that $\Sigma_XF:\Stone$
and the projection map $\Sigma_XF\ras X$ is surjective. The goal of this section is to define
a natural isomorphism $\cHH^n(X,F,M) \equiv H^n(X,M)$ if $M:X\ra \OAbG$. This is direct for $n = 0$, since both
groups are $\Pi_XM$. 

The general case follows by induction from the two next Lemmas.
We introduce the short exact sequence $0\ra M_x\ra M_x^{F_x}\ra L_x\ra 0$.

\begin{lemma}
 $H^n(X,M^F) = 0$ for $n>0$ and we have an exact sequence $H^{n-1}(X,M^F)\ra H^{n-1}(X,L)\ra H^n(X,M)\ra 0$ for $n>0$.
\end{lemma}

\begin{proof}
  The first claim follows from the fact that $\Pi_XK(M^F,n) = \Pi_XK(M,n)^F = \Pi_{\Sigma_XF}K(M\circ\pi_1,n)$
  is $(n-1)$ connected using Theorem \ref{cohom2} and the fact that $\Sigma_XF:\Stone$. The second claim
  follows then from the long exact sequence associated to the short exact sequence
  $0\ra M_x\ra M_x^{F_x}\ra L_x\ra 0$.
\end{proof}


\begin{lemma}
  $\cHH^n(X,F,M^F) = 0$ for $n>0$ and we have an exact sequence
  $\cHH^{n-1}(X,F,M^F)\ra\cHH^{n-1}(X,F,L)\ra \cHH^n(X,F,M)\ra 0$ for $n>0$.
\end{lemma}

\begin{proof}
  The first claim is Lemma \ref{cech2}. The second claim follows then from the long
  exact sequence associated to the short sequence $C(X,F,M) \ra C(X,F,M^F)\ra C(X,F,L)$ which
  is exact using Theorem \ref{cohom1}.
\end{proof}




\end{document}     
                                                                                  











