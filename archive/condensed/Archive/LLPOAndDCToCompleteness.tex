\begin{lemma}\label{LLPOAndDCToCompleteness}
Using dependent choice and LLPO, we can show completeness of propositional logic. 
\end{lemma}
\begin{proof}
  Let $B:\Boole$ satisfy $0\neq_B 1$. We will show there merely exists a map $B\to 2$. 
  Let $G$ be the set of generators of $B$. 
  
  We will use dependent choice on the the following $E_n,R_n$:
  \begin{itemize}
    \item 
  Let $E_n$ be the type consisting of 
  \begin{itemize}
    \item A map from the first $n$ generators to $2$, denoted $x_n:G_n \to 2$. 
    \item A proposition denoting that $0\neq 1$ in 
      \begin{equation}
        B_n := B/\big( \{g|g\in G_n, x_n(g) = 0\} \cup \{ \neg g| g\in G_n, x_n(g) = 1\}\big).
      \end{equation}
  \end{itemize}
  \item 
    And let $R_n:E_n \to E_{n+1} \to \mathcal U$ denote the relation that $x_{n+1}$ extends $x_n$. 
  \end{itemize} 
  Note that $E_0$ is inhabited as $0\neq 1$ in $B$. 
 % 
  Now suppose that $E_n$ is inhabited,  and let $x_n:G_n \to 2$. 
  Note that in $B_n$, we have $0\neq 1$ and thus $$\neg ((g =1)  \wedge ((\neg g) = 1))$$
  for all $g:B_n$.
  Therefore, we have $$\neg \neg (( g\neq 1) \vee ((\neg g) \neq 1)).$$
  Note that inequality is closed, and by LLPO, the disjunction of closed propositions is closed, 
  hence equivalent to it's double negation. 
  Therefore, we have $$(( g\neq 1) \vee ((\neg g) \neq 1))$$
  This holds in particular for $g$ the $n+1$'th generator. 
  Therefore, we have that $0\neq 1$ in $B_n/\{g\}$ or in $B_n/\{\neg g\}$. 
  Thus we can extend $x_n$ by letting $x_{n+1}(g) = 0$ or $x_{n+1}(g) = 1$ respectively. 
  
  By dependent choice, we get a map $x:G\to 2$. 
  We claim that for this map $x$, we have $0\neq 1$ in 
  \begin{equation}
    B' := B/\big( \{g|g\in G_n, x(g) = 0\} \cup \{ \neg g| g\in G, x(g) = 1\}\big).
  \end{equation}
  Note that $B'$ is the colimit of the sequence $B_n$ with projection maps $B_n \to B_{n+1}$. 
  Thus if $0=1$ in $B'$, $0=1$ in some $B_n$, which doesn't happen by assumption. 
  Therefore we have $0\neq 1$ in $B'$. 
  Furthermore, note that $B'$ is equivalent to a Boolean algebra with no generators, 
  as any generator in $B$ is sent to either $0$ or $1$ by the relations in $B'$. 

  But now any Boolean algebra with no generators and $0\neq 1$ is isomorphic to $2$. 
  Therefore $B'\simeq 2$, and the projection map $B\to B'$ gives a map $B \to 2$. 
  



\end{proof}



















































%%
%%
%%\begin{lemma}
%%  Using Dependent Choice and LLPO, we can show completeness of propositional logic
%%\end{lemma}
%%\begin{proof}
%%%  We will show for all $B:\Boole$ that $0\neq_B 1 \to ||Sp(B)||$.
%%%  We will first show this for $B:\Boole$ with finitely many generators. 
%%%
%%%  We will use induction on the number of generators of $B$. 
%%%  Suppose $B$ has $0$ generators, we claim that $B\simeq 2$. 
%%%  Any $r$ in the free algebra on $0$ generators satisfies $r=0\vee r=1$. 
%%%  If $r=0$, we can leave out $r$ in the relations of $B$. If $r=1$, we get a contradiction with $0\neq_B 1$. 
%%%  Thus we get that $0\neq_B1 \to ||Sp(B)||$ if $B$ has $0$ generators. 
%%%
%%%  Suppose that for any countably presented Boolean algebra $A$ on $n$ generators, we have $0\neq_A 1 \to ||Sp(A)||$. 
%%%  Let $B$ have $n+1$ generators, and suppose $0\neq_B 1$. 
%%%  Let $g$ be a generator of $B$.
%%%  Consider $B/\{g\}, B / \{\neg g\}$. 
%%%  Note that these are isomorphic to countably presented Boolean algebras on $n$ generators. 
%%%  We claim that at least one of them satisfies $0\neq 1$. We will use LLPO. 
%%%
%%%  Note that $0=1$ in $B/\{b\}$ iff $b=1$.
%%%  So it is sufficient to show that $b\neq 1$ or  $ (\neg b) \neq 1$ for all $b:B$. 
%%%  Note that inequality is a closed proposition, and by LLPO, the disjunction of closed propositions is closed. 
%%%  Therefore, it is sufficient to prove that $\neg \neg ( (b\neq 1) \vee ((\neg b) \neq 1))$, which 
%%%  follows from the fact that $\neg (b = 1) \wedge ((\neg b) = 1)$ is provable. 
%%
%%  We will show for all $B:\Boole$ that $0\neq_B 1 \to ||Sp(B)||$.
%%  Let $B_n\subseteq B$ be generated by the first $n$ generators of $B$
%%  and quotiented by the relations expressable in those generators. 
%%  We will merely give a compatible function $\Pi_{n:\mathbb N} (B_n \to 2)$. 
%%  We will use dependent choice, with $E_n$ given by compatible functions in $\Pi_{k\leq n} (B_n \to 2)$
%%
%%  First we claim that $B_0\simeq 2$. 
%%  Any $r$ in the free algebra on $0$ generators satisfies $r=0\vee r=1$. 
%%  If $r=0$, we can leave out $r$ in the relations of $B|_0$. If $r=1$, we get a contradiction with $0\neq_B 1$. 
%%  Thus $B_0 \simeq 2$, hence we have an inhabitant of $Sp(B)$.
%%
%%  Suppose we have defined a compatible function in $\Pi_{k\leq n} (B_k\to 2)$.
%%  Let $g$ be the $n+1$'th generator of $B$.
%%  Consider $B_{n+1}/\{g\}, B_{n+1} / \{\neg g\}$. 
%%  We shall show that at least one of these Boolean algebras is isomorphic to $B_n$. 
%%
%%  Recall that inequality is closed, and by LLPO, disjunctions of closed propositions are closed. 
%%  Therefore, $( (g\neq 1) \vee ((\neg g) \neq 1))$ is closed, and equivalent to it's double negation, 
%%  which is provable. As $0\neq 1$, we can show $\neg ( (g = 1) \wedge ( (\neg g) = 1))$. 
%%  Therefore $ (g\neq 1) \vee ((\neg g) \neq 1)$. 
%%  \begin{itemize}
%%    \item 
%%  If $g \neq 1$, we have that $B_{n+1} / \{g\} \simeq B_n$, and we can extend the map $B_n \to 2$ 
%%  to a map $B_{n+1} \to 2$ by sending $g$ to $0$. 
%%  \rednote{I'm less confident in the isomorphism $B_{n+1} / \{g\} \simeq B_n$, there can be more relations still 
%%  , we have that $B_{n+1}/\{g\} = B_n/something$, where something consists of the relations 
%%   of $B$ expressable with the first $n$ generators and $g_{n+1}$, but then replaces $g_{n+1}$ by $1$, 
%% these relations can be more than just the relations expressable in the first $n$ generators. }
%%    \item 
%%  Similarly, if $\neg g \neq 1$, we can extend the map $B_n \to 2$ to a map $B_{n+1} \to 2$ by sending $g $ to $1$. 
%%  \end{itemize}
%%  Thus there exists an extension of the map $B_n \to 2$ to a map $B_{n+1} \to 2$. 
%%  
%%  By dependent choice, we merely have a compatible function $\Pi_{n:\mathbb N} (B_n \to 2)$, 
%%  which is the same as a map $B\to 2$. 
%%  Hence $0\neq_B1 \to ||Sp(B)||$. 
%%  By \Cref{SpectrumEmptyIff01Equal}, this means that $\neg \neg Sp(B) \to ||Sp(B)||$. 
%%\end{proof}
%%
