For any rational $0\leq q \leq 1$, there is a (not necessarily unique) sequence $\alpha$ such that 

\begin{equation}q = \sum_{i : \N} \frac{\alpha(i)}{2^{i+1}}\end{equation}

One way to get such a sequence is by the following construction:
\begin{equation}\alpha(n)  = \begin{cases}
  1 \text{ if } q - \sum_{0\leq i < n} \frac{\alpha(i)}{2^{i+1}} > \frac 1 {2 ^ n + 1} \\
0 \text{ otherwise } 
\end{cases}
\end{equation} 
This construction defines a function $f: \mathbb Q \to 2^\N$. 
\begin{definition}
  A Cauchy sequence is a sequence of rational numbers $(q_n)_{n: \N}$ with $-1 \leq q_n \leq 2$ 
  for all $n:\N$
  such that for every $\epsilon: \mathbb Q_{>0}$, we have an $N:\N$, 
  such that whenever $n,m \geq N$, we have 
\begin{equation}
  | q_n - q_m | \leq \epsilon
\end{equation} 
\end{definition}
\begin{definition}
  The type of Cauchy reals is the subtype of Cauchy sequences under the quotient of the following 
  equivalence relation:
\begin{equation}
  p \sim_C q : = \forall (\epsilon : \mathbb Q_{>0} )\exists ( N :\N) \forall (n : \N) ((n \geq N) \to 
  (| p_n - q_n| \leq  \epsilon))
\end{equation}
\end{definition}


\begin{definition}
  The interval is defined as the subtype of Cantor space under the quotient of the equivalence relation
  $\sim_s$ such that for any finite binary sequence $x$, we have
  \begin{equation}(x,1,\overline 0) \sim_s (x ,0, \overline 1)\end{equation}
\end{definition} 



\begin{proposition}
  For $f$ as above, for any Cauchy sequence $p: \N \to \mathbb Q$ 
  the map $f\circ p: \N \to 2^\N$ can be extended to a map 
  $ext(p) : \Noo \to 2^\N$. 
 % 
  Furthermore, $ext(p)(\infty) \sim_s ext(q)(\infty)$ whenever $p \sim_C q$. 
\end{proposition} 
\begin{corollary}
  $2^\N/\sim_s$ is the completion of the rational interval. 
\end{corollary}

