%alg\documentclass[12pt,a4paper]{amsart}
\documentclass[10pt,a4paper]{article}
%\ifx\pdfpageheight\undefined\PassOptionsToPackage{dvips}{graphicx}\else%
%\PassOptionsToPackage{pdftex}{graphicx}
%\PassOptionsToPackage{pdftex}{color}
%\fi

%\usepackage{diagrams}

\usepackage{color}
\newcommand\coloremph[2][red]{\textcolor{#1}{\emph{#2}}}

\newcommand\greenemph[2][green]{\textcolor{#1}{\emph{#2}}}
\newcommand{\EMP}[1]{\emph{\textcolor{red}{#1}}}

%\usepackage[all]{xy}
\usepackage{url}
\usepackage{verbatim}
\usepackage{latexsym}
\usepackage{amssymb,amstext,amsmath,amsthm}
\usepackage{epsf}
\usepackage{epsfig}
% \usepackage{isolatin1}
\usepackage{a4wide}
\usepackage{verbatim}
\usepackage{proof}
\usepackage{latexsym}
%\usepackage{mytheorems}
\newtheorem{theorem}{Theorem}[section]
\newtheorem{corollary}{Corollary}[section]
\newtheorem{lemma}{Lemma}[section]
\newtheorem{proposition}{Proposition}[section]


\usepackage{float}
\floatstyle{boxed}
\restylefloat{figure}


%%%%%%%%%
\def\oge{\leavevmode\raise
.3ex\hbox{$\scriptscriptstyle\langle\!\langle\,$}}
\def\feg{\leavevmode\raise
.3ex\hbox{$\scriptscriptstyle\,\rangle\!\rangle$}}

%%%%%%%%%


\newcommand\myfrac[2]{
 \begin{array}{c}
 #1 \\
 \hline \hline 
 #2
\end{array}}


\newcommand{\nats}{\mathbb{N}}

\newcommand{\ODisc}{\mathsf{ODisc}}
\newcommand{\Open}{\mathsf{Open}}
\newcommand{\refl}{\mathsf{refl}}
\newcommand{\Noo}{\nats_{\infty}}
\newcommand\norm[1]{\left\lVert #1 \right\rVert}



\begin{document}

\title{Directed Univalence}

\author{}
\date{}
\maketitle

%\rightfooter{}

\section*{Introduction}

Following the argument of Barton and Commelin, we prove directed univalence from
our 4 axioms of Synthetic Stone Duality.

We define $\ODisc$ to be the type of sets that are sequential colimit of finite sets.
If $E$ in $\ODisc$ then $\norm{E}$ is in $\Open$, the type of {\em open} propositions, i.e.
propositions of the form $\exists_{n:\nats}\alpha n = 0$ for some binary sequence $\alpha$.
The propositions $\perp$ and $\top$ are open. 

\section{One direction}

Given two types $A$ and $B$ and $f:A\rightarrow B$ we define $X~p$ a family of types
over $\Open$ by
$$
X~p ~=~\Sigma_{b:B} (\Sigma_{a:A}f~a =_B b)^{\neg p}
$$
We have then $X~\top = B$ and $X~\perp = A$.

Given $a:A$ we define a section $s_a:\Pi_{p:\Open}X~p$ by
$s_a~p = (f~a,u)$ with $u~x = (a,\refl)$.

\section{The other direction}

Conversely, given a family $X:\Open\rightarrow\ODisc$, we define
$X~\perp~\rightarrow~ X~\top$ as follows.

We can view $\Noo$ to be the type of non increasing binary sequences.
We define $\alpha_n$ to be the sequence $\alpha_n~m = m<n$ and
$\alpha_{\infty}$ to be the constant sequence $1$.
We have a surjection $\Noo\rightarrow \Open$ defined by $\alpha\mapsto [\alpha] = \exists_n~\alpha~n = 0$.
Also, if $\alpha:\Noo$ we can consider the closed subset $[\alpha,\alpha_{\infty}]$ of $\Noo$.

We define $S:\Noo\rightarrow \Noo$ by $(S~\alpha)~0 = 1$ and $(S~\alpha)~(n+1) = \alpha~n$.
We have $S~\alpha_{\infty} = \alpha_{\infty}$ and $S~\alpha_n = \alpha_{n+1}$.

\medskip

We assume give $x_0 : X~\perp$.

We want to define from this an element in $X~\top$.

\medskip


We define $Z:\Noo\rightarrow \ODisc$ by $Z~\alpha = X~[\alpha]$.

We have $Z~\alpha_n = X~\top$ and $Z~\alpha_{\infty} = X~\perp$.

We consider the proposition $G~\alpha$ expressing that there exists
a section $s:\Pi_{u:[\alpha,\alpha_{\infty}]}Z~u$ such that $s~\alpha_{\infty} = x_0$.

This is an open proposition, since $[\alpha,\alpha_{\infty}]$ is Stone and $Z~u$ in $\ODisc$,
and $G~\alpha_{\infty}$ holds. Hence $G~\alpha_n$ holds
for $n$ big enough.

Also if we have two sections $s$ and $s'$ such that  $s~\alpha_{\infty} =  s'~\alpha_{\infty} = x_0$
then $s~\alpha_n = s'~\alpha_n$ for $n$ big enough.

So given any section $s$ such that $s~\alpha_{\infty} = x_0$ we have that $s~(S~\alpha_n) = s~\alpha_n$
for $n$ big enough, i.e. $u~n = s~\alpha_n$ is eventually constant, and, this constant value in $X~\top$
is independent of the section $s$.

In this way, we define a unique element of $X~\top$ from $x_0$, and this provides the map $X~\perp\rightarrow X~\top$.

\section{The two maps are inverses}

\end{document}     
                                                                                  
