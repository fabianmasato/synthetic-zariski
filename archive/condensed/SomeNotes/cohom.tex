%alg\documentclass[12pt,a4paper]{amsart}
\documentclass[10pt,a4paper]{article}
%\ifx\pdfpageheight\undefined\PassOptionsToPackage{dvips}{graphicx}\else%
%\PassOptionsToPackage{pdftex}{graphicx}
%\PassOptionsToPackage{pdftex}{color}
%\fi

%\usepackage{diagrams}

\usepackage{color}
\newcommand\coloremph[2][red]{\textcolor{#1}{\emph{#2}}}

\newcommand\greenemph[2][green]{\textcolor{#1}{\emph{#2}}}
\newcommand{\EMP}[1]{\emph{\textcolor{red}{#1}}}

%\usepackage[all]{xy}
\usepackage{url}
\usepackage{verbatim}
\usepackage{latexsym}
\usepackage{amssymb,amstext,amsmath,amsthm}
\usepackage{epsf}
\usepackage{epsfig}
% \usepackage{isolatin1}
\usepackage{a4wide}
\usepackage{verbatim}
\usepackage{proof}
\usepackage{latexsym}
%\usepackage{mytheorems}
\newtheorem{theorem}{Theorem}[section]
\newtheorem{corollary}{Corollary}[section]
\newtheorem{lemma}{Lemma}[section]
\newtheorem{proposition}{Proposition}[section]


\usepackage{float}
\floatstyle{boxed}
\restylefloat{figure}


%%%%%%%%%
\def\oge{\leavevmode\raise
.3ex\hbox{$\scriptscriptstyle\langle\!\langle\,$}}
\def\feg{\leavevmode\raise
.3ex\hbox{$\scriptscriptstyle\,\rangle\!\rangle$}}

%%%%%%%%%


\newcommand\myfrac[2]{
 \begin{array}{c}
 #1 \\
 \hline \hline 
 #2
\end{array}}

\newcommand\HH{\mathrm{H}}
\newcommand\cHH{\check{\mathrm{H}}}
\newcommand{\nats}{\mathbb{N}}
\newcommand{\ints}{\mathbb{Z}}
\newcommand{\reals}{\mathbb{R}}

\newcommand{\ODisc}{\mathsf{ODisc}}
\newcommand{\Stone}{\mathsf{Stone}}
\newcommand{\Aff}{\mathsf{Aff}}
\newcommand{\Scheme}{\mathsf{Scheme}}
\newcommand{\CHaus}{\mathsf{CHaus}}
\newcommand{\Open}{\mathsf{Open}}
\newcommand{\refl}{\mathsf{refl}}
\newcommand{\Noo}{\nats_{\infty}}
\newcommand{\Ab}{\mathsf{AbGrp}}
\newcommand{\ras}{\twoheadrightarrow}
\newcommand\norm[1]{\left\lVert #1 \right\rVert}



\begin{document}

\title{Cohomology}

\author{}
\date{}
\maketitle

%\rightfooter{}

\section*{Introduction}

If $X:\CHaus$ and $S\ras X$ with $S:\Stone$, we define $S_x$ the fiber of this map at $x:X$. We can then define the cochain complex
\[
\Pi_{x:X}\ints^{S_x}\rightarrow \Pi_{x:X}\ints^{S_x\times S_x}\rightarrow \Pi_{x:X}\ints^{S_x\times S_x\times S_x}\rightarrow \dots
\]
and the main result of this note is that the $n$th cohomology group of this cochain complex is
$\HH^n(X,\ints)$. Actually we can replace $\ints$ by a family of abelian groups $M_x$ over $X$
and consider the cochain complex
\[
\Pi_{x:X}M_x^{S_x}\rightarrow \Pi_{x:X}M_x^{S_x\times S_x}\rightarrow \Pi_{x:X}M_x^{S_x\times S_x\times S_x}\rightarrow \dots
\]
and we show that for a large class of such family, the $n$th cohomology group of this cochain complex is
$\HH^n(X,M)$. 

\medskip

It is interesting furthermore that essentially the same argument shows that if $X$ is a {\em separated}
scheme, and $S\ras X$ with $S$ an {\em affine} scheme, such that all fibers $S_x$ are {\em affine}
we have that the $n$th cohomology group of the cochain complex
\[
\Pi_{x:X}M_x^{S_x}\rightarrow \Pi_{x:X}M_x^{S_x\times S_x}\rightarrow \Pi_{x:X}M_x^{S_x\times S_x\times S_x}\rightarrow \dots
\]
is $\HH^n(X,M)$. The $n$th cohomology group of this cochain complex is exactly $\cHH^n(X,M)$, and so we
prove $\HH^n(X,M) = \cHH^n(X,M)$.


\section{Lemma about Stone spaces}

Using the characterisation of Stone spaces as compact Hausdorff spaces that are totally disconnected one shows
that $\Sigma_SF:\Stone$ if $S:\Stone$ and $F:S\rightarrow\Stone$, i.e. Stone spaces are closed by sum types.

From this follows the following result.

\begin{proposition}
  If we write $S = \varprojlim S_n$ with $S_n$ finite, with $p_n^m : S_m\rightarrow S_n$ for $n\leqslant m$
  and $p_n:S\rightarrow S_n$, 
  then we can find $F_n(x),~x:S_n$ family of finite sets $p_n^m:F_m(u)\rightarrow F_n(p_n^m(u))$
  and $p_n:F(x)\rightarrow F(p_n(x))$,
  such that $F(x) = \varprojlim F_n(p_n(x))$ for $x:S$.
\end{proposition}

\section{First result}

We show first that $\HH^1(S,M) = 0$ if $M$ is an abelian group in $\ODisc$ and $S:\Stone$.

Let $\chi$ be an element of $K(M,1)^S$, we want to show $\norm{\chi = *}$. We have $\Pi_{x:S}\norm{\chi(x) = *}$.
By local choice, we have $p:S'\ras S$ such that $\norm{\chi\circ p = *}$.

Since we want to prove a proposition, we can assume $\Pi_{y:S'}\chi(p(y)) = *$.
Let $S_x$ be the fiber at $x:S$ of $p$. For $y_0,y_1:S_x$ we have $\chi(p(y_0))) = *$ and $\chi(p(y_1)) = *$
and hence we have $v_x(y_0,y_1):M$. The map $v:\Pi_{x:X}M^{S_x\times S_x}$ is a cocycle, and we want to show
that it is a coboundary. For this we use the fact that $S'\ras S$ can be written as a sequential
limit of surjective maps $F'_n\ras F_n$ between {\em finite} sets.

 Since $M:\ODisc$ we have $M^{S'}\rightarrow M^{S'\times_S S'}\rightarrow\dots$ sequential limit of
 $M^{F'_n}\rightarrow M^{F'_n\times_{F_n} F'_n}\rightarrow\dots$. Since all these cochain complex are exact, so is
 the limit. It follows that $v$ is a coboundary, i.e. we can find $u:\Pi_{x:S}M^{S_x}$ such that
 $v_x(y_0,y_1) = u_x(y_1)-u_x(y_0)$.
 From this, we show $\chi = *$.

 \begin{corollary}\label{Stone1}
   If $0\rightarrow L\rightarrow M\rightarrow N\rightarrow 0$ is exact, with $L:\ODisc$ then
   so is $0\rightarrow L^S\rightarrow M^S\rightarrow N^S\rightarrow 0$ for any $S:\Stone$.
 \end{corollary}
 


\section{Refinement}

We can generalise the previous argument to the case where we have a family $M_x$ of abelian groups
for $x:S$. What we need that is that, if $F_x:\Stone$ for $x:S$ such that $\Pi_{x:S}\norm{F_x}$
then the cochain complex
\[
\Pi_{x:S}M_x^{F_x}\rightarrow \Pi_{x:S}M_x^{F_x\times F_x}\rightarrow \Pi_{x:S}M_x^{F_x\times F_x\times F_x}\rightarrow \dots
\]
is exact. Let us call $C(M,F)$ this context.

Let us write $E(M)$ this property of a function $M:S\rightarrow\Ab$. We have just seen that $E(M)$ implies
$\HH^1(S,M) = 0$.
It may be that $E(M)$ always holds if $M~x:\ODisc$ but I have not been able to show this. Instead, we give
closure properties showing that $E(M)$ holds for several families $M$.

We now write $G(M)$ the property that we have $E(M^H)$ for any $H:S\rightarrow\Stone$ such that
$\Pi_{x:S}\norm{H~x}$.

\begin{lemma}\label{Stone2}
  If $G(M)$ then we have $\HH^1(S,M^H)$  for any $H:S\rightarrow\Stone$ such that
$\Pi_{x:S}\norm{H~x}$.
\end{lemma}


Using that $\Stone$ is closed by sum types, we have that $G(M)$ holds if $M$ is the constant family $\lambda_x \ints$.
%(and more generally a constant family determined by an abelian group in $\ODisc$)
This means that the cochain complex $C(M,F)$ is exact and so is the cochain complex $C(M^H,F)$. Let $\Delta(M,H)$ be
determined by the exact sequence $0\rightarrow M_x\rightarrow M_x^{H_x}\rightarrow \Delta(M,H)_x\rightarrow 0$.

\begin{lemma}
  If we have $G(M)$ then we have $G(\Delta(M,H))$.
\end{lemma}

\begin{proof}
  Let $G:S\rightarrow \Stone$ with $\Pi_{x:S}\norm{Gx}$. By corollary \ref{Stone1}, we have a short exact sequence
  \[
  0\rightarrow M_x^{G_x\times F_x^k}\rightarrow M_x^{H_x\times G_x\times F_x^k}\rightarrow \Delta(M,H)_x^{G_x\times F_x^k}\rightarrow 0
  \]
  and then using Proposition \ref{Stone2}, we have
  \[
  0\rightarrow \Pi_S M^{G\times F^k}\rightarrow \Pi_S M^{H\times G\times F^k}\rightarrow \Pi_S \Delta(M,H)^{G\times F^k}\rightarrow 0
  \]
  and the result follows from the fact that we have $E(M^G)$ and $E(M^{G\times H}).$
\end{proof}



\section{Generalisation}

We show now by induction on $n$ the following result.

\begin{theorem}\label{Stone}
  If we have $G(M)$ then $\HH^n(S,M) = 0$ for $n>0$.
\end{theorem}

We have seen that this result holds for $n=1$.

We now assume that it holds for $0<k<n$, with $n>1$ and we show that it holds for $n$.

Let $\chi:\Pi_{x:S}K(M_x,n)$ represent a cohomology class. We want to show
$\norm{\chi = *}$ which is a proposition.
By local choice, we have $p:S'\ras S$ with $\norm{\chi\circ p = *}$.
We write $F_x$ the fiber of $p$ at $x:S$. We have that the image of $\chi(x)$
under the diagonal map $K(M_x,n)\rightarrow K(M_x,n)^{F_x}$ is $*$. The diagonal
map can be factorized as
$K(M_x,n)\rightarrow K(M_x^{F_x},n)\rightarrow K(M_x,n)^{F_x}$. But the
map $ K(M_x^{F_x},n)\rightarrow K(M_x,n)^{F_x}$ is an embedding since
$\HH^{k}(F_x,M_x) = 0$ for $0<k<n$ by induction. It follows that the $\chi(x)$
is already mapped to $*$ in $K(M_x^{F_x},n)$.

So $\chi$ maps to $0$ in $\HH^n(S,M_x^{F_x})$. Since $F_x$ is inhabited,
the diagonal map $M_x\rightarrow M_x^{F_x}$ is an embedding and we have a
short exact sequence
\[
0\rightarrow M_x\rightarrow M_x^{F_x}\rightarrow \Delta(M,F)_x\rightarrow 0
\]
We get as part of the associated long exact sequence
\[
\HH^{n-1}(S,\Delta(M,F)) \rightarrow \HH^n(S,M)\rightarrow \HH^n(S,M^F)
\]
Since we have $G(M)$ we also have $G(\Delta(M,F))$ and we have
by induction $\HH^{n-1}(S,\Delta(M,F)) = 0$. Hence $\chi = 0$ in $\HH^n(S,M)$.

\section{Application to compact Hausdorff spaces}

Let $X$ be in $\CHaus$. We can find $S:X\rightarrow\Stone$ such that $\Pi_{x:X}\norm{S_x}$, so that  the projection map
$p:\Sigma_XS\rightarrow X$ is surjective, and we associate the
cochain complex $C(X,M,S)$
\[
\Pi_{x:X}M_x^{S_x}\rightarrow \Pi_{x:X}M_x^{S_x\times S_x}\rightarrow \Pi_{x:X}M_x^{S_x\times S_x\times S_x}\rightarrow \dots
\]
for $M:X\rightarrow\Ab$.

\begin{theorem}
  If $G(M\circ p)$ then $\HH^n(X,M)$ for $n>0$ is the cohomology of the cochain complex $C(X,M,S)$.
\end{theorem}

 In particular, this holds if $M$ is the constant family $\lambda_x \ints$.

 \begin{proof}
   We have $S:\Stone$ and $p:S\ras X$. We write $F_x$ the fiber of $p$ at $x:X$.
   Using Theorem \ref{Stone}, we show that this holds for $n=1$, since we can associate
   to any $\chi:K(M,1)^X$ a cocycle in $\Pi_XS^{F_x^2}$, having $\norm{\chi = *}$ iff this
   cocycle is a coboundary, and we can associate in an inverse way an element in $K(M,1)^X$ from any
   cocycle in $\Pi_XS^{F_x^2}$.
   
   We use next the short exact sequence
\[
0\rightarrow M_x\rightarrow M_x^{F_x}\rightarrow \Delta(M,F)_x\rightarrow 0
\]
and the fact that we have $K(M_x,n)^{F_x} = K(M_x^{F_x},n)$
and $\HH^n(X,M^F) = 0$ for $n>0$ by Theorem \ref{Stone}.
Using the associated long exact sequence it follows that we have
\[
\HH^n(X,M) = \HH^{n-1}(X,\Delta(M,F)).
\]
Since this inductive relation also holds for the cochain cohomology, the result follows.
 \end{proof}
 
 \section{Some Examples}

 We deduce from this for instance that $\HH^n([0,1],\ints) = 0$ for $n>0$
 and that $\HH^1(\reals/\ints, \ints) = \ints$
 and $\HH^n(\reals/\ints, \ints) = 0$  for $n>1$.

 \section{Application to separated schemes}

 The same reasoning works for {\em separated} schemes. Let $X$ be a separated scheme and $M_x$ a family
 of w.q.c. abelian groups over $X$. We can find $S$ affine with a surjection $S\ras X$. Since
 $X$ is separated, each fiber $S_x$ is affine.
We associate the
cochain complex $C(X,M,S)$
\[
\Pi_{x:X}M_x^{S_x}\rightarrow \Pi_{x:X}M_x^{S_x\times S_x}\rightarrow \Pi_{x:X}M_x^{S_x\times S_x\times S_x}\rightarrow \dots
\]
for $M:X\rightarrow\Ab$. We define $\cHH^n(X,M)$ to be the $n$th cohomology group of this cochain complex
and we are going to show that it does not depend on the choice of $S\ras X$ and that
$\cHH^n(X,M) = \HH^n(X,M)$ for $n>0$. For this we show exactly like in the previous section that we have
\[
\HH^n(X,M) = \HH^{n-1}(X,\Delta(M,F)).
\]
Since we also have
\[
\cHH^n(X,M) = \cHH^{n-1}(X,\Delta(M,F)).
\]
and that we can show $\HH^1(X,M) = \cHH^1(X,M)$ like in the previous section (or like in Foundations),
we deduce $cHH^n(X,M) = \HH^n(X,M)$ for $n>0$.




\end{document}     
                                                                                  
