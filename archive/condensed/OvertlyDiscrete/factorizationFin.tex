\begin{definition}
  A proposition $p$ is open if there merely exists some $\alpha:\N_\infty$ such that 
  $p$ is equivalent to $\Sigma_{i:\N} \alpha(i) = 1$. 
\end{definition}
\begin{lemma}
  An countable disjunction of open propositions is open. 
\end{lemma}
\begin{proof}
  For $\alpha_j$ a sequence of sequences, 
  we have that $\bigvee_{j:\N}(\exists_{i:\N} \alpha_j(i) = 1)$ 
  is equivalent to there being some $i,j:\N$ with $\alpha_j(i) = 1$. 
  Denote $\alpha_j(i)$ by $\alpha(j,i)$. 
  Let $z:\N \to \N \times \N$ be the zigzag surjection. 
  Define 
  \begin{equation}
    \beta(k) = 
    \begin{cases}
      1 \text{ if } \alpha(z(k)) = 1\text{ for $k$ minimal}\\
      0 \text{ otherwise} 
    \end{cases}
  \end{equation}
  Note that $\beta(k)$ is $1$ for some $k:\N$ iff there are some $(i,j):\N \times \N$ 
  with $\alpha_j(i) = 1$. 
  Thus the disjunction of $\alpha_i$ is equivalent to 
  $\exists_{k:\N} \beta(k) = 1$. 
  As $\beta$ is $1$ at most once, $\beta:\N_\infty$ and we're done. 
\end{proof}




\begin{lemma}
  For $S$ Stone and $A\subseteq S$ closed, 
  the proposition $\exists_{x:S} A(x)$ is closed. 
\end{lemma}
\begin{proof}
  
\end{proof}




\begin{lemma}
  For $S$ Stone, the propositon $\neg S$ is open. 
\end{lemma}

\begin{proof}
  Note that for $S = Sp(B)$, we have that $\neg S$ iff $0=1$ is derivable in $B$. 
  Let $(\phi_i)|_{i:\N}$ be the countably many relations on the generators of $B$. 
  Remark that 
  \begin{equation}
    ((\phi_i)|_{i:\N} \vdash 0 = 1 )= \bigvee_{k:\N} ((\phi_i)_{i\leq k} \vdash 0 = 1)
  \end{equation}
  We claim that for any $k:\N$, $||(\phi_i)_{i\leq k} \vdash 0 = 1||$ is an open proposition. 
  If this claim holds, by the previous lemma we would then conclude that $0=1$ in $B$ is an open proposition
  and hence $\neg S$ is as well. 

  To show that $||(\phi_i)_{i\leq k} \vdash 0 = 1||$ is an open proposition, 
  we use that with finitely many formuas


\end{proof}
