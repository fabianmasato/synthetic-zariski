The language of homotopy type theory has proved to be appropriate to develop an internal language for higher toposes, for example with Synthetic Algebraic Geometry for the Zariski topos.
In this paper we use this for the higher topos corresponding to light condensed sets.
This seems to be an appropriate setting to develop synthetic topology (similar to Martin Escardo)
This consists of extending homotopy type theory with 4 axioms and use them to prove internal properties of light condensed sets.
We get an axiom system strong enough to prove Markov's principle, LLPO and the negation of WLPO. 
We also introduce types of open and closed propositions, inducing a topology on any type. 
This leads to a (synthetic) topological study of Stone and compact Hausdorff types. 
All functions are continuous, and the topology is as one would classically expect for compact separable Hausdorff space.
For example, any map from the unit interval to itself is continuous in the usual epsilon-delta sense.
We use the synthetic homotopy theory given by the higher types of homotopy type theory to make homotopical arguments.
As an application, we compute the cohomology of the interval and use this to prove Brouwer's fixed point theorem. 
