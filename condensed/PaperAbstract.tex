The language of homotopy type theory has proved to be appropriate as an internal language for various higher toposes, 
for example with Synthetic Algebraic Geometry for the Zariski topos.
In this paper we apply such techniques to the higher topos corresponding to the light condensed sets 
of Dustin Clausen and Peter Scholze.
This seems to be an appropriate setting to develop synthetic topology, similar to the work of 
Martín Escardó.
To reason internally about light condensed sets, we use homotopy type theory extended with 4 axioms.
Our axioms are strong enough to prove Markov's principle, LLPO and the negation of WLPO. 
We also define a type of open propositions, inducing a topology on any type. 
This leads to a (synthetic) topological study of Stone and compact Hausdorff spaces. 
Indeed all functions are continuous in the sense that they respect this induced topology, 
and this topology is as expected for these class of types.
For example, any map from the unit interval to itself is continuous in the usual epsilon-delta sense.
We also use the synthetic homotopy theory 
given by the higher types of homotopy type theory to define and 
work with cohomology.
As an application, we compute the cohomology of the interval and use this to prove Brouwer's fixed point theorem
internally. 
