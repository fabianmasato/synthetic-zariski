% latexmk -pdf -pvc main.tex
\documentclass{../util/zariski}


\title{A Foundation for Synthetic Stone Duality}

\begin{document}

\author{Felix Cherubini, Thierry Coquand, Freek Geerligs and Hugo Moeneclaey}

\maketitle

%\begin{abstract}
%In synthetic algebraic geometry (SAG) \cite{draft}, we study finitely presented algebras over a commutative ring. 
%In this work, we study countably presented Boolean algebras instead. 
%Where the finitely presented algebras over a commutative ring induce a Zariski topos, 
%%the opposite category of these 
%the countably presented Boolean algebras induce the topos of light condensed sets \cite{Scholze}. 
%\cite{draft} proposes an axiomatization of the Zariski topos in univalent homotopy type theory \cite{hott}. 
%In this work, we propose similar axioms, which we expect to be modelled by light condensed sets. 
%% Furthermore, spectra of countably presented Boolean algebras correspond to quotients of Cantor space
%% which is cool because reasons
%\end{abstract} 
%
\rednote{The following is a collection of notes on work in progress.}

\rednote{I'm cleaning up, not all results that were in older versions have a place yet.}
\begin{abstract}
  The language of homotopy type theory has proved to be appropriate as an internal language for various higher toposes, 
for example with Synthetic Algebraic Geometry for the Zariski topos.
In this paper we apply such techniques to the higher topos corresponding to the light condensed sets 
of Dustin Clausen and Peter Scholze.
This seems to be an appropriate setting to develop synthetic topology, similar to the work of 
Martín Escardó.
To reason internally about light condensed sets, we use homotopy type theory extended with 4 axioms.
Our axioms are strong enough to prove Markov's principle, LLPO and the negation of WLPO. 
We also define a type of open propositions, inducing a topology on any type. 
This leads to a (synthetic) topological study of Stone and compact Hausdorff spaces. 
Indeed all functions are continuous in the sense that they respect this induced topology, 
and this topology is as expected for these class of types.
For example, any map from the unit interval to itself is continuous in the usual epsilon-delta sense.
We also use the synthetic homotopy theory 
given by the higher types of homotopy type theory to define and 
work with cohomology.
As an application, we compute the cohomology of the interval and use this to prove Brouwer's fixed point theorem
internally. 

\end{abstract}


\tableofcontents

% Logic and Topology
% - Building blocks, explaining the types of Stone, Boole (and what we mean with countable)
% - Rules, stating the axioms and first consequences, including the omniscience principles
% - Topology on propositions, mentioning under what constructions open propositions are closed
% - Examples of closed/open propositions, explaining why Boole is discrete and Stone Hausdorff
% - Topology on Stone spaces, including the classification of closed spaces.
% - Compact Hausdorff spaces. 
% Directed Univalence
% - Tychonov
% - Directed univalence, link to Phoa
% Cohomology
% - Cohomology and the interval
% Appendix, 
%  - alternative formulations of axiom 2
%  - more details on technical constructions
%  - colimit presentation of countably presented Boolean algebras (I'm not sure where we actually use this)
%  - scott continuity instead of axiom 1 



%\section*{Introduction}
%TODO

\rednote{Right now I'm throwing texts relating our stuff to the literature here.}

In formal/point-free topology \cite{TODO}, 
we consider that a Boolean algebra $B$ represents a Stone space $Sp(B)$ and a map
$Sp(B') \to Sp(B)$ is represented by a map $B\rightarrow B'$; 
the map $Sp(B')\to Sp(B)$ is then said to be
{\em formally surjective} if the corresponding map $B\to B'$ is injective. 
In the topos of light condensed sets,this becomes a true duality, 
as we shall see in \Cref{FormalSurjectionsAreSurjections}.

\section*{Acknowledgements}
The idea to use the topological characterization of stone spaces as totally disconnected, compact Hausdorff spaces to prove \Cref{stone-sigma-closed} was explained to us by Martín Escardó.
We profited a lot from a discussion with Reid Barton and Johann Commelin. 
David Wärn noticed that Markov's principle (\Cref{MarkovPrinciple}) holds. 
At TYPES 2024, we had an interesting discussion with Bas Spitters on the topic of the article.


\section{Stone duality}
\subsection{Preliminaries}
%
%In this section, we introduce the type of countably presented Boolean algebras $\Boole$ and of Stone spaces $\Stone$. 
%Both of these types carry a natural category structure. 
%In later sections, we will axiomatize an anti-equivalence between these categories, 
%which is classically valid and called Stone duality. 
\begin{definition}
  A countably presented Boolean algebra $B$ is a Boolean algebra such that there merely are 
  countable sets $I,J$, 
  a set of generators $g_i,~{i\in I}$ and a set $f_j,~{j\in J}$ 
  of Boolean expressions over these generators 
  such that $B$ is equivalent to the quotient of the free Boolean 
  algebra over the generators by the relations
  $f_j=0$. We denote this algebra by $2[I]/(f_j)_{j:J}$.
\end{definition} 
\begin{remark}\label{BooleAsCQuotient}
By countable in the previous definition we mean sets that are 
merely equal to some decidable in $\N$. 
Note that any countably presented algebra is also merely of the form 
$2[\N]\rangle / (r_n)_{n:\N}$.
%, if we add dummy variables that we equate to $0$, and dummy relations that equate $0$ to itself.
% 0 is not special here right? 
\end{remark}


%We will call the family $(f_j)_{j\in J}$ as above a set of relations. 
%If $I,J$ are finite, we call $B$ a finitely presented Boolean algebra. 
%Once we have postulated the axiom of dependent choice, 
%in \Cref{secBooleAsColimits}
%we will be able to show that every countably presented algebra 
%is actually a colimit of a sequence of finitely presented Boolean algebras.
%They are therefore dual to pro-finite objects, which are used 
%in the theory of light condensed sets \cite{Scholze,Dagur,TODO}.

\begin{remark}
  We denote the type of countably presented Boolean algebras $\Boole$. 
  Note that this type does not depend on a choice of universe. 
  Also note $\Boole$ has a natural category structure. 
\end{remark}

\begin{example}
  If both the set of generators and relations are empty, we have the Boolean algebra $2$.
  The underlying set is $\{0,1\}$ and $0\neq_2 1$.
  Remark that $2$ is initial in $\Boole$. 
\end{example}
%\begin{remark}
%Note that any Boolean algebra must contain the elements $0,1$. 
%Therefore, $2$ is initial in $\Boole$. 
%\end{remark} 
%We can therefore use it to define points of objects in the category dual to that of countably presented Boolean algebras. 
%
%\subsection{Stone spaces}
\begin{definition}
  For $B$ a countably presented Boolean algebra, 
  we define $Sp(B)$ as the set of Boolean morphisms from $B$ to $2$. 
\end{definition}
\begin{definition}
  We define the predicate on types $\isSt$ by 
  \begin{equation}
    \isSt(X) := \sum\limits_{B : Boole} X = Sp(B)
  \end{equation} 
  A type $X$ is called \textit{Stone} if $\isSt(X)$ is inhabited.
\end{definition}

%\subsection{Examples}
\begin{example}
  \label{boolean-algebra-examples}
  \begin{enumerate}[(i)]
  \item There is only one Boolean map $2\to 2$, thus $Sp(2)$ is the singleton type $\top$. 
  \item   
    The tivial Boolean algebra is given by $2/(1)$. 
    We have $0=1$ in the trivial Boolean algebra. 
    As there cannot be a map from the trivial Boolean algebra into $2$ preserving both $0,1$, 
    the corresponding Stone space is the empty type $\bot$, 
  \item\label{ExampleBAunderCantor}   
    We denote by $C$ the Boolean algebra $2[\N]$.
    %given by $\N$ as a set of generators and no relations. We write $p_n$ for the generator corresponding to $n$.
    A morphism $C\to 2$ corresponds to a function $\mathbb N\to 2$, 
    which is a binary sequence. 
    The Stone space $Sp(C)$ of these binary sequences is denoted 
    $2^{\N}$ and called \notion{Cantor space}.
  \item\label{ExampleBAunderNinfty}
    We denote by $B_\infty$ for $C/(g_m\wedge g_n)_{m\neq n}$.
    A morphism $B_\infty\to 2$ corresponds to a function 
    $\mathbb N \to 2$ that hits $1$ at most once. 
    The corresponding Stone space is denoted by $\N_\infty$.
  \end{enumerate}
\end{example}

\begin{lemma}\label{ClosedPropAsSpectrum}
  For $\alpha:2^\N$, we have an equivalence of propositions: 
  $$
    (\forall_{n:\N} \alpha(n) = 0 )\leftrightarrow Sp(2/(\alpha(n))_{n:\N}).
  $$
\end{lemma}
\begin{proof}
  There is at most one $x:2\to 2$, and it can only satisfy 
  $x(\alpha(n)) = 0$ for all $n:\N$ iff 
  $\alpha(n) \neq_2 1$ for all $n:\N$. 
  As $2$ has underlying set $\{0,1\}$, we have $(\alpha(n) \neq_2 1) \to (\alpha(n) =_2 0)$. 
\end{proof}

\begin{remark}\label{BinftyTermsWriting}
%  In \Cref{N-co-fin-cp}, we will show 
  It can be shown that $B_\infty$ is equivalent to the Boolean algebra on 
  subsets of $\N$ which are finite or co-finite. 
  Under this equivalence, the generator $g_n$ is sent to the singleton $\{n\}$. 
  Because of this, we have that any $b:B_\infty$ can be written 
  either as $\bigvee_{i\in I_0} p_i$ or as $\bigwedge_{i\in I_0} \neg p_i$ for some finite $I_0\subseteq \N$. 
\end{remark}


%\begin{remark}
%  As Boolean algebras are rings, any relation of the form $f=g$ with both $f,g$ Boolean expressions 
%  can be written as $h=0$ with $h=f-g$ a Boolean expression. 
%\end{remark} 




\input{Rules/Axioms.tex}
\subsection{Anti-equivalence of $\Boole$ and $\Stone$}

\begin{remark}\label{SpIsAntiEquivalence}
Stone types will take over the role of affine scheme from \cite{draft}, 
and we repeat some results here. 
Analogously to Lemma 3.1.2 of \cite{draft}, 
for $X$ Stone, Stone duality tells us that $X = Sp(2^X)$. 
%
Proposition 2.2.1 of \cite{draft} now says that 
$Sp$ gives a natural equivalence 
\begin{equation}
   Hom_{\Boole} (A, B) = (Sp(B) \to Sp(A))
\end{equation}
Therefore $Sp$ is an embedding from $\Boole$ to any universe of types, and $\isSt$ is a proposition.

Its image, $\Stone$ also has a natural category structure.
By the above and Lemma 9.4.5 of \cite{hott}, the map $Sp$ defines a dual equivalence of categories between $\Boole$ and $\Stone$.
\end{remark}

\begin{lemma}\label{SpectrumEmptyIff01Equal}
  For $B:\Boole$, we have $0=_B1$ iff $\neg Sp(B)$.
\end{lemma}
\begin{proof}
  If $0=_B1$, there is no map $B\to 2$ respecting both $0$ and $1$, thus $\neg Sp(B)$. 
  Conversely, if $\neg Sp(B)$, then 
  $Sp(B)$ equals $\bot$, the spectrum of the trivial Boolean algebra. 
  As $Sp$ is an embedding, $B$ is equivalent to the trivial Boolean algebra, hence $0=_B1$. 
\end{proof}
\begin{corollary}\label{LemSurjectionsFormalToCompleteness}
 For $S:\Stone$, we have that $\neg \neg S \to || S ||$
\end{corollary}
\begin{proof}
  Let $B:\Boole$ and suppose $\neg \neg Sp(B)$. 
  Let $f:2 \to B$. If $f(0) = f(1)$ then $0=1$ in $B$, thus $\neg Sp(B)$, 
  contradicting our assumption. Hence $f(0)\neq f(1)$. 
  Hence by case distinction on $2$ we can show that $f$ is injective. 
%  $f x = f y$ implies $ x= y$. 
%  Thus $f$ is injective and 
  By \Cref{SurjectionsAreFormalSurjections} the map $Sp(B) \to Sp(2)$ is surjective, 
  thus $Sp(B)$ is merely inhabited. 
\end{proof} 

%\begin{corollary}\label{MoreConcreteCompleteness}
%  By the above and propositional completeness, we have that $||Sp(B)||$ iff $0\neq_B1$. 
%\end{corollary}



%SurjectionsFormalSurjections%We conclude this section on the anti-equivalence of Stone and $\Boole$ by a relating surjections to injections. 
%SurjectionsFormalSurjections%This theorem is actually equivalent to completeness of propositional logic, which we'll discuss in 
%SurjectionsFormalSurjections%\Cref{NotesOnAxioms}. 
%SurjectionsFormalSurjections%
%SurjectionsFormalSurjections%\begin{theorem}\label{FormalSurjectionsAreSurjections}
%SurjectionsFormalSurjections%  Let $f:A\to B$ be a map of countably presented Boolean algebras. 
%SurjectionsFormalSurjections%  If $f$ is injective, then the corresponding map $(\cdot) \circ f : Sp(B) \to Sp(A)$ is surjective. 
%SurjectionsFormalSurjections%\end{theorem}
%SurjectionsFormalSurjections%
%SurjectionsFormalSurjections%\begin{proof}
%SurjectionsFormalSurjections%  Assume $f$ injective and let $x:Sp(A)$.
%SurjectionsFormalSurjections%  By \Cref{FiberConstruction}, we have that $\left(\sum\limits_{y:Sp(B)} y\circ f = x \right) = Sp(B/R) $
%SurjectionsFormalSurjections%  for $R=f(G)$ for some countable $G\subseteq A$ with $x(g) = 0$ for all $g\in G$. 
%SurjectionsFormalSurjections%  By propositional completeness and \Cref{SpectrumEmptyIff01Equal}, 
%SurjectionsFormalSurjections%  it's sufficient to show that $0\neq_{B/R}1$. 
%SurjectionsFormalSurjections%  Note that $0=_{B/R} 1$ iff 
%SurjectionsFormalSurjections%  $1 =_B \bigvee R_0$ for some $R_0\subseteq R$ finite. 
%SurjectionsFormalSurjections%  But then $$1 = \bigvee f(G_0) = f(\bigvee  G_0)$$ for some $G_0\subseteq G$ finite. 
%SurjectionsFormalSurjections%  And as $f$ is injective, $\bigvee G_0 = 1$. 
%SurjectionsFormalSurjections%  However, 
%SurjectionsFormalSurjections%  $$
%SurjectionsFormalSurjections%  x(\bigvee G_0) = 
%SurjectionsFormalSurjections%  x(\bigvee_{g\in G_0} g ) = \bigvee_{g \in G_0} x(g) = \bigvee_{g\in G_0} 0 = 0$$
%SurjectionsFormalSurjections%  And as $x(1) = 1$, we get a contradiction. Therefore $0\neq_{B/R} 1$ as required. 
%SurjectionsFormalSurjections%\end{proof}  
%SurjectionsFormalSurjections%The converse to the above theorem is true as well, regardless of propositional completeness:
%SurjectionsFormalSurjections%\begin{lemma}\label{SurjectionsAreFormalSurjections}
%SurjectionsFormalSurjections%If $f:A\to B$ is a map in $\Boole$ and $(\cdot) \circ f :Sp(B) \to Sp(A)$ is surjective, 
%SurjectionsFormalSurjections%$f$ is injective. 
%SurjectionsFormalSurjections%\end{lemma}
%SurjectionsFormalSurjections%\begin{proof}
%SurjectionsFormalSurjections%  Suppose precomposition with $f$ is surjective. 
%SurjectionsFormalSurjections%  Let $a:A$ be such that $f(a)= 0$. 
%SurjectionsFormalSurjections%  By assumption, for every $x:A\to 2$, there is a $y:B\to 2$ with $y\circ f = x$. 
%SurjectionsFormalSurjections%  Consequentely $x(a) = y(f(a)) = y(0) = 0$. 
%SurjectionsFormalSurjections%  So $x(a) = 0$ for every $x:Sp(A)$. 
%SurjectionsFormalSurjections%  Thus $Sp(A) = Sp(A/\{a\})$, and as $Sp$ is an embedding, 
%SurjectionsFormalSurjections%  $A \simeq A/\{a\}$, and $a = 0$ in $A$. 
%SurjectionsFormalSurjections%  So whenever $f(a) = 0$, we have $a=0$. Thus $f$ is injective. 
%SurjectionsFormalSurjections%\end{proof}

\subsection{Principles of omniscience}
In constructive mathematics, we do not assume the law of excluded middle (LEM).
There are some principles called principles of omniscience that are weaker than LEM, which can be used to describe 
how close a logical system is to satisfying LEM.
References on these principles include \cite{HannesDiener, ReverseMathsBishop}.
In this section, we will show that two of them (MP and LLPO) hold, 
and one (WLPO) fails in our system.

\begin{theorem}[The negation of the weak lesser principle of omniscience ($\neg$WLPO)]\label{NotWLPO}
  \begin{equation}
    \neg \forall_{\alpha:2^\N} 
    ((\forall_{n:\N} \alpha(n) = 0 ) \vee \neg (\forall_{n:\N} \alpha(n) = 0))
  \end{equation}
%  We cannot decide for general $\alpha:2^\N$, whether $\forall_{n:\mathbb N} \alpha(n) = 0$.
%  It is not the case that the statement %There is no method which given $\alpha:2^\mathbb N$ decides whether 
%  $\forall_{n:\mathbb N} \alpha(n) = 0$ is decidable for general $\alpha:2^\mathbb N$. 
\end{theorem}
\begin{proof}
%  Such a decision method is a function 
  Let $f:2^\mathbb N \to 2$ such that 
  $f(\alpha) = 0$ iff $\forall_{n:\mathbb N} \alpha (n)= 0$. 
  By \Cref{AxStoneDuality}, there is some $c:C$ with 
  $f(\alpha) = 0 \leftrightarrow \alpha(c) = 0$. 
  We can express $c$ using finitely many generators $(g_n)_{n\leq N}$. 
  Now consider $\beta,\gamma:2^\N$ given by 
  $\beta(g_n) = 0$ for all $n:\mathbb N$ and
  $\gamma(g_n) = 0$ iff $n\leq N$. 
  As $\beta, \gamma$ are equal on $(g_n)_{n\leq N}$, we have $\beta(c) = \gamma(c)$. 
  However, $f(\beta) = 0$ and $f(\gamma) = 1$, giving a contradiction as required. 
%  We thus have a contradiction, thus a decision method as required doesn't exist. 
\end{proof}

The following result is due to David W\"arn:
\begin{theorem}[Markov's principle (MP)]\label{MarkovPrinciple}
  For $\alpha:\Noo$, we have that 
  \begin{equation}
    (\neg (\forall_{n:\mathbb N} \alpha (n)= 0)) \to \Sigma_{n:\mathbb N} \alpha (n)= 1
  \end{equation}
\end{theorem}
\begin{proof}
  By \Cref{ClosedPropAsSpectrum}, we have that $\neg(\forall_{n:\N} \alpha(n) = 0)$ implies that 
  $Sp(2/(\alpha(n))_{n:\N}$ is empty. 
%  We will show that the spectrum of $2/(\alpha(n))_{n:\N}$ is empty. 
%  Suppose $x:2\to 2$, if  $x(\alpha(n)) = 0$, we get $\alpha(n) \neq 1$. 
%  Thus if $\neg (\forall_{n:\N} \alpha(n) = 0$, we have $\neg Sp(2/(\alpha(n))_{n:\N})$.
  Hence $2/(\alpha(n))_{n:\N}$ is trivial by \Cref{SpectrumEmptyIff01Equal}. 
  Then there is a finite subset $N_0\subseteq \N$ with $\bigvee_{i:N_0} \alpha(i) = 1$. 
  As $\alpha(i) \in \{0,1\}$ and $\alpha(i) = 1$ for at most one $i:\N$, 
  there exists an unique $n\in\mathbb N$ with $\alpha(n) = 1$. 
%  Assume $\neg (\forall_{n:\mathbb N} \alpha (n)= 0)$.
%  It is sufficient to show that $2/\{\alpha(n)|n\in\N\}$ is the trivial Boolean algebra. 
%  It will then follow that there is a finite subset $N_0\subseteq \N$ 
%  with $\bigvee_{i:N_0} \alpha(i) = 1$.
%  As $\alpha(i) \in \{0,1\}$ and $\alpha(i) = 1$ for at most one $i$, it then follows that 
%  there exists an unique $n\in\mathbb N$ with $\alpha(n) = 1$. 
%%
%  To show that $2/\{\alpha(n)|n\in\N\}$ is trivial, we will show it has an empty spectrum. 
%  Suppose $x: 2 \to 2$ is such that $x(\alpha(n)) = 0$ for every $n:\N$. 
%  As $x(1) = 1$, we must have for every $n:\N$ that $\alpha(n) \neq 1$. 
%  But then $\alpha(n) = 0$, contradicting our assumption. 
%  We get a contradicition and there thus there are no points in the spectrum of $2/\{\alpha(n)|n\in\N\}$ as required. 
\end{proof}

\begin{corollary}
  For $\alpha:2^\mathbb N$, we have that 
  \begin{equation}
    (\neg (\forall_{n:\mathbb N} \alpha (n)= 0)) \to \Sigma_{n:\mathbb N} \alpha (n)= 1
  \end{equation}
\end{corollary}
\begin{proof}
  Given $\alpha:2^\mathbb N$, consider the sequence $\alpha':\Noo$ satisfying $\alpha'(n) = 1$ iff 
  $n$ is minimal with $\alpha(n) = 1$. Then apply the above theorem.
\end{proof}

\begin{theorem}[The lesser limited principle of omniscience (LLPO)]\label{LLPO}
  For $\alpha:\N_\infty$, 
  we have that 
  \begin{equation}\label{eqnLLPO}
    \forall_{k:\N} \alpha(2k) = 0  \vee \forall_{k:\N} \alpha(2k+1) = 0
  \end{equation}
\end{theorem}
\begin{proof}
%
%  We first will define a map $f:B_\infty \to B_\infty \times B_\infty$. 
%  Because of \Cref{rmkMorphismsOutOfQuotient}, it is sufficient to define $f$ on $(p_n)_{n:\N}$ with 
%  $f(p_n) \wedge f(p_m) = (0,0)$ for $n\neq m$. 
%  To define $f(p_n)$, we use a case distinction on whether $n$ is odd or even. 
  Define $f:B_\infty \to B_\infty \times B_\infty$ as follows:
  \begin{equation}\label{eqnLLPOProofMap}
    f(p_n) =\begin{cases}
      (p_k,0) \text{ if } n = 2k\\
      (0,p_k) \text{ if } n = 2k+1\\
    \end{cases}
  \end{equation}
  Note that $f$ is well-defined as map in $\Boole$. 
 % , can make a case distinction on parity. 
%  By making a case distinction on $n,m$ being odd or even, 
%  we can see that 
%  $f(p_n) \wedge f(p_m) = (0,0)$ when $n\neq m$, thus $f$ is well-defined. 
  We claim $f$ is injective. Assume $f(x) = 0$, 
  to see that $x=0$, we make a case distinction on whether $x$ corresponds to a finite or a cofinite set as in \Cref{BinftyTermsWriting}.
%
%  We also claim it is injective.
%  Now let $x:B_\infty$ with $f(x) = 0$. 
  We denote $E,O\subseteq \N$ for the even and odd numbers respectively. 
%  and we make a case distincition based on \Cref{BinftyTermsWriting}.
  \begin{itemize}
    \item Suppose 
      $x = \bigvee_{i\in I_0} g_i$ with $I_0$ finite. 
      Then 
      $$f(x) = (\bigvee_{i\in I_0 \cap E } g_{\frac i2} , \bigvee_{i\in I_0 \cap O } g_{\frac {i-1}2} ) = (0,0)$$
      As $g_j\neq 0$ for all $j\in\N$, we must have $I_0 \cap E = \emptyset = I_0 \cap O$. 
      Thus $I_0= \emptyset$, and $x = 0$. 
    \item Suppose 
%      Let $x$ correspond to a cofinite subset of $\N$. Write 
      $x = \bigwedge_{j\in J} \neg g_j$ with $J$ finite. % for $J$ finite. 
      We will derive a contradiction. %, from which we can conclude that $x=0$ after all. 
      Note that   
      $$f(x) = (\bigwedge_{j\in J \cap E } \neg g_j , \bigwedge_{j\in J \cap O } \neg g_j ) = (0,0)$$
%      As $f(x) = (0,0)$, we have that 
%      $\bigwedge_{j\in J \cap E } \neg p_j =0$ and
%      $\bigwedge_{j\in J \cap O } \neg p_j  = 0$.
      However, any finite meet of negations corresponds to a cofinite set, hence is nonzero. 
      We get a contradiction and conclude $x=0$. 
%      However, any finite meet of negations will correspond to a cofinite set,
%      in particular it will not correspond to the empty set, and thus not be $0$.
%      Thus $f(x)\neq 0$, contradicting the assumption that $f(x) = 0$, hence $x=0$ ex falso. 
  \end{itemize}
%  In both cases, we conclude $x=0$, thus $f$ is injective. 
  By \Cref{SurjectionsAreFormalSurjections},
%  \Cref{FormalSurjectionsAreSurjections}, 
  $f$ corresponds to a surjection 
  $s:\Noo + \Noo \to \Noo$.
  Thus for $\alpha : \Noo$, 
  there exists some $x:\Noo + \Noo$ such that $s x = \alpha$. 
  If $x = inl(\beta)$, 
  for any $k:\N$, we have that 
  $$\alpha (g_{2k+1}) = s(x) (g_{2k+1}) = x(f(g_{2k+1})) = inl(\beta) (0,g_k)  = \beta(0) = 0.$$
  Similarly, if $x = inr(\beta)$, we have $\alpha(g_{2k}) = 0$ for all $k:\N$. 
  Thus \Cref{eqnLLPO} holds for $\alpha$ as required. 
\end{proof}
As the following shows, our use of \Cref{SurjectionsAreFormalSurjections} was non-trivial: 
%The use of \Cref{FormalSurjectionsAreSurjections}, and hence of propositional completeness, 
%was helpful in the above proof, as the following shows:
\begin{lemma}
  The function $f$  as in \Cref{eqnLLPOProofMap} does not have a retraction. 
\end{lemma}
\begin{proof}
  Suppose $r:B_\infty \times B_\infty \to B_\infty$ is a retraction of $f$. 
  Note that $r(0,1):B_\infty$ is expressable using only finitely many generators $(g_n)_{n\leq N}$
  Note that $r(0,1) \geq r(0,g_k) = g_{2k+1}$ for all $k:\N$. 
  As a consequence, $r(0,1)$ cannot be of the form $\bigvee_{i\in I_0} g_i$, and by \Cref{BinftyTermsWriting}, 
  $r(0,1)$ corresponds to a cofinite subset of $\N$. % = \bigwedge_{i:I_0} \neg p_i$, where $i\leq N$ for $i\in I_0$. 
  By similar reasoning so does $r(1,0)$.% corresponds to a cofinite subset of $\N$. 
  But the intersection of cofinite subsets is cofinite, while 
  $$r(0,1) \wedge r(1,0) = r( (1,0) \wedge (0,1)) = r(0,0) = 0$$
  which gives a contradiction. Thus no retraction exists. 
\end{proof}


%We finish with an equivalent formulation of LLPO:
%
%
%\begin{lemma}\label{corAlternativeLLPO}
%  Let $(\phi_n)_{n:\N}, (\psi_m)_{m:\N}$ be families of decidable propositions indexed over $\N$.
%  We then have 
%  \begin{equation}
%    (\forall_{n:\N} \forall_{m:\N} (\phi_n \vee \psi_m) )
%    \leftrightarrow
%    ((\forall_{n:\N} \phi_n) \vee (\forall_{m:\N} \psi_m) )
%  \end{equation}
%\end{lemma}
%\begin{proof}
%  See \cite{HannesDiener, ReverseMathsBishop}
%\end{proof}
%\begin{proof}
%  Note that the implication from right to left in the above equation always holds.
%  Assume that for all $m,n:\mathbb N$ we have $\phi_n\vee \psi_m$ 
%  Consider the sequence $\alpha:2^\mathbb N$ where $\alpha(2n) = 0$ iff $\phi_n$ and 
%  $\alpha(2m+1) = 0$ iff $\psi_m$. 
%  Let $\beta:\Noo$ be such that $\beta(i) = 1$ iff $i$ is minimal with $\alpha(i) = 1$
%  By LLPO, we have that 
%  $\beta$ is $0$ on all odd entries or on all even entries. 
%  Suppose that $\beta$ hits $0$ on all odd entries. 
%  We will show $\psi_m$ for all $m:\N$. 
%  As $\beta(2m+1) = 0$, there are two options:
%  \begin{itemize}
%    	\item If $\alpha(l)=0$ for all $l\leq 2m+1$. Then in particular $\alpha(2m+1)=0$ and $\psi_m$ holds.
%	\item Otherwise there is some $l<2m+1$ with $\beta(l) = 1$. 
%  As $\beta$ hits $0$ on odd entries, $l$ is even. 
%  So $\alpha(2n) = 1$ for $n = \frac{l}2$, meaning that $\neg \phi_n$. 
%  By assumption, $\phi_n \vee \psi_m$ holds, hence $\psi_m$ must hold. 
%  Thus for all $m:\N$, we have $\psi_m$ if $\beta$ hits $0$ on all odd entries. 
%  By a symmetric argument, if $\beta$ hits $0$ on all even entries, we have $\phi_n$ for all $n:\N$. 
%  We conclude that 
%  $((\forall_{n:\N} \phi_n) \vee (\forall_{m:\N} \psi_m) )$ 
%  as required. 
%  \end{itemize}
%\end{proof}
%
%\begin{remark}
%Note that the above statement implies LLPO as $\alpha(2n) =0 \vee \alpha(2m+1) =0$ for all $n,m:\mathbb N$ if $\alpha:\Noo$. 
%\end{remark}


\section{Topology of Stone spaces}
%In this section, we will define the types of open and closed propositions. 
%These will allow us to define a (synthetic) topology  \cite{SyntheticTopologyLesnik} on any type.
%We will study this topology on Stone types in particular.
%
\subsection{Open and closed propositions}
In this section we will introduce a topology on the type of propositions, and 
study their logical properties.
We think of open and closed propositions respectively as countable disjunctions and conjunctions of decidable propositions.
Such a definition is universe-independent, and can be made internally.

\begin{definition}
A proposition $P$ is open (resp. closed) if there exists some $\alpha:2^\N$ such that $P \leftrightarrow \exists_{n:\mathbb N} \alpha_n = 0$ (resp. $P \leftrightarrow \forall_{n:\mathbb N} \alpha_n = 0$). We denote by $\Open$ and $\Closed$ the types of open and closed propositions.
\end{definition}

\begin{remark}\label{rmkOpenClosedNegation}
  The negation of an open proposition is closed, 
  and by MP (\Cref{MarkovPrinciple}), the negation of a closed proposition is open %. 
%  Also by MP, we have 
  and both open, closed propositions are $\neg\neg$-stable. 
%  and $\neg \neg P \to P$ whenever $P$ is open or closed. 
%  By the negation of WLPO (\Cref{NotWLPO}), 
  By $\neg$WLPO (\Cref{NotWLPO}), 
  not every closed proposition is decidable. 
  Therefore, not every open proposition is decidable. 
  % Both therefore and similarly can be used here, by a similar proof we can show it, or we can use that 
  % if $P$ is closed and $\neg P$ is decidable, so is $\neg \neg P = P$. 
  Every decidable proposition is both open and closed.
%  and in \Cref{ClopenDecidable} we shall see the converse. 
\end{remark}
\begin{lemma}
  We have the following results on open and closed propositions:
  \begin{itemize}
%    \item Closed propositions are closed under finite disjunctions. 
    \item Closed propositions are closed under countable conjunctions. 
    \item Open propositions are closed under finite conjunctions. 
    \item Open propositions are closed under countable disjunctions. 
  \end{itemize}
\end{lemma}
\begin{proof}
%  By Proposition 1.4.1 of \cite{HannesDiener}, LLPO (\Cref{LLPO}) is equivalent to the statement that 
%  the disjunction of two closed propositions are closed. 
  The statements have similar proofs, and we only present the proof that closed propositions are closed under 
  countable conjunctions. 
  Let $(P_n)_{n:\N}$ be a countable family of closed propositions. 
  By countable choice, for each 
  $n:\N$ we have an $\alpha_n:2^\N $ 
  such that $P_n \leftrightarrow \forall_{m:\N} \alpha_{n,m} =0$. 
  Consider a surjection $s:\N \twoheadrightarrow \N \times \N$, and let 
%  Let 
%  $$\beta_k = \alpha_{s(k)}.$$
  $\beta_k = \alpha_{s(k)}.$
  Note that $\forall_{k:\N} \beta_k = 0$ if and only if 
%  $\forall_{m,n:\N}\alpha_{m,n} = 0$, which happens if and only if 
  $\forall_{n:\N} P_n$. 
%  Hence the countable conjunction of closed propositions is closed. 
\end{proof}
\begin{remark}
  LLPO (\Cref{LLPO}) is equivalent to the statement that for $P,Q$ open, we have 
  $(\neg P \vee \neg Q) \leftrightarrow \neg (P\wedge Q)$. 
\end{remark}
%\begin{proof}
%  Assuming the above statement, let $\alpha:\Noo$, and consider 
%  the open propositions 
%  $\exists_{k:\mathbb N}\alpha_{2k+1} = 1, \exists_{k:\N} \alpha_{2k} = 1$. 
%  As $\alpha:\Noo$ there's at most one $n:\mathbb N$ with $\alpha_n =1$, so their conjunction is false. 
%  By the above statement, it follows one of them is false, leading to the statement of LLPO. 
%  
%  Now suppose LLPO holds, and assume $\neg (\exists_{n:\mathbb N} \alpha_n = 1\wedge \exists_{n:\mathbb N} \beta_n = 1$
%  for $\alpha,\beta :\Noo$. Then the sequence $\gamma:2^\N$ given by $\gamma_{2n} = \alpha_n, \gamma_{2n+1} = \beta_n$
%  can hit $1$ at most once and thus induces a sequence in $\Noo$. LLPO then shows that 
%  $\neg \forall_{n:\mathbb N} \alpha_n = 1\vee \neg \forall_{n:\mathbb N} \beta_n=1$ as required. 
%\end{proof}
\begin{corollary}
  Closed propositions are closed under finite disjunctions.
\end{corollary}
\begin{proof}
  Closed propositions are negations of open propositions. 
  As the conjunction of two open propositions is open, LLPO gives that 
  the disjunction of two closed propositions is closed. 
\end{proof}
We will use the above properties silently from now on. 
%OneBigLemma#
%OneBigLemma#\rednote{Phrase the following lemmas as one big lemma, 
%OneBigLemma#and use them silently without reference, also we should just state $\neg\neg$-stability instead of referring to the above all the time. }
%OneBigLemma#
%OneBigLemma#\begin{lemma}\label{ClosedCountableConjunction}
%OneBigLemma#  Closed propositions are closed under countable conjunctions.
%OneBigLemma#\end{lemma}
%OneBigLemma#\begin{proof}
%OneBigLemma#  Let $(P_n)_{n:\N}$ be a countable family of closed propositions. 
%OneBigLemma#  By countable choice, for each 
%OneBigLemma#  $n:\N$ we have an $\alpha_n:2^\N $ 
%OneBigLemma#  such that $P_n \leftrightarrow \forall_{m:\N} \alpha_{n,m} =0$. 
%OneBigLemma#  Consider a surjection $s:\N \twoheadrightarrow \N \times \N$, and let 
%OneBigLemma#%  Let 
%OneBigLemma#%  $$\beta_k = \alpha_{s(k)}.$$
%OneBigLemma#  $\beta_k = \alpha_{s(k)}.$
%OneBigLemma#  Note that $\forall_{k:\N} \beta_k = 0$ if and only if 
%OneBigLemma#%  $\forall_{m,n:\N}\alpha_{m,n} = 0$, which happens if and only if 
%OneBigLemma#  $\forall_{n:\N} P_n$. 
%OneBigLemma#  Hence the countable conjunction of closed propositions is closed. 
%OneBigLemma#\end{proof} 
%OneBigLemma#Using similar arguments, we can show the following two lemmas:
%OneBigLemma#\begin{lemma}\label{OpenCountableDisjunction}
%OneBigLemma#  Open propositions are closed under countable disjunctions. 
%OneBigLemma#\end{lemma}
%OneBigLemma#\begin{lemma}\label{OpenFiniteConjunction}
%OneBigLemma#Open propositions are closed under finite conjunctions. 
%OneBigLemma#\end{lemma}
%OneBigLemma#%\begin{proof}
%OneBigLemma#%We use \Cref{ClosedFiniteDisjunction} and the fact that $\neg(P\lor Q) \leftrightarrow \neg P \land \neg Q$.
%OneBigLemma#%\end{proof}
%OneBigLemma#%\begin{proof}
%OneBigLemma#%  Similar to the previous lemma. 
%OneBigLemma#%\end{proof}
%OneBigLemma#\begin{lemma}\label{ClosedFiniteDisjunction} 
%OneBigLemma#  Closed propositions are closed under finite disjunctions. 
%OneBigLemma#\end{lemma}
%OneBigLemma#\begin{proof}
%OneBigLemma#  This statement is equivalent to LLPO (\Cref{LLPO}) by  
%OneBigLemma#  Proposition 1.4.1 of \cite{HannesDiener}. 
%OneBigLemma#%  , LLPO is equivalent to the statement that 
%OneBigLemma#%  for $(\phi_n)_{n:\N}, (\psi_m)_{m:\N}$ families of decidable propositions indexed over $\N$, we have
%OneBigLemma#%  \begin{equation}
%OneBigLemma#%    (\forall_{n:\N} \forall_{m:\N} (\phi_n \vee \psi_m) )
%OneBigLemma#%    \leftrightarrow
%OneBigLemma#%    ((\forall_{n:\N} \phi_n) \vee (\forall_{m:\N} \psi_m) )
%OneBigLemma#%  \end{equation}
%OneBigLemma#%%  $(\forall_{n:\N} \alpha(n) = 0 )\vee (\forall_{n:\N} \beta(n) = 0 )$ is closed for any $\alpha,\beta:2^\N$.
%OneBigLemma#%%  By \Cref{corAlternativeLLPO}, the statement is equivalent to 
%OneBigLemma#%%  $ \forall_{n:\N}  \forall_{m:\N}  (\alpha(n) = 0 \vee \beta(m) = 0)$, 
%OneBigLemma#%  The latter which is a countable conjunction of decidable propositions, 
%OneBigLemma#%  hence closed by \Cref{ClosedCountableConjunction}.
%OneBigLemma#\end{proof}
%OneBigLemma#
\begin{corollary}\label{ClopenDecidable}
  If a proposition is both open and closed, it is decidable. 
\end{corollary}
\begin{proof}
  If $P$ is open and closed, %$\neg P$, and hence 
  $P\vee \neg P$ is open, 
  hence $\neg\neg$-stable and provable. 
%  and we conclude by $\neg\neg$-stability of open propositions. 
%  but open propositions are $\neg\neg$-stable by \Cref{rmkOpenClosedNegation} so we can conclude.
%  hence 
 % equivalent to $\neg \neg (P \vee \neg P)$ by \Cref{rmkOpenClosedNegation}.
 % As the latter proposition is provable, we may conclude $P$ is decidable. 
%  
%  If $P$ is open and closed, $P\vee \neg P$ is open, hence
%  equivalent to $\neg \neg (P \vee \neg P)$, which is provable. 
\end{proof}


%\begin{lemma}\label{OpenFiniteConjunction}
%  Open propositions are closed under finite conjunctions. 
%\end{lemma}
%\begin{proof}
%  We need to show that for any $\alpha,\beta:2^\N$, the following proposition is open:
%  \begin{equation}\label{eqnConjunctionOpen}
%    (\exists_{n:\N} \alpha(n) = 0 )\wedge(\exists_{n:\N} \beta(n) = 0 )
%  \end{equation}
%  Consider $\gamma:2^\N$ given by 
%  $\gamma(l) = 1$ iff there exist some $k,k'\leq l$ with 
%  $\alpha(k) = \beta(k') = 0$. 
%  As we only need to check finitely many combinations 
%  of $k,k'$, this is a decidable property for each $l:\N$ and $\gamma$ is well-defined. 
%  Then $\exists_{k:\N}\gamma(k)=0$ if and only if \Cref{eqnConjunctionOpen} holds.
%\end{proof}

\begin{lemma}\label{ClosedMarkov}
  For $(P_n)_{n:\N}$ a sequence of closed propositions, we have 
  $\neg \forall_{n:\N} P_n \leftrightarrow  \exists_{n:\N} \neg P_n$. 
\end{lemma}
\begin{proof}
  Both $\neg \forall_{n:\N} P_n$ and $\exists_{n:\N} \neg P_n$ are open, hence $\neg\neg$-stable.
  The equivalence follows. 
%  and the equivalence follows. 
%  from which the equivalence follows. 
%  We have that $\forall_{n:\N}P_n$ is closed and $\exists_{n:\N} \neg P_n$ is open by \Cref{OpenCountableDisjunction}, therefore both are $\neg\neg$-stable by \Cref{rmkOpenClosedNegation} and we can conclude.
%It is always the case that $\exists_{n:\N}\neg P_n \to \neg \forall_{n:\N} P_n$. 
  %For the converse direction,
  %note that $\neg \exists_{n:\N} \neg P_n(x) \to \forall_{n:\N} \neg \neg P_n(x).$
  %By \Cref{rmkOpenClosedNegation}, $\neg \neg  P_n(x)\leftrightarrow P_n(x)$ for all $n:\N$. 
  %It follows that 
  %$\neg \forall_{n:\N} P_n(x)\to 
  %\neg \neg \exists_{n:\N} \neg P_n(x).$
  %As $\exists_{n:\N}\neg P_n(x)$ is a countable disjunction of open propositions, 
  %it is open by \Cref{OpenCountableDisjunction} and thus equivalent to 
  %$\neg\neg\exists_{n:\N} \neg P_n(x)$ by \Cref{rmkOpenClosedNegation}.
  %We conclude that $\neg \forall_{n:\N} P_n \to \exists_{n:\N} \neg P_n$ as required. 
\end{proof} 

%\begin{lemma}\label{OpenDependentSums}
%  Open propositions are closed under dependent sums.
%\end{lemma}
%\begin{proof}
%  \rednote{If we show that Open propositions are exactly the overtly discrete ones, this is implied by $\Sigma$-closure}
%  First note that for $D$ a decidable proposition, and $X:D \to \Open$,
%  by case splitting on $D$, we can see 
%  $\Sigma_{d:D} X(d)$ is open.
%%
%  Then note that for $P$ an open proposition, 
%  there exists a sequence of decidable propositions $A_n$ with 
%  $P = \exists_{n:\N} A_n $.
%%
%  So for $Y : P \to Open $, the dependent sum $\Sigma_P Y$ is given by 
%  $\exists_{n:\N} (\Sigma_{a:A_n} Y(n,a))$,
%  which is a countable disjunction of open propositions, 
%  hence open by \Cref{OpenCountableDisjunction}.
%\end{proof}
%
%We will see the same holds for closed propositions in \Cref{ClosedDependentSums}.
%
%\begin{remark}\label{ImplicationOpenClosed}
%  If $P$ is open, $P \to \bot$ is only open if $P$ is decidable, which is not in general the case. 
%  Thus $\Open$ is not closed under dependent products. Neither is $\Closed$. 
%  However, as $(P\to Q)  \to \neg \neg (\neg P \vee Q)$,
%  we have that if $P$ is open and $Q$ is closed, then $P\to Q$ is closed, and similarly $Q\to P$ is open.
%\end{remark}
\begin{lemma}\label{ImplicationOpenClosed}
  If $P$ is open %(resp. closed) 
  and $Q$ is closed % (resp. open) 
  then $P\to Q$ is closed. % (resp. open). 
  If $P$ is closed and $Q$ open, then $P\to Q$ is open. 
\end{lemma}
\begin{proof}
  Note that $\neg P \vee Q$ is closed. Using $\neg\neg$-stability
  we can show $(P\to Q) \leftrightarrow (\neg P \vee Q)$. 
  The other proof is similar. 
%  and we conclude by 
%  Assume $P$ open and $Q$ closed, the other proof is similar. 
%  Note that $(\neg P \vee Q) \to (P \to Q)$ and 
%  $(P\to Q)\to \neg\neg(\neg P \vee Q)$. 
%  By \Cref{rmkOpenClosedNegation} it follows that 
%  $(\neg P \vee Q)\leftrightarrow (P \to Q)$, and using \Cref{ClosedFiniteDisjunction}, 
%  we can conclude that $P\to Q$ is closed. 
\end{proof}
%
%The following question was asked by Bas Spitters at TYPES 2024:


\subsection{$\Open$ and $\ODisc$} %Open propositions are Overtly discrete and Stone propositions.}
\begin{lemma}
  Whenever $P$ is a proposition and overtly discrete, $P$ is open. 
\end{lemma}
\begin{proof}
  If $P$ is overtly discrete, then $P\leftrightarrow \exists_{n:\N} P_n$. 
  As every $P_n$ is finite, it is decidable. 
  Hence $P$ is a countable disjunction of decidable propositions, hence open.% by \Cref{OpenCountableDisjunction}. 
\end{proof}
\begin{lemma}
  Whenever $P$ is a an open proposition, it is overtly discrete.
\end{lemma}
\begin{proof}
  Suppose $P\leftrightarrow \exists_{n:\N} \alpha_n = 1$. 
  Let $P_n = \exists_{k\leq n} (\alpha_k = 1)$, which is a decidable proposition, hence a finite set. 
  Then the colimit of $P_n$ is $P$. 
\end{proof} 
\begin{corollary}\label{PropOpenIffOdisc}
  A proposition is open iff it is overtly discrete.
\end{corollary}
\begin{proof}
  Immediate by the above two lemmas. 
\end{proof}
\begin{corollary}\label{OpenDependentSums}
  Open propositions are closed under dependent sums. 
\end{corollary}
\begin{proof}
  Immediate from \Cref{OdiscSigma} and \Cref{PropOpenIffOdisc}.
\end{proof}
\begin{remark}
  Note that the sequential colimit commutes with the propositional truncation, thus for $B:\ODisc$, we have 
  $||B||:\ODisc$. 
\end{remark}
%\begin{corollary}
%  Open propositions are closed under countable disjunctions. 
%\end{corollary}
%\begin{proof}
%  Clearly $\N:\ODisc$, and for $P:\N \to \Open$, and by the above 
%  $||\Sigma_{n:\N} P_n||:\ODisc$. 
%\end{proof} 

%\begin{corollary}\label{OpenFiniteConjunction}
%  Open propositions are closed under finite conjunctions. 
%\end{corollary}
%\begin{proof}
%  A conjunction of propositions is a product, which is a dependent sum. 
%\end{proof}
%FollowsAlsoFromNextLemma%\begin{lemma}\label{ODiscEqualityOpen}
%FollowsAlsoFromNextLemma%  Whenever $B$ is overtly discrete and $a,b:B$, the proposition $a=_B b$ is open. 
%FollowsAlsoFromNextLemma%\end{lemma}
%FollowsAlsoFromNextLemma%\begin{proof}
%FollowsAlsoFromNextLemma%  For $a,b:B$ there is some $n:\N$ with $a',b':B_n$ and $\iota_n(a') = a,\iota_n(b') = b$.
%FollowsAlsoFromNextLemma%  By \Cref{rmkEqualityColimit}, we have that $a=_B b$ iff 
%FollowsAlsoFromNextLemma%  there is some $m\geq n$ with $\iota_n^m (a') = \iota_n^m(b')$. 
%FollowsAlsoFromNextLemma%  As equality in finite sets is decidable, this is a countable disjunction of decidable propositions, hence open. 
%FollowsAlsoFromNextLemma%%  \rednote{That reference can also be a direct cite}.
%FollowsAlsoFromNextLemma%\end{proof}

\begin{lemma}
  A type $B$ is overtly discrete iff it is the quotient of a countable set by an open equivalence relation. 
\end{lemma}
\begin{proof}
  Is $B:\ODisc$, then $B= \Sigma{n:\N} B_n/\sim_B$ where $\sim_B$ is the reflexive closure of  
  $(n,b)\sim(m,\iota_n^m b)$ for $n\leq m$. 
%
  Conversely, assume $B= D/R$ with $D\subseteq \N$ decidable and $R$ open. 
  By countable choice we get $\alpha_{(\cdot,\cdot)}:D \to D \to 2^\N$ such that 
  $R(x,y)\leftrightarrow \exists_{n:\N}\alpha(n) = 1$. 
  Define $D_n = (D \cap \N_{\leq n})$, and $R_n : D_n \to D_n \to 2$ so that $R_n(x,y)$ checks whether
  $\alpha_{(x,y)}(k) =1$ for some $k\leq n$. 
  Note that $B_n = D_n/R_n$ is a finite set, and has colimit $B$. 
\end{proof}
\begin{remark}\label{BooleEpiMono}
  In particular equality in overtly discrete types is open. 
  By \Cref{OdiscSigma}, it follows that any $g:B\to C$ in $\Boole$ has an overtly discrete kernel, 
  which must therefore be enumerable. Hence $B/Ker(g)$ is in $\Boole$. 
  By uniqueness of epi-mono factorizations and \Cref{SurjectionsAreFormalSurjections}, the factorization 
  $B\twoheadrightarrow B/Ker(g) \hookrightarrow C$ corresponds to 
  $Sp(C) \twoheadrightarrow Sp(B/Ker(g)) \hookrightarrow Sp(B)$. 
\end{remark}


\subsection{$\Closed$ and $\Stone$}
%\begin{lemma}\label{BooleEqualityOpen}
%  Whenever $B:\Boole$, $a,b:B$ the proposition $a=_Bb$ is open. 
%\end{lemma}
%\begin{proof}
%  Let $G,R$ be the generators and relations of $B$. 
%  Let $a,b$ be represented by $x,y$ in the free Boolean algebra on $G$. 
%  Now let $R_n$ denote the first $n$ elements of $R$. 
%  Note that $a=b$ iff there exists some $n:\N$ with $x-y \leq \bigvee_{r\in R_n} r$. 
%  Furthermore, inequality is decidable in the free Boolean algebra, hence
%  $a=b$ is a countable disjunction of decidable propositions, hence open. 
%\end{proof}

\begin{corollary}\label{TruncationStoneClosed}
  Whenever $S:\Stone$, $||S||$ is closed. 
\end{corollary}
\begin{proof}
  By \Cref{SpectrumEmptyIff01Equal}, $\neg S$ is equivalent to $0=_B 1$, which is open by the above. 
  Hence $\neg \neg S$ is a closed proposition, and by \Cref{LemSurjectionsFormalToCompleteness}, so is $||S||$. 
\end{proof}
%\begin{remark}\label{ExplicitTruncationStoneClosed}
%  \rednote{New check later}
%  The above lemma and corollary actually show that if we have an explicit 
%  presentation of a Stone space as $S = Sp(2[G] / R)$, 
%  we can construct an explicit sequence $\alpha:2^\N$ such that $||S|| \leftrightarrow \forall_{n:\N} \alpha(n) = 0$. 
%\end{remark}


\begin{corollary}\label{PropositionsClosedIffStone}
  A proposition $P$ is closed iff it is a Stone space. 
\end{corollary}
\begin{proof}
  By the above, if $S$ is both a Stone space and a proposition, it is closed. 
  By \Cref{ClosedPropAsSpectrum}, any closed proposition is Stone. 
\end{proof}

\begin{lemma}\label{StoneEqualityClosed}
  Whenever $S:\Stone$, and $s,t:S$, the proposition $s=t$ is closed. 
\end{lemma}
\begin{proof}
  Suppose $S= Sp(B)$ and let $G$ be the generators of $B$. 
  Note that $s=t$ iff $s(g) =_2 t(g)$ for all $g:G$. 
  As $G$ is countable, and equality in $2$ is decidable, 
  $s=t$ is a countable conjunction of decidable propositions, hence 
  closed. 
\end{proof}
%
The following question was asked by Bas Spitters at TYPES 2024:
\begin{corollary}
  For $S:\Stone$ and $x,y,z:S$ 
  \begin{equation}\label{Apartness}
  x \neq y \to (x\neq z \vee y \neq z)
  \end{equation}
\end{corollary}
\begin{proof}
  As $x\neq y$, we can show that $\neg ( x = z \wedge y = z)$. 
  This in turn implies $\neg \neg ( x \neq  z \vee y \neq  z)$. 
  As, $x\neq z$ and $y \neq z$ are both open propositions, by \Cref{OpenCountableDisjunction} so is their disjunction. 
  By \Cref{rmkOpenClosedNegation}, that disjunction is double negation stable and \Cref{Apartness} follows. 
\end{proof}
\begin{remark}
  If \Cref{Apartness} holds in a type, we say that it's inequality is an apartness relation. 
  By a similar proof as above, it can be shown that in our setting inequality is an apartness relation 
  as soon as equality is open or closed. 
\end{remark}

\subsection{Types as spaces}
The subobject $\Open$ of the type of propositions induces a topology on every type. 
This is the viewpoint taken in synthetic topology. 
We will follow the terminology of \cite{SyntheticTopologyLesnik}, 
other references include \cite{SyntheticTopologyEscardo}%, TODOSortOutTaylorsReferences}.
%Defining a topology in this way has some benefits, which we summarize in this section. 

\begin{definition}
  Let $T$ be a type, and let $A\subseteq T$ be a subtype. 
  We call $A\subseteq T$ open or closed iff $A(t)$ is open or closed respectively for all $t:T$.
\end{definition}

\begin{remark}
  It follows immediately that the pre-image of an open by any map of types sends is open, so that any map is continuous. 
%  This is only relevant for a space if the topology we defined above matches the topology one would expect. 
  In \Cref{StoneClosedSubsets}, we shall see that the resulting topology is as expected for second countable Stone spaces.
  In \Cref{IntervalTopologyStandard}, we shall see that the same for the unit interval. 
\end{remark}



%\begin{remark}
%  Phao's principle is a special case of directed univalence. 
%\end{remark}
%\begin{proof}
%  \rednote{TODO}
%\end{proof}

\subsection{The topology on Stone spaces}
\begin{theorem}\label{StoneClosedSubsets}
  Let $A\subseteq S$ be a subset of a Stone space. The following are equivalent:
  \begin{enumerate}[(i)]
    \item There exists a map $\alpha:S \to 2^\N$ such that 
      $A (x) \leftrightarrow \forall_{n:\N} \alpha_{x,n} = 0$ for any $x:S$. 
    \item There exists a family 
      $(D_n)_{n:\N}$ 
      of decidable subsets of $S$ such that $A = \bigcap_{n:\N} D_n$. 
    \item There exists a Stone space $T$ and some embedding $T\to S$ which image is $A$
    \item There exists a Stone space $T$ and some map $T\to S$ which image is $A$. 
    \item $A$ is closed.
  \end{enumerate}
\end{theorem}
\begin{proof}
\item 
  \begin{itemize}
  \item 
    $(i)\leftrightarrow (ii)$. $D_n$ and $\alpha$ can be defined from each other by 
     $D_n(x) \leftrightarrow (\alpha_{x,n} = 0)$. Then observe that
     \begin{equation}
     x\in \bigcap_{n:\N} D_n \leftrightarrow 
      \forall_{n:\mathbb N} (\alpha_{x,n} = 0) 
     \end{equation}
     
   \item $(ii) \to (iii)$. Let $S=Sp(B)$. 
%      By Stone duality, we have $d_n,~n:\N$ terms of $B$ such that $D_n = \{x:S| x(d_n) = 1\}$. 
%      Let $C = B/(\neg d_n)_{n:\N}$.
%      Then the map $Sp(C) \to S$ is as desired because
%      $$Sp(C) = \{x:S| \forall_{n:\N} x(\neg d_n) =0\}  = \bigcap_{n:\N} D_n.$$
      By \Cref{AxStoneDuality}, we have $(d_n)_{n:\N}$ in $B$ such that $D_n = \{x:S\ |\ x(d_n) = 0\}$. 
      Let $C = B/(d_n)_{n:\N}$.
      Then $Sp(C) \to S$ is as desired because:
      $$Sp(C) = \{x:S\ |\ \forall_{n:\N} x(d_n) =0\}  = \bigcap_{n:\N} D_n.$$
%      By \Cref{SurjectionsAreFormalSurjections}, t
%      The quotient map $B \twoheadrightarrow C$
%      corresponds to a map $\iota:Sp(C) \hookrightarrow  S$. 
%      For $s:S$, $s$ lies in the image of this map iff 
%      for all $n:\N$ we have  $s(\neg d_n) = 0$, 
%      \begin{equation}
%        x\in \iota(Sp(C)) \leftrightarrow x(\neg d_n) = 0 \leftrightarrow x(d_n) = 1 \leftrightarrow x\in D_n
%      \end{equation}
%      Thus the image of $\iota$ is given by $\bigcap_{n:\N} D_n$. 
   \item $(iii) \to (iv)$. Immediate.
   \item $(iv) \to (ii)$. Assume $f:T\to S$ corresponds to $g:B\to C$ in $\Boole$. 
     By \Cref{BooleEpiMono}, $f(T) = Sp(B/Ker(g))$, and 
%     by \Cref{OdiscQuotientCountableByOpen}
     there is a surjection $d:\N\to Ker(g)$. Denote by $D_n$ the decidable subset of $S$ corresponding to $d_n$. Then we have that $Sp(B/Ker(g)) = \bigcap_{n:\N} D_n$. 
%     Note that the factorization 
%     $B\twoheadrightarrow B/Ker(g) \hookrightarrow C$ 
%     corresponds to the factorization 
%     $S\twotheadleftarrow f(T) \hookleftarrow T$
%%     We have a surjection $B\twoheadrightarrow B/Ker(g)$ in $\Boole$, 
%%     which corresponds to the inclusion $f(T) = Sp(B/Ker(g)) \subseteq S$. 
%     Ans $f(T) = Sp(B/Ker(g))$
%
%     Therefore $B/Ker(g):\Boole$, 
%     and the quotient map $B\twoheadrightarrow B/Ker(g)$ induces an inclusion 
%     $Sp(B/Ker(g)) \hookrightarrow Sp(B)$ with 
%     $f\in Sp(B/Ker(g))\leftrightarrow \forall_{n:\N}f(d_n) = 0 $. 
%     
%
%
%
%
%
%oldProof%     \rednote{TODO 
%oldProof%       The order of untracating is important in this proof, 
%oldProof%     and I struggle a bit with stressing this in a way this is clear (and concise). 
%oldProof%    Check with fresh eyes later. }
%oldProof%      Let $f:T\to S$ be a map between Stone spaces. 
%oldProof%      Assume $S = Sp(A), T = Sp (B)$. 
%oldProof%%      For this proof, we work with explicit presentations for $A,B$. 
%oldProof%%
%oldProof%      Let $G$ be a countable set of generators of $A$. 
%oldProof%      Assume also we have countable sets of generators and relations for $B$. 
%oldProof%%
%oldProof%      Following \Cref{FiberConstruction}, using $G$, for each $x:S$, we can construct 
%oldProof%      a countable set $I_x\subseteq B$ such that $$Sp(B/I_x) = (\Sigma_{y:T} f(y) = x) .$$
%oldProof%      By \Cref{ExplicitTruncationStoneClosed}, we can construct a sequence 
%oldProof%      $\alpha_x$ such that this type is inhabited iff $\forall_{n:\N} \alpha_x(n) = 0$,
%oldProof%      as required. 
%
%      Recall that the propositional truncation of a Stone is closed, as it is the negation of $0=1$ in the underlying 
%      Boolean algebra, which is open as it f
%
%
%      the core idea of the proof was that the closed proposition corresponds to checking equality in the underlying BA, 
%      which was closed as 
%
%
%
%
%      Note that $x$ in the image of $f$ iff $0\neq_{B/I_x} 1$. 
%      At this point, we have generators and relations of $B/I_x$ as data.
%      Hence using the proof of \Cref{BooleEqualityOpen}, we can construct a sequence 
%      $\alpha_x:2^\N$ such that $0 =_{B/I_x}1\leftrightarrow \exists_{n:\N} \alpha_x(n) = 0$. 
%      And for $\beta_x(n) = 1-\alpha_x(n)$, we conclude that 
%      \begin{equation}
%        x\in f(T) \leftrightarrow \forall_{n:\N} \beta_x(n) = 0
%      \end{equation}
%      Note that we did not use any choice axioms in the proof of this implication,
%      as we untruncated our assumptions before we specified $x$. 
   \item $(i) \to (v)$. By definition.
   \item $(v) \to (iv)$.
     %As $A$ is closed, it corresponds to a map $a:S\to \Closed$. 
     We have a surjection $2^\N\to\Closed$ defined by $\alpha \mapsto \forall_{n:\mathbb N} \alpha_n = 0.$
     \Cref{AxLocalChoice} gives us that there merely exists $T, e, \beta_\cdot$ as follows:
     \begin{equation}
       \begin{tikzcd}
         T \arrow[r,"\beta"] \arrow[d, two heads,swap,"e"] & 2^\mathbb N 
         \arrow[d,two heads] \\
         S \arrow[r,swap,"A"] & \Closed
       \end{tikzcd} 
     \end{equation} 
     Define $B(x) \leftrightarrow \forall_{n:\mathbb N} \beta_{x,n} = 0$. 
     As $(i) \to (iii)$ by the above, $B$ is the image of some Stone space. 
     Note that $A$ is the image of $B$, 
     thus $A$ is the image of some Stone space. 
\end{itemize} 
\end{proof} 

\begin{remark}\label{ClosedInStoneIsStone}
%Using condition $(iii)$, 
The previous result implies that closed subtype of Stone spaces are Stone.
\end{remark}

\begin{corollary}\label{InhabitedClosedSubSpaceClosed}
  For $S:\Stone$ and $A\subseteq S$ closed, we have 
  $\exists_{x:S} A(x)$ is closed. 
\end{corollary}
\begin{proof}
  By \Cref{ClosedInStoneIsStone}, $\Sigma_{x:S}A(x)$ is Stone, 
  so its truncation is closed by \Cref{TruncationStoneClosed}.
\end{proof}

\begin{corollary}\label{ClosedDependentSums}
  Closed propositions are closed under dependent sums. 
\end{corollary}
\begin{proof}
  Let $P:\Closed$ and $Q:P \to \Closed$. 
  Then $\Sigma_{p:P} Q(p) \leftrightarrow \exists_{p:P} Q(p)$.
  As $P$ is Stone by \Cref{PropositionsClosedIffStone}, 
  \Cref{InhabitedClosedSubSpaceClosed} gives that $\Sigma_{p:P} Q(p)$ is closed. 
\end{proof}
\begin{remark}\label{ClosedDominance}\label{ClosedTransitive}
  Analogously to \Cref{OpenTransitive} and \Cref{OpenDominance}, it follows that 
  closedness is transitive and $\Closed$ forms a dominance. 
\end{remark}

%We can get a dual to completeness.


%ImpliedByBooleEpiMono%\begin{lemma}\label{DualCompleteness}
%ImpliedByBooleEpiMono%Let $A$ and $B$ be c.p. boolean algebra with a map:
%ImpliedByBooleEpiMono%\[i:Sp(A)\to Sp(B)\] 
%ImpliedByBooleEpiMono%The following are equivalent:
%ImpliedByBooleEpiMono%\begin{enumerate}[(i)]
%ImpliedByBooleEpiMono%\item The induced map $B\to A$ is surjective.
%ImpliedByBooleEpiMono%\item The map $i$ is an embedding.
%ImpliedByBooleEpiMono%\item The map $i$ is a closed embedding.
%ImpliedByBooleEpiMono%\end{enumerate}
%ImpliedByBooleEpiMono%\end{lemma}
%ImpliedByBooleEpiMono%
%ImpliedByBooleEpiMono%\begin{proof}
%ImpliedByBooleEpiMono%  \item
%ImpliedByBooleEpiMono%\begin{itemize}
%ImpliedByBooleEpiMono%\item[$(i)\to (ii)$] Immediate.
%ImpliedByBooleEpiMono%\item[$(ii)\to (iii)$] By \Cref{StoneEqualityClosed} the fibers of $i$ are closed in $Sp(A)$, so by \Cref{ClosedInStoneIsStone} they are Stone, so they are closed by \Cref{PropositionsClosedIffStone}.
%ImpliedByBooleEpiMono%\item[$(iii)\to (i)$] By we have that $Sp(A) = \cap_{n:\N}D_n$ for $D_n$ decidable in $Sp(B)$. Assuming $D_n$ correponds to $b_n:B$ though duality, we then have that $A=B/ (b_n)_{n:\N}$ and the induced map is the quotient map:
%ImpliedByBooleEpiMono%$$B\to B/(b_n)_{n:\N}$$
%ImpliedByBooleEpiMono%which is surjective.
%ImpliedByBooleEpiMono%\end{itemize}
%ImpliedByBooleEpiMono%\end{proof}

\begin{lemma}\label{StoneSeperated}
  Assume $S:\Stone $ with $F,G:S \to \Closed$ be such that $F\cap G = \emptyset$. 
  Then there exists a decidable subset $D:S \to 2$ such $F\subseteq D, G \subseteq \neg D$. 
\end{lemma}
\begin{proof}
%  \rednote{Too shorten this (and some other proofs), I've removed some negations and pretended  $D:S\to 2$ is given by $\{x:S|x(d) = 0\}$ instead of $\{x:S|x(d) = 1\}$ }
  Assume $S = Sp(B)$. 
  By \Cref{StoneClosedSubsets}, for all $n:\N$ there is $f_n,g_n:B$ such that 
  $x\in F$ if and only if $x(f_n) = 0$ for all $n:\N$ and 
  $y\in G$ if and only if $y(g_m) = 0$ for all $m:\N$. 
%  $x\in F$ iff $x(f_n) = 1$ for all $n:\N$ and 
%  $y\in G$ iff $y(g_m) = 1$ for all $m:\N$. 
%
%  Denote $R\subseteq B$ for $\{\neg f_n|n:\N\} \cup\{\neg g_n|n:\N\}$. 
  Denote $R\subseteq B$ for $\{f_n|n:\N\} \cup\{g_n|n:\N\}$. 
  Note that $Sp(B/R) = F \cap G=\emptyset$, so by \Cref{SpectrumEmptyIff01Equal}
%  Note that any $x:Sp(B/R)$ is such %gives a map $x:B\to 2$ such that
%  $x(g_n)= x(f_n) = 1$ for all $n:\N$, hence $x\in F \cap G$. 
%  As $F\cap G = \emptyset $, it follows that $Sp(B/R)$ is empty.
%
  there exists finite sets $I,J\subseteq \N $ such that 
%  $$1 =_B \left(\left(\bigvee_{i\in I} \neg f_i\right) \vee \left(\bigvee_{j\in J} \neg g_j\right)\right).$$
%  $$1 =_B \left(\left(\bigvee_{i\in I}  f_i\right) \vee \left(\bigvee_{j\in J}  g_j\right)\right).$$
  $1 =_B ((\bigvee_{i\in I}  f_i) \vee (\bigvee_{j\in J}  g_j)).$
%
  Let $y\in F$, then $y(f_i) = 0$ for all $i\in I$, hence
%  $$1 =_2 y(1) = y\left(\bigvee_{j\in J} g_j\right).$$
  $y(\bigvee_{j\in J} g_j) = 1 $
  And if $x\in G$, we have 
%  $x\left(\bigvee_{j\in J} g_j\right) = 0$. 
  $x(\bigvee_{j\in J} g_j) = 0$. 
  Thus we can define the required $D$ by 
  $D(x) \leftrightarrow x(\bigvee_{j\in J} g_j) = 0$.
  %$$D = \{x:S | x\left(\bigvee_{j\in J} g_j \right) = 0\}$$.
  %, we have $F\subseteq D, G\subseteq \neg D$. 
  
%  Let $y\in G$. Then $y(\neg g_j) = 0$ for all $j \in J$. 
%  Let $y\in G$. Then $y(g_j) = 0$ for all $j \in J$. 
%  Hence 
%  $$
%  1 
%%  = y(1) 
%  =_2
%%  y(\bigvee_{i\in I} \neg f_i) = y (\neg (\bigwedge_{i\in I} f_i))
%  y\left(\left(\bigvee_{i\in I}  f_i\right) \vee \left(\bigvee_{j\in J}  g_j\right)\right)
%  = 
%%  \left(y\left(\bigvee_{i\in I}  f_i\right)\right) \vee \left(\bigvee_{j\in J} y(g_j)\right)
%%  = 
%  y\left(\bigvee_{i\in I}  f_i\right)
%  $$
%%  Thus $y(\bigwedge_{i\in I} f_i) = 0$. 
%%  Note that if $x\in F$, we have $x(f_i) = 1$ for all $i\in I$, hence 
%%  $x(\bigwedge_{i\in I} f_i) = 1$. 
%%  Thus for $D$ corresponding to $\bigwedge_{i\in I} f_i$, we have that 
%%  $F\subseteq D, G\subseteq \neg D$ as required. 
\end{proof} 

%RedundantByOpenInCHaus%\begin{corollary}\label{StoneOpenSubsets}
%RedundantByOpenInCHaus%  Let $A\subseteq S$ be a subset of a Stone space, then 
%RedundantByOpenInCHaus%  $A$ is open iff there exists some countable family $D_n,~n:\N$ of decidable subsets of $S$ with 
%RedundantByOpenInCHaus%  $A = \bigcup_{n:\N} D_n$. 
%RedundantByOpenInCHaus%\end{corollary}
%RedundantByOpenInCHaus%\begin{proof}
%RedundantByOpenInCHaus%  By \Cref{rmkOpenClosedNegation}, $A$ is open iff $\neg A$ is closed and $A = \neg \neg A$. 
%RedundantByOpenInCHaus%  By \Cref{StoneClosedSubsets}, $\neg A$ is closed iff 
%RedundantByOpenInCHaus%  $\neg A = \bigcap_{n\in \N} E_n$ for some countably family of decidable subsets $E_n,~n:\N$. 
%RedundantByOpenInCHaus%  Thus $\neg \neg A = \neg (\bigcap_{n\in \N} E_n)$. 
%RedundantByOpenInCHaus%  By MP (\Cref{MarkovPrinciple}), we have that 
%RedundantByOpenInCHaus%  $\neg (\bigcap_{n\in \N} E_n)= \bigcup_{n\in \N} \neg E_n$. 
%RedundantByOpenInCHaus%  Thus $D_n := \neg E_n$ is as required. 
%RedundantByOpenInCHaus%\end{proof}


\section{Compact Hausdorff spaces}
\begin{definition}
  A type $X$ is called a compact Hausdorff space if its identity types are closed propositions and there exists some $S:\Stone$ and a surjection $S\twoheadrightarrow X$.
\end{definition}

%This means that compact Hausdorff spaces are precisely quotients of Stone spaces by closed equivalence relations.

\subsection{Topology on compact Hausdorff spaces}

\begin{lemma}\label{CompactHausdorffClosed}
  Let $X:\CHaus$ with $S:\Stone$ and a surjective map $q:S\twoheadrightarrow X$.
  Then $A\subseteq X$ is closed if and only if it is the image of a closed subset of $S$ by $q$. 
\end{lemma}
\begin{proof}
%  If $A$ is closed, then it's pre-image under any map is also closed. 
%  In particular for $q:S\to X$ the quotient map, $q^{-1}(A)$ is closed. 
  As $q$ is surjective, we have $q(q^{-1}(A)) = A$.
  If $A$ is closed, so is $q^{-1}(A)$ and 
  hence $A$ is the image of a closed subtype of $S$. 
  Conversely, let $B\subseteq S$ be closed. 
%  Then for any $s:S$, the subtype $\{t:S| B(s) \wedge s \sim t\} \subseteq S$ is closed. 
%  Hence by 
  Define $A'\subseteq S$ by 
  $$A'(s) := \exists_{t:S} (B(t) \wedge q(s) = q(t)).$$
  As $B(t)$ and $q(s) = q(t)$ are closed, by \Cref{ClosedCountableConjunction} and \Cref{InhabitedClosedSubSpaceClosed}, 
  we have that $A'$ is closed. 
  Also $A'$ factors through $q$ as a map $A: X\to \Closed$.
  Furthermore, $A'(s) \leftrightarrow (q(s)\in q(B))$. 
  Hence $A=q(B)$. 
%  Therefore $A(x)$ iff $x$ is in the image of $B$. 
\end{proof}

\begin{remark}\label{InhabitedClosedSubSpaceClosedCHaus}
  Let $X:\Chaus$.
  From \Cref{StoneClosedSubsets}, it follows that $A\subseteq X$ is closed if and only if it is the image of a map 
  $T\to X$ for some $T:\Stone$. 
  If $A$ is closed, it follows from \Cref{InhabitedClosedSubSpaceClosed} that $\exists_{x:X} A(x)$ is closed as well. 
\end{remark}
%\begin{corollary}
%  For $X:\CHaus$ a subtype $A\subseteq X$ is closed iff it is the image of 
%  a map $T\to X$ for some $T:\Stone$. 
%\end{corollary}
%\begin{proof}
%  Directly from the above and \Cref{StoneClosedSubsets}.
%\end{proof}
%WhyDidWeNeedThis%\begin{remark}
%WhyDidWeNeedThis%  It is not the case that every closed subset of a compact Hausdorff space can be written 
%WhyDidWeNeedThis%  as countable intersection of decidable subsets. 
%WhyDidWeNeedThis%  In \Cref{UnitInterval}, we shall introduce the unit interval $[0,1]$ as a compact Hausdorff space with many closed 
%WhyDidWeNeedThis%  subsets, but only two decidable subsets. 
%WhyDidWeNeedThis%  In \Cref{ConnectedComponent}, we shall actually see that whenever every singleton of a compact Hausdorff space $X$
%WhyDidWeNeedThis%  can be written as countable intersection of decidable subsets, $X$ is Stone. 
%WhyDidWeNeedThis%  \rednote{Actually, we'll see that $Sp(2^X)$ and $X$ are bijective sets, 
%WhyDidWeNeedThis%    which only implies that $X$ is Stone if $2^X:\Boole$, but this depends on our definition of countable, 
%WhyDidWeNeedThis%see \Cref{CountabilityDiscussion}}
%WhyDidWeNeedThis%\end{remark}


\begin{corollary}\label{AllOpenSubspaceOpen}
  For $U\subseteq X$ an open subset of a compact Hausdorff space, we have that the proposition 
  $\forall_{x:X} U(x)$ is open. 
\end{corollary}
\begin{proof}
  As $U$ is open, $\neg U$ is closed. 
  So $\exists_{x:X} \neg U(x)$ is closed by \Cref{InhabitedClosedSubSpaceClosedCHaus}. 
  Using \Cref{rmkOpenClosedNegation}, it follows that 
  $\neg (\exists_{x:X} \neg U(x))$ is open. 
  Furthermore, it is equivalent to $\forall_{x:X} \neg \neg U(x)$, 
  which is equivalent to $\forall_{x:X} U(x)$ by \Cref{rmkOpenClosedNegation}.
\end{proof}

\begin{lemma}\label{CHausFiniteIntersectionProperty}
  Given $X:\Chaus$ and $C_n:X\to \Closed$ closed subsets such that $\bigcap_{n:\N} C_n =\emptyset$, there is some $k:\N$ 
  with $\bigcap_{n\leq k} C_n  = \emptyset$. 
\end{lemma}
\begin{proof}
  By \Cref{CompactHausdorffClosed} it is enough to prove the result when $X$ is Stone, and by \Cref{StoneClosedSubsets} we can assume $C_n$ decidable.
  So assume 
  $X=Sp(B)$ and $c_n:B$ such that: 
  $$C_n = \{x:B\to 2\ |\ x(c_n) = 0\}.$$ 
  Then the set of maps $B\to 2$ sending all $c_n$ to $0$ is given by: 
  $$Sp(B/(c_n)_{n:\N})%\ |\ n:\N\}) 
  \simeq \bigcap_{n:\N} C_n = \emptyset .$$
  Hence 
%  $0=_{B/(\neg c_n)_{n:\N}}1$ 
  $0=1$ in $B/(c_n)_{n:\N}$ %\ |\ n:\N\}$, 
  and there is some $k:\N$ with 
  $\bigvee_{n\leq k} c_n = 1$, which also means that: 
  $$\emptyset = Sp(B/(c_n)_{n\leq k}) %\ |\ n\leq k\})  
  \simeq \bigcap_{n\leq k} C_n $$
  as required.
\end{proof}

\begin{corollary}\label{ChausMapsPreserveIntersectionOfClosed}
  Let $X,Y:\CHaus$ and $f:X \to Y$. 
  Suppose $(G_n)_{n:\N}$ is a decreasing sequence of closed subsets of $X$. 
  Then $f(\bigcap_{n:\N} G_n) = \bigcap_{n:\N}f(G_n)$. 
\end{corollary}
\begin{proof}
  It is always the case that $f(\bigcap_{n:\N} G_n) \subseteq \bigcap_{n:\N} f(G_n)$. 
  For the converse direction, suppose that $y \in f(G_n)$ for all $n:\N$. 
  We define $F\subseteq X$ closed by $F=f^{-1}(y)$. 
  Then for all $n:\N$ we have that $F\cap G_n$ is merely inhabited and therefore non-empty. 
  By \Cref{CHausFiniteIntersectionProperty} this implies that $\bigcap_{n:\N}(F\cap G_n) \neq \emptyset$. 
  By \Cref{InhabitedClosedSubSpaceClosedCHaus} and \Cref{rmkOpenClosedNegation}, we have that $\bigcap_{n:\N} (F\cap G_n)$ is merely inhabited. Thus $y\in f(\bigcap_{n:\N} G_n)$ as required. 
\end{proof}

\begin{corollary}\label{CompactHausdorffTopology}
Let $A\subseteq X$ be a subset of a compact Hausdorff space and $p:S\twoheadrightarrow X$ be a surjective map with $S:\Stone$. Then $A$ is closed (resp. open) if and only if there exists a sequence $(D_n)_{n:\N}$ of decidable in $S$ such that $A = \bigcap_{n:\N} p(D_n)$ (resp. $A = \bigcup_{n:\N} \neg p(D_n)$).
%\begin{itemize}
%  \item $A$ is closed iff %if and only if 
%    it can be written as $\bigcap_{n:\N} p(D_n)$
%for some $D_n\subseteq S$ decidable. 
%  \item $A$ is open iff %if and only if 
%    it can be written as $\bigcup_{n:\N} \neg p(D_n)$
%for some $D_n\subseteq S$ decidable.
%\end{itemize}  
\end{corollary}
\begin{proof}
  The characterization of closed sets follows from characterization (ii) in \Cref{StoneClosedSubsets}, 
  \Cref{CompactHausdorffClosed} 
  and \Cref{ChausMapsPreserveIntersectionOfClosed}. 
%  The characterization of open sets 
  For open sets we use \Cref{rmkOpenClosedNegation} and
  \Cref{ClosedMarkov}.
\end{proof}
%
\begin{remark}
  For $S:\Stone$, there is a surjection $\N\twoheadrightarrow 2^S$. 
  It follows that for any $X:\CHaus$ there is a surjection from $\N$ to a basis of $X$. 
  Classically this means that $X$ is second countable. 
\end{remark}
%It follows that compact Hausdorff spaces are second countable:
%\begin{corollary}
%  Any $X:\Chaus$ is has a topological basis which is countable.
%\end{corollary}
%\begin{proof}
%  By \Cref{CompactHausdorffTopology}, 
%  a basis is given by the images of the decidable subsets of some $S:\Stone$. 
%  By \cref{ODiscBAareBoole}, $2^S$ is 
%  overtly discrete so we have a surjection $\N\to 2^S$.
%  \end{proof}
%
\begin{lemma}\label{CHausSeperationOfClosedByOpens}
 Assume $X:\CHaus$ and $A,B\subseteq X$ closed such that $A\cap B=\emptyset$. 
  Then there exist $U,V\subseteq X$ open such that $A\subseteq U$, $B\subseteq V$ and $U\cap V=\emptyset$. 
\end{lemma}
\begin{proof}
  Let $q:S\to X$ be a surjective map with $S:\Stone$.
  As $q^{-1}(A)$ and $q^{-1}(B)$ are closed, 
  by \Cref{StoneSeperated}, there is some $D:S \to 2$ such that
  $q^{-1}(A) \subseteq D, q^{-1}(B) \subseteq \neg D$. 
  Note that $q(D)$ and $q(\neg D)$ are closed by \Cref{CompactHausdorffClosed}. 
  As $q^{-1}(A) \cap \neg D  =\emptyset$, we have that 
  $A\subseteq \neg q(\neg D)$. As $A\cap B = \emptyset$, we have that 
  $A\subseteq U:= \neg q(\neg D) \cap \neg B$.
  Similarly, $B\subseteq V:=\neg  q(D) \cap \neg A$. 
  Then $U$ and $V$ are disjoint because $\neg q(D)\cap \neg q(\neg D) \subset \neg (q(D)\cup q(\neg D)) = \neg X = \emptyset$.
\end{proof}


\subsection{Compact Hausdorff spaces are stable under dependent sums}

\begin{lemma}\label{StoneAsClosedSubsetOfCantor}
A type $X$ is Stone iff it is merely a closed in $2^\N$.
\end{lemma}
\begin{proof}
  By \Cref{BooleAsCQuotient}, any $B:\Boole$ is can be written as $C/(r_n)_{n:\N}$.
  By \Cref{BooleEpiMono}, the quotient map induces an embedding $Sp(B)\hookrightarrow Sp(C)= 2^\N$, 
  which is closed by 
  by \Cref{StoneClosedSubsets}.
\end{proof}

%
%\begin{proof}
%Any countably presented boolean algebra $B$ is enumerable, which gives a surjective morphism:
%$$ 2[\N]\to B$$
%so that by \Cref{DualCompleteness} we merely have a closed embedding:
%$$ Sp(B)\to 2^\N$$
%\end{proof}
\rednote{Can we maybe combine the next two Lemmas?}
\begin{lemma}\label{SigmaStoneCompactHausdorff}
Assume $S:\Stone$ and $T:S\to \Stone$. Then $\Sigma_{x:S}T(x)$ is Compact Hausdorff.
\end{lemma}

\begin{proof}
  By \Cref{ClosedDependentSums} and \Cref{StoneEqualityClosed}, the identity types in $\Sigma_{x:S}T(x)$ are closed.
  By \Cref{StoneAsClosedSubsetOfCantor} %, there is a surjection 
%  $\Sigma_{y:2^\N}(\cdot)(y):(2^\N \to \Closed) \twoheadrightarrow \Stone$. 
%  By \Cref{AxLocalChoice}, it follows we have some $S':\Stone$, a surjection $q:S'\twoheadrightarrow S$ and 
%
  we have for each $x:S$ that 
  $\exists_{A:2^\N\to \Closed} T(x) = \Sigma_{y:2^\N}A(y)$. 
  Using \Cref{AxLocalChoice} we get $S':\Stone$ with a surjective map:
  $q:S'\to S$ 
%$$q:S'\to S$$
and:
$ C : S'\to (2^\N\to\Closed)$
%$ C : (S'\times 2^\N)\to\Closed$
  such that for all $x:S'$ we have 
  $T(q(x)) = \Sigma_{y:2^\N}C(x,y).$
This gives a surjective map:
%$$ \Sigma_{x:S'}\Sigma_{y:2^\N} C(x,y)\twoheadrightarrow \Sigma_{x:S}T(x)$$
$$ \Sigma_{c:(S'\times 2^\N)}C(c)\twoheadrightarrow \Sigma_{x:S}T(x)$$
%By \Cref{StoneClosedUnderPullback}, $S'\times 2^\N$ is Stone, 
%as $C(x,y):\Closed$ for any $(x,y):S'\times 2^\N$, by $\Cref{ClosedInStoneIsStone}$, it follows
The source is Stone by \Cref{StoneClosedUnderPullback} and \Cref{ClosedInStoneIsStone} so we can conclude.
\end{proof}

\begin{lemma}
Assume $X:\CHaus$ and $T:X\to \CHaus$. Then $\Sigma_{x:X}T(x)$ is Compact Hausdorff.
\end{lemma}

\begin{proof}
By \Cref{ClosedDependentSums} we have that identity type in $\Sigma_{x:X}T(x)$ are closed.
%
We know that for any $x:X$ we have $\exists_{Y:\Stone} S'\twoheadrightarrow C(x)$. 
Consider the quotient map $q:S \twoheadrightarrow  X$ with $S:\Stone$. 
By \Cref{AxLocalChoice} we get $S':\Stone$ with a surjective map: $e:S'\to S$
such that for all $x:S'$ we have $Y(x):\Stone$ and a surjective map $Y(x)\to T(q(e(x)))$. 
This gives a surjective map:
$$ \Sigma_{x:S'}Y(x)\to \Sigma_{x:X}T(x)$$
By \Cref{SigmaStoneCompactHausdorff} we have a surjective map from a Stone space to the source so we can conclude.
\end{proof}

\subsection{Stone spaces are stable under dependent sums}
We will show that Stone spaces are precisely totally disconnected compact Hausdorff spaces. 
We will use this to prove that a dependent sum of Stone spaces is Stone.

\begin{lemma}\label{AlgebraCompactHausdorffCountablyPresented}
Assume $X:\Chaus$, then $2^X$ is countably presented.
\end{lemma}

\begin{proof}
  Consider some quotient map $q:S\twoheadrightarrow X$ with $S:\Stone$. 
%First we choose $S\to X$ surjective with $S$ Stone and prove that $2^X$ is an open subalgebra of $2^S$.
%
  This induces an injection of Boolean algebras $2^X \hookrightarrow 2^S$.
  Note that $a:S\to 2$ lies in $2^X$ iff %for all $s,t:S$, 
  $$\forall_{s,t:S}\ q(s) =_X q(t) \to a(s) =_2a(t).$$
  As equality in $X$ is closed and equality in $2$ is decidable, so \Cref{ImplicationOpenClosed}
  tells us that the implication is open for every $s,t:S$. 
  By \Cref{AllOpenSubspaceOpen}, we conclude that 
%  $\forall_{s:S} \forall_{t:S} ((q(s) =_X q(t)) \to (a(s) =_2 a(t)))$ is open. 
%  Hence 
  $2^X$ is an open subalgebra of $2^S$. 
  Therefore, it is in $\ODisc$ by
  \Cref{PropOpenIffOdisc} and \Cref{OdiscSigma} 
  and in $\Boole$ by \Cref{ODiscBAareBoole}.
%
%
%  \rednote{It might be nice to show that Boolean algebras are countably presented iff they are overtly discrete}
%Now we prove that open subalgebras of countably presented agebras are countably presented. Assume $U\subset 2[\N] / F$ such a subalgebra. We have that $U$ is equivalent to the algebra generated by the $s:2[\N]$ such that $[s]\in U$ quotiented by the relation $s=t$ for all $s,t:2[\N]$ such that $[s],[t]\in U$ and $[s]=[t]$.
%
%Using that $2[\N]$ is countable and that $[s]=[t]$ is open by \Cref{BooleEqualityOpen}, we see that $U$ is generated by variables and relations each indexed by an open in $\N$. But by \Cref{OpenInNAreDecidableInN} any open in $\N$ is countable, so $U$ is countably presented.
\end{proof}
\begin{definition}
For all $X:\Chaus$ and $x:X$,
  we define $Q_x$ the connected component of $x$
  as the intersection of all decidable subsets of $X$ containing $x$. 
\end{definition}

\begin{lemma}\label{ConnectedComponentClosedInCompactHausdorff}
For all $X:\CHaus$ with $x:X$, we have that $Q_x$ is a countable intersection of decidables in $X$.
\end{lemma}
\begin{proof}
%  By \Cref{AlgebraCompactHausdorffCountablyPresented} we have that $2^X$ is countably presented, 
%  therefore we can enumerate the elements of $2^X$, say as $(D_n)_{n:\N}$. 
  By \Cref{AlgebraCompactHausdorffCountablyPresented},
  we can enumerate the elements of $2^X$, say as $(D_n)_{n:\N}$. 
  Define $E_n$ for $n:\N$ as $D_n$ if $x\in D_n$ and $X$ otherwise. 
  Then $\cap_{n:\N}E_n = Q_x$.
\end{proof}
%

\begin{lemma}\label{ConnectedComponentSubOpenHasDecidableInbetween}
  Assume $X:\Chaus$ with $x:X$ and suppose $U\subseteq X$ open with $Q_x\subseteq U$. 
  Then we have some decidable $E\subseteq X$ with $x\in E$ and $E\subseteq U$. 
\end{lemma}
\begin{proof}
  By \Cref{ConnectedComponentClosedInCompactHausdorff}, 
  we have $Q_x = \bigcap_{n:\N}D_n$ with $D_n\subseteq X$ decidable. 
  If $Q_x \subseteq U$, we have that 
  $$Q_x\cap \neg U = \bigcap_{n:\N} (D_n \cap \neg U) = \emptyset.$$
  By \Cref{CHausFiniteIntersectionProperty} there is some $k:\N$ with 
  $$(\bigcap_{n\leq k} D_n )\cap \neg U  = \bigcap_{n\leq k} (D_n \cap \neg U) = \emptyset.$$
  Therefore $\bigcap_{n\leq k} D_n \subseteq \neg\neg U$, which equals $U$ by \Cref{rmkOpenClosedNegation}. So $\bigcap_{n\leq k} D_n$ gives us the desired decidable subset.
\end{proof}

\begin{lemma}\label{ConnectedComponentConnected}
Assume $X:\Chaus$ with $x:X$. Then any map $Q_x\to 2$ is constant.
\end{lemma}
\begin{proof}
Assume given a separation $Q_x = A\cup B$ with $A,B$ disjoint and decidable in $Q_x$. Assume $x\in A$. 
%We will show $B=\emptyset$. 
%
By \Cref{ConnectedComponentClosedInCompactHausdorff}, $Q_x\subseteq X$ is closed. 
Using \Cref{ClosedTransitive}, it follows that $A,B\subseteq X$ are closed and disjoint.
By \Cref{CHausSeperationOfClosedByOpens} there exist $U,V$ disjoint open such that $A\subseteq U$ and $B\subseteq V$. 
%
By \Cref{ConnectedComponentSubOpenHasDecidableInbetween} we have a decidable $D$ such that $Q_x\subseteq D\subseteq U\cup V$. 
Note that $D\cap U = D \cap (\neg V):=E$ is clopen, hence decidable by \Cref{ClopenDecidable}.
Remark that $x\in E$, hence $B\subseteq Q_x \subseteq E$ but $B \cap E = \emptyset$, hence $B=\emptyset$. 
%Then we define $E = D\cap U$. 
%We have that $E$ is open, it is closed as $E=D\cap \neg V$, therefore it is decidable by \Cref{ClopenDecidable}.
%
%Then $Q_x\subset E$ with $E$ decidable and $B\cap E = \emptyset$. 
%But then $Q_x\cap B = \emptyset$ and $B=\emptyset$.
\end{proof}

\begin{lemma}\label{StoneCompactHausdorffTotallyDisconnected}
Let $X:\CHaus$, then $X$ is Stone if and only $\forall_{x:X}\ Q_x=\{x\}$.
\end{lemma}

\begin{proof}
  By \Cref{AxStoneDuality}, it is clear that for all $x:S$ with $S:\Stone$ we have that $Q_x=\{x\}$.
%
  Conversely, assume $X:\CHaus$ such that $\forall_{x:X}\ Q_x = \{x\}$.
  We claim that the evaluation map $e:X \to Sp(2^X)$ is both injective and surjective, hence an equivalence. 
%  \item 
    Let $x,y:X$. If $f(x) = f(y)$ for all $f:2^X$, then $y \in Q_x$, hence $x=y$ by assumption. 
    Thus $e$ is injective. 
%  \item 
    Let $q:S\twoheadrightarrow X$ be a surjective map. 
    It induces an injection $2^X \hookrightarrow 2^S$, which by \Cref{SurjectionsAreFormalSurjections}
    induces a surjection $Sp(2^S) \twoheadrightarrow Sp(2^X)$. 
    Note that $e\circ q$ factors as $S\simeq Sp(2^S) \twoheadrightarrow Sp(2^X)$. 
    It follows that $e$ is surjective. 
%
%For the converse, we show that the map:
%\[X\to Sp(2^X)\]
%is an equivalence and conclude by \Cref{AlgebraCompactHausdorffCountablyPresented}. 
%
%Surjectivity always hold, indeed considering $q:S\to X$ surjective with $S$ Stone, we have that $2^X\subset 2^S$ as so that the by \Cref{FormalSurjectionsAreSurjections} the map:
%$$S = Sp(2^S)\to Sp(2^X)$$
%is surjective and it factors though $X$.
%
%Now let us prove injectivity. Assume $x,y:X$ having the same image in $Sp(2^X)$. This means that any map in $X\to 2$ has the same value on $x$ and $y$, so $x\in Q_y$ and by hypothesis $x=y$.
\end{proof}

\begin{theorem}
  \label{stone-sigma-closed}
Assume $S:\Stone$ and $T:S\to\Stone$. Then $\Sigma_{x:S}T(x)$ is Stone.
\end{theorem}

\begin{proof}
By \Cref{SigmaCompactHausdorff} we have that $\Sigma_{x:S}T(x)$ is compact Hausdorff. 
By \Cref{StoneCompactHausdorffTotallyDisconnected} 
it is enough to show that for all $x:S$ and $y:T(x)$ 
we have that $Q_{(x,y)}$ is a singleton.
%
Assume $(x',y')\in Q_{(x,y)}$, then for any map $f:S\to 2$ we have that:
$$ f(x) = f\circ \pi_1(x,y) = f\circ \pi_1(x',y') = f(x')$$
so that $x'\in Q_x$ and since $S$ is Stone by \Cref{StoneCompactHausdorffTotallyDisconnected} we have that $x=x'$.
%
Therefore we have $Q_{(x,y)}\subseteq \{x\}\times T(x)$. Assume $z,z':Q_{(x,y)}$, then for any map $g:T(x)\to 2$ we have that $g(z)=g(z')$ by \Cref{ConnectedComponentConnected}. Since $T(x):\Stone$, we conclude $z=z'$ by \Cref{StoneCompactHausdorffTotallyDisconnected}.
%Therefore we have $Q_{(x,y)}\subseteq \{x\}\times T_{x}$. 
%By \Cref{ConnectedComponentConnected} we have an inhabited 
%connected subtype of a Stone space. 
%Then any map $T_x\to 2$ is constant on $Q_{(x,y)}$ and by 
%\Cref{StoneCompactHausdorffTotallyDisconnected} we conclude that it is a singleton.
\end{proof}




\section{The Unit interval}
\rednote{Active WIP}
\subsection{The unit interval as Compact Hausdorff space}
In this section we will introduce the unit interval $I$ as compact Hausdorff space. 
The definition is based on \cite{Bishop}. 
%We will then calculate the cohomology of $I$. 
%For a proof that the unit interval corresponds to the definition using Cauchy sequences, 
%we refer to the appendix. 


%\subsection{The Cauchy reals}
The goal of this section is to introduce the real numbers in a constructive setting, 
following the definition given in \cite{Bishop} with some small adaptations. 
We will later use this definition to show that the interval $[0,1]$ is compact Hausdorff in the sense 
of \Cref{dfnCompactHausdorff}. 

We will assume we are given natural and rational numbers, with decidable (in)equalities
working as expected. 

\begin{definition}
  A \textbf{Cauchy sequence} is a sequence $x : \N \to \mathbb Q$ such that
  for any $n,m:\N$, we have %$0\leq x_n \leq 1$ and 
$|x_n-x_m| \leq (\frac12)^n + (\frac12)^m$. 
\end{definition}
\begin{remark}
  If $x$ is a cauchy sequence and $q$ a rational number, the 
  sequence $(x-q)_n = (x_n-q)$ is also Cauchy.
\end{remark}

Following \cite{Bishop}, we define inequality relations between Cauchy sequences and
rational numbers. 
\begin{definition}
  For $x$ a Cauchy sequence and $q$ a rational number, we define 
  \begin{itemize}
%    \item $x> q = \Sigma(n:\N) x_n > q + {\frac12}^n$. %for some $n:\N$. 
%    \item $x< q = \Sigma(n:\N) x_n < q - {\frac12}^n$. %for some $n:\N$. 
    \item $x\leq  q = \Pi_{n:\N} x_n \leq q+(\frac12)^n$. 
    \item $x\geq  q = \Pi_{n:\N} x_n \geq q-(\frac12)^n$. 
  \end{itemize}
\end{definition}
%\begin{lemma}
%  For $x$ Cauchy and $q$ rational, we have that 
%  $x\leq q$ iff for each $n:\N$, we have a $N_n:\N$ with 
%  $x_m> q-(\frac12)^n$ for all $m \geq N_n$. 
%\end{lemma}
\begin{lemma}\label{ComparisonLemma}
  For $x$ a Cauchy sequence and $q$ a rational number, we have
  $x \leq q \vee x \geq q$. 
\end{lemma}
\begin{proof}
  For rational numbers, we have decidable inequalities, 
  therefore $\geq 0 \vee q \leq 0$. 
  It follows that 
  $ \forall (n:\N) \forall (m:\N) q \geq -(\frac12)^n \vee q \leq (\frac12)^m$. 
  Now by \Cref{TODO}, we may conclude 
  $ (\forall (n:\N) q \geq -(\frac12)^n ) \vee (\forall (m:\N) q \leq (\frac12)^m)$
  as required.
\end{proof}


%%%\begin{definition}
%%%  A Cauchy sequence $x$ is \textbf{nonnegative} if $x_n \geq -(\frac12)^n$
%%%  for every $n:\N$. 
%%%  $x$ is \textbf{nonpositive} if $x_n \leq (\frac12)^n$
%%%  for every $n:\N$. 
%%%\end{definition} 
%%%%\begin{lemma}
%%%%  A Cauchy sequence is nonnegative iff there exists an $N$ such that $x_n \geq -(\frac12)^N$
%%%%  for all $n\geq N$. 
%%%%  A Cauchy sequence is nonpositive iff there exists an $N$ such that $x_n \leq (\frac12)^N$
%%%%  for all $n\geq N$. 
%%%%\end{lemma}
%%%%\begin{proof}
%%%%  Assume $x$ is nonnegative. Thus for every $n:\N$, we have $x_n\geq -(\frac12)^n \geq -(\frac12)^0$. 
%%%%  Thus $N$ can taken to be $0$. 
%%%%%
%%%%  Conversely, as $x$ is Cauchy, we have
%%%%  for all $m :\N$ that  
%%%%%  \begin{equation}- (\frac12)^m -(\frac12)^n \leq    x_m-x_n \leq (\frac12)^m + (\frac12)^n \end{equation}
%%%%  \begin{equation}- (\frac12)^m -(\frac12)^n \leq    x_n-x_m \leq (\frac12)^m + (\frac12)^n \end{equation}
%%%%  If in addition there is an $N$ such that whenever $m\geq N$, we have 
%%%%  $x_m \geq -(\frac12)^N$, so $-x_m \leq (\frac12)^N$, 
%%%%  so $x_n -x_m \leq x_n - (\frac12)^N$. 
%%%%  Therefore, 
%%%%  \begin{equation}- (\frac12)^m -(\frac12)^n \leq    x_n-x_m \leq x_n-(\frac12)^N \end{equation}
%%%%  Thus 
%%%%  \begin{equation}- (\frac12)^m -(\frac12)^n  + (\frac12)^N \leq x_n \end{equation}
%%%%  As $N \geq N$, we have in particular 
%%%%  \begin{equation}- (\frac12)^N -(\frac12)^n  + (\frac12)^N \leq x_n \end{equation}
%%%%  \begin{equation} - (\frac12)^n  \leq x_n \end{equation}
%%%%  thus $x$ is nonnegative. 
%%%%
%%%%  The nonpositive case goes similar. 
%%%%\end{proof}   
%%% 
%%%
%%%\begin{lemma}
%%%  A Caucy sequence is nonnegative or nonpositive. 
%%%\end{lemma}

%\begin{lemma}
%  For any Cauchy sequence $p$, we have 
%  $(\forall (n:\N) p_n \leq (\frac12)^n) \vee (\forall (n:\N) p_n \geq -(\frac12)^n)$. 
%\end{lemma}
%\begin{proof}
%We 
%\end{proof}

\begin{definition}
Given two Cauchy sequences $p = (p_n)_{n\in\N}, q=(q_n)_{n\in\N}$, 
we define the proposition $p \sim_C  q$ as 
\begin{equation}
  p \sim_C q : = \forall (n,m : \N) ((| p_n - q_m| \leq  (\frac12)^n + (\frac12)^m))
\end{equation}
\end{definition}

%\begin{remark}
%  Note that $p\sim_C q$ is equivalent to 
%\begin{equation}
%  \forall (n : \N) | p_n - q_n| \leq  (\frac12)^{n-1}
%\end{equation}
%The equivalence doesn't hold, unless you cut off initial segments (which shouldn't matter). 
%\end{remark} 

\begin{definition}
  The type of \textbf{Cauchy reals} is given by 
  the type of Cauchy sequences modulo $\sim_C$.
\end{definition}

We claim that the inequality in \Cref{TODO} extends to a well-defined 
inequality between Cauchy reals and rational numbers. 

Furthermore, we claim that 
$\Pi_{x:\mathbb R} \Pi_{q:\mathbb Q} x \leq q \vee x \geq q$. 

%\begin{lemma}
%  For any Cauchy real $r$ any Cauchy sequence $p$ representing $r$, 
%  we have 
%  \begin{equation}
%    (\forall (n:\N) p_n \leq (\frac12)^n) \vee (\forall (n:\N) p_n \geq (\frac12)^n)
%  \end{equation}
%
%\end{lemma}

\begin{definition}
  A Cauchy sequence in the interval is a Cauchy sequence $x$ such that 
  for any $n:\N$, we have $0\leq x_n \leq 1$. 
 % 
  The interval of Cauchy reals is given by the type of Cauchy sequences in the interval 
  modulo $\sim_C$. We denote it by $[0,1]$. 
\end{definition}  


%------------------content of file BinaryClosedEquivalence.tex...
%We want to show that the interval of Cauchy reals is Compact Hausdorff. 
%Informally, to any binary sequence $\alpha : \N \to 2$, 
%we can associate a Cauchy sequence 
%$cs(\alpha)$, given by 
%\begin{equation}\label{eqnBinaryEncode}
%  (cs(\alpha))_n = \sum\limits_{i = 0 }^n \frac {\alpha(i)}{2^{i+1}}
%\end{equation}
%and we are going to give a closed relation on Cantor space such that 
%two binary sequences are equivalent iff they correspond to the same Cauchy reals. 
\begin{example}
  Let $n:\N$, we denote $C_n = 2[n]$ for the free Boolean algebra on $n$ generators 
  and no relations. 
  Note that $Sp(C_n) = 2^n$ corresponds to the space of finite binary sequences. 
\end{example}
Now we introduce some notation:
\begin{definition}
  \item Given an infinite binary sequence $\alpha:2^\N$ and a natural number $n : \N$  
    we denote $\alpha|_n: 2^n$ for the 
    restriction of $\alpha$ to a finite sequence of length $n$. 
  \item We denote $\overline 0, \overline 1$ 
    for the binary sequences which are constantly $0$ and $1$ respectively. 
  \item We denote $0,1$ for the sequences of length $1$ hitting $0,1$ respectively. 
  \item If $x$ is a finite sequence and $y$ is any sequence, 
    denote $x\cdot y$ for their concatenation. 
\end{definition} 
Now we'll give a definition for when two finite binary sequences of length $n$ correspond 
to real numbers whose distance is $\leq (\frac12)^n$.
Informally, we want for every finite sequence $s$ that 
$(s \cdot 0 \cdot \overline 1)$ and  $(s \cdot 1 \cdot \overline 0)$ are equivalent. 

\begin{definition}
  Let $n:\N$ and let $s,t : 2^n$. 
  We say $s,t$ are $n$-near, and write $s\sim_n t$ if 
  there merely exists some $m:\N$ and some $u:2^m$, such that 
 \begin{equation}\label{EqnNearness}
   \big(
     (s = (u\cdot 0\cdot \overline 1)|_n) \vee (s = (u \cdot 1 \cdot \overline 0) |_n)
   \big)
    \wedge 
   \big(
     (t = (u\cdot 0\cdot \overline 1)|_n) \vee (t = (u \cdot 1 \cdot \overline 0) |_n)
   \big)
  \end{equation} 
\end{definition}
\begin{remark}\label{nearnessProperties}
\item As we're dealing with finite sequences, $s\sim_n t$ is decidable. 
\item Given any $s:2^n$, using $m=n, u = s$ above, we can show that $s\sim_n s$. 
  So $n$-nearness is reflexive. 
\item \Cref{EqnNearness} is symmetric in $s$ and $t$. Hence $n$-nearness is symmetric.
\item Note that $0\cdot 0\sim_2 0\cdot 1 \sim_2 1\cdot 0 \sim_2 1\cdot 1$, 
  but $0\cdot 0\nsim_2 1\cdot 1$. %is not $2$-near to $1\cdot 1$. 
  Thus $n$-nearness is not in general transitive. 
\end{remark}
\begin{definition}
  Let $\alpha, \beta: 2^\N$, we define $a\sim_I\beta$ as 
  $\forall_{n:\N} (\alpha|_n \sim_n \beta|_n)$. 
\end{definition}
\begin{lemma}\label{IntervalFiberSizeAtMost2}
  Whenever $\alpha,\beta,\gamma:2^\N$, are such that 
  $\alpha\sim_I \beta, \beta\sim_I \gamma$, 
  at least two of $\alpha,\beta,\gamma$ are equal. 
\end{lemma}
\begin{proof}
  We will show that $\beta = \gamma \vee \alpha = \gamma \vee \alpha = \beta$. 
  By \Cref{StoneEqualityClosed} and \Cref{ClosedFiniteDisjunction}, this is a closed proposition. 
  By \Cref{rmkOpenClosedNegation}, we can instead show the double negation. 
  To this end, assume that none of $\alpha,\beta,\gamma$ are equal. 
  By \Cref{MarkovPrinciple}, there exist indices $i,j,k\in \N$ with 
  \begin{equation}
    \beta(i) \neq \gamma(i), \alpha(j) \neq \gamma(j), \alpha(k) \neq \beta(k)
  \end{equation}
  Let $n:=\max(i,j,k) + 2$. 
  As $\alpha\sim_I \beta$, we have $\alpha|_n\sim_n\beta|_n$. 
  By assumption $\alpha|_n \neq \beta|_n$, so WLOG we may assume that 
  we have some $m: \N, u:2^m$ with 
  \begin{equation}
    \alpha|_n = (u\cdot 0 \cdot \overline 1) |_n , \beta|_n = (u \cdot 1 \cdot \overline 0)|_n.
  \end{equation}
  As $\alpha(k) \neq \beta(k) $ and $n\geq k+2$, 
  it follows in particular that $m\leq n-2$ and hence 
  $\beta(n-1) = 0$.% and $\beta(m) = 1$. 

  As also $\beta\sim_I \gamma$, we have $\beta|_n \sim_n \gamma|_n$.
  So there exists some $m':\N, u':2^m$ with 
 \begin{equation} %\label{EqnNearness}
   \big(
     (\beta|_{n} = (u'\cdot 0\cdot \overline 1)|_n) \vee (\beta|_{n} = (u' \cdot 1 \cdot \overline 0) |_n)
   \big)
    \wedge 
   \big(
     (\gamma|_{n} = (u'\cdot 0\cdot \overline 1)|_n) \vee (\gamma|_{n} = (u' \cdot 1 \cdot \overline 0) |_n)
   \big).
  \end{equation} 
  Similarly as above, we have that $m'\leq n-2$, and as $\beta(n-1) = 0$, it follows that 
  $\beta|_{n} = (u' \cdot 1 \cdot \overline 0) |_n$. 
  Now as $\beta(i)\neq \gamma(i)$ with $i<n$, we have that $\beta|_n \neq \gamma|_n$, hence 
  $\gamma|_{n} = (u'\cdot 0\cdot \overline 1)|_n$. 
  Now we have $m,m'\leq n-2$ and $u:2^m, u':2^{m'}$ such that 
  \begin{equation}
    (u\cdot 1 \cdot \overline 0)|_n = \beta|_n = (u'\cdot 1 \cdot \overline 0)|_n
  \end{equation}
  Note that $\beta(m') = 1$. 
  But also $\beta(l) = 0$ for all $l$ with $m<l<n$
  Therefore $m'\leq m$. 
  By similar reasoning, $m\leq m'$. We conclude $m=m'$. 
  As a consequence, $u = u'$, but then 
  $\gamma|_n = \alpha|_n$, contradicting that $\alpha(j)\neq \gamma(j) $ for $j<n$. 
  Hence we arrive at a contradiction, as required. 
\end{proof}


\begin{corollary}
  $\sim_I$ is a closed equivalence relation on $2^\N$. 
\end{corollary}
\begin{proof}
  By \Cref{nearnessProperties}, $\sim_I$ is a countable conjunction of decidable propositions. 
  Also by \Cref{nearnessProperties}, $\sim_n$ is reflexive and symmetric for all $n:\N$, thus
  $\sim_I$ is reflexive and symmetric as well. 
  Finally $\sim_I$ is transitive as a consequence of \Cref{IntervalFiberSizeAtMost2}.
\end{proof}
\begin{definition}
  We define $\I:\Chaus$ as $\I= 2^\N/\sim_I$. 
\end{definition}
%------------------
%\subsection{The topology of the interval}
We have defined the interval as a certain quotient of Cantor space, 
in \Cref{IntervalvsCauchyInterval}, a proof is provided for the following theorem:
\begin{theorem}
  Let $I'$ denote the interval of Cauchy real numbers.
  Then the map $2^\N\to I'$ given by 
  \begin{equation}
    \alpha \mapsto \bigsum\limits_{i\in \N} \frac{\alpha(i)} {2^{i+1}}
  \end{equation}
  is well-defined and induces an equivalence $I\simeq I'$. 
\end{theorem}



%\begin{lemma}
  $b$ sends $\sim_n$ equivalent binary sequences to $\sim_C$ equivalent Cauchy sequences. 
\end{lemma}
\begin{proof}
  Let $\alpha, \beta$ be binary sequences.
  We claim that $|b(\alpha)_n - b(\beta)_n| \leq (\frac12)^{n+1}$ 
  whenever $\text{near}_n(\alpha, \beta)$. 
  It will follow that if $\alpha\sim_n \beta$, then 
  $b(\alpha)\sim_C b(\beta)$. 

  Let $n:\N$ and assume $m:\N$ with $m\leq n$ and 
  let $z$ be a sequence of length $m$ such that 
  $\alpha|_n = z\cdot 1 \cdot \overline 0|_n$ and $\beta|_n = z \cdot 0 \cdot \overline q |_n$. 
  then $b(\alpha)_n = \sum_{i\leq m} \frac{z(i)}{2^{i+1}} + (\frac12)^{m+2}$ and 
  $b(\beta)_n = \sum_{i\leq m} \frac{z(i)}{2^{i+1}} + \sum\limits_{m+2 \leq i \leq n}(\frac12)^{i+1}$. 
  Thus 
  $b(\alpha)_n - b(\beta)_n = (\frac12)^{m+2} - \sum\limits_{m+2 \leq i \leq n}(\frac12)^{i+1} = 
  (\frac12)^{n+1}$, 
  which is smaller than required. 
\end{proof}  

\begin{lemma}
  Whenever $b(\alpha) \sim_C b(\beta)$, 
  we have $\alpha \sim_n \beta$. 
\end{lemma}
\begin{proof}
  Assume $b(\alpha) \sim_Cb (\beta)$. 
  Let $n:\N$. 
  We shall show that $\text{near}_n(\alpha , \beta)$. 

  As we're only checking finitely many entries, 
  we either have $\alpha|_n = \beta|_n$, 
  or there exists a smallest $m\leq n$ with 
  $\alpha(m) \neq \beta(m)$. 

  If $\alpha|_n = \beta|_n$, we have $\text{near}_n(\alpha,\beta)$ and are done. 
  WLOG assume $\alpha(m) = 1, \beta(m) = 0$ for $m$ minimal. 

  Now note that 
  \begin{equation} 
    b(\alpha)_{k+1} - b(\beta)_{k+1} = 
    b(\alpha)_{k} - b(\beta)_{k} + 
    \frac{\alpha(k+1) - \beta(k+1)}{2^{k+2}}.
  \end{equation}

  For $k>m$, we have that 
  \begin{equation}
  |b(\alpha)_k - b(\beta)_k |= 
  |(\frac12)^{m+1} + \sum\limits_{i=m+1}^k \frac{ \alpha(i) -\beta(i)}{2^{i+1}}|. 
  \end{equation}
  Note that the right summand is always $\leq (\frac12)^{m+1}$. 
  Therefore, we can leave out the absolute value function. 

  We claim that for every $k\geq m+1$, we have $\alpha(k) = 0, \beta(k) = 1$. 
  We will use induction. 
  Suppose that for every $m <i<j$, we have $\alpha(i) = 0$, and $\beta(i) = 1$. 
  Then 
  \begin{equation}
    b(\alpha)_{j-1} - b(\beta)_{j-1} = 
    (\frac12)^{m+1} + 
    \sum\limits_{i=m+1}^{j-1} \frac{ -1}{2^{i+1}} 
    = (\frac12)^{j}
  \end{equation}
   
  \begin{itemize}
    \item 
      we claim that $\alpha(j) = 0$ 
      Suppose $\alpha(j) = 1$. 
      Then $\alpha(j) -\beta(j) \geq 0$. 
      And for $j + 2$, we have that 
  \begin{align}
    &b(\alpha)_{j+2} - b(\beta)_{j+2}
    \\
    =  
    &(b(\alpha)_{j-1} - b(\beta)_{j-1}) + 
    &\frac{\alpha(j)-\beta(j)}{2^{j+1}} +  
    &\frac{\alpha(j+1) - \beta_(j+1)}{2^{j+2}}
    +
    &\frac{\alpha(j+2) - \beta_(j+2)}{2^{j+3}}
    \\
    \geq  
      & (\frac12)^j + &0 
    + &\frac{-1}{2^{j+2}} 
    + &\frac{-1}{2^{j+3}} 
    \\
      > &(\frac12)^{j+1}
  \end{align}
  which contradicts $b(\alpha) \sim_Cb(\beta)$, 
  which would require that $|b(\alpha)_{j+2} - b(\beta_{j+2} | \leq (\frac{12})^{j+2}+ (\frac12)^{j+2} = (\frac12)^{j+1}$. 
  Therefore $\alpha(j) \neq 1$, and thus $\alpha(j) = 0$. 
    \item 
      We also claim that $\beta(i) = 1$. 
      If $\beta(i) = 0$, we also have 
      $\alpha(j) -\beta(j) \geq 0$, and the rest of the proof is similar as above. 
  \end{itemize}
\end{proof}


%\begin{lemma}
  The map $b: 2^\N \to [0,1]$ is surjective. 
\end{lemma}
\begin{proof}
  First, suppose we have a function 
  $d:\Pi_{x:\mathbb R} \Pi_{q: \mathbb Q} (x \leq q + x \geq q)$
  Then we could recursively define 
  $$\alpha(n) = \begin{cases}
    0 \text{ if } d(x - \sum\limits_{i<n} \frac{\alpha(i)}{2^{i+1}} , \frac{1}{2^{n+1}}) = inl(\cdot) \\
    1 \text{ otherwise}
  \end{cases}
  $$
%  Recall that inequality between rational numbers is decidable, therefore we can define
%  $$\alpha(n) = \begin{cases}
%    0 \text{ if } |x_n - \sum\limits_{i<n} \frac{\alpha(i)}{2^{i+1}}| \leq  \frac{1}{2^{n+1}} \\
%    1 \text{ otherwise}
%  \end{cases}
%  $$
  Note that 
  $$\alpha(n) = \begin{cases}
    0 \text{ if } d(x - b(\alpha)_{n-1} , \frac{1}{2^{n+1}}) = inl(\cdot) \\
    1 \text{ otherwise}
  \end{cases}
  $$
  We'll show by induction that $b(\alpha)_n \leq x$ for every $n:\N$. 
  First $b(\alpha)_0 = 0 \leq x$. 
  Assuming, $b(\alpha)_k \leq x$, for $b(\alpha)_{k+1}$, 
  there are two cases:
  \begin{itemize}
    \item 
     if $d(x -  b(\alpha)_k, \frac{1}{2^{n+1}}) = inl(\cdot)$, 
     then $b(\alpha)_{k+1} = b(\alpha)_k$, which is $\leq x$ by induction hypothesis. 
   \item 
     Otherwise, $ x - b(\alpha)_k \geq (\frac12)^{k+1}$
     So $x-b(\alpha)_k - (\frac12)^{k+1} \geq 0$, 
     and $b(\alpha)_{k+1} = b(\alpha)_k + (\frac12)^{k+1}$. 
     So $x-b(\alpha)_{k+1} \geq 0$, and $b(\alpha)_{k+1} \leq x$ as required. 
 \end{itemize}
 So by induction $b(\alpha)_n\leq x$ for every $n:\N$. 
 Therefore, $|x-b(\alpha)_n| = x-b(\alpha)_n$. 
  
  We shall also show by induction that 
  $ x- b(\alpha)_n \leq (\frac12)^{n+1} $
  for every natural number $n:\N$. 
%
  For $n = 0$, this follows from the assumption that $x\leq 1$. 
%
  Suppose that $ x- b(\alpha)_k  \leq (\frac12)^{k+1} $. 
  We make a case distinction on the form of $d(x-b(\alpha)_k, (\frac12)^{k+2})$.
  \begin{itemize}
    \item 
      If $d(x-b(\alpha)_k , (\frac12)^{k+2}) = inl(\cdot)$, 
      then $  x-b(\alpha)_k  \leq (\frac12)^{k+2}$, 
      and $b(\alpha)_{k+1} = b(\alpha)_k$, 
      and $x-b(\alpha)_{k+1}  \leq (\frac12)^{k+2}$ as well, 
      as required. 
    \item 
      Otherwise, we must have
      $ x- b(\alpha)_k  \geq (\frac12)^{k+2}$, 
      and $b(\alpha)_{k+1} = b(\alpha)_k + (\frac12)^{k+1}$.
      By induction hypothesis, we have 
      $x-b(\alpha)_k \leq (\frac12)^{k+1}$. 
      Thus \begin{equation}
        x-b(\alpha)_{k+1} = x - b(\alpha)_k - (\frac12)^{k+1}
        \leq (\frac12)^{k+1} - (\frac12)^{k+2} = (\frac12)^{k+2}
      \end{equation}
      as required. 
  \end{itemize}
  
  By induction, we conclude that 
  $ | b(\alpha)_n - x |  \leq (\frac12)^{n+1} $
  for every $n:\N$. 
  Therefore $b(\alpha)$ converges to $x$. 

  We may conclude that $\Pi_{x:[0,1]} \Pi_{q: \mathbb Q} (x \leq q + x \geq q)$ implies that 
  we can give for each $x: [0,1]$ a binary sequence $\alpha$ with $b(\alpha) = x$. 
  As we have the propositional trunctation of the premise by \Cref{ComparisonLemma}, 
  we may conclude that for each $x:[0,1]$ there merely exists $\alpha$ with $b(\alpha) = x$. 
  Therefore $b$ is surjective. 
\end{proof}



%
%
%
%\begin{theorem}
%  The interval of Cauchy reals is isomorphic to $2^\N / \sim_t$. 
%\end{theorem} 
%\begin{proof}
%  This follows from the fact that $b:2^\N$ is such that $\alpha\sim_n \beta$ iff $b(\alpha)\sim_t b(\beta)$. 
%  and for every Cauchy real, there is a binary sequence being sent to it, so the composition of $b$ and the 
%  quotient from Caucy sequences to Cauchy real is a surjection. 
%\end{proof}
%
%\begin{corollary}
%  The interval is compact Hausdorff. 
%\end{corollary}

\subsection{Order on the interval}
\begin{definition}
  For $n:\N$ we define 
  $cs_n:2^n \to \mathbb Q$ by 
  \begin{equation}
    cs_n(a) = \sum\limits_{i=0}^{n-1} \frac{a(i)} {2^{i+1}}
  \end{equation}
  And for $\alpha:2^\N$, we define the sequence $cs(\alpha) : \N \to \mathbb Q$ by 
  \begin{equation}
    cs(\alpha)_n = cs_n(\alpha|_n)
  \end{equation}
\end{definition}
\begin{remark}\label{rmkPropertiesCSn}
  $cs_n$ gives a bijection between $2^n$ and it's image 
  $\{\frac{k}{2^n}|0\leq k \leq 2^{n}-1\}\subseteq \mathbb Q$.
%  of rational numbers of the form  
%  $\frac{k}{2^n}$ for $0\leq k \leq 2^n-1$. 
  This observation has some corollaries: 
  \begin{itemize}
    \item In particular, each $cs_n$ is injective. 
    \item Furthermore, whenever $a\neq b:2^n$, we must have that 
      \begin{equation} 
        |cs_n(a)-cs_n(b)|\geq \frac{1}{2^n}.
      \end{equation}
    \item It is known that $\bigcup_{n:\N} \{\frac{k}{2^n}|0\leq k \leq 2^{n}-1\}$ 
      lies dense in the interval of Cauchy reals $[0,1]$. 
      It follows that $cs$ induces a surjection from Cantor space to $[0,1]$. %the interval of Cauchy reals. 
      We claim without proof it in fact induces an equivalence between $\I$ and $[0,1]$.
%      between $I$ and the interval of Cauchy reals. 
  \end{itemize}
  Finally, let us repeat a well-known identity for all $m<n$ on such sums, which we'll make some use of 
  \begin{equation}
   \sum\limits_{i = m}^{n-1} \frac{1}{2^{i+1}} = \frac{1}{2^{m}} - \frac{1}{2^n}
  \end{equation}
\end{remark}
\begin{lemma}\label{CauchyApproxLemma}
  Let $n:\N$ and  $s,t:2^n$. Assume there is some $ m \leq n$ with $cs_m(s|_m) = cs_m(t|_m) + \frac{1}{2^m}$, and 
  at the same time $cs_n(s) -cs_n(t)\leq \frac{1}{2^n}$. 
  Then there is some $k< m$ and some $u:2^k$ such that 
  \begin{equation}
    (s = u \cdot 1 \cdot \overline 0|_n)
    \wedge 
    (t = u \cdot 0 \cdot \overline 1|_n)
  \end{equation}
\end{lemma}
\begin{proof}
%  By injectivity of $cs_m$, 
  By assumption, we have that $s|_m \neq t|_m$. 
  Then there must be some smallest number $k<m$ such that 
  $s(k) \neq t(k)$. As $k$ is minimal, we have $s|_k = t|_k = : u$. 
%  WLOG, we assume that $s(m) = 1, t(m) = 0$. 
  It follows for all $l\leq n$ that 
%  We thus have for all $k<l\leq n$ that 
%  \begin{align}
%    cs_l(s|_l) &= 
%    cs_k(u|_k) + \sum\limits_{i = k}^{l-1} \frac{s(i)}{2^{i+1}}\\
%    cs_l(t|_l) &= 
%    cs_k(u|_k) + \sum\limits_{i = k}^{l-1} \frac{t(i)}{2^{i+1}}
%  \end{align}
%  And thus 
  \begin{align}
    cs_l(s|_l)-cs_l(t|_l) 
    = \sum\limits_{i = k}^{l-1} \frac{s(i)-t(i)}{2^{i+1}}
%    =\frac{s(k)-t(k)}{2^{k+1}} + \sum\limits_{i = k+1}^{l-1} \frac{s(i)-t(i)}{2^{i+1}}
  \end{align}
  Note that as $s(i),t(i) \in \{0,1\}$, we must have %that $s(i) -t(i) \in \{-1,0,1\}$. 
  $|s(i)-t(i)| \leq 1$. 
  Hence for any $k'<l$, we have 
  \begin{equation}
    \left|\sum\limits_{i = k'}^{l-1} \frac{s(i)-t(i)}{2^{i+1}}\right|
    \leq 
    \sum\limits_{i = k'}^{l-1} \frac{1}{2^{i+1}}
    = 
    \frac{1}{2^{k'}} - \frac{1}{2^{l}}
  \end{equation}
  Note that using the two equations above for $l=m$ and $k'=k+1$ we have:
  \begin{align}
    cs_m(s|_m) -cs_m(t|_m) 
    =&
    \frac{s(k)-t(k)}{2^{k+1}} + \sum\limits_{i = k+1}^{m-1} \frac{s(i)-t(i)}{2^{i+1}} \\
    \leq& 
    \frac{s(k)-t(k)}{2^{k+1}} + \left(\frac{1}{2^{k+1}} - \frac{1}{2^{m}}\right)
  \end{align}
  As the left hand side should equal $\frac{1}{2^m}$,
  we must have that $s(k)-t(k) \neq -1$. 
  As $s(k) \neq t(k)$ it follows that $s(k) = 1, t(k) = 0$.
  But now 
  \begin{equation}
    cs_n(s) -cs_n(t) 
    =
    \frac{1}{2^{k+1}} + \sum\limits_{i = k+1}^{n-1} \frac{s(i)-t(i)}{2^{i+1}}
    \geq 
    \frac{1}{2^{k+1}} - \left(\frac{1}{2^{k+1}} - \frac{1}{2^n} \right)
    =
    \frac{1}{2^{n}}
  \end{equation}
  And as $cs_n(s)-cs_n(t) \leq \frac{1}{2^n}$ as well, we get that 
  $cs_n(s)-cs_n(t) = \frac{1}{2^n}$. 
  Note that this lower bound is only reached if $s(i)-t(i) = -1$ for all $k<i<n$. 
  Hence for those $i$, we have $s(i) = 0, t(i) = 1$. 
  Thus 
  \begin{equation}
    s = (u \cdot 1\cdot \overline 0) |_n \wedge 
    t = (u \cdot 0\cdot \overline 1) |_n.
  \end{equation}
\end{proof}

 
\begin{corollary}\label{alternativeSimByCauchyDistance}
  Let $n:\N$ and let $s,t:2^n$. Then 
  \begin{equation}
    s\sim_n t \leftrightarrow |cs_n(s) - cs_n(t)| \leq \frac{1}{2^{n}}.
  \end{equation} 
\end{corollary}

\begin{proof}
  \item  
    Assume $ s \sim_n t$. If $s=t$, we have $cs_n(s) - cs_n(t) = 0$, 
    otherwise, we may without loss of generality assume there is some $m<n$ and some $u:2^m$ such that 
  \begin{equation}
    (s = u \cdot 0 \cdot \overline 1|_n) \wedge ( t = u \cdot 1 \cdot \overline 0 |_n) . 
  \end{equation}
  Then 
  \begin{align}
    cs_n(s) &= 
    cs_m(u) + 0 + \sum\limits_{i = m+1}^{n-1} \frac{1}{2^{i+1}}\\
    cs_n(t) &= 
    cs_m(u) + \frac{1}{2^{m+1}} + 0  
  \end{align}
  And hence 
  \begin{equation}
    cs_n(t) - cs_n(s) = \frac{1}{2^{m+1}} - \sum\limits_{i = m+1}^{n-1} \frac{1}{2^{i+1}} = \frac{1}{2^n}
  \end{equation}
  Thus in all cases, from $s\sim_n t$, we can conclude that 
  \begin{equation}
    |cs_n(s) -cs_n(t) |\leq \frac{1}{2^n}
  \end{equation}
  \item 
  Conversely, assume that $|cs_n(s) - cs_n(t)| \leq \frac{1}{2^n}$. 
  If $s = t$, it is clear that $s \sim_n t$.
  If $s\neq t$, there must be some smallest number $m<n$ such that 
  $s(m) \neq t(m)$. As $m$ is minimal, we have $s|_m = t|_m = : u$. 
  WLOG, we assume that $s(m) = 1, t(m) = 0$. 
  Then $cs_m(s|_{m+1})  = cs_{m+1}(t|_{m+1}) + \frac{1}{2^{m+1}}$
  and by \Cref{CauchyApproxLemma} it follows that 
  \begin{equation}
    s = (u \cdot 1\cdot \overline 0) |_n \wedge 
    t = (u \cdot 0\cdot \overline 1) |_n.
  \end{equation}
  and thus we can conclude $s\sim_n t$ as required. 
\end{proof}


Inspired by Definitions 2.7 and 2.10 \Cite{Bishop}, 
we define inequality on $\I$ as follows:
\begin{definition}
  Let $\alpha,\beta:2^\N$. 
  We define $\alpha\leq_\I \beta$ and $\alpha<_\I\beta$ as follows:
  \begin{align}
  \alpha\leq_\I\beta : = \forall_{n:\N} \left( cs(\alpha)_n \leq cs(\beta)_n + \frac {1} {2^n}\right)\\ 
    \alpha   <_\I \beta : = \exists_{n:\N} \left( cs(\alpha)_n < cs(\beta)_n - \frac {1} {2^n}\right)
%    \\\rednote{Can become n\pm1, \leq ,<, +\frac1{2^n+2} }
\end{align}
\end{definition}
\begin{lemma}
  $\leq_\I$ respects $\sim_\I$. 
\end{lemma}
\begin{proof}
  We will show that whenever $\alpha\leq_\I \gamma$ and $\alpha\sim_\I\beta$, we have $\beta\leq_\I\gamma$. 
  The other proof obligation goes similarly. 
%  The proof is similar to $\alpha'\leq_\I\gamma'$ and $\gamma'\sim_\I\beta'$, we have $\alpha'\leq_\I\beta'$.


  As $\beta\leq_\I\gamma$ is closed, by \Cref{rmkOpenClosedNegation} it is double negation stable. 
  By \Cref{MarkovPrinciple}, the negation is that there is some 
  $N:\N$ with 
  $cs(\beta)_N > cs(\gamma)_N + \frac{1}{2^N}.$
  As $\alpha\leq_\I\gamma$, we have 
  $cs(\gamma)_N + \frac{1}{2^N}\geq cs(\alpha)_N $. 
  Thus $cs(\beta)_N > cs(\alpha)_N$ and therefore $cs(\beta)_N = cs(\alpha)_N+\frac{1}{2^N}$ using  $\alpha\sim_\I\beta$.
%  Yet as $\alpha\sim_\I\beta$, from \Cref{alternativeSimByCauchyDistance}
%  we have $cs(\beta)_n \leq cs(\alpha)_n + \frac{1}{2^n}$ for all $n:\N$. 
%  Therefore, by \Cref{CauchyApproxLemma}, for $n\geq N$, we may conclude that 
  It follows that 
  $$
  cs(\alpha)_N+\frac{1}{2^N} > cs(\gamma)_N + \frac{1}{2^N} \geq cs(\alpha)_N
  $$
  From \Cref{rmkPropertiesCSn}, we must have
  $cs(\gamma)_N  + \frac{1}{2^N} = cs(\alpha)_N$, otherwise the distance 
  between $cs(\gamma)_N$ and $cs(\alpha)_N$ 
  would be smaller than $\frac{1}{2^N}$.
%  Using again $\alpha\sim_\I\beta$ and \Cref{CauchyApproxLemma}, 
%  for $n\geq N$ we get 
%  $cs(\beta)_n = cs(\alpha)_n + \frac{1}{2^n}$.
  As $cs(\alpha)_n \leq cs(\gamma)_n + \frac{1}{2^n}$ for all $n\geq N$, 
  \Cref{CauchyApproxLemma} gives that 
  $\alpha\sim_\I\gamma$. But also $\beta\sim_\I\gamma$. 
  But now $\alpha,\beta,\gamma$ are all distinct yet related by $\sim_\I$, contradicting 
  \Cref{IntervalFiberSizeAtMost2}. 
\end{proof}

\begin{remark}\label{NegationOfGeq}
  By \Cref{MarkovPrinciple}, we have that $\neg (\alpha \leq \beta) \leftrightarrow (\beta <_\I \alpha)$. 
  It follows immediately that $<_\I$ also respects $\I$. 
  Therefore, $\leq_\I, <_\I$ induce relations $\leq,<$ on $\I$.
  As the order in $\mathbb Q$ is decidable, $\leq, <$ are closed and open respectively. 
\end{remark} 

\begin{lemma}\label{IntervalOrderLeqOrGeq}
  For any $x,y:\I$, we have $x\leq y \vee y \leq x$. 
\end{lemma}
\begin{proof}
  Note that $x\leq y \vee y \leq x$ is the disjunction of two closed propositions, hence by 
  \Cref{ClosedFiniteDisjunction} and \Cref{rmkOpenClosedNegation} we can show it's double negation instead. 
  By the above remark, the negation implies that $x>y$ and $y<x$. We will show this is a contradiction. 
  Let $\alpha,\beta:2^\N$ correspond to $x,y$ and assume $n,m:\N$ with 
  $cs(\alpha)_n < cs(\beta)_n-\frac{1}{2^n}$ and 
  $cs(\beta)_m < cs(\alpha)_m-\frac{1}{2^m}$. 
  WLOG assume $n<m$. In this case for $\gamma$ any of $\alpha,\beta$, we have
  $$0\leq cs(\gamma)_m - cs(\gamma)_n = \sum_{i = n}^{m-1} \frac{\gamma(i)}{2^{i+1}}\leq \frac{1}{2^n}-\frac{1}{2^m}$$
  While at the same time, we have 
  \begin{align}
    cs(\beta)_m - cs(\beta)_n &\leq cs(\alpha)_m -\frac{1}{2^m} - cs (\beta)_n \\
                              & = (cs(\alpha)_m-cs(\alpha)_n)  +      (cs(\alpha)_n -cs(\beta)_n) - \frac{1}{2^m}\\
                              & \leq (\frac{1}{2^n} - \frac{1}{2^m}) -\frac{1}{2^n}               - \frac{1}{2^m}\\
                              &<0
  \end{align}
  giving a contradiction as required. 
\end{proof}

\begin{remark}\label{rmkMapOutOfLeqGeq}
  From \Cref{alternativeSimByCauchyDistance} we have $((x\leq y) \wedge (y \leq x )) \leftrightarrow (x = y)$. 
  So in order to define a map $(x \leq y) \vee (y \leq x) \to P$, we need to define a map 
  $f:x\leq y \to P$ and a map $g:y \leq x \to P$ such that $f|_{x = y} = g|_{x=y}$. 
\end{remark}
\rednote{These properties are nice but not necessary and paused WIP:}
\rednote{It is no used for Bouwer's fixed point theorem}
\begin{corollary}\label{inequality-lesser-greater-than}
    For $x,y:\I$ we have $(x\leq y \wedge x \neq y) \leftrightarrow (x < y)$. 
    Also $(x\neq y) \leftrightarrow (x < y + x > y)$. 
\end{corollary} 
\begin{proof}
    By $(x<y)\leftrightarrow \neg (y\leq x)$
    It's also immediate from the definitions that $x<y$ implies $x\neq y$. 
    As $((x\leq y) \wedge (y \leq x )) \leftrightarrow (x = y)$, 
    if $x\leq y \wedge x \neq y$, we have $\neg (y \leq x)$, hence $x<y$. 
\end{proof}

%    \item $(\exists_{y:I}(x\leq y \wedge y \leq z ))\leftrightarrow (x \leq z)$. 
%    \item $(\exists_{y:I}(x<y \wedge y < z ))\leftrightarrow (x < z)$. 

\begin{lemma}
  Whenever $x,y:\I$ satisfy $x<y$, there is some $z:\I$ with  $x<z \wedge z< y$. 
\end{lemma} 

%
%\rednote{TODO}
%For any $x,y:I$ we have 
%\begin{itemize}
%  \item TODO $x\leq y \wedge x \neq y \to x < y$. 
%\end{itemize}



\subsection{The topology of the interval}


\begin{definition}
  Let $a,b:\I$. 
  Following standard notation, we denote
  \begin{equation}
    [a,b]:= \Sigma_{x:\I} (a\leq x \wedge x \leq b),
  \end{equation}
  we call subsets of $\I$ of this form closed intervals. 
%
  We also denote 
  \begin{align}
    (-\infty,a) &:= \Sigma_{x:\I} (x < a)   \\
    (a,\infty) &:= \Sigma_{x:\I} (a < x)  \\
    (-\infty,\infty) &:= \I  \\
    (a,b) &:= \Sigma_{x:\I} (a < x \wedge x < b),
  \end{align}
  We call subsets of $\I$ of these forms open intervals. 
\end{definition}
\begin{remark}
  Note that closed intervals and open intervals are closed and open respectively. 
\end{remark}


%\begin{lemma}\label{IntervalQuotientMapIntersectionCommute}
%  Let $D_n:2^\N \to 2$ be a sequence of decidable subsets with $D_{n+1}\subseteq D_n$.
%  For $p$ the quotient map $2^\N \to I$, we have that 
%  $p(\bigcap_{n:\N} D_n) = p(\bigcap_{n:\N} D_n)$
%\end{lemma}
%\begin{proof}
%  It is always the case that $$p(\bigcap_{n:\N} D_n) \subseteq \bigcap_{n:\N} p(D_n).$$
%  For the converse direction, let $(\bigcap_{n:\N} p(D_n))(x)$. 
%  We will show that $ \neg \neg (p(\bigcap D_n)) (x)$, which is sufficient by \Cref{rmkOpenClosedNegation}. 
%%
%  As $(\bigcap_{n:\N} p(D_n))(x)$, there exists some $y\in D_0$ with $p(y) = x$. 
%%
%  If $x\notin p(\bigcap_{n:\N} D_n)$, we cannot have for all $n:\N$ that $y_0 \in  D_n$. 
%  By Markov, there must exist some $k:\N$ with $\neg D_k(y_0)$. 
%  As $D_{n+1}\subseteq D_n$ for all $n:\N$, it follows that $y_0\notin D_n$ for all $n\geq k$. 
%%
%  As $x\in \bigcap_{n:\N}p(D_n)$, there is however some $y_k\in D_k$ with $p(y_k) = x$. 
%  By a similar argument, we have some $l>k$ with $y_k\notin D_l$, and some $y_l$ with $p(y_l) = x, y_l \in D_l$. 
%  But now we have that $y_0, y_k, y_l:2^\N$ are all distinct, but $p(y_0) = p(y_k) = p(y_l) = x$. 
%  This contradicts \Cref{IntervalFiberSizeAtMost2}, and we're done. 
%\end{proof}


\begin{lemma}\label{ImageDecidableClosedInterval}
  For $p:2^\N \to \I$ the quotient map and $D\subseteq 2^\N$ decidable, we have $p(D)$ a finite union of closed intervals. 
\end{lemma}
\begin{proof}
  We will show the above if there exists some $n:\N, u:2^n$ such that $D(\alpha) \leftrightarrow \alpha|_n = u$.
  This is sufficient, as any decidable subset of $2^\N$ can be written as finite union of such decidable subsets. 
  We claim that $p(D) = [p(u\cdot \overline 0) , p(u \cdot \overline 1)]$. 
\item 
  We will first show that $p(D) \subseteq [p(u\cdot \overline 0) , p(u \cdot \overline 1)]$. 
  Suppose $D(\alpha)$. Then 
  Then $\alpha|_n = u$ and hence 
%  for $m\leq n$ we have 
%  \begin{equation}
%    cs(\alpha)_m = cs_m(u|_m) = cs(u\cdot \overline 0)_m= cs(u\cdot \overline 1)_m
%  \end{equation}
%  For $m>n$, we have that 
%  \begin{align}
%    cs(u\cdot \overline 1)_m =
%    cs_n(u) +\sum_{i = n} ^{m-1} \frac{1}{2^{i+1}}
%    \\
%    cs(\alpha)_m =
%    cs_n(u) +\sum_{i = n} ^{m-1} \frac{\alpha(i)}{2^{i+1}}
%    \\
%    cs(u\cdot \overline 0)_m = 
%    cs_n(u) +\sum_{i = n} ^{m-1} \frac{0}{2^{i+1}}
%  \end{align} 
%  Hence for all $m:\N$, we have 
  \begin{equation}
    cs(u\cdot \overline 1)_m \geq 
    cs(\alpha)_m \geq 
    cs(u\cdot\overline 0)_m
  \end{equation}
 which implies that $p(u\cdot \overline 1) \geq_\I p(\alpha) \geq_\I p(u\cdot\overline 0)$, as required. 
\item 
  To show that $[p(u\cdot \overline 0) , p(u \cdot \overline 1)]\subseteq p(D)$, 
  Suppose
  $(u\cdot \overline 0) \leq_\I \alpha \leq_\I (u \cdot \overline 1)$. 
  It is sufficient to show that 
  $$(\alpha|_n = u )\vee (\alpha \sim_\I u \cdot \overline 0 )\vee (\alpha \sim_\I u \cdot \overline 1).$$
  As this is a disjunction of closed propositions, by \Cref{ClosedFiniteDisjunction} it's closed, and by 
  \Cref{rmkOpenClosedNegation}, we can instead show the double negation. 
  So suppose that none of the disjoints hold. 
  As $\alpha|_n \neq u$, there is some minimal $m$ with $\alpha(m) \neq u(m)$. 
  We assume that $\alpha(m) = 1, u(m) = 0$, the other case goes similarly. 
  Then for all $k:\N$, we have 
  $cs(\alpha)_k \geq cs(u \cdot \overline 1)|_k$. 
  As also 
  $(u\cdot \overline 1)\geq_\I \alpha$, we have 
  $$cs(u \cdot \overline 1)|_k + \frac{1}{2^k} \geq cs(\alpha)_k \geq cs(u\cdot \overline 1)_k,$$
  From which it follows that $|cs(u\cdot\overline 1)_k - cs(\alpha)_k|\leq \frac{1}{2^k}$. 
  Hence $(u\cdot \overline 1)|_k \sim_k \alpha|_k$ by \Cref{alternativeSimByCauchyDistance}. 
  Hence $x\sim_\I (a\cdot\overline 1)$, contradicting our assumption as required. 
\end{proof}

\begin{lemma}\label{complementClosedIntervalOpenIntervals}
  The complement of a finite union of closed intervals is 
  a finite union of open intervals. 
\end{lemma}
\begin{proof}
  We'll use induction on the amount of closed intervals. 
  The empty union of closed intervals is empty, and hence it's complement is $\I$, which is an open interval.  
  Let $(C_i)_{i<k}$ be a finite set of closed intervals with $\neg (\bigcup_{i<k}C_i)$ 
  a finite union of open intervals $\bigcup_{j<l} O_i$. 
  Suppose $C_{k}$ is closed. We need to show that 
  $\neg (\bigcup_{i\leq k} C_i)$ is also a finite union of open intervals. 
  First note that in general, 
  $(\neg (A \vee B ))\leftrightarrow (\neg A \wedge \neg B)$
  hence 
  $$
  \neg (\bigcup_{i\leq k} C_i)
  = 
  \neg ((\bigcup_{i<k} C_i) \cup C_k) 
  =
  (\neg (\bigcup_{i<k} C_i) )\cap (\neg C_k) 
  $$
  And by the induction hypothesis and distributivity, this equals 
  $$
  (\bigcup_{j<l} O_i) ) \cap (\neg C_k) 
  =
  \bigcup_{j<l} (O_i \cap (\neg C_k) )
  $$
  So we need to show that the intersection of an open interval and the negation of a closed interval is a 
  finite union of open intervals. We assume or open intervals are of the form $(a,b)$ for $a,b:\I$. 
  The other cases are very similar. 
  So let $a,b,c,d:\I$ and consider 
  $U = (a,b) \cap (\neg [c,d])$. 
  Then 
  \begin{align} 
    U(x) &= \Sigma_{x:\I}  
  (a < x \wedge x < b) \wedge ( x < c \vee d < x)\\
  &= \Sigma_{x:\I}
  (a < x \wedge x < b \wedge x < c ) \vee ( d < x \vee a<x \wedge x < b)\\
  &= 
  \Sigma_{x:\I}
  (a < x \wedge x < b \wedge x < c ) 
  \cup 
  \Sigma_{x:\I}
  ( d < x \vee a<x \wedge x < b)
  \end{align} 
  We will show that 
  $U' = \Sigma_{x:\I}(a < x \wedge x < b \wedge x < c ) $ is an open interval. 
  By a similar argument, the other part will be as well, meaning that $U$ is the union of two open intervals. 
  Consider that $b\leq c \vee c \leq b$. 
  If $b \leq c$, $(x<b \wedge x< c) \leftrightarrow x<b$ and $U' = (a,b)$
  If $c \leq b$, $(x<b \wedge x< c) \leftrightarrow x<c$ and $U' = (a,c)$
  If $b=c$, these open intervals agree, hence from \Cref{rmkMapOutOfLeqGeq} we can conclude that $U'$ is an open interval. 
  We conclude that $U$ is the union of two open intervals as required. 
\end{proof}
%
\begin{lemma}
  Every open $U\subseteq \I$ can be written as countable union of open intervals.
\end{lemma} 
\begin{proof}
%  Let $U\subseteq I$ open, then $\neg U$ is closed and $U = \neg \neg U$ by \Cref{rmkOpenClosedNegation}. 
%  By \Cref{StoneClosedSubsets}, \Cref{CompactHausdorffClosed} and \Cref{ChausMapsPreserveIntersectionOfClosed}
  %\Cref{IntervalQuotientMapIntersectionCommute}, 
  By \Cref{CompactHausdorffTopology}
  there is some sequence of decidable subsets $D_n\subseteq 2^\N$ 
%  with $\neg U = \bigcap_{n:\N} p(D_n)$. 
%  Thus $U = \neg \bigcap_{n:\N} p(D_n)$. 
%  By \Cref{ClosedMarkov}, it follows that 
  with $U = \bigcup_{n:\N} \neg p(D_n)$. 
  By \Cref{ImageDecidableClosedInterval}, each $p(D_n)$ is a finite union of closed intervals, 
  and by \Cref{complementClosedIntervalOpenIntervals} it follows that each $\neg p(D_n)$ is a finite union of open intervals. 
  We conclude that $U$ is a countable union of open intervals as required. 
\end{proof}
%
%%  $\neg U$ is a countable intersection of finite unions of closed intervals. 
%%  Thus $\neg\neg U$ is a countable union of finite intersections of complements of closed intervals. 
%%  As complements of closed intervals are finite unions of open intervals (TODO), 
%%  and finite intersections of such things are still finite unions of open intervals, 
%%  it follows that $\neg\neg U$ is a countable union of open intervals. 
%%  By \Cref{rmkOpenClosedNegation}, $\neg \neg U = U$ and we're done. 
%%  \rednote{Lotta handwaving here, definitely not finished} 
%\end{proof}
%

\begin{remark}\label{IntervalTopologyStandard}
  It follows that the topology of $\I$ is generated by open intervals, 
  which corresponds to the standard topology on $\I$. 
  Hence our notion of continuity corresponds with the $\epsilon,\delta$-definition of continuity one would expect. 
  Thus every function $f:\I\to \I$ in the system we presented is continuous in the $\epsilon,\delta$-sense. 
\end{remark}


\section{Cohomology}
In this section we compute $H^1(S,\Z) = 0$ for all $S$ Stone, and show that $H^1(X,\Z)$ for $X$ compact Hausdorff can be computed using \v{C}ech cohomology. We use this to compute $H^1(\I,\Z)=0$. 

\begin{remark}
We only work with the first cohomology group with coefficients in $\Z$ as it is sufficient for the proof of Brouwer's fixed-point theorem, but the results could be extended to $H^n(X,A)$ for $A$ any family of countably presented abelian groups indexed by $X$.
\end{remark}

\begin{remark}
We write $\mathrm{Ab}$ for the type of abelian groups and if $G:\mathrm{Ab}$ we write $\B G$ for the delooping of $G$ \cite{hott,davidw23}. This means that $H^1(X,G)$ is the set truncation of $X \to \B G$. 
\end{remark}

\subsection{\v{C}ech cohomology}

\begin{definition}
Given a type $S$, types $T_x$ for $x:S$ and $A:S\to\mathrm{Ab}$, we define $\check{C}(S,T,A)$ as the chain complex
\[
\begin{tikzcd}
     \prod_{x:S}A_x^{T_x} \ar[r,"d_0"] & \prod_{x:S}A_x^{T_x^2}\ar[r,"d_1"] &  \prod_{x:S}A_x^{T_x^3}
\end{tikzcd}
\]
where the boundary maps are defined as
\begin{align*}
d_0(\alpha)_x(u,v) =&\ \alpha_x(v)-\alpha_x(u)\\
d_1(\beta)_x(u,v,w) =&\ \beta_x(v,w) - \beta_x(u,w) + \beta_x(u,v)
\end{align*}
\end{definition}

\begin{definition}
Given a type $S$, types $T_x$ for $x:S$ and $A:S\to\mathrm{Ab}$, we define its \v{C}ech cohomology groups by
\[
  \check{H}^0(S,T,A) = \mathrm{ker}(d_0)\quad \quad \quad \check{H}^1(S,T,A) = \mathrm{ker}(d_1)/\mathrm{im}(d_0)
\]
We call elements of $\mathrm{ker}(d_1)$ cocycles and elements of $\mathrm{im}(d_0)$ coboundaries.
\end{definition}

This means that $\check{H}^1(S,T,A) = 0$ if and only if $\check{C}(S,T,A)$ is exact at the middle term. Now we give three general lemmas about \v{C}ech complexes.

\begin{lemma}\label{section-exact-cech-complex}
Assume a type $S$, types $T_x$ for $x:S$ and $A:S\to\mathrm{Ab}$ with $t:\prod_{x:S}T_x$. Then $\check{H}^1(S,T,A)=0$.
\end{lemma}

\begin{proof}
Assume given a cocycle, i.e. $\beta:\prod_{x:S}A_x^{T_x^2}$ such that for all $x:S$ and $u,v,w:T_x$ we have that $\beta_x(u,v)+\beta_x(v,w) = \beta_x(u,w)$. We define $\alpha:\prod_{x:S}A_x^{T_x}$ by $\alpha_x(u) = \beta_x(t_x,u)$. Then for all $x:S$ and $u,v:T_x$ we have that $d_0(\alpha)_x(u,v) =  \beta_x(t_x,v) - \beta_x(t_x,u) = \beta_x(u,v)$ so that $\beta$ is a coboundary.
\end{proof}

\begin{lemma}\label{canonical-exact-cech-complex}
Given a type $S$, types $T_x$ for $x:S$ and $A:S\to\mathrm{Ab}$, we have that $\check{H}^1(S,T,\lambda x.A_x^{T_x})=0$.
\end{lemma}

\begin{proof}
Assume given a cocycle, i.e. $\beta:\prod_{x:S}A_x^{T_x^3}$ such that for all $x:S$ and $u,v,w,t:T_x$ we have that $\beta_x(u,v,t)+\beta_x(v,w,t) = \beta_x(u,w,t)$. We define $\alpha:\prod_{x:S}A_x^{T_x^2}$ by $\alpha_x(u,t) = \beta_x(t,u,t)$. Then for all $x:S$ and $u,v,t:T_x$ we have that $d_0(\alpha)_x(u,v,t) = \beta_x(t,v,t) - \beta_x(t,u,t) = \beta_x(u,v,t)$ so that $\beta$ is a coboundary.
\end{proof}

\begin{lemma}\label{exact-cech-complex-vanishing-cohomology}
Assume a type $S$ and types $T_x$ for $x:S$ such that $\prod_{x:S}\propTrunc{T_x}$ and $A:S\to\mathrm{Ab}$ such that $\check{H}^1(S,T,A) = 0$.
Then given $\alpha:\prod_{x:S}\B A_x$ with $\beta:\prod_{x:S} (\alpha(x) = *)^{T_x}$, we can conclude $\alpha = *$.
\end{lemma}

\begin{proof}
We define $g : \prod_{x:S} A_x^{T_x^2}$ by $g_x(u,v) = \beta_x(v) - \beta_x(u)$.
It is a cocycle in the \v{C}ech complex, so that by exactness there is $f:\prod_{x:S}A_x^{T_x}$ such that for all $x:S$ and $u,v:T_x$ we have that $g_x(u,v)= f_x(v) - f_x(u)$.
Then we define $\beta' : \prod_{x:S}(\alpha(x)=*)^{T_x}$ by $\beta'_x(u) = \beta_x(u) - f_x(u)$
so that for all $x:S$ and $u,v:T_x$ we have that $\beta'_x(u) = \beta'_x(v)$ is equivalent to $f_x(v) - f_x(u) = \beta_x(v) - \beta_x(u)$, which holds by definition. So $\beta'$ is constant on each $T_x$ and therefore gives $\prod_{x:S} (\alpha(x)=*)^{\propTrunc{T_x}}$. By $\prod_{x:S}\propTrunc{T_x}$ we conclude $\alpha = *$.
\end{proof}


\subsection{Cohomology of Stone spaces}

%
%%\subsection{Needed results}
%
%%\rednote{Should probably be moved elsewhere}
%
\begin{lemma}\label{finite-approximation-surjection-stone}
Assume given $S:\Stone$ and $T:S\to\Stone$ such that $\prod_{x:S}\propTrunc{T(x)}$.
Then there exists a sequence of finite types $(S_k)_{k:\N}$ with limit $S$ 
%\rednote{Should the maps in the sequence be mentioned? (Maps $p_k$ are mentioned below)}
%\begin{equation}
%\begin{tikzcd}
%S_0 & S_1 \ar[l,"p_0"]& S_2\ar[l,"p_1"] & \cdots\ar[l]\\
%\end{tikzcd}
%\end{equation}
%such that: 
%\[\mathrm{lim}_kS_k = S\]
and a compatible sequence of families of finite types $T_k$ over $S_k$
with $\prod_{x:S_k}\propTrunc{T_k(x)}$ and 
$\mathrm{lim}_k\left(\sum_{x:S_k}T_k(x)\right) = \sum_{x:S}T(x)$. 
%
%Given $S:\Stone$ and $T:S\to\Stone$ such that $\prod_{x:S}\propTrunc{T(x)}$, there exists a sequence of finite types $(S_k)_{k:\N}$
%\rednote{Should the maps in the sequence be mentioned? (Maps $p_k$ are mentioned below)}
%%\begin{equation}
%%\begin{tikzcd}
%%S_0 & S_1 \ar[l,"p_0"]& S_2\ar[l,"p_1"] & \cdots\ar[l]\\
%%\end{tikzcd}
%%\end{equation}
%such that: 
%\[\mathrm{lim}_kS_k = S\]
%and for each $k:\N$ we have a family of finite types $T_k(x)$ for $x:S_k$ such that $\prod_{x:S_k}\propTrunc{T_k(x)}$ with maps $T_{k+1}(x) \to T_k(p_k(x))$ such that:
%\[\mathrm{lim}_k\left(\sum_{x:S_k}T_k(x)\right) = \sum_{x:S}T(x)\]
\end{lemma}

\begin{proof}
By theorem \Cref{stone-sigma-closed} and the usual correspondence between surjections and families of inhabited types, a family of inhabited Stone spaces over $S$ correspond to a Stone space $T$ with a surjection $T\to S$. Then we conclude using \Cref{ProFiniteMapsFactorization}.
%\rednote{ \@ Hugo This follows from \Cref{ProFiniteMapsFactorization} and \Cref{stone-sigma-closed} 
%  and considering the surjection $(\Sigma_{x:S} T(x)) \to S$, but we discussed whether it might be easier to 
%  refactor the proof where you use the above or make a remark after \Cref{stone-sigma-closed}}
\end{proof}

\begin{lemma}\label{cech-complex-vanishing-stone}
Assume given $S:\Stone$ with $T:S\to\Stone$ such that $\prod_{x:S}\propTrunc{T_x}$. Then we have that $\check{H}^1(S,T,\Z) = 0$.
\end{lemma}


\begin{proof}
We apply \cref{finite-approximation-surjection-stone} to get $S_k$ and $T_k$ finite. Then by \cref{scott-continuity} we have that $\check{C}(S,T,\Z)$ is the sequential colimit of the $\check{C}(S_k,T_k,\Z)$. By \cref{section-exact-cech-complex} we have that each of the $\check{C}(S_k,T_k,\Z)$ is exact, and a sequential colimit of exact sequences is exact.
\end{proof}

\begin{lemma}\label{eilenberg-stone-vanish}
Given $S:\Stone$, we have that $H^1(S,\Z) = 0$. 
\end{lemma}

\begin{proof}
Assume given a map $\alpha:S\to \B\Z$. We use local choice to get $T:S\to\Stone$ such that $\prod_{x:S}\propTrunc{T_x}$ with $\beta:\prod_{x:S}(\alpha(x)=*)^{T_x}$. Then we conclude by \cref{cech-complex-vanishing-stone} and \cref{exact-cech-complex-vanishing-cohomology}.
\end{proof}

\begin{corollary}\label{stone-commute-delooping}
For any $S:\Stone$ the canonical map $\B(\Z^S) \to (\B\Z)^S$ is an equivalence.
\end{corollary}

\begin{proof}
This map is always an embedding. To show it is surjective it is enough to prove that $(\B\Z)^S$ is connected, which is precisely \Cref{eilenberg-stone-vanish}.
\end{proof}


\subsection{\v{C}ech cohomology of compact Hausdorff spaces}

\begin{definition}
A \v{C}ech cover consists of $X:\CHaus$ and $S:X\to\Stone$ such that $\prod_{x:X}\propTrunc{S_x}$ and $\sum_{x:X}S_x:\Stone$.
\end{definition}

By definition any compact Hausdorff space $X$ is part of a \v{C}ech cover $(X,S)$.

\begin{lemma}\label{cech-eilenberg-0-agree}
Given a \v{C}ech cover $(X,S)$ and $A:X\to\mathrm{Ab}$, we have an isomorphism $H^0(X,A) = \check{H}^0(X,S,A)$ natural in $A$.
\end{lemma}

\begin{proof}
By definition an element in $\check{H}^0(X,S,A)$ is a map $f:\prod_{x:X}A_x^{S_x}$
such that for all $u,v:S_x$ we have $f(u)=f(v)$. Since $A_x$ is a set and the $S_x$ are merely inhabited, this is equivalent to $\prod_{x:X}A_x$. Naturality in $A$ is immediate.
\end{proof}

\begin{lemma}\label{eilenberg-exact}
Given a \v{C}ech cover $(X,S)$ we have an exact sequence
\[H^0(X,\lambda x.\Z^{S_x}) \to H^0(X,\lambda x.\Z^{S_x}/\Z) \to H^1(X,\Z)\to 0\]
\end{lemma}

\begin{proof}
We use the long exact cohomology sequence associated to
\[0 \to \Z \to \Z^{S_x} \to \Z^{S_x}/\Z\to 0\]
We just need $H^1(X,\lambda x.\Z^{S_x}) = 0$ to conclude. But by \cref{stone-commute-delooping} we have that $H^1(X,\lambda x.\Z^{S_x}) = H^1\left(\sum_{x:X}S_x,\Z\right)$ which vanishes by \cref{eilenberg-stone-vanish}.
\end{proof}

\begin{lemma}\label{cech-exact}
Given a \v{C}ech cover $(X,S)$ we have an exact sequence
\[\check{H}^0(X,S,\lambda x.\Z^{S_x}) \to \check{H}^0(X,S,\lambda x.\Z^{S_x}/\Z) \to \check{H}^1(X,S,\Z)\to 0\]
\end{lemma}

\begin{proof}
For $n=1,2,3$, we have that $\Sigma_{x:X}S_x^n$ is Stone so that  $H^1(\Sigma_{x:X}S_x^n, \Z) = 0$ by \cref{eilenberg-stone-vanish}, giving short exact sequences
\[0\to \Pi_{x:X}\Z^{S_x^n} \to \Pi_{x:X}(\Z^{S_x})^{S_x^n}\to \Pi_{x:X}(\Z^{S_x}/\Z)^{S_x^n}\to 0\]
They fit together in a short exact sequence of complexes
\[0 \to \check{C}(X,S,\Z) \to \check{C}(X,S,\lambda x.\Z^{S_x}) \to \check{C}(X,S,\lambda x.\Z^{S_x}/\Z)\to 0\]
But since $\check{H}^1(X,\lambda x.\Z^{S_x}) = 0$ by \cref{canonical-exact-cech-complex}, we conclude using the associated long exact sequence.
\end{proof}

\begin{theorem}\label{cech-eilenberg-1-agree}
Given a \v{C}ech cover $(X,S)$, we have that $H^1(X,\Z) = \check{H}^1(X,S,\Z)$
\end{theorem}

\begin{proof}
By applying \cref{cech-eilenberg-0-agree}, \cref{eilenberg-exact} and \cref{cech-exact} we get that $H^1(X,\Z)$ and $\check{H}^1(X,S,\Z)$ are cokernels of isomorphic maps, so they are isomorphic.
\end{proof}

This means that \v{C}ech cohomology does not depend on $S$.

\subsection{Cohomology of the interval}
%
%Recall that we denote $C_n=2^n$ with a binary relation $\sim_n$ on $C_n$ such that for all $x,y:2^\N$ we have that:
%\[\left(\forall(n:\N).\ x|_n\sim_n y|_n\right) \leftrightarrow x=_\I y\]
%
%\begin{lemma}\label{description-Cn-simn}
%We have that $(C_n,\sim_n)$ is equivalent to $(\Fin(2^n),\lambda x,y.\ |x-y|\leq 1)$.
%\end{lemma}
\begin{remark}\label{description-Cn-simn}
  Recall from \Cref{def-cs-Interval} that 
  there is a binary relation $\sim_n$ on $2^n=:\I_n$ such that 
  $(2^n,\sim_n)$ is equivalent to  $(\Fin(2^n),\lambda x,y.\ |x-y|\leq 1)$
  and for $\alpha,\beta:2^\N$ we have $(cs(\alpha) = cs(\beta)) \leftrightarrow 
  \left(\forall_{n:\N}\alpha|_n \sim_n \beta|_n\right)$. 
\end{remark}

We define $\I_n^{\sim2} = \Sigma_{x,y:\I_n}x\sim_n y$ and $\I_n^{\sim3} = \Sigma_{x,y,z:\I_n}x\sim_n y \land y\sim_n z\land x\sim_n z$.

\begin{lemma}\label{Cn-exact-sequence}
For any $n:\N$ we have an exact sequence
\[0\to \Z\overset{d_0}{\longrightarrow} \Z^{\I_n} \overset{d_1}{\longrightarrow} \Z^{\I_n^{\sim2}} \overset{d_2}{\longrightarrow} \Z^{\I_n^{\sim3}}\]
where $d_0(k) = (\_\mapsto k)$ and
\begin{eqnarray}
 d_1(\alpha)(u,v) &=& \alpha(v)-\alpha(u)\nonumber\\
 d_2(\beta)(u,v,w) &=& \beta(v,w)-\beta(u,w)+\beta(u,v).\nonumber
\end{eqnarray}
\end{lemma}

\begin{proof}
It is clear that the map $\Z\to \Z^{\I_n}$ is injective as $\I_n$ is inhabited, so the sequence is exact at $\Z$. Assume a cocycle $\alpha:\Z^{\I_n}$, meaning that for all $u,v:\I_n$, if $u\sim_nv$ then $\alpha(u)=\alpha(v)$. Then by \cref{description-Cn-simn} we see that $\alpha$ is constant, so the sequence is exact at $\Z^{\I_n}$.

Assume a cocycle $\beta:\Z^{\I_n^{\sim2}}$, meaning that for all $u,v,w:\I_n$ such that $u\sim_nv$, $v\sim_nw$ and $u\sim_nw$ we have that $\beta(u,v)+\beta(v,w) = \beta(u,w)$. %This is equivalent to asking $\beta(u,u)=0$ and $\beta(u,v) = -\beta(v,u)$.
Using \cref{description-Cn-simn} to pass along the equivalence between $2^n$ and $\Fin(2^n)$, we define $\alpha(k) = \beta(0,1)+\cdots+\beta(k-1,k)$.
We can check that $\beta(k,l) = \alpha(l)-\alpha(k)$, so that $\beta$ is indeed a coboundary and the sequence is exact at $\Z^{\I_n^{\sim2}}$.
\end{proof}

\begin{proposition}\label{cohomology-I}
We have that $H^0(\I,\Z) = \Z$ and $H^1(\I,\Z) = 0$.
\end{proposition}

\begin{proof}
Consider $cs:2^\N\to\I$ and the associated \v{C}ech cover $T$ of $\I$ defined by: 
\[T_x = \Sigma_{y:2^\N} (x=_\I cs(y))\]
Then for $l=2,3$ we have that $\mathrm{lim}_n\I_n^{\sim l} = \sum_{x:\I} T_x^l$. By \cref{Cn-exact-sequence} and stability of exactness under sequential colimit, we have an exact sequence
\[ 0\to \Z\to \mathrm{colim}_n \Z^{\I_n} \to \mathrm{colim}_n \Z^{\I_n^{\sim2}}\to \mathrm{colim}_n \Z^{\I_n^{\sim3}}\]
By \cref{scott-continuity} this sequence is equivalent to
\[ 0\to \Z\to \Pi_{x:\I}\Z^{T_x} \to  \Pi_{x:\mathbb{I}}\Z^{T_x^2} \to  \Pi_{x:\mathbb{I}}\Z^{T_x^3}\]
So it being exact implies that $\check{H}^0(\I,T,\Z) = \Z$ and $\check{H}^1(\I,T,\Z) = 0$.
We conclude by \cref{cech-eilenberg-0-agree} and \cref{cech-eilenberg-1-agree}.
\end{proof}

\begin{remark}
We could carry a similar computation for $\mathbb{S}^1$, by approximating it with $2^n$ with $0^n\sim_n1^n$ added. We would find $H^1(\mathbb{S}^1,\Z)=\Z$. We will give an alternative, more conceptual proof in the next section.
\end{remark}


\subsection{Brouwer's fixed-point theorem}

Here we consider the modality defined by localising at $\I$ as explained in \cite{modalities}. It is denoted by $L_\I$. We say that $X$ is $\I$-local if $L_\I(X) = X$ and that it is $\I$-contractible if $L_\I(X)=1$.

\begin{lemma}\label{Z-I-local}
$\Z$ and $2$ are $\I$-local.
\end{lemma}

\begin{proof}
By \cref{cohomology-I}, from $H^0(\I,\Z)=\Z$ we get that the map $\Z\to \Z^\I$ is an equivalence, so $\Z$ is $\I$-local. We see that $2$ is $\I$-local as it is a retract of $\Z$.
\end{proof}

\begin{remark}
Since $2$ is $\I$-local, we have that any Stone space is $\I$-local.
\end{remark}

\begin{lemma}\label{BZ-I-local}
$\B\Z$ is $\I$-local.
\end{lemma}

\begin{proof}
Any identity type in $\B\Z$ is a $\Z$-torsor, so it is $\I$-local by \cref{Z-I-local}. So the map $\B\Z\to \B\Z^{\I}$ is an embedding. From $H^1(\I,\Z)=0$ we get that it is surjective, hence an equivalence.
\end{proof}

\begin{lemma}\label{continuously-path-connected-contractible}
Assume $X$ a type with $x:X$ such that for all $y:X$ we have $f:\I\to X$ such that $f(0)=x$ and $f(1)=y$. Then $X$ is $\I$-contractible.
\end{lemma}

\begin{proof}
%First we prove that the map:
%\[\eta_X:X\to L_\I(X)\] 
%is surjective. Indeed its fiber are $\I$-contractible, but for any type $F$ we have a map:
%\[L_\I(F) \to L_\mathbb{F}(\propTrunc{F}) = \propTrunc{F}\] 
For all $y:X$ we get a map $g:\I\to L_\I(X)$ such that $g(0) = [x]$ and $g(1)=[y]$. Since $L_\I(X)$ is $\I$-local this means that $\prod_{y:X}[x]=[y]$. We conclude $\prod_{y:L_\I(X)}[x]=y$ by applying the elimination principle for the modality.
\end{proof}

\begin{corollary}\label{R-I-contractible}
We have that $\R$ and $\mathbb{D}^2=\{(x,y):\mathbb R^2\ \vert\ x^2+y^2\leq 1\}$ are $\I$-contractible.
\end{corollary}

\begin{proposition}\label{shape-S1-is-BZ}
$L_\I(\R/\Z) = \B\Z$.
\end{proposition}

\begin{proof}
As for any group quotient, the fibers of the map $\R\to\R/\Z$ are $\Z$-torsors, so we have an induced pullback square
\[
\begin{tikzcd}
\R\ar[r]\ar[d] & 1\ar[d] \\
\R/\Z\ar[r] & \B\Z
\end{tikzcd}
\]
Now we check that the bottom map is an $\I$-localisation. Since $\B\Z$ is $\I$-local by \cref{BZ-I-local}, it is enough to check that its fibers are $\I$-contractible. Since $\B\Z$ is connected it is enough to check that $\R$ is $\I$-contractible. This is \cref{R-I-contractible}.
\end{proof}

\begin{remark}
By \cref{BZ-I-local}, for any $X$ we have that $H^1(X,\Z) = H^1(L_{\I}(X),\Z)$, so that by \cref{shape-S1-is-BZ} we have that $H^1(\R/\Z,\Z) = H^1(\B\Z,\Z) = \Z$.
\end{remark}

We omit the proof that $\mathbb{S}^1=\{(x,y):\R^2\ \vert\ x^2+y^2=1\}$ is equivalent to $\R/\Z$.
The equivalence can be constructed using trigonometric functions, which exist by Proposition 4.12 in \cite{Bishop}.

\begin{proposition}
\label{no-retraction}
The map $\mathbb{S}^1\to \mathbb{D}^2$ has no retraction.
\end{proposition}

\begin{proof}
By \cref{R-I-contractible} and \cref{shape-S1-is-BZ} we would get a retraction of $\B\Z\to 1$, so $\B\Z$ would be contractible.
\end{proof}

\begin{theorem}[Intermediate value theorem]
  \label{ivt}
  For any $f: \I\to \I$ and $y:\I$ such that $f(0)\leq y$ and $y\leq f(1)$,
  there exists $x:\I$ such that $f(x)=y$.
\end{theorem}

\begin{proof}
  By \Cref{InhabitedClosedSubSpaceClosedCHaus}, the proposition $\exists_{x:\I}\, f(x)=y$ is closed and therefore $\neg\neg$-stable, so we can proceed with a proof by contradiction.
  If there is no such $x:\I$, we have $f(x)\neq y$ for all $x:\I$.
  By \cref{LesserOpenPropAndApartness} we have that $a<b$ or $b<a$ for all distinct numbers $a,b:\I$. So the following two sets cover $\I$
  \[
    U_0:= \{x:\I\mid f(x)<y\} \quad\quad
    U_1:= \{x:\I\mid y<f(x)\}
    \]
  Since $U_0$ and $U_1$ are disjoint, we have $\I=U_0+U_1$ which allows us to define a non-constant function $\I\to 2$, which contradicts \Cref{Z-I-local}.
\end{proof}

\begin{theorem}[Brouwer's fixed-point theorem]
  For all $f:\mathbb{D}^2\to \mathbb{D}^2$ there exists $x:\mathbb{D}^2$ such that $f(x)=x$.
\end{theorem}

\begin{proof}
  As above, by \Cref{InhabitedClosedSubSpaceClosedCHaus}, we can proceed with a proof by contradiction,
  so we assume $f(x)\neq x$ for all $x:\mathbb{D}^2$.
  For any $x:\mathbb{D}^2$ we set $d_x= x-f(x)$, so we have that one of the coordinates of $d_x$ is invertible.
  Let $H_x(t) = f(x) + t\cdot d_x $ be the line through $x$ and $f(x)$.
  The intersections of $H_x$ and $\partial\mathbb{D}^2=\mathbb{S}^1$ are given by the solutions of an equation quadratic in $t$. By invertibility of one of the coordinates of $d_x$, there is exactly one solution with $t> 0$.
  We denote this intersection by $r(x)$ and the resulting map $r:\mathbb D^2\to\mathbb S^1$ has the property that it preserves $\mathbb{S}^1$.
  Then $r$ is a retraction from $\mathbb{D}^2$ onto its boundary $\mathbb{S}^1$, which is a contradiction by \Cref{no-retraction}.
\end{proof}

\begin{remark}
In constructive reverse mathematics \cite{HannesDiener}, it is known that both the intermediate value theorem and Brouwer's fixed-point theorem are equivalent to LLPO. But LLPO does not hold in real cohesive homotopy type theory, so \cite{shulman-Brouwer-fixed-point} prove a variant of the statement involving a double negation.
\end{remark}



\appendix
%\section{Technical details}
\section{Some notes on $\Noo$}
Recall that we defined $B_\infty$ as the quotient of the freely generated algebra 
over $p_n,~n\in\N$ by the relations $\{p_n \wedge p_m | n\neq m\}$. 

\begin{lemma}\label{N-co-fin-cp}
  The Boolean algebra of co-finite subsets of $\N$
  is equivalent to $B_\infty$. 
\end{lemma}
\begin{proof}
  Let $f:B_\infty \to \N_{(co)fin}$ be induced by sending $p_n$ to $\{n\}$. 
  Note that whenever $n\neq m$, we have 
  $f(p_n)\wedge f(p_m) = \{n\} \cap \{m\} = \emptyset$, 
  thus $f$ respects the relations of $B_\infty$ and is well-defined.

  Define $g:N_{(co)fin)} \to B_\infty$ as follows:
  \begin{itemize}
    \item On a finite subset $I$, we define $g(I) = \bigvee_{i\in I} p_i$, 
    \item On a cofinite subset $J$, we define $g(J) = \bigwedge _{i \in J^C} \neg p_i$. 
  \end{itemize}
  Note that in these cases we indeed have $I,J^C$ are finite, so these are well-defined elements. 
  We must show that $g$ is a Boolean morphism. 

  \begin{itemize}
    \item 
      By deMorgan's laws, $g$ preserves $\neg$:
      for $I$ finite we have
      \begin{equation}
      \neg g(I) = \neg (\bigvee_{i\in I} p_i) = \bigwedge_{i\in I} \neg p_i = g(I^C)
      \end{equation}
      And for $J$ cofinite, we apply similar reasoning. 
    \item To see that $g$ preserves $\vee$, we need to check three cases
      \begin{itemize}
        \item If both $I,J$ are finite, then 
        \begin{equation} 
          g(I \cup J) = \bigvee_{i\in I \cup J} p_i= \bigvee_{i\in I} p_i \vee \bigvee_{j\in J} p_j 
          = g(I) \vee g(J)
        \end{equation}
        and we're done. 
      \item If both $I,J$ are cofinite, we have
        \begin{equation}
          g(I) \vee g(J) = 
          \bigwedge_{i \in I^C} \neg p_i \vee 
          \bigwedge_{j \in J^C} \neg p_j 
          = 
          \bigwedge_{i\in I^C} 
          \bigwedge_{j \in J^C}(\neg p_i \vee  \neg p_j) 
        \end{equation}
        Now note that in $B_\infty$, we have 
        \begin{equation}
          \neg p_i \vee \neg p_j = \neg ( p_i \wedge p_j) = 
          \begin{cases}
            \neg p_i \text{ if } i = j\\
            1 \text{ if } i \neq j  
          \end{cases}
        \end{equation}
        Therefore, we can leave out the case that $i\neq j$ in the calculation of the above meet, and
        \begin{equation}
          \bigwedge_{i\in I^C} 
          \bigwedge_{j \in J^C}(\neg p_i \vee  \neg p_j)  
          = 
          \bigwedge_{i \in (I^C \cap J^C)} \neg p_i
          = 
          \bigwedge_{i \in (I \cup J)^C} \neg p_i 
        \end{equation}
        as $I\cup J$ must also be cofinite, this equals 
          $ g( I \cup J)$. 
        \item 
          If $I$ is finite and $J$ cofinite, we have 
          that $I\cup J$ is cofinite, hence 
          \begin{equation}
            g(I\cup J) = \bigwedge_{k\in (I \cup J)^C} \neg p_k
            = \bigwedge_{k \in (J^C -I)} \neg p_k
          \end{equation}
          Now note that 
          whenever $i\neq k$, we have 
          \begin{equation}
            p_i = (p_i \wedge \neg p_k) \vee (p_i \wedge p_k) = 
            (p_i \wedge \neg p_k) \vee 0 = p_i \wedge \neg p_k
          \end{equation}
          Hence by absorption
          \begin{equation} 
            (p_i \vee \neg p_k)  =
              \begin{cases}
                1 \text{ if } i = k \\
                \neg p_k \text{ if } i \neq k
              \end{cases}
          \end{equation}
          As for all $k\in J^C-I$ and all $i\in I$ we have $k\neq i$, we may thus write
          \begin{equation}\label{eqnCofiniteHelper1}
            \bigwedge_{k \in (J^C - I)} \neg p_k = 
            \bigwedge_{k \in (J^C - I)} (\neg p_k \vee (\bigvee_{i\in I} p_i))
          \end{equation}
          We now note that 
          \begin{equation}\label{eqnCofiniteHelper2}
            1=\bigwedge_{i\in I} 1 = \bigwedge_{i\in I} (\neg p_i \vee (\bigvee_{i\in I} p_i)).
          \end{equation}
          Taking the meet of the expressions in \Cref{eqnCofiniteHelper1} and \Cref{eqnCofiniteHelper2}, 
          we see that 
          \begin{equation}
            \bigwedge_{k \in (J^C - I)} \neg p_k = 
            \bigwedge_{j \in J^C} (\neg p_j \vee (\bigvee_{i\in I} p_i))
          \end{equation}
          And using distributivity rules, we can see that 
          \begin{equation}
            \bigwedge_{j \in J^C} (\neg p_j \vee (\bigvee_{i\in I} p_i))
            = 
            (\bigwedge_{j \in J^C} \neg p_k) \vee (\bigvee_{i\in I} p_i)
          \end{equation}
          From which we may conclude that $g(I\cup J) = g(I) \cup g(J)$. 
      \end{itemize}
    \item The case for $\wedge$ is completely dual to the case for $\vee$. 
  \end{itemize}
We conclude that $g$ is a Boolean morphism. 
Furthermore, it is easy to see that $g$ and $f$ are each other's inverse, 
thus the Boolean algebras are isomorphic. 
\end{proof}
\begin{remark}\label{AppendixCofiniteOrFinite}
  As a consequence of the above proof, any $b:B_\infty$ corresponds either to 
  \begin{itemize}
    \item a finite set $I$, in which case $b = \bigvee_{i\in I} p_i$. 
    \item a cofinite set $J$, in which case $b = \bigwedge_{j\in J^C} \neg p_j$. 
  \end{itemize}
  We will call $b$ finite/cofinite respectively. 
\end{remark}
\begin{remark}
Recall that $\Noo$ is defined as the spectrum of $B_\infty$. 
If $\alpha:\Noo$ satisfies $\alpha(p_n) = 1$, then $\alpha(p_m) = 0$ for all $n\neq m$. 
Therefore, for each $n:\N$, there is an unique map $\chi_n$ with $\chi_n(p_n) = 1$. 
There is also the point $\chi_\infty : \Noo$ which is unique 
with the property that $ \chi_\infty(p_n) = 0$ for all $n:\N$. 
We will call decidable subsets of $\Noo$ finite/cofinite iff their corresponding elements of $B_\infty$ are. 
\end{remark}
\begin{lemma}\label{FiniteDecidableSubsetsCharacterization}
  Finite decidable subsets of $\Noo$ are of the form 
  $\{\chi_i | i \in I\}$ for some finite $I\subseteq \N$. 
\end{lemma}
\begin{proof}
  Let $d= \bigvee_{i\in I} p_i$. 
  Clearly whenever $i\in I$, we have $\chi_i(d) = 1$. 
%
  Now suppose $f:B_\infty \to 2$ is such that $f(d) = 1$. 
  Then $\bigvee_{i\in I}(f(p_i)) = 1$, hence it is not the case that $f(p_i) = 0$ for all $i\in I$. 
  Now as $I$ is finite and $f(p_i) = 0 \vee f(p_i) = 1$ for all $i\in I$, 
  there must exist some (necessarily unique) $i\in I$ with $f(p_i) = 1$. Hence $f = \chi_i$. 
%
  Thus $f(d) = 1$ iff there is some $i\in I$ with $f = \chi_i$. 
\end{proof}
\begin{corollary}\label{CoFiniteDecidableSubsetsCharacterization}
  Cofinite decidable subsets of $\Noo$ are of the form
  $\neg \{\chi_i | i \in I\}$ for $I\subseteq\N$ finite. 
\end{corollary}
\begin{proof}
  Let $D$ be a cofinite decidable subset. Then $\neg D$ is a finite decidable subset, 
  By the above lemma it follows that $\neg D = \{\chi_i | i\in I\}$. 
  As $\neg \neg D = D$, the result follows. 
\end{proof}
\begin{corollary}
 Any a decidable subset $D\subseteq\Noo$ is cofinite iff $\chi_\infty\in D$. 
\end{corollary}
\begin{proof}
  This follows from the observation that $\chi_\infty \in \neg \{\chi_i | i \in I\}$ for $I\subseteq \N$, 
  the observation that all decidable subsets are either finite or cofinite, 
  and the characterization of finite a cofinite decidable subsets in 
  \Cref{FiniteDecidableSubsetsCharacterization} and 
  \Cref{CoFiniteDecidableSubsetsCharacterization}.
\end{proof}
\begin{corollary}
  If $U\subseteq \Noo$ is open and $\chi_\infty \in U$, there exists some $n\in \N$ such that 
  $\{\chi_k | k\geq n\} \subseteq U$. 
\end{corollary}
\begin{proof}
  If $U$ is open, by \Cref{StoneOpenSubsets}, it is a countable union of decidable subsets. 
  One of these must contain $\chi_\infty$, hence be cofinite and 
  of the form $\neg \{ \chi_i | i \in I\}$ for some finite $I\subseteq \N$.
  As $I$ is finite, there is some $n:\N $ with $n>i$ for all $i\in I$. 
  For all $k\geq n$, we have that $k\notin I$, hence $\chi_k \in \neg \{\chi_i | i \in I\}\subseteq U$ as required. 
\end{proof}



%
%\begin{lemma}
%  For all decidable subsets $D:\Noo\to 2$,
%  with $D$ non-empty, there exists some $n:\N$ with $\chi_n \in D$. 
%\end{lemma}
%\begin{proof}
%  We make a case distinction based on \Cref{AppendixCofiniteOrFinite}. 
%  \begin{itemize}
%    \item 
%      If $D$ corresponds to a finite $d:B_\infty$, but is non-empty, then 
%      $d=\bigvee_{i\in I} p_i$ for $I\subseteq \N$ finite and non-empty. 
%      If $I$ is finite (as in \Cref{dfnFinite}) and non-empty, 
%      $I\simeq Fin_k$ for some $k\neq 0$. 
%      In particular, there is a map $1 \to I$,
%      hence a term $i:I$. 
%      Then $\chi_i(d) = 1$, hence $\chi_i \in D$. 
%    \item 
%      If $D$ corresponds to some cofinite $d:B_\infty$, we have 
%      $d = \bigvee_{i\in I} \neg p_i$ for some $I\subseteq \N$ finite. 
%      Then there is some 
%\end{proof}
%




\section{Cocompleteness of $\Boole$}
\rednote{TODO, is $\Boole$ closed under countable limits? 
  It has finite colimits, as it has pushouts and initial object.
  It should also have sequential colimits (TODO). 
  Is a countable coproduct the sequential colimit of it's initial finite coproducts? 
}
\begin{lemma}\label{BoolePushouts}
  Countably presented Boolean algebras are closed under pushout. 
\end{lemma} 
\begin{proof}
  Let $A,B,C:\Boole$, and suppose $f:A\to B, g:A \to C$ are Boolean morphisms. 
  Let $G_A, G_B,G_C$ be the underlying countable sets of generators for $B,C$ and 
  let $R_A,R_B,R_C$ be the underlying countable sets of relations. 
  Consider $P$ the Boolean algebra generated by $G_B\sqcup G_C$ under the relations 
  $R_B\cup R_C \cup F$ where $F$ is the set of expressions $f(a)-g(a), a\in G_A$.
  
  Note that as the generators of $B$ are included in those of $P$, 
  and all relations of $B$ are included in those of $P$, there is a map $h:B\to P$. 
  Similarly there is a map $i:C\to P$. 
  We now claim that the following is a pushout square:
  \begin{equation}\begin{tikzcd}
    A \arrow[r,"f"] \arrow[d,"g"] & B \arrow[d,"h"]\\
    C \arrow[r,"i"] & P
  \end{tikzcd}\end{equation}  
  Suppose $\beta:B \to X, \gamma:C\to X$ are such that $\beta\circ f = \gamma \circ h$. 
  $\beta,\gamma$ then induce maps on the generators of $P$. 
  These maps respect $F$ as $\beta\circ f=\gamma\circ h$, and they must respect $R_B,R_C$ as they are maps out of $B,C$. 
  Therefore, $\beta,\gamma$ induce a map $e:P\to X$, such that 
  $e(b) = \beta(b)$ for $b:G_B$ and $e(c)=\gamma(c)$ for $c:G_C$. 
  Furthermore, any map $P\to X$ with this property must agree with $e$ on all the generators of $P$, 
  and therefore equal $e$. Thus $e$ is the unique extension $P\to X$. 
  Thus $P$ the above square is indeed a pushout. 
\end{proof}

For some proofs in this paper, 
\rednote{(right now the counter's at two)}
we'd like a very concrete description of the fiber of a map of Stone spaces. 
The following construction turns out to be particularly useful. 
\begin{lemma}\label{FiberConstruction}
  Let $A,B:\Boole$, let $G$ be an explicit countable set of generators for $A$, and let 
  $f:A \to B, x:A\to 2$. 
  Define the countable set 
  \begin{equation}
    G' = \{a | a\in G, x(a) = 0\} \cup \{\neg a | a \in G, x(a) = 1\}
  \end{equation} 
  For $R = f(G')$,
%  Then we can construct a countable set $R\subseteq B$ such that 
  the pushout of $f$ and $x$ is given by $B/R$. 
\end{lemma}  
\begin{proof}
We consider the following pullback in the category of Stone spaces:
  \begin{equation}\begin{tikzcd}
    \sum\limits_{y:Sp(B)} y\circ f = x \arrow[d] \arrow[r] \arrow["\lrcorner"{pos=0.125}, phantom, dr] 
    & \top \arrow[d,"x"]\\
    Sp(B) \arrow[r,"(\cdot) \circ f"] & Sp(A)
  \end{tikzcd}  \end{equation}
Dual to this square, we have the following pushout in the category of Boolean algebras,
where $Sp(P) \simeq  (\sum\limits_{y:Sp(B)} y \circ f = x)$:
  \begin{equation}\begin{tikzcd}
    A \arrow[d,"x"'] \arrow[r,hook,"f"] \arrow[rd,phantom,"\ulcorner"{pos=0.125}] & B\arrow[d]\\
    2 \arrow[r] & P
  \end{tikzcd}\end{equation} 
  Following \Cref{BoolePushouts}, 
  the pushout $P$ is given by $B/R$ with $R$ the relations $f(a) -x(a)$ 
  where $a$ ranges over the generators of $A$.
  Note that $x(a) \in \{0,1\}$. 
  If $x(a)=0$, then $f(a)-x(a) = f(a)$, 
  and if $x(a) = 1$, then $f(a) -x(a) = \neg f(a) = f(\neg a)$. 
  So we can define the subset $G'\subseteq A$ given by 
  \begin{equation}
    G' = \{a | a\in G, x(a) = 0\} \cup \{\neg a | a \in G, x(a) = 1\}
  \end{equation} 
  $G'$ is in bijection with $G$, hence countable. 
  Furthermore, $x(g) = 0$ for all $g\in G'$. 
  And $R = f(G')$.
\end{proof}





%\begin{lemma}\label{BooleCoEqualizers}
%  Countably presented Boolean algebras are closed under coequalizers.
%\end{lemma}
%\begin{proof}
%  Let $f,g:A\to B$ be Boolean morphisms.
%  Define $C = B/R$, where $R$ is given by the relations $fa-ga,~a\in G_A$, for $G_A$ the set of generators of $A$.
%  Suppose that we have a map $x:B\to D$ with $xf = gf$. Then $x$ respects $R$, and thus defines a map $y:C \to D$. 
%  Furthermore, any map $C\to D$ extending $x$ agrees with $y$ on the generators of $C$, 
%  and is thus equal to $y$. Therefore $C$ is the coequalizer of $f,g$. 
%\end{proof}


%%
%%\begin{corollary}\label{CoCompletenessBoole}
%%  The category of countably presented Boolean algebras contains all finite colimits. 
%%\end{corollary}
%%\begin{proof}
%%  Recall that $\Boole$ has an initial object given by $2$. 
%%  By \Cref{BoolePushouts}, 
%%%  it is therefore closed under coproducts. 
%%%  By \Cref{BooleCoEqualizers}, 
%%  it follows that $\Boole$ contains all finite colimits. 
%%\end{proof}


\section{Some notes on our axioms}
\label{NotesOnAxioms}
\subsection{Alternatives to propositional completeness}
In \Cref{Axioms}, we have chosen to present propositional completeness as an axiom. 
However, assuming Stone duality, we could have made some other choices, 
and left propositional completeness as a theorem. 
What's more, assuming the axiom of Dependent choice,
the axiom is equivalent to LLPO. 
In this section, we will show these equivalences. 

\begin{theorem}\label{AlternativesToAxiom2}
  Assuming Stone duality, the following are equivalent:
  \begin{enumerate}[(i)]
    \item For $S$ Stone, we have $\neg \neg S \to ||S||$. 
    \item For $S$ Stone, we have that $||S||$ is closed. 
    \item A map $f:A \to B$ in $\Boole$ is injective iff the map $(\cdot) \circ f : Sp(B) \to Sp(A)$ is surjective. 
  \end{enumerate}
\end{theorem}
\begin{proof}
  We assume that $S= Sp(B)$. 
  Note the proof of \Cref{SpectrumEmptyIff01Equal} only uses Stone duality. 
  The proof of \Cref{BooleEqualityOpen} only relies on the definition of $\Boole$.
  Hence the argument in \Cref{TruncationStoneClosed}, which shows $(i)\to (ii)$ only relies on Stone duality. 
  Furthermore, the argument that closed propositions are double negation stable (\Cref{rmkOpenClosedNegation})
  only used \Cref{MarkovPrinciple}, which followed from Stone duality as well. 
  Hence if $||S||$ is closed, we have $\neg \neg ||S|| \leftrightarrow ||S||$, thus $(ii) \to (i)$. 
  $(i)\to (iii)$ is \Cref{FormalSurjectionsAreSurjections}. 
  By the above discussion, we also have that $\neg \neg S$ iff $0\neq_B 1$. 
  Note that $0\neq_B 1$ iff the map $2\to B$ is injective. 
  Furthermore, $||S||$ iff the map $S \to \top $ is surjective. 
  Hence $(iii) \to (i)$. 
\end{proof} 

\begin{lemma}\label{LLPOAndDCToCompleteness}
Assuming dependent choice, Stone duality, and that closed propositions are closed under disjunctions, 
we can show propositional completeness. 
\end{lemma}
\begin{proof}
  Let $B:\Boole$ satisfy $0\neq_B 1$. We will show there merely exists a map $B\to 2$. 
  Let $G$ be the set of generators of $B$. 
  We will use dependent choice on the the following $E_n,R_n$:
  \begin{itemize}
    \item 
  Let $E_n$ be the type consisting of 
  \begin{itemize}
    \item A map from the first $n$ generators of $B$ to $2$, denoted $x_n:G_n \to 2$. 
    \item A proposition denoting that $0\neq_{B_n} 1$ for $B_n$ given by:
      \begin{equation}
        B_n := B/\big( \{g|g\in G_n, x_n(g) = 0\} \cup \{ \neg g| g\in G_n, x_n(g) = 1\}\big).
      \end{equation}
  \end{itemize}
  \item 
    And let $R_n:E_n \to E_{n+1} \to \mathcal U$ denote the relation that $x_{n+1}$ extends $x_n$. 
  \end{itemize} 
  Note that $E_0$ is inhabited as $0\neq_B 1$. Assume $E_n$.
%  Now assume $x_n:G_n\to 2$ witnesses $E_n$. 
  As $0\neq_{B_n}1$, for all $g:B_n$, we can show 
%  we have $$\neg ((g =1)  \wedge ((\neg g) = 1)).$$
% % 
%%  Now suppose that $E_n$ is inhabited,  and let $x_n:G_n \to 2$. 
%%  Note that in $B_n$, we have $0\neq 1$ and thus $$\neg ((g =1)  \wedge ((\neg g) = 1))$$
%%  for all $g:B_n$.
%  Therefore, we have 
  $$\neg \neg (( g\neq 1) \vee ((\neg g) \neq 1)).$$
  By \Cref{BooleEqualityOpen}, and \Cref{rmkOpenClosedNegation}, 
  (which could be shown using Stone Duality)
  and the assumption that 
  closed statements are closed under disjunction, we have that the above statement is equivalent to 
  $(g \neq 1) \vee ((\neg g) \neq 1)$. 
  This holds in particular for $g$ the $n+1$'th generator of $B$. 
  Therefore, we have that $0\neq 1$ in $B_n/\{g\}$ or in $B_n/\{\neg g\}$. 
  Thus we can extend $x_n$ by letting $x_{n+1}(g) = 0$ or $x_{n+1}(g) = 1$ respectively. 
  
  By dependent choice, we get a map $x:G\to 2$. 
  We claim that for this map $x$, we have $0\neq 1$ in 
  \begin{equation}
    B' := B/\big( \{g|g\in G, x(g) = 0\} \cup \{ \neg g| g\in G, x(g) = 1\}\big).
  \end{equation}
  Note that $B'$ is the colimit of the sequence $B_n$ with projection maps $B_n \to B_{n+1}$. 
  Thus if $0=1$ in $B'$, $0=1$ in some $B_n$, which doesn't happen by assumption. 
  Therefore we have $0\neq 1$ in $B'$. 
  Furthermore, note that $B'$ is equivalent to a Boolean algebra with no generators, 
  as any generator in $B$ is sent to either $0$ or $1$ by the relations in $B'$. 
%
  But now any Boolean algebra with no generators and $0\neq 1$ is isomorphic to $2$. 
  Therefore $B'\simeq 2$, and the projection map $B\to B'$ gives a map $B \to 2$. 
  
\end{proof}

\begin{corollary}
Assuming dependent choice and Stone duality, TFAE:
\begin{enumerate}[(i)]
  \item For $S$ Stone, we have $\neg \neg S \to ||S||$. 
  \item LLPO.
  \item The disjunction of two closed propositions is closed. 
\end{enumerate}
\end{corollary}
\begin{proof}
  $(i) \to (ii)$ is \Cref{LLPO}, $(ii) \to (iii)$ is \Cref{ClosedFiniteDisjunction}, 
  and $(iii) \to (i)$ is \Cref{LLPOAndDCToCompleteness}
\end{proof}
\rednote{
  @Hugo, you mentioned that axiom 2 was independent from the other axioms. 
This might be a good place to reference to that proof}


\subsection{The formulation of local choice}

\begin{lemma}
  TFAE:
  \begin{itemize}
\item  Whenever $S$ Stone and $E\twoheadrightarrow S$ surjective, then there is some $T$ Stone,
    a surjection $T \twoheadrightarrow S$ and a map $T\to E$ 
    such that the following diagram commutes:
    \begin{equation}\begin{tikzcd}
      & E \arrow[d,""',two heads]\\
      T \arrow[ru,dashed]  \arrow[r,two heads, dashed ] &S %& \arrow[l, "", two heads, dashed] T\arrow[lu, ""',dashed ]
    \end{tikzcd}\end{equation}  
\item
  Whenever we have $S:\Stone$, $E,F$ arbitrary types, a map $f:S \to F$ and a 
  surjection $e:E \twoheadrightarrow F$, 
  there exists a Stone space $T$, a cover $T\twoheadrightarrow S$ and an arrow $T\to E$ making the following diagram commute:
    \begin{equation}\begin{tikzcd}
      T \arrow[d,dashed, two heads ] \arrow[r,dashed]&  E \arrow[d,""',two heads, "e"]\\
      S  \arrow[r, "f"] & F
    \end{tikzcd}\end{equation}  
\end{itemize} 
\end{lemma}
\begin{proof}
  By considering $f=id$ we can see that the second statement implies the first. 

  For the converse, let $S,E,F,e,f$ be as in the second statement. 
  As $e$ is surjective, whenever $s:S$, we there merely exists some $b:E$ with $e(b) = f(s)$. 
  This induces an element $(s,b):S\times_F E$. 
  Thus the projection $S\times_F E \to S$ is surjective, and 
  the first axiom provides us with a $T$ as required. 
    \begin{equation}\begin{tikzcd}
     & S \times_F E \arrow[r] \arrow[d,two heads] &  E \arrow[d,""',two heads, "e"]\\
       T \arrow[r, two heads ,dashed ] \arrow[ru,dashed]& 
       S  \arrow[r, "f"] & F
    \end{tikzcd}\end{equation}  
    
\end{proof}

\subsection{Scott continuity implies Stone duality}

\section{Countability}\label{CountabilityDiscussion}
In the system presented in this paper, 
one of the fundamental building blocks are countably presented Boolean algebras. 
There are several definitions of countable, which are not necissarily constructively equivalent. 

\begin{definition}
  A type $T$ is enumerable iff there exists a surjection $\N \to 1 + T$. 
\end{definition}
\begin{definition}\label{dfnFinite}
  A type $T$ is finite if there exists some $k:\N$ with $T\simeq Fin_k$. 
\end{definition}
\begin{definition}
  A type $T$ is strongly countable if
  $T$ is finite or merely isomorphic to $\N$.
\end{definition}
\begin{definition}
  A type $T$ is subcountable iff it is merely isomorphic to a decidable subset of $\N$. 
\end{definition}


\begin{lemma}
  Every strongly countable type is subcountable. 
\end{lemma}
\begin{proof}
  Note that $Fin_k$ and $\N$ are both isomorphic to a decidable subset of $\N$. 
\end{proof}
\begin{lemma}
  Every subcountable type is enumerable. 
\end{lemma}
\begin{proof}
  For $A\subseteq \N$ decidable, define $f:\N \to 1 + \Sigma_{n:\N} A(n)$ by 
  $$
  f(n) = 
  \begin{cases}
    inl(*) \text{ if } \neg A(n)\\
    inr(n) \text{ if } A(n)
  \end{cases}
  $$
\end{proof} 


\begin{lemma}\label{OpenSubsetNAreSubCountable}
  Any open subset of $\N$ is subcountable. 
\end{lemma} 
\begin{proof}
  Let $A:\N \to \Open$. 
  By countable choice, there exists a map $\alpha_{(\cdot)}:\N \to \Noo$ such that 
  $\exists_{m:\N} \alpha_n(m) = 0 \leftrightarrow A (n)$. 
  Define $B\subseteq \N \times \N$ by 
  $B(m,n) = (\alpha_{n}(m) = 0)$. 
  Note that $A(n) \leftrightarrow || \Sigma_{m:\N } B(m,n) ||$
\end{proof}

\begin{lemma}\label{OpenSubsetEnumerableAreEnumerable}
  Any open subset of an enumerable type is enumerable. 
\end{lemma}
\begin{proof}
  Let $A$ be enumerable and let $P:A \to Open$.
  We will show that $\Sigma_{a:A} P a$ is enumerable. 
  Let $s:\N \to 1 + A$ surjective. 
  Define $s':\N \to Open$ by 
  $$
  s'(n) = 
  \begin{cases}
    \bot \text { if } s(n) = inl(*)\\
    P(a) \text { if } s(n) = inr(a)
  \end{cases}
  $$ 
  By countable choice, we get a map 
  $\alpha_{(\cdot)}: \N \to 2^\N$  such that 
  $(\exists_{m:\N} \alpha_n(m) = 0) \leftrightarrow s'(n)$. 
  Note that $\alpha_n(m) = 1$ iff 
  $s'(n)$ which happens iff $s(n) = inr(a_n)$ for some $a_n:A$ with $P(a_n)$. 
  Therefore, we can define 
  $z:\N \times \N \to 1 + \Sigma_{a:A} P a$ by 
  \begin{equation}
    z(m,n) = 
    \begin{cases}
      inl(*) \text{ if } \alpha_{n}(m) = 0 \\
      a_n  \text{ if } \alpha_{n}(m) = 1 \text{ and $a_n$ as above}
    \end{cases}
  \end{equation}
  Note that if $a:A$ satisfies $P(a)$, there is some $n:\N$ with $s(n) = inr(a)$. 
  And as $P(a)$, there exists some $m:\N$ with $\alpha_n(m) = 1$. 
  Hence $z(m,n) = a$. 
  Thus $z$ is surjective. 
  As $\N \times \N \simeq \N$, we conclude that 
  $\sum_{a:A} P a$ is enumerable. 
\end{proof}

\begin{lemma}
  Any open subset of $\N$ is subcountable. 
\end{lemma}
\begin{proof}
  Let $A:\N \to \Open$. 
  By countable choice, we get a map $\alpha_{\cdot}: \N \to \Noo$ such that 
  $A(n) \leftrightarrow \Sigma_{m:\N} \alpha_n (m) = 1$. 
  Define $B:\N \times \N \to 2$ by $B(m,n) = \alpha_n(m)$. 
  We then have a bijection $\Sigma_{n:\N} A(n) \to \Sigma_{(n,m) : \N \times \N} B(m,n)$ sending 
  $(n,(m,p))$ to $(n,m,p)$.
\end{proof}



\begin{lemma}\label{OpenSubsetOfNNotDecidable}
  It is not the case that for every $P:\N \to \Open$, 
  $||\Sigma_{n:\N} P(n)||$ is decidable.
  % The subset being decidable could be interpreted as 
  %    that P(n) is decidable for all $n:\N$,
  % or that \Sigma_{n:\N} P(n) + \neg \Sigma_{n:\N} P(n) 
  % or that || \Sigma_{n:\N} P(n) || + \neg ||\Sigma_{n:\N} P(n) ||
\end{lemma}
\begin{proof}
  For $p$ any open proposition and $P(n) = p$ constantly, we have 
  $||\Sigma_{n:\N} P(n)||\leftrightarrow p$. 
  As not every open proposition is decidable (\Cref{rmkOpenClosedNegation}), 
  not every $||\Sigma_{n:\N} P(n)||$ is decidable. 
\end{proof}

\begin{lemma}\label{StronglyCountableTruncationDecidable}
  For every strongly countable type $A$, $||A||$ is decidable. 
\end{lemma}
\begin{proof}
  For a proposition, being decidable is a proposition. 
  Hence we may untruncate the definition of strongly open. 
  If $A \simeq \N$ or $A\simeq Fin_k$ for $k\neq 0$, we have $||A||$. 
  If $A \simeq Fin_0$, then $\neg ||A||$. 
\end{proof}

\begin{corollary}
  Not every enumerable type is strongly countable.
\end{corollary}
\begin{proof}
  If every enumerable type is strongly countable, 
  by \Cref{OpenSubsetEnumerableAreEnumerable}, every open subset of open subsets of $\N$ is strongly countable. 
  By \Cref{StronglyCountableTruncationDecidable}, the truncation of the corresponding type is decidable, which 
  contradicts\Cref{OpenSubsetOfNNotDecidable}.
\end{proof}

\begin{remark}
  Every enumerably represented Boolean algebra has an enumerable underlying set. 
  and every enumerable Boolean algebra is enumerably represented. 
\end{remark}

\section{Representing Stone spaces as limits}
\subsection{Countably presented algebras as sequential colimits}\label{secBooleAsColimits}
\begin{definition}
  We define a type $E$ to be Overtly Discrete iff it is the colimit of an $\N$-indexed sequence of finite sets. 
\end{definition} 
%
%\begin{definition}
%  A sequence in a category is a diagram of shape $\N$, 
%  where $\N$ carries the natural structure of a poset. 
%\end{definition}
\begin{lemma}
  Every countably presented Boolean algebra is overtly discrete.
\end{lemma}
%\begin{lemma}\label{lemProFinitePresentation}
%  For every countably presented Boolean algebra $B$
%  there merely exists a sequence of finitely presented Boolean algebras 
%  whose colimit in the category of Boolean algebras is $B$. 
%\end{lemma}
\begin{proof}
  Consider $\langle G \rangle \langle\langle R \rangle\rangle$ a countable presentation of a Boolean algebra $B$. 
  We will show there exists a diagram of shape $\N$ taking values in Boolean algebras 
  with $\langle G\rangle / R$ as the colimit.
  \paragraph{The diagram}
  Let $R_n$ be the first $n$ terms in $R$. 
  Note that each of these finitely many terms uses only finitely many symbols from $G$.
  Let $G_n$ be the finite set of terms used in $R_n$, unioned with the finite set of the first $n$ elements of $G$. 
  Define for each $n\in\N$ the finitely presented Boolean algebra $B_n = \langle G_n \rangle  \langle R_n \rangle$. 
  If $n\leq m$, then \Cref{rmkMorphismsOutOfQuotient} gives us a map $B_n \to B_m$ 
  as $G_n \subseteq G_{n+1}$ and $R_n \subseteq R_{n+1}$. 
  Thus $(B_n)_{n\in \N}$ gives us a diagram of shape $\N$
  with values in finitely presented algebras. 

  \paragraph{The colimit}
  As $G_n\subseteq G$ and $R_n \subseteq R$, 
  \Cref{rmkMorphismsOutOfQuotient} also gives us a map $B_n\to \langle G \rangle \langle R \rangle$. 
  We claim the resulting cocone is a colimit. 

  Suppose we have a cocone $C$ on the diagram $(B_n)_{n\in\N}$. 
  We need to show that there exists a map $\langle G \rangle / R\to C$ and
  we need to show this map is unique as map between cocones. 
  \begin{itemize}
    \item To show there exists a map $\langle G \rangle / R \to C$, 
      we use remark \Cref{rmkMorphismsOutOfQuotient} again. 
      Let $g\in G$ be the $n$'th element of $G$, 
      note that $g\in G_n$, and consider the image of $g$ under the map $B_n \to C$. 
      This procedure defines a function from $G$ to the underlying set of $C$. 
      Let $\phi \in R$ be the $n$'th element of $R$, 
      note that $\phi \in R_n$, and the map $B_n \to C$ must send $\phi$ to $0$. 
      Thus the function from $G$ to the underlying set of $C$ also sends $\phi$ to $0$. 
      This thus defines a map $\langle G \rangle / R \to C$. 
    \item To show uniqueness, consider that any map of cocones $\langle G \rangle / \langle R \rangle \to C$ 
      must take the same values on all $g\in G_n$ for all $n\in\N$. 
      Now all $g\in G$ occur in some $G_n$, so any map of cocones $\langle G \rangle /  \langle R \rangle \to C$ 
      takes the same values for all $g\in G$. 
      \Cref{rmkMorphismsOutOfQuotient} now tell us that these values uniquely determine the map. 
  \end{itemize}
\end{proof}
\begin{remark}
  Conversely, any colimit of a sequence of finite Boolean algebras 
  is a countably presented Boolean algebra with 
  as underlying sets of generators and relations the countable union of the finite sets of 
  generators and relations, which are both countable. 
\end{remark}
\begin{lemma}\label{lemFinitelyPresentedBACompact}
  For any finitely presented Boolean algebra $A$,
  and any sequence $(B_n)_{n:\N}$ of Boolean algebras with colimit $B$
  we have that the set $B^A$ is the colimit of the sequence of sets $(B_n^A)_{n:\N}$. 
\end{lemma}  
\begin{proof}
  First note that $B^A$ forms a cocone on $(B_n^A)_{n:\N}$ 
  because any map $A \to B_n$ induces a map $A \to B$. 
  Let $C$ be a cocone on $(B_n^A)_{n:\N}$. 
  We shall show there is an unique morphism of cocones $B^A \to C$. 
  \begin{itemize}
    \item For existence, let $f:B^A$. 
      As $A$ is finitely presented, we write $A = \langle G \rangle / \langle R \rangle$ with $G$ finite.
      By \Cref{rmkMorphismsOutOfQuotient}, $f$ is uniquely determined by it's values on $g\in G$. 
      As $G$ is finite, so is it's image $f(G)\subseteq B$. 
      But any finite subset of $B$ already occurs in $B_n$ for some $n\in\N$. 
      Consequently, the image of $f$ is already contained in some $B_n$. 
      Thus there is some $f_n:(B_n^A)$ such that postcomposing 
      $f_n$ with the map $B_n \to B$ gives back $f$. 
      The image of $f_n$ under the map $(B_n^A) \to C$ is how we define the image of $f$. 
      This is well-defined by the cocone conditions on $C$. 
    \item 
      For uniqueness, by function extensionality maps $B^A \to C$ are uniquely determined by their values on 
      $f:B^A$. By the above, the value of $f$ is uniquely determined by it's value on $B_n$ for 
      any $n$ with the image of $f$ in $B_n$. Thus there is at most one morphism of cocones $B^A \to C$. 
  \end{itemize}
\end{proof}
\begin{remark}\label{rmkEqualityColimit}
  In the above proof, we used that any element $b\in B$ already occurs in some $B_n$. 
  However, please note that it is not necessarily the case that it occurs uniquely in $B_n$, 
  there might be multiple elements in $B_n$ which can all be sent to $b$ in the end. 

  In case our sequence comes from the construction in \Cref{lemProFinitePresentation}, 
  we can see that whenever there are two elements in 
  $B_n$ corresponding to $b\in B$, they will become equal in $B_m$ for some $m\geq n$. 
  The reason is that if $b \sim_{\langle R \rangle} c$, there is a finite subset $R_0 \subseteq R$ such that 
  $b\sim_{\langle R_0 \rangle} c$, which will occur in some $R_m$. 

  One could wonder whether this property holds for general colimits of sequences. 
  In general, if we assume $B$ is the colimit of an arbitrary sequence $(B_n)_{n:\N}$, 
  and there exist some $B_n$ with two elements corresponding to the same element in $B$, 
  Theorem 7.4 from \cite{SequentialColimitHoTT} says that there merely exists some $m\geq n$
  such that they are already equal in $B_m$. 
\end{remark}

%For our next lemma on this presentation of sequences we need the axiom of dependent choice. 
%\begin{axiomNum}[Dependent choice]\label{axDependentChoice}
%  Given a family of types $(E_n)_{n:\N}$ and 
%  a relation 
%  $R_n:E_n\rightarrow E_{n+1}\rightarrow {\mathcal U}$ such that
%  for all $n$ and $x:E_n$ there exists $y:E_{n+1}$ with $p:R_n~x~y$ 
%  then given $x_0:E_0$ there exists
%  $u:\Pi_{n:\N}E_n$ and $v:\Pi_{n:\N}R_n~(u~n)~(u~(n+1))$ and $u~0 = x_0$.
%\end{axiomNum}
\begin{lemma}[Using dependent choice]\label{lemDecompositionOfColimitMorphisms}
  Let $B,C$ be countably presented Boolean algebras, 
  and suppose we have a morphism $f:B\to C$.
  There exists sequences of finitely presented Boolean algebras 
  $(B_n)_{n:\N}, (C_n)_{n:\N}$ with colimits $B,C$ respectively
  and compatible maps of Boolean algebras $f_n:B_n \to C_n$, 
  such that $f$ is the induced morphism $B\to C$.
\end{lemma}
\begin{proof}
  Let $(B_n)_{n:\N}, (C_n)_{n:\N}$ be 
  sequences of finitely presented Boolean algebras with colimits $B$ and $C$. 
  We will take a subsequence of $(C_n)_{n:\N}$, using the axiom of dependent choice above. 

  Our family of types $E_k$ as in \Cref{axDependentChoice} 
  will be strictly increasing sequences $(n_i)_{i\leq k}$ of natural numbers together with a finite family of maps 
  $(f_i: B_{i} \to C_{n_i})_{i\leq k}$ such that
  for all $0\leq i<k$ the following diagram commutes:
  \begin{equation}\label{eqnDecompositionOfColimitMorphisms}
    \begin{tikzcd}
      B_{i} \arrow[r] \arrow[d, "f_i"]& B_{{i+1}} \arrow[r] \arrow[d,"f_{i+1}"]& B \arrow[d,"f"] \\
      C_{n_i} \arrow[r] & C_{n_{i+1}} \arrow[r] & C 
    \end{tikzcd}
  \end{equation}
  Our relation $R_k$ will tell whether the second sequence extends the first one. 
%
  By \Cref{lemFinitelyPresentedBACompact} 
  there exists some $n_0:\N$ 
  such that $B_0 \to B \to C$ factors as 
  \begin{equation}
    \begin{tikzcd}
      B_{0} \arrow[r] \arrow[d, "f_0"]& B \arrow[d,"f"] \\
      C_{n_0} \arrow[r] & C 
    \end{tikzcd}
  \end{equation}
  Because our goal is a proposition, we can untracate this existence to data. 
  This data will form our $x_0:E_0$. %from \Cref{axDependentChoice}. 
%
  Now suppose we have $(f_i: B_{i} \to C_{n_i})_{i\leq k}$ for some $k\geq 0$ 
  such that
  for all $0\leq i<k$ the diagram of \Cref{eqnDecompositionOfColimitMorphisms} commutes.
  We shall show that in this case there exists an $n_{k+1}, f_{k+1}$ 
  making the same diagram commute for $i = k$. 
  Consider $B_{{k}+1}\to B \to C$. By the same argument as for $B_0$, we have a factorization 
  \begin{equation}
    \begin{tikzcd}
    B_{k+1} \arrow[r]  \arrow[d,"f'_{k+1}"]& B \arrow[d,"f"]\\
    C_{n'_{k+1}} \arrow[r] & C
    \end{tikzcd}
  \end{equation}
  Note that we may assume $n'_{k+1} > n_k$.
  Note that it is not necessarily the case that 
  $f'_{k+1}$ is compatibly with $f_k$, meaning the left square in the following diagram needn't commute:
  \begin{equation}
    \begin{tikzcd}
      B_{k} \arrow[r] \arrow[d, "f_k"]& B_{{k+1}}  \arrow[r] \arrow[d,"f'_{k+1}"] & B \arrow[d,"f"] \\
      C_{n_k} \arrow[r] & C_{n'_{k+1}} \arrow[r]  & C 
    \end{tikzcd}
  \end{equation}
  However, both $f'_{k+1}, f_k$ induce the same map $B_{k} \to C$. 
  Recall by \Cref{rmkMorphismsOutOfQuotient} this map is induced by it's value on finitely many elements. 
  By \Cref{rmkEqualityColimit}, it follows there is an $n_{k+1} \geq {n'_{k+1}}$ 
  such that for $f_{k+1}$ the composition of $f'_{k+1}:B_{k+1} \to C_{n'_{k+1}}$ and 
  the map $C_{n'_{k+1}} \to C_{n_{k+1}}$, the following diagram does commute:
  \begin{equation}
    \begin{tikzcd}
      B_{k} \arrow[d,"f_k"]\arrow[r] & B_{{k+1}} \arrow[rd, "f_{k+1}"] \arrow[rr] & & B \arrow[d,"f"] \\
      C_{n_k} \arrow[r] & C_{n'_{k+1}} \arrow[r] & C_{n_{k+1}} \arrow[r] & C 
    \end{tikzcd}
  \end{equation}
  Now by dependent choice for the above $x_0, R_n, E_n$, we get a sequence $(f_i:B_i \to C_{n_i})$  for some 
  strictly increasing sequence $n_i$ of natural numbers. 
  Note that for such a sequence $(n_i)_{i:\N}$, 
  $(C_{n_i})_{i:\N}$ converges to $C$. Also $(B_i)_{i:\N}$ still converges to $B$. 
  Futhermore, by construction the map that sequence $f_i$ induces from $B \to C$ shares all values with $f$
  and thus is equal to $f$. 
  Thus our sequence $f_i$ is as required. 
\end{proof}
\begin{remark}\label{rmkEpiMonoFactorizationCommutes}
  For $f,(f_i)_{i:\N}$ as above, whenever $f_n(x) = 0$, we have $f_{n+1}(x \circ \iota_{n,n+1}) = 0$
  for $\iota_{n,n+1}$ the map $A_n \to A_{n+1}$. 
  By \Cref{rmkMorphismsOutOfQuotient}, $\iota_{n,n+1}$ induces a map $A_n/Ker(f_n)\to A_{n+1}/Ker(f_{n+1})$. 
  This induced map is such that the following diagram commutes:
  \begin{equation}\begin{tikzcd}
    A_n \arrow[d, two heads] \arrow[r, "\iota_{n,n+1}"] & A_{n+1} \arrow[d,two heads]\\
    A_n /Ker(f_n) \arrow[d,hook] \arrow[r] & A_{n+1} /Ker(f_{n+1}) \arrow[d,hook] \\
    B_n \arrow[r] & B_{n+1}
  \end{tikzcd}\end{equation}  
  As the induced maps be epi's / mono's  is epi /mono, the colimit of the sequence 
  $A_n / Ker(f_n)$ will fit into an epi-mono factorization of $f$ and thus be iso to $A/Ker(f)$. 
  Thus the epi-mono factorization of the colimit is the colimit of the epi-mono factorizations. 
\end{remark}
\begin{remark}\label{rmkIsoEpiMonoMapColimit}
  Whenever $f:B \to C$ is an iso, any sequence with $B$ as colimit, also has $C$ as colimit. 
  Thus any iso can be represented this way as sequence of iso's. 
  Conversely, any sequence of isomorphisms induces an isomorphism of their colimits. 

  It follows from \Cref{rmkEpiMonoFactorizationCommutes} that when $f$ is epi/mono, 
  we can say that $f$ can be induced by a sequence 
  $(f_i)_{i\in \N}$ with all $f_i$ epi/mono. 
\end{remark}





\rednote{WIP}

In this section, we will work with countably presented Boolean algebras $B$ 
as colimit of some sequence of finitely presented (and hence finite) Boolean algebras $B_n$. 
If $n\leq m$, we'll denote $i^m_n:B_n\to B_m$ and $\iota_n:B_n\to B$ for the maps in this sequence. 

For $S=Sp(B)$, we have that $S$ is the limit of the induced sequence $S_n = Sp(B_n)$. 
For $n\leq m$, we will denote the restriction maps sometimes as $ \cdot |^m_n : S_m \to S_n$ and $\cdot|_n :S\to S_n$, 
and sometimes as $\pi_m^n:S_m \to S_n, \pi_n : S\to S_n$. 

\begin{lemma}\label{DecidableSubsetOfStoneApproximation}
  Let $D:S\to 2$ be a decidable subset. 
  Then there exists a sequence $D_n:S_n \to 2$ such that for all $s:S$, we have 
  $D(s) \leftrightarrow \forall_{n:\N} D_n(s|_n)$. 
  Furthermore, this sequence is such that for $m\leq n$, and $s':S_n$, we have $D_n(s')\to D_m(s'|_m)$.
\end{lemma}
\begin{proof}
  By Stone duality, there is a $b:B$ with $D(s) \leftrightarrow s(b) = 1$. 
  As $B$ is the colimit of the sequence $B_n$, there merely exists some $m:\N, b':B_m$ with 
  $\iota_m(b') = b$. 
  Note that $s|_m (b') = s(\iota_m(b'))$ for all $s:S$. 
  Define $D' :S_m \to 2 $ by $D'(s')\leftrightarrow s'(b') = 1$. 
  For $n:\N$, define $D_n:S_n \to 2$ by 
  \begin{equation}
    D_n = \begin{cases}
      S_n \text{ if } n < m \\
      D'\circ \pi_n^m \text { if } n\geq m
    \end{cases}
  \end{equation}
  Then $\forall_{n:\N} D_n(s_n)$ iff $D'(s|_m)$ which happens iff $s|_m(b') = 1$, 
  which happens iff $s(\iota_m(b')) = s(b) = 1$, which happens iff $D(s)$.
  Also clearly for $m\leq n$ and $s':S_n$, we have that $D_n(s')\to D_m(s'|_m)$. 
  Hence the sequence $D_n$ is as required. 
\end{proof} 
\begin{corollary}
  For $P:S\to \Closed$ a closed subset, there exists a sequence
  of decidable subsets $P_k:S_k \to 2$ such that $P(s) \leftrightarrow \forall_{k:\N} P_k(s|_k)$. 
\end{corollary}
\begin{proof}
  By \Cref{StoneClosedSubsets}, we merely have some sequence of decidable subsets $D^n:S\to 2$ 
  with 
  \begin{equation}
    P(s) \leftrightarrow \forall_{n:\N} D^n(s)
  \end{equation}
  By applying countable choice to the \Cref{DecidableSubsetOfStoneApproximation}, there exists for each $n$ a sequence 
  $D^n_k: S_k\to 2$ such that for each $s:S$, we have 
  \begin{equation}
    D^n(s) \leftrightarrow \forall_{k:\N}D^n_k(s|_k).
  \end{equation}
  Furthermore, $D^n_k$ is decreasing in $k$, by which we mean that if 
  for $k\leq l$ and $s'\in S_l$, we have 
  \begin{equation}D^n_l(s')\to D^n_k(s'|_k).\end{equation}
  %
%  WLOG may assume that $D^{n+1} \subseteq D^n$, in which case we can also assume that 
%  $D^n_k$ is decreasing in $n$, so 
%  \begin{equation}D^{n+1}_k\subseteq D^{n}_k\subseteq S_k\end{equation}
  %
  Now define decidable subsets $P_k:S_k \to 2$ as follows: 
  \begin{equation}
    P_k = \bigcap_{n\leq k} D^n_k
  \end{equation}
  We claim that $P(s) \leftrightarrow \forall_{k:\N} P_k(s|_k)$. 
  \begin{itemize}
    \item Assuming $P(s)$, we have for each $n:\N$ that $D^n(s)$, and hence for all $k:\N$ that 
      $D^n_k(s|_k)$, in particular if $n\leq k$. Hence $P_k(s|_k)$ as required.
    \item Assuming that $\forall_{k:\N} P_k(s|_k)$, 
      we have for each $k:\N$ and each $n\leq k$ that $D^n_k(s|_k)$. 
      To show $P(s)$, it is sufficient to show that for any $m,l:\N$, we have $D^m_l(s|_l)$.
      \begin{itemize}
        \item If $m < l$, this is obvious. 
        \item If $l\leq m$, we can use that $D^m_m(s|_m)$
          and that $D^n_k$ is decreasing in $k$, to conclude 
          $D^m_l(s|_l)$.
      \end{itemize} 
  \end{itemize}
  So $P_n:S_n\to 2$ is as required. 

\end{proof}

%
%
%
%\begin{lemma}
%  Let $S:\Stone$ be represented as limit of a sequence of finite Stone spaces $S_n$. 
%  Let $P:S \to \Closed$. 
%  Then there merely exists a function $P_{\cdot} : \Pi_{n:\N} S_n \to 2$
%  such that $P(s) \leftrightarrow \forall_{n:\N} P_n(s|_n)$. 
%\end{lemma}  
%\begin{proof}
%%  
%  By \Cref{StoneClosedSubsets}, there merely exists some sequence $D_n:S\to 2$
%  such that $P(s) \leftrightarrow \forall_{n:\N} D_n(s)$. 
%  WLOG we may assume $D_0 = \top$ and $D_{n+1}\subseteq D_n$ for all $n:\N$. 
%%  
%  By Stone duality, each $D_n$ corresponds to some $b_n:B$. 
%  Now each $b_n:B$ already occurs in some $B_{m}$.
%  By countable choice, we can find a sequence of natural numbers $(m_n)_{n:\N}$, 
%  and elements $b_{m_n}:B_{m_n}$ such that $i_{m_n}(b_{m_n}) = b_n$. 
%%
%%
%%
%%  Then each $b_{m_n}:B_{m_n}$ corresponds to a decidable set 
%%  $D_{m_n}:S_{m_n} \to 2$. 
%  We may assume that $(m_n)_{n:\N}$ is a strictly increasing sequence with $m_0 = 0$. 
%  In this case, there exists a function $f:\N \to \N$ such that 
%  for all $k:\N$, we have $m_{f(k)} \leq k \leq m_{f(k)+1}$. 
%  % f 0       = 0, 
%  % f (n + 1) | k  < f(n) + 1 = f(n)
%  %           | k >= f(n) + 1 = f(n) + 1 
%  Then we define 
%  \begin{equation} 
%    P_k = e_k(b_{m_{f(k)}}).
%  \end{equation}
%  We claim that for any $s:S$, we have $P(s) \leftrightarrow \forall_{n:\N} P_n(s|_n)$. 
%  \begin{itemize}
%    \item If $P(s)$, then for any $n:\N$, we have $D_n(s)$, meaning that $s(b_n) = 1$, 
%      now
%
%    \item Suppose that for all $k:\N$, we have $P_k(s|_k)$.
%    To show that $P(s)$, we will show that for all $n:\N$ we have $D_n(s)$. 
%    Let $n:\N$, then $D_n(s)$ iff $s(b_n) = 1$, which is the case as soon as there is some $m:\N$ with 
%    $s|_m (b') = 1$ for some $b'$ with $\iota_a$
%%    As $P_k(s|_k)$, we have $s|_k ( b_{m_{f(k)}} ) = 1$ , hence $s (b_n) = 1$
%          
%  \end{itemize}
%
%  
%\end{proof}


\begin{lemma}
  \rednote{Countability and Boolean doesn't matter once you have the representations, 
  to get the representations with $v_n\circ u_n = 0$, you might need countable choice.}
  Let $A,B,C$ be countably presented Boolean algebras, represented by sequences $(A_n)_{n:\N}, (B_n)_{n:\N},(C_n)_{n:\N}$. 
  Let $u,v,(u_n)_{n:\N},(v_n)_{n:\N}$ be as in the following diagram of Boolean algebras:
  \begin{equation}
    \begin{tikzcd}
      A_\infty \arrow[r,"u_\infty"] & B_\infty  \arrow[r,"v_\infty"] & C_\infty
      \\
      A_n \arrow[u,"i_n^\infty"] \arrow[r,"u_n"] & B_n \arrow[u,"j_n^\infty"] \arrow[r,"v_n"] & C_n \arrow[u,"k^\infty_n"]
      \\
      A_m \arrow[u,"i_m^n"] \arrow[r,"u_m"] & B_m \arrow[u,"j_m^n"] \arrow[r,"v_m"] & C_m \arrow[u,"k_m^n"]
    \end{tikzcd} 
  \end{equation} 
  Furthermore, assume that $v\circ u = 0$ and $v_n \circ u_n = 0$ for all $n:\N$.
  Then $Ker(v)/Im(u)$ is the colimit of the sequence $Ker(v_n)/Im(u_n)$. 
\end{lemma}
\begin{proof}
  First, we will note what the maps in this sequence are, which by some abuse of notation also gives
  the cocone maps. 
\paragraph{If $n\leq m$, there are maps $Ker(v_n)/Im(u_n)\to Ker(v_m) / Im(u_m)$.}
Let $x,y\in B_n, a \in A_n$ be such that $x - y  = u_n(a)$, 
then $j_n^m(x) - j_n^m(y) = j_n^m(u_n(a)) = u_m(i_n^m(a))$.
Thus whenever $x,y\in B_n$ are such that $x \sim_{Im(u_n)} y$, we have that 
$j_n^m(x) \sim_{Im(u_m)} j_n^m(y)$. 
%
%
Furthermore, if $x\in Ker(v_n)$, then $v_n(x) = 0$, thus 
\begin{equation}
  v_m(j_n^m(x)) = k_n^m(v_n(x)) = k_n^m(0) = 0
\end{equation} 
and hence $j_n^m(x) \in Ker(v_m)$. 
Thus $j_n^m$ induces a map $\iota_n^m:Ker(v_n)/Im(u_n) \to Ker(v_m)/Im(u_m)$, 
with $\iota^n_m([x]) = [j_n^m(x)]$ for $x\in Ker(v_n)$. 

\paragraph{These maps are compatible (in particular, the maps $j_n^\infty$ form a cocone).}. 
If $k\leq n \leq m$, we have that $j_n^m \circ j_k^n = j_k^m$.
We thus have that $\iota_n^m \circ \iota_k^n = \iota_k^m$.

\paragraph{Given any cocone $\kappa_n : Ker(v_n)/Im(u_n)\to K$, 
  there exists an extension $\kappa_\infty(v_\infty)/Im(u_\infty)\to K$}
      If $\kappa_n$ forms a cocone, this means that for $n\leq m$ we have 
      $\kappa_m = \kappa_n \circ \iota_m^n$.

      We shall give a map $\kappa_\infty:Ker(v_\infty)/Im(u_\infty) \to K$ satisfying 
      $\kappa_\infty \circ \iota_n^\infty= \kappa_n$ for all $n:\N$.
      We're going to define a map $k:Ker(v_\infty) \to K$.
%%
      Let $x\in Ker(v_\infty)$. Then $x\in B_\infty$ and $v\infty(x) = 0$. 
      As $B_\infty$ is the colimit of the sequence $B_n$, 
      there is some $n:\N$ and some $x':B_n$ with $v_n(x') = 0$. 
      We'd like to define $k(x) = \kappa_n([x'])$. We need to check this definition doesn't depend on $n$. 
      \begin{itemize}
        \item \textbf{$k$ is well-defined} 
      Assume $n\leq m$ are such that we have $x':B_n, x'':B_m$ with $v_n(x') = 0, v_m(x'') = 0$ 
      and $j_n(x') = j_m(x'') = x$. 
      Then there exists some $l\geq m\geq n$ with 
      $j_n^l (x') = j_m^l(x'')$, hence 
      $$\iota_n^l[x'] = \iota_m^l[x'']$$ and thus 
      $$\kappa_l(\iota_n^l[x']) = \kappa_l(\iota_m^l[x''])$$
      But now $\kappa_l\circ \iota_n^l = \kappa_n$ and $\kappa_l\circ \iota_m^l = \kappa_m$, hence 
      $$\kappa_n([x']) = \kappa_m([x''])$$
      Thus $k$ is well-defined. 
      \item \textbf{$k$ respects $Im(u)$}
        Let $x,y:B$ and let $a:A$ be such that are such that $x-y = u(a)$ and $v(x) = v(y) = 0$.
        Then there is some $n:\N$, and some $x',y':B_n, a':A_n$ with $x'-y'= u_n(a'), v_n(x') = v_n(y') = 0$ 
        and $j_n(x') = x, j_n(y') = y, i_n(a') = a$. 
        Hence $[x'] = [y']$, hence $\kappa_n([x']) = \kappa_n([y'])$.
        Thus $k(x) = k(y)$. 
      \end{itemize}
      We conclude that $k$ induces a map $\kappa_\infty:Ker(v)/Im(u) \to K$. 
    \paragraph{$\kappa$ is colimiting}
    Let $\kappa_n$ be as above, and suppose that 
    $\lambda \circ \iota_n^\infty = \kappa_n$. 
    Then for all $n:\N$, $x':B_n$ such that $j_n(x) = x$, we have 
    $$\lambda ([x]) = \lambda \circ \iota_n([x']) = \kappa_n([x']) = k(x)$$
    As such $n,x'$ always exist, it follows that 
    $\lambda([x]) = k(x)$, hence $\lambda[x] = \kappa[x]$, so $\lambda = \kappa$ as required. 
\end{proof} 





%
%\section{Topology}
%
\subsection{Closed subtypes}

\begin{definition}%
  \label{closed-proposition}\label{closed-subtype}
  \begin{enumerate}[(a)]
  \item
    A \notion{closed proposition} is a proposition
    which is merely of the form $x_1 = 0 \land \dots \land x_n = 0$
    for some elements $x_1, \dots, x_n \in R$.
  \item
    Let $X$ be a type.
    A subtype $U : X \to \Prop$ is \notion{closed}
    if for all $x : X$, the proposition $U(x)$ is closed.
  \item
    For $A$ a finitely presented $R$-algebra
    and $f_1, \dots, f_n : A$,
    we set
    $V(f_1, \dots, f_n) \colonequiv
    \{\, x : \Spec A \mid f_1(x) = \dots = f_n(x) = 0 \,\}$.
  \end{enumerate}
\end{definition}

Note that $V(f_1, \dots, f_n) \subseteq \Spec A$ is a closed subtype
and we have $V(f_1, \dots, f_n) = \Spec (A/(f_1, \dots, f_n))$.

\begin{proposition}[using \axiomref{sqc}]%
  There is an order-reversing isomorphism of partial orders
  \begin{align*}
    \text{f.g.-ideals}(R) &\xrightarrow{{\sim}} \Omega_{cl} \\
    I &\mapsto (I = (0))
  \end{align*}
  between the partial order of finitely generated ideals of $R$
  and the partial order of closed propositions.
\end{proposition}

\begin{proof}
  For a finitely generated ideal $I = (x_1, \dots, x_n)$,
  the proposition $I = (0)$ is indeed a closed proposition,
  since it is equivalent to $x_1 = 0 \land \dots \land x_n = 0$.
  It is also evident that we get all closed propositions in this way.
  What remains to show is that
  \[ I = (0) \Rightarrow J = (0)
     \qquad\text{iff}\qquad
     J \subseteq I
     \rlap{\text{.}}
  \]
  For this we use synthetic quasicoherence.
  Note that the set $\Spec R/I = \Hom_{\Alg{R}}(R/I, R)$ is a proposition
  (has at most one element),
  namely it is equivalent to the proposition $I = (0)$.
  Similarly, $\Hom_{\Alg{R}}(R/J, R/I)$ is a proposition
  and equivalent to $J \subseteq I$.
  But then our claim is just the equation
  \[ \Hom(\Spec R/I, \Spec R/J) = \Hom_{\Alg{R}}(R/J, R/I) \]
  which holds by \Cref{spec-embedding},
  since $R/I$ and $R/J$ are finitely presented $R$-algebras
  if $I$ and $J$ are finitely generated ideals.
\end{proof}

\begin{lemma}[using \axiomref{sqc}]%
  \label{ideals-embed-into-closed-subsets}
  We have $V(f_1, \dots, f_n) \subseteq V(g_1, \dots, g_m)$
  as subsets of $\Spec A$
  if and only if
  $(g_1, \dots, g_m) \subseteq (f_1, \dots, f_n)$
  as ideals of $A$.
\end{lemma}

\begin{proof}
  The inclusion $V(f_1, \dots, f_n) \subseteq V(g_1, \dots, g_m)$
  means a map $\Spec (A/(f_1, \dots, f_n)) \to \Spec (A/(g_1, \dots, g_m))$
  over $\Spec A$.
  By \Cref{spec-embedding}, this is equivalent to
  a homomorphism $A/(g_1, \dots, g_m) \to A/(f_1, \dots, f_n)$,
  which in turn means the stated inclusion of ideals.
\end{proof}

\begin{lemma}[using \axiomref{loc}, \axiomref{sqc}, \axiomref{Z-choice}]%
  \label{closed-subtype-affine}
  A closed subtype $C$ of an affine scheme $X=\Spec A$ is an affine scheme
  with $C=\Spec (A/I)$ for a finitely generated ideal $I\subseteq A$.
\end{lemma}

\begin{proof}
  By \axiomref{Z-choice} and boundedness,
  there is a cover $D(f_1),\dots,D(f_l)$, such that
  on each $D(f_i)$, $C$ is the vanishing set of functions
  \[ g_1,\dots,g_n:D(f_i)\to R\rlap{.} \]
  By \Cref{ideals-embed-into-closed-subsets},
  the ideals generated by these functions
  agree in $A_{f_i f_j}$,
  so by \Cref{fg-ideal-local-global},
  there is a finitely generated ideal $I\subseteq A$,
  such that $A_{f_i}\cdot I$ is $(g_1,\dots,g_n)$
  and $C=\Spec A/I$.
\end{proof}

\subsection{Open subtypes}

While we usually drop the prefix ``qc'' in the definition below,
one should keep in mind, that we only use a definition of quasi compact open subsets.
The difference to general opens does not play a role so far,
since we also only consider quasi compact schemes later.

\begin{definition}%
  \label{qc-open}
  \begin{enumerate}[(a)]
  \item A proposition $P$ is \notion{(qc-)open}, if there merely are $f_1,\dots,f_n:R$,
    such that $P$ is equivalent to one of the $f_i$ being invertible.
  \item Let $X$ be a type.
    A subtype $U:X\to\Prop$ is \notion{(qc-)open}, if $U(x)$ is an open proposition for all $x:X$.
  \end{enumerate}
\end{definition}

\begin{proposition}[using \axiomref{loc}, \axiomref{sqc}]%
  \label{open-iff-negation-of-closed}
  A proposition $P$ is open
  if and only if
  it is the negation of some closed proposition
  (\Cref{closed-proposition}).
\end{proposition}

\begin{proof}
  Indeed, by \Cref{generalized-field-property},
  the proposition $\inv(f_1) \lor \dots \lor \inv(f_n)$
  is the negation of ${f_1 = 0} \land \dots \land {f_n = 0}$.
\end{proof}

\begin{proposition}[using \axiomref{loc}, \axiomref{sqc}]%
  \label{open-union-intersection}
  Let $X$ be a type.
  \begin{enumerate}[(a)]
  \item The empty subtype is open in $X$.
  \item $X$ is open in $X$.
  \item Finite intersections of open subtypes of $X$ are open subtypes of $X$.
  \item Finite unions of open subtypes of $X$ are open subtypes of $X$.
  \item Open subtypes are invariant under pointwise double-negation.
  \end{enumerate}
  Axioms are only needed for the last statement.
\end{proposition}

In \Cref{open-subscheme} we will see that open subtypes of open subtypes of a scheme are open in that scheme.
Which is equivalent to open propositions being closed under dependent sums.

\begin{proof}[of \Cref{open-union-intersection}]
  For unions, we can just append lists.
  For intersections, we note that invertibility of a product
  is equivalent to invertibility of both factors.
  Double-negation stability
  follows from \Cref{open-iff-negation-of-closed}.
\end{proof}

\begin{lemma}%
  \label{preimage-open}
  Let $f:X\to Y$ and $U:Y\to\Prop$ open,
  then the \notion{preimage} $U\circ f:X\to\Prop$ is open.
\end{lemma}

\begin{proof}
  If $U(y)$ is an open proposition for all $y : Y$,
  then $U(f(x))$ is an open proposition for all $x : X$.
\end{proof}

\begin{lemma}[using \axiomref{loc}, \axiomref{sqc}]%
  \label{open-inequality-subtype}
  Let $X$ be affine and $x:X$, then the proposition
  \[ x\neq y \]
  is open for all $y:X$.
\end{lemma}

\begin{proof}
  We show a proposition, so we can assume $\iota: X\to \A^n$ is a subtype.
  Then for $x,y:X$, $x\neq y$ is equivalent to $\iota(x)\neq\iota(y)$.
  But for $x,y:\A^n$, $x\neq y$ is the open proposition that $x-y\neq 0$.
\end{proof}

The intersection of all open neighborhoods of a point in an affine scheme,
is the formal neighborhood of the point.
We will see in \Cref{intersection-of-all-opens}, that this also holds for schemes.

\begin{lemma}[using \axiomref{loc}, \axiomref{sqc}]%
  \label{affine-intersection-of-all-opens}
  Let $X$ be affine and $x:X$, then the proposition
  \[ \prod_{U:X\to \Open}U(x)\to U(y) \]
  is equivalent to $\neg\neg (x=y)$.
\end{lemma}

\begin{proof}
  By \Cref{open-union-intersection}, $\neg\neg (x=y)$ implies $\prod_{U:X\to \Open}U(x)\to U(y)$.
  For the other implication,
  $\neg (x=y)$ is open by \Cref{open-inequality-subtype}, so we get a contradiction.
\end{proof}

We now show that our two definitions (\Cref{affine-open}, \Cref{qc-open})
of open subtypes of an affine scheme are equivalent.

\begin{theorem}[using \axiomref{loc}, \axiomref{sqc}, \axiomref{Z-choice}]%
  \label{qc-open-affine-open}
  Let $X=\Spec A$ and $U:X\to\Prop$ be an open subtype,
  then $U$ is affine open, i.e. there merely are $h_1,\dots,h_n:X\to R$ such that
  $U=D(h_1,\dots,h_n)$.
\end{theorem}

\begin{proof}
  Let $L(x)$ be the type of finite lists of elements of $R$,
  such that one of them being invertible is equivalent to $U(x)$.
  By assumption, we know
  \[\prod_{x:X}\propTrunc{L(x)}\rlap{.}\]
  So by \axiomref{Z-choice}, we have $s_i:\prod_{x:D(f_i)}L(x)$.
  We compose with the length function for lists to get functions $l_i:D(f_i)\to\N$.
  By \Cref{boundedness}, the $l_i$ are bounded.
  Since we are proving a proposition, we can assume we have actual bounds $b_i:\N$.
  So we get functions $\tilde{s_i}:D(f_i)\to R^{b_i}$,
  by append zeros to lists which are too short,
  i.e. $\widetilde{s}_i(x)$ is $s_i(x)$ with $b_i-l_i(x)$ zeros appended.

  Then one of the entries of $\widetilde{s}_i(x)$ being invertible,
  is still equivalent to $U(x)$.
  So if we define $g_{ij}(x)\colonequiv \pi_j(\widetilde{s}_i(x))$,
  we have functions on $D(f_i)$, such that
  \[
    D(g_{i1},\dots,g_{ib_i})=U\cap D(f_i)
    \rlap{.}
  \]
  By \Cref{affine-open-trans}, this is enough to solve the problem on all of $X$.
\end{proof}

This allows us to transfer one important lemma from affine-opens to qc-opens.
The subtlety of the following is that while it is clear that the intersection of two
qc-opens on a type, which are \emph{globally} defined is open again, it is not clear,
that the same holds, if one qc-open is only defined on the other.

\begin{lemma}[using \axiomref{loc}, \axiomref{sqc}, \axiomref{Z-choice}]%
  \label{qc-open-trans}
  Let $X$ be a scheme, $U\subseteq X$ qc-open in $X$ and $V\subseteq U$ qc-open in $U$,
  then $V$ is qc-open in $X$.
\end{lemma}

\begin{proof}
  Let $X_i=\Spec A_i$ be a finite affine cover of $X$.
  It is enough to show, that the restriction $V_i$ of $V$ to $X_i$ is qc-open.
  $U_i\colonequiv X_i\cap U$ is qc-open in $X_i$, since $X_i$ is qc-open.
  By \Cref{qc-open-affine-open}, $U_i$ is affine-open in $X_i$,
  so $U_i=D(f_1,\dots,f_n)$.
  $V_i\cap D(f_j)$ is affine-open in $D(f_j)$, so by \Cref{affine-open-trans},
  $V_i\cap D(f_j)$ is affine-open in $X_i$.
  This implies $V_i\cap D(f_j)$ is qc-open in $X_i$ and so is $V_i=\bigcup_{j}V_i\cap D(f_j)$.
\end{proof}

\begin{lemma}[using \axiomref{loc}, \axiomref{sqc}, \axiomref{Z-choice}]%
  \label{qc-open-sigma-closed}
  \begin{enumerate}[(a)]
  \item qc-open propositions are closed under dependent sums:
    if $P : \Open$ and $U : P \to \Open$,
    then the proposition $\sum_{x : P} U(x)$ is also open.
  \item Let $X$ be a type. Any open subtype of an open subtype of $X$ is an open subtype of $X$.
  \end{enumerate}
\end{lemma}

\begin{proof}
  \begin{enumerate}[(a)]
  \item Apply \Cref{qc-open-trans} to the point $\Spec R$.
  \item Apply the above pointwise.
  \end{enumerate}
\end{proof}

\begin{remark}
  \Cref{qc-open-sigma-closed} means that
  the (qc-) open propositions constitute a \notion{dominance}
  in the sense of~\cite{rosolini-phd-thesis}.
\end{remark}

The following fact about the interaction of closed and open propositions
is due to David Wärn.

\begin{lemma}%
  \label{implication-from-closed-to-open}
  Let $P$ and $Q$ be propositions
  with $P$ closed and $Q$ open.
  Then $P \to Q$ is equivalent to $\lnot P \lor Q$.
\end{lemma}

\begin{proof}
  For any propositions $P$, $Q$, it is the case that $\lnot P \lor Q$ implies $P \to Q$ and that $P \to Q$ implies $\lnot \lnot (\lnot P \lor Q)$.
  When $P$ is closed and $Q$ is open, we have that $\lnot P$ is open by \Cref{open-iff-negation-of-closed}, so $\lnot P \lor Q$ is also an open proposition and thus double-negation stable by \Cref{open-union-intersection}.
\end{proof}


%
%\subsection{Compact Hausdorff}
%\begin{definition}
  A type $X$ is called a compact Hausdorff space if its identity types are closed propositions and there exists some $S:\Stone$ and a surjection $S\twoheadrightarrow X$.
\end{definition}

%This means that compact Hausdorff spaces are precisely quotients of Stone spaces by closed equivalence relations.

\subsection{Topology on compact Hausdorff spaces}

\begin{lemma}\label{CompactHausdorffClosed}
  Let $X:\CHaus$ with $S:\Stone$ and a surjective map $q:S\twoheadrightarrow X$.
  Then $A\subseteq X$ is closed if and only if it is the image of a closed subset of $S$ by $q$. 
\end{lemma}
\begin{proof}
%  If $A$ is closed, then it's pre-image under any map is also closed. 
%  In particular for $q:S\to X$ the quotient map, $q^{-1}(A)$ is closed. 
  As $q$ is surjective, we have $q(q^{-1}(A)) = A$.
  If $A$ is closed, so is $q^{-1}(A)$ and 
  hence $A$ is the image of a closed subtype of $S$. 
  Conversely, let $B\subseteq S$ be closed. 
%  Then for any $s:S$, the subtype $\{t:S| B(s) \wedge s \sim t\} \subseteq S$ is closed. 
%  Hence by 
  Define $A'\subseteq S$ by 
  $$A'(s) := \exists_{t:S} (B(t) \wedge q(s) = q(t)).$$
  As $B(t)$ and $q(s) = q(t)$ are closed, by \Cref{ClosedCountableConjunction} and \Cref{InhabitedClosedSubSpaceClosed}, 
  we have that $A'$ is closed. 
  Also $A'$ factors through $q$ as a map $A: X\to \Closed$.
  Furthermore, $A'(s) \leftrightarrow (q(s)\in q(B))$. 
  Hence $A=q(B)$. 
%  Therefore $A(x)$ iff $x$ is in the image of $B$. 
\end{proof}

\begin{remark}\label{InhabitedClosedSubSpaceClosedCHaus}
  Let $X:\Chaus$.
  From \Cref{StoneClosedSubsets}, it follows that $A\subseteq X$ is closed if and only if it is the image of a map 
  $T\to X$ for some $T:\Stone$. 
  If $A$ is closed, it follows from \Cref{InhabitedClosedSubSpaceClosed} that $\exists_{x:X} A(x)$ is closed as well. 
\end{remark}
%\begin{corollary}
%  For $X:\CHaus$ a subtype $A\subseteq X$ is closed iff it is the image of 
%  a map $T\to X$ for some $T:\Stone$. 
%\end{corollary}
%\begin{proof}
%  Directly from the above and \Cref{StoneClosedSubsets}.
%\end{proof}
%WhyDidWeNeedThis%\begin{remark}
%WhyDidWeNeedThis%  It is not the case that every closed subset of a compact Hausdorff space can be written 
%WhyDidWeNeedThis%  as countable intersection of decidable subsets. 
%WhyDidWeNeedThis%  In \Cref{UnitInterval}, we shall introduce the unit interval $[0,1]$ as a compact Hausdorff space with many closed 
%WhyDidWeNeedThis%  subsets, but only two decidable subsets. 
%WhyDidWeNeedThis%  In \Cref{ConnectedComponent}, we shall actually see that whenever every singleton of a compact Hausdorff space $X$
%WhyDidWeNeedThis%  can be written as countable intersection of decidable subsets, $X$ is Stone. 
%WhyDidWeNeedThis%  \rednote{Actually, we'll see that $Sp(2^X)$ and $X$ are bijective sets, 
%WhyDidWeNeedThis%    which only implies that $X$ is Stone if $2^X:\Boole$, but this depends on our definition of countable, 
%WhyDidWeNeedThis%see \Cref{CountabilityDiscussion}}
%WhyDidWeNeedThis%\end{remark}


\begin{corollary}\label{AllOpenSubspaceOpen}
  For $U\subseteq X$ an open subset of a compact Hausdorff space, we have that the proposition 
  $\forall_{x:X} U(x)$ is open. 
\end{corollary}
\begin{proof}
  As $U$ is open, $\neg U$ is closed. 
  So $\exists_{x:X} \neg U(x)$ is closed by \Cref{InhabitedClosedSubSpaceClosedCHaus}. 
  Using \Cref{rmkOpenClosedNegation}, it follows that 
  $\neg (\exists_{x:X} \neg U(x))$ is open. 
  Furthermore, it is equivalent to $\forall_{x:X} \neg \neg U(x)$, 
  which is equivalent to $\forall_{x:X} U(x)$ by \Cref{rmkOpenClosedNegation}.
\end{proof}

\begin{lemma}\label{CHausFiniteIntersectionProperty}
  Given $X:\Chaus$ and $C_n:X\to \Closed$ closed subsets such that $\bigcap_{n:\N} C_n =\emptyset$, there is some $k:\N$ 
  with $\bigcap_{n\leq k} C_n  = \emptyset$. 
\end{lemma}
\begin{proof}
  By \Cref{CompactHausdorffClosed} it is enough to prove the result when $X$ is Stone, and by \Cref{StoneClosedSubsets} we can assume $C_n$ decidable.
  So assume 
  $X=Sp(B)$ and $c_n:B$ such that: 
  $$C_n = \{x:B\to 2\ |\ x(c_n) = 0\}.$$ 
  Then the set of maps $B\to 2$ sending all $c_n$ to $0$ is given by: 
  $$Sp(B/(c_n)_{n:\N})%\ |\ n:\N\}) 
  \simeq \bigcap_{n:\N} C_n = \emptyset .$$
  Hence 
%  $0=_{B/(\neg c_n)_{n:\N}}1$ 
  $0=1$ in $B/(c_n)_{n:\N}$ %\ |\ n:\N\}$, 
  and there is some $k:\N$ with 
  $\bigvee_{n\leq k} c_n = 1$, which also means that: 
  $$\emptyset = Sp(B/(c_n)_{n\leq k}) %\ |\ n\leq k\})  
  \simeq \bigcap_{n\leq k} C_n $$
  as required.
\end{proof}

\begin{corollary}\label{ChausMapsPreserveIntersectionOfClosed}
  Let $X,Y:\CHaus$ and $f:X \to Y$. 
  Suppose $(G_n)_{n:\N}$ is a decreasing sequence of closed subsets of $X$. 
  Then $f(\bigcap_{n:\N} G_n) = \bigcap_{n:\N}f(G_n)$. 
\end{corollary}
\begin{proof}
  It is always the case that $f(\bigcap_{n:\N} G_n) \subseteq \bigcap_{n:\N} f(G_n)$. 
  For the converse direction, suppose that $y \in f(G_n)$ for all $n:\N$. 
  We define $F\subseteq X$ closed by $F=f^{-1}(y)$. 
  Then for all $n:\N$ we have that $F\cap G_n$ is merely inhabited and therefore non-empty. 
  By \Cref{CHausFiniteIntersectionProperty} this implies that $\bigcap_{n:\N}(F\cap G_n) \neq \emptyset$. 
  By \Cref{InhabitedClosedSubSpaceClosedCHaus} and \Cref{rmkOpenClosedNegation}, we have that $\bigcap_{n:\N} (F\cap G_n)$ is merely inhabited. Thus $y\in f(\bigcap_{n:\N} G_n)$ as required. 
\end{proof}

\begin{corollary}\label{CompactHausdorffTopology}
Let $A\subseteq X$ be a subset of a compact Hausdorff space and $p:S\twoheadrightarrow X$ be a surjective map with $S:\Stone$. Then $A$ is closed (resp. open) if and only if there exists a sequence $(D_n)_{n:\N}$ of decidable in $S$ such that $A = \bigcap_{n:\N} p(D_n)$ (resp. $A = \bigcup_{n:\N} \neg p(D_n)$).
%\begin{itemize}
%  \item $A$ is closed iff %if and only if 
%    it can be written as $\bigcap_{n:\N} p(D_n)$
%for some $D_n\subseteq S$ decidable. 
%  \item $A$ is open iff %if and only if 
%    it can be written as $\bigcup_{n:\N} \neg p(D_n)$
%for some $D_n\subseteq S$ decidable.
%\end{itemize}  
\end{corollary}
\begin{proof}
  The characterization of closed sets follows from characterization (ii) in \Cref{StoneClosedSubsets}, 
  \Cref{CompactHausdorffClosed} 
  and \Cref{ChausMapsPreserveIntersectionOfClosed}. 
%  The characterization of open sets 
  For open sets we use \Cref{rmkOpenClosedNegation} and
  \Cref{ClosedMarkov}.
\end{proof}
%
\begin{remark}
  For $S:\Stone$, there is a surjection $\N\twoheadrightarrow 2^S$. 
  It follows that for any $X:\CHaus$ there is a surjection from $\N$ to a basis of $X$. 
  Classically this means that $X$ is second countable. 
\end{remark}
%It follows that compact Hausdorff spaces are second countable:
%\begin{corollary}
%  Any $X:\Chaus$ is has a topological basis which is countable.
%\end{corollary}
%\begin{proof}
%  By \Cref{CompactHausdorffTopology}, 
%  a basis is given by the images of the decidable subsets of some $S:\Stone$. 
%  By \cref{ODiscBAareBoole}, $2^S$ is 
%  overtly discrete so we have a surjection $\N\to 2^S$.
%  \end{proof}
%
\begin{lemma}\label{CHausSeperationOfClosedByOpens}
 Assume $X:\CHaus$ and $A,B\subseteq X$ closed such that $A\cap B=\emptyset$. 
  Then there exist $U,V\subseteq X$ open such that $A\subseteq U$, $B\subseteq V$ and $U\cap V=\emptyset$. 
\end{lemma}
\begin{proof}
  Let $q:S\to X$ be a surjective map with $S:\Stone$.
  As $q^{-1}(A)$ and $q^{-1}(B)$ are closed, 
  by \Cref{StoneSeperated}, there is some $D:S \to 2$ such that
  $q^{-1}(A) \subseteq D, q^{-1}(B) \subseteq \neg D$. 
  Note that $q(D)$ and $q(\neg D)$ are closed by \Cref{CompactHausdorffClosed}. 
  As $q^{-1}(A) \cap \neg D  =\emptyset$, we have that 
  $A\subseteq \neg q(\neg D)$. As $A\cap B = \emptyset$, we have that 
  $A\subseteq U:= \neg q(\neg D) \cap \neg B$.
  Similarly, $B\subseteq V:=\neg  q(D) \cap \neg A$. 
  Then $U$ and $V$ are disjoint because $\neg q(D)\cap \neg q(\neg D) \subset \neg (q(D)\cup q(\neg D)) = \neg X = \emptyset$.
\end{proof}


%
%
%%\begin{definition}
  A type $X$ is called a compact Hausdorff space if its identity types are closed propositions and there exists some $S:\Stone$ and a surjection $S\twoheadrightarrow X$.
\end{definition}

%This means that compact Hausdorff spaces are precisely quotients of Stone spaces by closed equivalence relations.

\subsection{Topology on compact Hausdorff spaces}

\begin{lemma}\label{CompactHausdorffClosed}
  Let $X:\CHaus$ with $S:\Stone$ and a surjective map $q:S\twoheadrightarrow X$.
  Then $A\subseteq X$ is closed if and only if it is the image of a closed subset of $S$ by $q$. 
\end{lemma}
\begin{proof}
%  If $A$ is closed, then it's pre-image under any map is also closed. 
%  In particular for $q:S\to X$ the quotient map, $q^{-1}(A)$ is closed. 
  As $q$ is surjective, we have $q(q^{-1}(A)) = A$.
  If $A$ is closed, so is $q^{-1}(A)$ and 
  hence $A$ is the image of a closed subtype of $S$. 
  Conversely, let $B\subseteq S$ be closed. 
%  Then for any $s:S$, the subtype $\{t:S| B(s) \wedge s \sim t\} \subseteq S$ is closed. 
%  Hence by 
  Define $A'\subseteq S$ by 
  $$A'(s) := \exists_{t:S} (B(t) \wedge q(s) = q(t)).$$
  As $B(t)$ and $q(s) = q(t)$ are closed, by \Cref{ClosedCountableConjunction} and \Cref{InhabitedClosedSubSpaceClosed}, 
  we have that $A'$ is closed. 
  Also $A'$ factors through $q$ as a map $A: X\to \Closed$.
  Furthermore, $A'(s) \leftrightarrow (q(s)\in q(B))$. 
  Hence $A=q(B)$. 
%  Therefore $A(x)$ iff $x$ is in the image of $B$. 
\end{proof}

\begin{remark}\label{InhabitedClosedSubSpaceClosedCHaus}
  Let $X:\Chaus$.
  From \Cref{StoneClosedSubsets}, it follows that $A\subseteq X$ is closed if and only if it is the image of a map 
  $T\to X$ for some $T:\Stone$. 
  If $A$ is closed, it follows from \Cref{InhabitedClosedSubSpaceClosed} that $\exists_{x:X} A(x)$ is closed as well. 
\end{remark}
%\begin{corollary}
%  For $X:\CHaus$ a subtype $A\subseteq X$ is closed iff it is the image of 
%  a map $T\to X$ for some $T:\Stone$. 
%\end{corollary}
%\begin{proof}
%  Directly from the above and \Cref{StoneClosedSubsets}.
%\end{proof}
%WhyDidWeNeedThis%\begin{remark}
%WhyDidWeNeedThis%  It is not the case that every closed subset of a compact Hausdorff space can be written 
%WhyDidWeNeedThis%  as countable intersection of decidable subsets. 
%WhyDidWeNeedThis%  In \Cref{UnitInterval}, we shall introduce the unit interval $[0,1]$ as a compact Hausdorff space with many closed 
%WhyDidWeNeedThis%  subsets, but only two decidable subsets. 
%WhyDidWeNeedThis%  In \Cref{ConnectedComponent}, we shall actually see that whenever every singleton of a compact Hausdorff space $X$
%WhyDidWeNeedThis%  can be written as countable intersection of decidable subsets, $X$ is Stone. 
%WhyDidWeNeedThis%  \rednote{Actually, we'll see that $Sp(2^X)$ and $X$ are bijective sets, 
%WhyDidWeNeedThis%    which only implies that $X$ is Stone if $2^X:\Boole$, but this depends on our definition of countable, 
%WhyDidWeNeedThis%see \Cref{CountabilityDiscussion}}
%WhyDidWeNeedThis%\end{remark}


\begin{corollary}\label{AllOpenSubspaceOpen}
  For $U\subseteq X$ an open subset of a compact Hausdorff space, we have that the proposition 
  $\forall_{x:X} U(x)$ is open. 
\end{corollary}
\begin{proof}
  As $U$ is open, $\neg U$ is closed. 
  So $\exists_{x:X} \neg U(x)$ is closed by \Cref{InhabitedClosedSubSpaceClosedCHaus}. 
  Using \Cref{rmkOpenClosedNegation}, it follows that 
  $\neg (\exists_{x:X} \neg U(x))$ is open. 
  Furthermore, it is equivalent to $\forall_{x:X} \neg \neg U(x)$, 
  which is equivalent to $\forall_{x:X} U(x)$ by \Cref{rmkOpenClosedNegation}.
\end{proof}

\begin{lemma}\label{CHausFiniteIntersectionProperty}
  Given $X:\Chaus$ and $C_n:X\to \Closed$ closed subsets such that $\bigcap_{n:\N} C_n =\emptyset$, there is some $k:\N$ 
  with $\bigcap_{n\leq k} C_n  = \emptyset$. 
\end{lemma}
\begin{proof}
  By \Cref{CompactHausdorffClosed} it is enough to prove the result when $X$ is Stone, and by \Cref{StoneClosedSubsets} we can assume $C_n$ decidable.
  So assume 
  $X=Sp(B)$ and $c_n:B$ such that: 
  $$C_n = \{x:B\to 2\ |\ x(c_n) = 0\}.$$ 
  Then the set of maps $B\to 2$ sending all $c_n$ to $0$ is given by: 
  $$Sp(B/(c_n)_{n:\N})%\ |\ n:\N\}) 
  \simeq \bigcap_{n:\N} C_n = \emptyset .$$
  Hence 
%  $0=_{B/(\neg c_n)_{n:\N}}1$ 
  $0=1$ in $B/(c_n)_{n:\N}$ %\ |\ n:\N\}$, 
  and there is some $k:\N$ with 
  $\bigvee_{n\leq k} c_n = 1$, which also means that: 
  $$\emptyset = Sp(B/(c_n)_{n\leq k}) %\ |\ n\leq k\})  
  \simeq \bigcap_{n\leq k} C_n $$
  as required.
\end{proof}

\begin{corollary}\label{ChausMapsPreserveIntersectionOfClosed}
  Let $X,Y:\CHaus$ and $f:X \to Y$. 
  Suppose $(G_n)_{n:\N}$ is a decreasing sequence of closed subsets of $X$. 
  Then $f(\bigcap_{n:\N} G_n) = \bigcap_{n:\N}f(G_n)$. 
\end{corollary}
\begin{proof}
  It is always the case that $f(\bigcap_{n:\N} G_n) \subseteq \bigcap_{n:\N} f(G_n)$. 
  For the converse direction, suppose that $y \in f(G_n)$ for all $n:\N$. 
  We define $F\subseteq X$ closed by $F=f^{-1}(y)$. 
  Then for all $n:\N$ we have that $F\cap G_n$ is merely inhabited and therefore non-empty. 
  By \Cref{CHausFiniteIntersectionProperty} this implies that $\bigcap_{n:\N}(F\cap G_n) \neq \emptyset$. 
  By \Cref{InhabitedClosedSubSpaceClosedCHaus} and \Cref{rmkOpenClosedNegation}, we have that $\bigcap_{n:\N} (F\cap G_n)$ is merely inhabited. Thus $y\in f(\bigcap_{n:\N} G_n)$ as required. 
\end{proof}

\begin{corollary}\label{CompactHausdorffTopology}
Let $A\subseteq X$ be a subset of a compact Hausdorff space and $p:S\twoheadrightarrow X$ be a surjective map with $S:\Stone$. Then $A$ is closed (resp. open) if and only if there exists a sequence $(D_n)_{n:\N}$ of decidable in $S$ such that $A = \bigcap_{n:\N} p(D_n)$ (resp. $A = \bigcup_{n:\N} \neg p(D_n)$).
%\begin{itemize}
%  \item $A$ is closed iff %if and only if 
%    it can be written as $\bigcap_{n:\N} p(D_n)$
%for some $D_n\subseteq S$ decidable. 
%  \item $A$ is open iff %if and only if 
%    it can be written as $\bigcup_{n:\N} \neg p(D_n)$
%for some $D_n\subseteq S$ decidable.
%\end{itemize}  
\end{corollary}
\begin{proof}
  The characterization of closed sets follows from characterization (ii) in \Cref{StoneClosedSubsets}, 
  \Cref{CompactHausdorffClosed} 
  and \Cref{ChausMapsPreserveIntersectionOfClosed}. 
%  The characterization of open sets 
  For open sets we use \Cref{rmkOpenClosedNegation} and
  \Cref{ClosedMarkov}.
\end{proof}
%
\begin{remark}
  For $S:\Stone$, there is a surjection $\N\twoheadrightarrow 2^S$. 
  It follows that for any $X:\CHaus$ there is a surjection from $\N$ to a basis of $X$. 
  Classically this means that $X$ is second countable. 
\end{remark}
%It follows that compact Hausdorff spaces are second countable:
%\begin{corollary}
%  Any $X:\Chaus$ is has a topological basis which is countable.
%\end{corollary}
%\begin{proof}
%  By \Cref{CompactHausdorffTopology}, 
%  a basis is given by the images of the decidable subsets of some $S:\Stone$. 
%  By \cref{ODiscBAareBoole}, $2^S$ is 
%  overtly discrete so we have a surjection $\N\to 2^S$.
%  \end{proof}
%
\begin{lemma}\label{CHausSeperationOfClosedByOpens}
 Assume $X:\CHaus$ and $A,B\subseteq X$ closed such that $A\cap B=\emptyset$. 
  Then there exist $U,V\subseteq X$ open such that $A\subseteq U$, $B\subseteq V$ and $U\cap V=\emptyset$. 
\end{lemma}
\begin{proof}
  Let $q:S\to X$ be a surjective map with $S:\Stone$.
  As $q^{-1}(A)$ and $q^{-1}(B)$ are closed, 
  by \Cref{StoneSeperated}, there is some $D:S \to 2$ such that
  $q^{-1}(A) \subseteq D, q^{-1}(B) \subseteq \neg D$. 
  Note that $q(D)$ and $q(\neg D)$ are closed by \Cref{CompactHausdorffClosed}. 
  As $q^{-1}(A) \cap \neg D  =\emptyset$, we have that 
  $A\subseteq \neg q(\neg D)$. As $A\cap B = \emptyset$, we have that 
  $A\subseteq U:= \neg q(\neg D) \cap \neg B$.
  Similarly, $B\subseteq V:=\neg  q(D) \cap \neg A$. 
  Then $U$ and $V$ are disjoint because $\neg q(D)\cap \neg q(\neg D) \subset \neg (q(D)\cup q(\neg D)) = \neg X = \emptyset$.
\end{proof}


%%\subsection{Intersection of closed in compact Hausdorff}

\begin{lemma}
  In a compact Hausdorff, closed sets are closed under intersection. 
\end{lemma}
\begin{proof}
  
\end{proof}



\begin{lemma}
  Any Stone space merely is a closed subspace of Cantor space. 
\end{lemma}
\begin{proof}
  Let $S$ be a Stone space, and let it's underlying Boolean algebra $B$ be generated by 
  $(b_n)_{n:\N}$ under quotient of the relations ${\phi_i}_{i:\N}$. 
  Then $S = \{ x: 2^\N | \forall_{i:\mathbb n} x(\phi_i) = 0\}$, 
\end{proof}


\begin{lemma}
  For, $D\subseteq 2^\N$ decidable, $\sim$ a closed equivalence relation on $2^\N$,
  the set $$\{x:S | \exists y : D (x\sim y)\}$$ is closed. 
\end{lemma}
\begin{proof}
  Let $x:S$. We need to show that $\exists (y:S) D(y) \wedge x \sim y$ is a closed proposition. 

  Note first that as $\sim $ is closed, $x \sim \cdot $ is a closed subset of $S$. 
  Therefore, $x\sim \cdot = \bigcap_{n:\N} E_n$ for $(E_n)_{n:\N}$ a
  countable family of decidable subsets of $S$, without losing generality, 
  we may even assume that $E_n \subseteq E_m$ whenever $m\geq n$. 
  We thus need to show that 
  $$
  \exists (y:S) D(y) \wedge (\bigcap_{n:\N} E_n)(y) 
  = 
  \exists (y:S) (\bigcap_{n:\N} D \cap  E_n)(y) 
  $$
  is closed. 
%
  Now we claim that 
  $\exists (y:S) D(y) \wedge E_n(y)$ is closed for all $n:\N$. 
  There merely exists an $m:\N$ such that both $D$ and $E_n$ only consider 
  the first $m$ entries of a sequence. 
  
\end{proof}



\begin{lemma}
  For $S$ Stone, $D\subseteq S$ decidable, 
  $\sim$ a decidable equivalence relation on $S$,
  the set $\{x:S | \exists y : D (x\sim y)\}$ is closed. 
\end{lemma}
\begin{proof}
  Let $B = 2^S$, so $S = Sp(B)$. 
  As $D$ is decidable, 
%  there is some $b:B$ such that $D(y) := (y(b) = 1)$. 
  there is some $n:\N$ such that $D(y)$ only depends on $y|_n$. 

  As $\sim$ is decidable, there is a finite set $I_0\subseteq \N$,
  such that $x\sim y = \prod_{i:I_0} x(i) = y(i)$. 

  Thus 
  $$
   \exists (y : D) (x\sim y) = 
  || \Sigma(y:2^\N) y(b) = 1 \wedge \prod(i:I_0) x(i) = y(i)||
  $$
\end{proof}



\begin{lemma}
  Let $S$ Stone, then $D\subseteq S$ is closed iff 
  $D\subseteq S\subseteq 2^{\N}$ is closed. 
\end{lemma}
\begin{proof}
  Follows immediately from countable intersection of basic clopen. 
\end{proof}




%%  Let $A,B\subseteq X$ be two closed subsets of a compact Hausdorff space $X = S/ \sim$. 
%%  If we know that closed subsets contain are exactly those containing their limits this is very easy right? 
%%  Then any sequence has it's limit both in $A$ and $B$. 
%%\begin{lemma}
%%  Whenever $x_n$ is a convergent sequence, so is $f(x_n)$. 
%%\end{lemma}
%%\begin{proof}
%%  Follows immediately from \Cref{sequenceConvergentIffLimit}.
%%\end{proof}
%%
%%
%%\begin{lemma}
%%  In a compact Hausdorff, whenever a subset $A$ contains all of its limit points, it is closed. 
%%\end{lemma}
%%
%%\begin{proof}
%%  Suppose $A\subseteq X$ contains all of it's limit points. We will show that $f^{-1}(A)$ is closed. 
%%  Let $(x_n)_{n:\N}$ be a sequence in $f^{-1}(A)$ with limit $l$, 
%%  then 
%%  $(f(x_n))_{n:\N}$ is a sequence in $A$ with limit $f(l)$. 
%%  $A$ contains $f(l)$ by assumption. 
%%  Therefore $l\in f^{-1}(A)$. 
%%  Thus every sequence in $f^{-1}(A)$ with a limit has its limit in $f^{-1}(A)$. 
%%\end{proof}
%%
%%\begin{lemma}
%%  In a Stone space, whenever a subset $A$ contains all of its limit points, it is closed. 
%%\end{lemma}
%%\begin{proof}
%%  Let $A \subseteq S$ contain all of it's limit points. 
%%  We will show $A$ is a countable intersection of decidable subsets of $S$, hence closed. 
%%  As $S$ is a subset of Cantor space, we may assume it is Cantor space. 
%%  Thus $A$ is a set of binary sequences. 
%%
%%  We will denote $D_n$ be the set of initial segments of length $n$ occuring in $A$. 
%%  We claim this is well defined, that's not a problem, as it's the image of an operation. 
%%
%%  Counterexample : $A = \{ \overline 0 | p\}$ which contains all of it's limit points
%%  (any sequence in $A$ must be $\overline 0$ constantly, which has a limit if the sequence exists in $A$). 
%%  However, $D_n$ is not decidable. 
%%  Also $A$ is not the intersection of countably many decidable sets I believe. 
%%  Unless off course $p$ is of the form $\alpha=0$, but those are not the only propositions.
%%  For example, the proposition $\beta\neq 0$ cannot be written in that form for general $\beta$. 
%%\end{proof}

%
%\subsection{Open propositions}
%\input{OvertlyDiscrete/FactorizationFin}
%
%\section{Analysis}
%
%\subsection{Convergence}
%\input{Convergence/convergenceClosed}
%\paragraph{Extensional convergence }  
\begin{definition}
  Let $B_\infty$ be the Boolean algebra on countably many generators $(p_n)_{n:\N}$ 
  over the equivalence $p_n\wedge p_m = 0 $ whenever $n \neq m$. 
\end{definition} 
\begin{definition}
  We denote $\Noo$ be the spectrum of $B_\infty$. 
\end{definition} 
\begin{lemma}
  $B_\infty$ is isomorphic with the Boolean algebra of 
  finite/cofinite subsets of $\N$. 
\end{lemma}
\begin{proof}
  To go from $B_\infty$ to subsets of $\N$, we send
  the generators $p_n$ to the singleton $\{n\}$, which are clearly finite. 
  We call the induced Boolean operation $f$. 

  To go from finite/cofinite subsets of $\N$ to $B_\infty$,
  a finite subset $I$ of $\N$ is sent to the element 
  $\bigvee_{i \in I} p_i$, and a cofinite subset $J$ is sent to the element 
  $\bigwedge_{i \in J^C} \neg p_i$.  
  We call this function $g$ and we need to show that $g$ is a Boolean morphism. 
  \begin{itemize}
    \item By deMorgan's laws, $g$ preserves $\neg$. 
    \item To see that $g$ respects $\vee$, we need to check three cases
      \begin{itemize}
        \item If both $I,J$ are finite, then 
        \begin{equation} 
          g(I \cup J) = \bigvee_{i\in I \cup J} p_i= \bigvee_{i\in I} p_i \vee \bigvee_{j\in J} p_j 
        \end{equation}
      \item If both $I,J$ are cofinite, we have
        \begin{equation}
          g(I) \vee g(J) = 
          \bigwedge_{i \in I^C} \neg p_i \vee 
          \bigwedge_{j \in J^C} \neg p_j 
          = 
          \bigwedge_{i\in I^C} 
          \bigwedge_{j \in J^C}(\neg p_i \vee  \neg p_j) 
        \end{equation}
        Now note that $\neg p_i \vee \neg p_j = \neg ( p_i \wedge p_j)$, which 
        is $1$ if $i \neq j$ and $p_i$ if $i =j$. 
        We can leave $1$ out of the meet, and we are left with the intersection of $I^C$ and $J^C$, so
        \begin{equation}
          g(I) \vee g(J) = 
          \bigwedge_{i \in (I^C \cap J^C)} \neg p_i
          = 
          \bigwedge_{i \in (I \cup J)^C} \neg p_i 
        \end{equation} 
        as the union of $I$ and $J$ is also cofinite, this equals 
          $ g( I \cup J)$. 
        \item If $I$ is finite and $J$ cofinite, we have 
        \begin{equation}
        g(I) \vee g(J) = (\bigvee_{i\in I} p_i) \vee (\bigwedge_{j \in J^C} \neg p_j)
        = \bigwedge_{j \in J^C} (\bigvee_{i \in I}( p_i \vee \neg p_j))
        \end{equation}
        If $i\neq j$, then $p_i\wedge p_j = 0$, hence $\neg p_j \geq p_i$ and $p_i \vee \neg p_j  = \neg p_j$
        If $i = j$, then $p_i \vee \neg p_j = 1$.
%        \begin{equation}
%        g(I) \vee g(J) = 
%        = \bigwedge_{j \in J^C} (\bigvee_{i \in I-J}( p_i \vee \neg p_j))
%        \end{equation}
%
%        \item If $I$ is cofinite and $J$ is finite, we have that $I \cup J$ is cofinite.
%        Thus 
%        \begin{equation}
%          g(I \cup J) = \bigwedge_{i \in (I \cup J)^C} \neg p_i
%        \end{equation}
%
      \end{itemize}
    \item The case for $\wedge$ is completely dual to the case for $\vee$. 
  \end{itemize}
We conclude that $g$ is a Boolean morphism. 
Furthermore, $g$ and $f$ are clearly inverses, thus the Boolean algebras are isomorphic. 
\end{proof}

  \begin{lemma}\label{lemBinftyNormalForm}
  Any element of $B_\infty$ can be written as 
  either $\bigvee_{i\in I} p_i$  or
  as $\bigwedge_{j\in J} \neg p_j$ 
  for finite $I,J\subseteq \N$. 
\end{lemma}
\begin{proof}
  Remark that whenever $n \neq m$, we have that 
  $\neg p_n \geq p_m$ as $p_m \wedge p_n = 0$. 
\end{proof}
There is canonical embedding $\N \hookrightarrow \Noo$, 
wich sends $n$ to the unique function $\chi_{n}$ sending $p_n$ to $1$. 
We denote $\infty \in \Noo$ for the function which is constantly $0$. 
By \Cref{PropMarkov}, if an element is not $\infty$, 
it comes from the embedding $\N \hookrightarrow \Noo$. 
\begin{lemma}\label{LemmaOpensContainingInfty}
  Let $U$ be an open subset of $\Noo$ containing $\infty$.
  Then there merely exists an $N:\N$ such that whenever $n\geq N$, 
  $\chi_n\in U$ as well. 
\end{lemma}
\begin{proof}
  It is sufficient to prove the lemma for $U$ a basic open. 
  Assume $b : B_\infty $ is such that 
  $U = \{ \phi: B_\infty \to 2| \phi(b) = 1\}$.
  Assume furthermore that $\infty \in U$.
%  so $U$ contains the function sending every $p_i$ to $0$. 
  by \Cref{lemBinftyNormalForm}, $b$ can have two forms.
  If $b = \vee_{i\in I} p_i$, then as $\infty(b) = 0$, 
  we must have $I = \emptyset$, and thus $b = 0$, 
  which means $U$ is empty, contradicting $\infty\in U$. 
  Therefore, 
  $b$ must be of the form $\wedge_{j \in J} \neg p_j$. 
  Note that for $N = \max J + 1$, whenever $n>J$, 
  $\chi_n$  sends $b$ to $1$. 
  Thus $\chi_n \in U$ as well, and we are done. 
\end{proof}

\begin{definition}
  Let $\alpha$ be a sequence in $X$, we say that $\alpha$
  is convergent iff there exists an extension. 
  \begin{equation}\begin{tikzcd}
    \N \arrow[r, "\alpha"] \arrow[d,hook]  & X \\
    \Noo \arrow[ru,dashed]
  \end{tikzcd}\end{equation}  
\end{definition}  



\begin{proposition}
  A sequence is convergent iff it has a limit
\end{proposition}
\begin{proof}
  Let $\alpha$ be convergent, with extension $\overline \alpha$.
  we claim that $\overline \alpha(\infty)$ is a limit of $\alpha$.
  Let $U \subseteq X$ be an open containing $x$. 
  As $\overline\alpha^{-1}(U)$ is an open subset of $\Noo$ containing $\infty$,
  \Cref{LemmaOpensContainingInfty} tells us there exists some $N$ such that $[N,\infty]\subseteq \overline \alpha^{-1}(U)$. 
  Thus there exists an $N$ such that for $n\geq N$, we have $\alpha(n) \in U$, as required. 

  Conversely, suppose $\alpha$ has limit $x$. 
  Assume $X = Sp(B)$, and let $b\in B$. Then $b$ corresponds to a decidable subset $U\subseteq X$.
  For any decidable subset $U \subseteq X$, we have 
  $\alpha^{-1}(U)$ a decidable subset of $\N$. 
  We claim that $\alpha^{-1}(U)$ is either finite or cofinite. 
  As $U$ is decidable, we can decide wheter $x\in U$. If $x\in U$, $\alpha^{-1}(U)$ is cofinite, as 
  $\alpha(n) \in U$ for all $n \geq N$ for some $N$. 
  If $x\notin U$, we have $x\in U^C$, which is also decidable and therefore $\alpha^{-1}(U^C)$ is cofinite. 
  As $\alpha^{-1}(U) ^ C = \alpha^{-1}(U^C)$, it follows that $\alpha^{-1}(U)$ is finite. 
  Thus $\alpha^{-1}(U)$ is finite or cofinite for any decidable subset $U\subseteq X$. 
  Finite and cofinite subsets of $\N$ correspond to elements of $B_\infty$. 
  Therefore, $\alpha$ induces a map $B \to B_\infty$, which corresponds to a map 
  $\overline \alpha: \Noo \to X$. 

  We claim that $\overline \alpha$ extends $\alpha$. 
  Denote $\iota$ for the map $\N \to \Noo$. 
  We need to show that $\overline \alpha \circ \iota = \alpha$. 
  By definition, we have that $(\overline \alpha \circ \iota)^{-1}(U) = \alpha^{-1}(U)$ 
  for any decidable $U\subseteq X$. 
\end{proof}

\begin{lemma}
  Whenever $S = Sp(B)$ Stone, $f,g: A \to S$, and $f^{-1}(U) = g^{-1}(U)$ for any decidable 
  $U\subseteq S$, we have $f = g$. 
\end{lemma}
\begin{proof}
  By our assumption, we have for all $a:A$ that $f(a) \in U \iff g(a) \in U$ for 
  any decidable $U\subseteq X$. Such $U$ correspond to $b:B$.
  and $f(a) \in U \iff f(a)(b)  = 1$. 
  So the functions $f(a),g(a):B \to 2$ are such that 
  $f(a) (b) = g(a) (b)$ for all $b:B$. 
  This holds for all $a:A$ and by two uses of function extensionality we may conclude 
  $f=g$. 
\end{proof}



%
%\subsection{The interval}
%%The goal of this section is to define the interval $[-2,2]_\mathbb R$ as a scheme. 
We assume $\N, \mathbb Q$ have been defined in HoTT
with linear propostional order relations $<,\leq, > ,\geq$ playing nicely together 
and standard algebraic operations. 
From these, we can define the subtype $\mathbb Q_{>0}=\sum_{q : \mathbb Q} (q>0)$, 
and the absolute-value function $|\cdot|$ on $\mathbb Q$. 

\begin{definition}
  A pre-Cauchy sequence is a sequence of rational numbers $(q_n)_{n: \N}$ with $-2 \leq q_n \leq 2$ 
  for all $n:\N$
%  together with a term of type
  such that for every $\epsilon: \mathbb Q_{>0}$, we have an $N_\epsilon:\N$, 
  such that whenever $n,m \geq N_\epsilon$, we have 
\begin{equation}
%  \forall \epsilon : \mathbb Q_{>0} \Sigma N : \N \forall m,n : \N (m,n \geq N) \to 
  | q_n - q_m | \leq \epsilon
\end{equation} 
\end{definition}

\begin{definition}
Given two pre-Cauchy sequences $p = (p_n)_{n\in\N}, q=(q_n)_{n\in\N}$, 
we define the proposition $p \sim_C  q$ as 
%for all $\epsilon : \mathbb Q_{>0}$ there exists an $N :\N$ such that whenever $n \geq N$, we have
\begin{equation}
  p \sim_C q : = \forall (\epsilon : \mathbb Q_{>0} )\exists ( N :\N) \forall (n : \N) ((n \geq N) \to 
  (| p_n - q_n| \leq  \epsilon))
\end{equation}
\end{definition}
Note that $\sim_C$ defines an equivalence relation on pre-Cauchy sequences. 
\begin{definition}
We define the type of Cauchy sequences as the type of pre-Cauchy sequences quotiented by $\sim_C$. 
\end{definition}

%\begin{definition}
%  A binary sequence consists of an initial segment $I \subseteq \N$
%  and a function $x:I \to 2$. 
%If $I$ is (in)finite, we call the binary sequence (in)finite as well. 
%\end{definition} 
%
%For $x$ a finite binary sequence and $y$ any binary sequence, 
%we'll denote $(x,y)$ for their concatenation, 
%and $\overline x$ for the infinite sequence repeating $x$. 
%
Denote $T = \{-1,0,1\}$. 
\begin{lemma}
  $T^\N$ is a scheme. 
\end{lemma}
\begin{proof}
  Sketch: partition $2^\N$ as follows: 
  For $\alpha: 2^\N$, we'll make a sequence $\beta: T^\N$.
  consider for each $n$ the $n$'th block of 2 entries in $\alpha$
  if both are $0$, $\beta(n) = 0$. 
  If the first is $1$, $\beta(n) = -1$
  If first is $0$ and the second is $1$, then $\beta(n) = 1$. 
  This is a closed equivalence relation. 
\end{proof} 

Consider the relation $\sim_s$ on $T^{\N}$, 
such that for any finite binary sequence $x$, we have 
\begin{align}
  (x,1,\overline 0) &\sim_t (x ,0, \overline 1) \\
  (x,-1,\overline 0) &\sim_t (x ,0, \overline {-1})\\
  (x,1,\overline {-1}) &\sim_t (x , \overline 0) \\
  (x,-1,\overline {1}) &\sim_t (x , \overline 0) 
\end{align} 
\begin{lemma}
$\sim_t$ induces a closed equivalence relation on $2^\N$. 
\end{lemma}
\begin{proof}
  TODO
\end{proof} 

\begin{proposition}\label{propTernaryCauchy}
  $T^\N/ \sim_t$ is isomorphic to the type of Cauchy sequences. 
\end{proposition} 
\begin{definition}%Construction might be better than definition here, but WIP so who cares. 
  For $\alpha: T^\N$, define the rational sequence $tri(\alpha)$ by 
  \begin{equation} (tri(\alpha))_n :  = \sum\limits_{0 \leq i \leq n} \frac{\alpha(i)} { 2^{i}} \end{equation}  
  This sequence is pre-Cauchy with $N_\epsilon$ given by the first $n$ with $(\frac12)^n<\epsilon$. 
\end{definition}  
%
%  Also, whenever $\alpha\sim_t \beta$, we have 
%  $tri(\alpha) \sim_C tri(\beta)$. 
%  Therefore $tri$ induces a function from $T^\N / \sim_t$ to Cauchy sequences. 
\begin{definition}
  Given a pre-Cauchy sequence $p$, 
  we will define a $T$-sequence $\alpha  = c(p): T^\N$.
  Consider any $i:\N$, and suppose $\alpha(j)$ has been defined for $0 \leq j<i$. 
%
  Let $\epsilon_i = (\frac12)^{i+1}$. %Placeholder value.
  Define $N_i:= N_{\epsilon_i}$. %is such that for $n,m \geq N_i$, we have $|p_n - p_m| < \epsilon_i$. 
%
  Consider 
  \begin{equation}
    \widetilde p_i = p_N - \sum\limits_{0\leq j < i} \frac {\alpha(j)}{2^{j}}.
  \end{equation}
  As the order on $\mathbb Q$ is total, we can define 
  \begin{equation}
    \alpha(i) = \begin{cases}
    \phantom{-} 1  \text{ if } \widetilde p_i \geq    (\frac12)^{i} \\
    -1             \text{ if } \widetilde p_i \leq  - (\frac12)^{i} \\
    \phantom{-} 0 \text{ otherwise } 
    \end{cases} 
  \end{equation}  
\end{definition} 
We shall now prove the following four things: 
\begin{itemize}
  \item 
    $c(tri(\alpha)) \sim_t \alpha$ for any $\alpha: T^n$.
  \item 
    $tri(c(p)) \sim_C p$ for any pre-Caucy sequence $p$. 
  \item 
    Whenever $p \sim_C q$, we have $c(p)\sim_t c(q)$. 
  \item 
    Whenever $\alpha \sim_t \beta$, we have $tri(\alpha) \sim_C tri(\beta)$. 
\end{itemize}
It follows that $c$ and $tri$ are maps between Cauchy sequences and $T^\N /\sim_t$ 
which are each other's inverse, proving Proposition \ref{propTernaryCauchy}
\begin{lemma} $tri(c(p)) \sim_C p$ for any pre-Caucy sequence $p$. 
\end{lemma} 



\begin{proof}
  Let $\epsilon>0$ be given, consider $n:\N$ such that
  $(\frac12)^n < \epsilon$. 
  We claim that for $m\geq N_n$, we have that $|p_m- tri(c(p))_m| < \epsilon$. 

  By definition $p_{N_n} $  
\end{proof} 






%In this section we will introduce the unit interval $I$ as compact Hausdorff space. 
The definition is based on \cite{Bishop}. 
%We will then calculate the cohomology of $I$. 
%For a proof that the unit interval corresponds to the definition using Cauchy sequences, 
%we refer to the appendix. 


%\subsection{The Cauchy reals}
The goal of this section is to introduce the real numbers in a constructive setting, 
following the definition given in \cite{Bishop} with some small adaptations. 
We will later use this definition to show that the interval $[0,1]$ is compact Hausdorff in the sense 
of \Cref{dfnCompactHausdorff}. 

We will assume we are given natural and rational numbers, with decidable (in)equalities
working as expected. 

\begin{definition}
  A \textbf{Cauchy sequence} is a sequence $x : \N \to \mathbb Q$ such that
  for any $n,m:\N$, we have %$0\leq x_n \leq 1$ and 
$|x_n-x_m| \leq (\frac12)^n + (\frac12)^m$. 
\end{definition}
\begin{remark}
  If $x$ is a cauchy sequence and $q$ a rational number, the 
  sequence $(x-q)_n = (x_n-q)$ is also Cauchy.
\end{remark}

Following \cite{Bishop}, we define inequality relations between Cauchy sequences and
rational numbers. 
\begin{definition}
  For $x$ a Cauchy sequence and $q$ a rational number, we define 
  \begin{itemize}
%    \item $x> q = \Sigma(n:\N) x_n > q + {\frac12}^n$. %for some $n:\N$. 
%    \item $x< q = \Sigma(n:\N) x_n < q - {\frac12}^n$. %for some $n:\N$. 
    \item $x\leq  q = \Pi_{n:\N} x_n \leq q+(\frac12)^n$. 
    \item $x\geq  q = \Pi_{n:\N} x_n \geq q-(\frac12)^n$. 
  \end{itemize}
\end{definition}
%\begin{lemma}
%  For $x$ Cauchy and $q$ rational, we have that 
%  $x\leq q$ iff for each $n:\N$, we have a $N_n:\N$ with 
%  $x_m> q-(\frac12)^n$ for all $m \geq N_n$. 
%\end{lemma}
\begin{lemma}\label{ComparisonLemma}
  For $x$ a Cauchy sequence and $q$ a rational number, we have
  $x \leq q \vee x \geq q$. 
\end{lemma}
\begin{proof}
  For rational numbers, we have decidable inequalities, 
  therefore $\geq 0 \vee q \leq 0$. 
  It follows that 
  $ \forall (n:\N) \forall (m:\N) q \geq -(\frac12)^n \vee q \leq (\frac12)^m$. 
  Now by \Cref{TODO}, we may conclude 
  $ (\forall (n:\N) q \geq -(\frac12)^n ) \vee (\forall (m:\N) q \leq (\frac12)^m)$
  as required.
\end{proof}


%%%\begin{definition}
%%%  A Cauchy sequence $x$ is \textbf{nonnegative} if $x_n \geq -(\frac12)^n$
%%%  for every $n:\N$. 
%%%  $x$ is \textbf{nonpositive} if $x_n \leq (\frac12)^n$
%%%  for every $n:\N$. 
%%%\end{definition} 
%%%%\begin{lemma}
%%%%  A Cauchy sequence is nonnegative iff there exists an $N$ such that $x_n \geq -(\frac12)^N$
%%%%  for all $n\geq N$. 
%%%%  A Cauchy sequence is nonpositive iff there exists an $N$ such that $x_n \leq (\frac12)^N$
%%%%  for all $n\geq N$. 
%%%%\end{lemma}
%%%%\begin{proof}
%%%%  Assume $x$ is nonnegative. Thus for every $n:\N$, we have $x_n\geq -(\frac12)^n \geq -(\frac12)^0$. 
%%%%  Thus $N$ can taken to be $0$. 
%%%%%
%%%%  Conversely, as $x$ is Cauchy, we have
%%%%  for all $m :\N$ that  
%%%%%  \begin{equation}- (\frac12)^m -(\frac12)^n \leq    x_m-x_n \leq (\frac12)^m + (\frac12)^n \end{equation}
%%%%  \begin{equation}- (\frac12)^m -(\frac12)^n \leq    x_n-x_m \leq (\frac12)^m + (\frac12)^n \end{equation}
%%%%  If in addition there is an $N$ such that whenever $m\geq N$, we have 
%%%%  $x_m \geq -(\frac12)^N$, so $-x_m \leq (\frac12)^N$, 
%%%%  so $x_n -x_m \leq x_n - (\frac12)^N$. 
%%%%  Therefore, 
%%%%  \begin{equation}- (\frac12)^m -(\frac12)^n \leq    x_n-x_m \leq x_n-(\frac12)^N \end{equation}
%%%%  Thus 
%%%%  \begin{equation}- (\frac12)^m -(\frac12)^n  + (\frac12)^N \leq x_n \end{equation}
%%%%  As $N \geq N$, we have in particular 
%%%%  \begin{equation}- (\frac12)^N -(\frac12)^n  + (\frac12)^N \leq x_n \end{equation}
%%%%  \begin{equation} - (\frac12)^n  \leq x_n \end{equation}
%%%%  thus $x$ is nonnegative. 
%%%%
%%%%  The nonpositive case goes similar. 
%%%%\end{proof}   
%%% 
%%%
%%%\begin{lemma}
%%%  A Caucy sequence is nonnegative or nonpositive. 
%%%\end{lemma}

%\begin{lemma}
%  For any Cauchy sequence $p$, we have 
%  $(\forall (n:\N) p_n \leq (\frac12)^n) \vee (\forall (n:\N) p_n \geq -(\frac12)^n)$. 
%\end{lemma}
%\begin{proof}
%We 
%\end{proof}

\begin{definition}
Given two Cauchy sequences $p = (p_n)_{n\in\N}, q=(q_n)_{n\in\N}$, 
we define the proposition $p \sim_C  q$ as 
\begin{equation}
  p \sim_C q : = \forall (n,m : \N) ((| p_n - q_m| \leq  (\frac12)^n + (\frac12)^m))
\end{equation}
\end{definition}

%\begin{remark}
%  Note that $p\sim_C q$ is equivalent to 
%\begin{equation}
%  \forall (n : \N) | p_n - q_n| \leq  (\frac12)^{n-1}
%\end{equation}
%The equivalence doesn't hold, unless you cut off initial segments (which shouldn't matter). 
%\end{remark} 

\begin{definition}
  The type of \textbf{Cauchy reals} is given by 
  the type of Cauchy sequences modulo $\sim_C$.
\end{definition}

We claim that the inequality in \Cref{TODO} extends to a well-defined 
inequality between Cauchy reals and rational numbers. 

Furthermore, we claim that 
$\Pi_{x:\mathbb R} \Pi_{q:\mathbb Q} x \leq q \vee x \geq q$. 

%\begin{lemma}
%  For any Cauchy real $r$ any Cauchy sequence $p$ representing $r$, 
%  we have 
%  \begin{equation}
%    (\forall (n:\N) p_n \leq (\frac12)^n) \vee (\forall (n:\N) p_n \geq (\frac12)^n)
%  \end{equation}
%
%\end{lemma}

\begin{definition}
  A Cauchy sequence in the interval is a Cauchy sequence $x$ such that 
  for any $n:\N$, we have $0\leq x_n \leq 1$. 
 % 
  The interval of Cauchy reals is given by the type of Cauchy sequences in the interval 
  modulo $\sim_C$. We denote it by $[0,1]$. 
\end{definition}  


%------------------content of file BinaryClosedEquivalence.tex...
%We want to show that the interval of Cauchy reals is Compact Hausdorff. 
%Informally, to any binary sequence $\alpha : \N \to 2$, 
%we can associate a Cauchy sequence 
%$cs(\alpha)$, given by 
%\begin{equation}\label{eqnBinaryEncode}
%  (cs(\alpha))_n = \sum\limits_{i = 0 }^n \frac {\alpha(i)}{2^{i+1}}
%\end{equation}
%and we are going to give a closed relation on Cantor space such that 
%two binary sequences are equivalent iff they correspond to the same Cauchy reals. 
\begin{example}
  Let $n:\N$, we denote $C_n = 2[n]$ for the free Boolean algebra on $n$ generators 
  and no relations. 
  Note that $Sp(C_n) = 2^n$ corresponds to the space of finite binary sequences. 
\end{example}
Now we introduce some notation:
\begin{definition}
  \item Given an infinite binary sequence $\alpha:2^\N$ and a natural number $n : \N$  
    we denote $\alpha|_n: 2^n$ for the 
    restriction of $\alpha$ to a finite sequence of length $n$. 
  \item We denote $\overline 0, \overline 1$ 
    for the binary sequences which are constantly $0$ and $1$ respectively. 
  \item We denote $0,1$ for the sequences of length $1$ hitting $0,1$ respectively. 
  \item If $x$ is a finite sequence and $y$ is any sequence, 
    denote $x\cdot y$ for their concatenation. 
\end{definition} 
Now we'll give a definition for when two finite binary sequences of length $n$ correspond 
to real numbers whose distance is $\leq (\frac12)^n$.
Informally, we want for every finite sequence $s$ that 
$(s \cdot 0 \cdot \overline 1)$ and  $(s \cdot 1 \cdot \overline 0)$ are equivalent. 

\begin{definition}
  Let $n:\N$ and let $s,t : 2^n$. 
  We say $s,t$ are $n$-near, and write $s\sim_n t$ if 
  there merely exists some $m:\N$ and some $u:2^m$, such that 
 \begin{equation}\label{EqnNearness}
   \big(
     (s = (u\cdot 0\cdot \overline 1)|_n) \vee (s = (u \cdot 1 \cdot \overline 0) |_n)
   \big)
    \wedge 
   \big(
     (t = (u\cdot 0\cdot \overline 1)|_n) \vee (t = (u \cdot 1 \cdot \overline 0) |_n)
   \big)
  \end{equation} 
\end{definition}
\begin{remark}\label{nearnessProperties}
\item As we're dealing with finite sequences, $s\sim_n t$ is decidable. 
\item Given any $s:2^n$, using $m=n, u = s$ above, we can show that $s\sim_n s$. 
  So $n$-nearness is reflexive. 
\item \Cref{EqnNearness} is symmetric in $s$ and $t$. Hence $n$-nearness is symmetric.
\item Note that $0\cdot 0\sim_2 0\cdot 1 \sim_2 1\cdot 0 \sim_2 1\cdot 1$, 
  but $0\cdot 0\nsim_2 1\cdot 1$. %is not $2$-near to $1\cdot 1$. 
  Thus $n$-nearness is not in general transitive. 
\end{remark}
\begin{definition}
  Let $\alpha, \beta: 2^\N$, we define $a\sim_I\beta$ as 
  $\forall_{n:\N} (\alpha|_n \sim_n \beta|_n)$. 
\end{definition}
\begin{lemma}\label{IntervalFiberSizeAtMost2}
  Whenever $\alpha,\beta,\gamma:2^\N$, are such that 
  $\alpha\sim_I \beta, \beta\sim_I \gamma$, 
  at least two of $\alpha,\beta,\gamma$ are equal. 
\end{lemma}
\begin{proof}
  We will show that $\beta = \gamma \vee \alpha = \gamma \vee \alpha = \beta$. 
  By \Cref{StoneEqualityClosed} and \Cref{ClosedFiniteDisjunction}, this is a closed proposition. 
  By \Cref{rmkOpenClosedNegation}, we can instead show the double negation. 
  To this end, assume that none of $\alpha,\beta,\gamma$ are equal. 
  By \Cref{MarkovPrinciple}, there exist indices $i,j,k\in \N$ with 
  \begin{equation}
    \beta(i) \neq \gamma(i), \alpha(j) \neq \gamma(j), \alpha(k) \neq \beta(k)
  \end{equation}
  Let $n:=\max(i,j,k) + 2$. 
  As $\alpha\sim_I \beta$, we have $\alpha|_n\sim_n\beta|_n$. 
  By assumption $\alpha|_n \neq \beta|_n$, so WLOG we may assume that 
  we have some $m: \N, u:2^m$ with 
  \begin{equation}
    \alpha|_n = (u\cdot 0 \cdot \overline 1) |_n , \beta|_n = (u \cdot 1 \cdot \overline 0)|_n.
  \end{equation}
  As $\alpha(k) \neq \beta(k) $ and $n\geq k+2$, 
  it follows in particular that $m\leq n-2$ and hence 
  $\beta(n-1) = 0$.% and $\beta(m) = 1$. 

  As also $\beta\sim_I \gamma$, we have $\beta|_n \sim_n \gamma|_n$.
  So there exists some $m':\N, u':2^m$ with 
 \begin{equation} %\label{EqnNearness}
   \big(
     (\beta|_{n} = (u'\cdot 0\cdot \overline 1)|_n) \vee (\beta|_{n} = (u' \cdot 1 \cdot \overline 0) |_n)
   \big)
    \wedge 
   \big(
     (\gamma|_{n} = (u'\cdot 0\cdot \overline 1)|_n) \vee (\gamma|_{n} = (u' \cdot 1 \cdot \overline 0) |_n)
   \big).
  \end{equation} 
  Similarly as above, we have that $m'\leq n-2$, and as $\beta(n-1) = 0$, it follows that 
  $\beta|_{n} = (u' \cdot 1 \cdot \overline 0) |_n$. 
  Now as $\beta(i)\neq \gamma(i)$ with $i<n$, we have that $\beta|_n \neq \gamma|_n$, hence 
  $\gamma|_{n} = (u'\cdot 0\cdot \overline 1)|_n$. 
  Now we have $m,m'\leq n-2$ and $u:2^m, u':2^{m'}$ such that 
  \begin{equation}
    (u\cdot 1 \cdot \overline 0)|_n = \beta|_n = (u'\cdot 1 \cdot \overline 0)|_n
  \end{equation}
  Note that $\beta(m') = 1$. 
  But also $\beta(l) = 0$ for all $l$ with $m<l<n$
  Therefore $m'\leq m$. 
  By similar reasoning, $m\leq m'$. We conclude $m=m'$. 
  As a consequence, $u = u'$, but then 
  $\gamma|_n = \alpha|_n$, contradicting that $\alpha(j)\neq \gamma(j) $ for $j<n$. 
  Hence we arrive at a contradiction, as required. 
\end{proof}


\begin{corollary}
  $\sim_I$ is a closed equivalence relation on $2^\N$. 
\end{corollary}
\begin{proof}
  By \Cref{nearnessProperties}, $\sim_I$ is a countable conjunction of decidable propositions. 
  Also by \Cref{nearnessProperties}, $\sim_n$ is reflexive and symmetric for all $n:\N$, thus
  $\sim_I$ is reflexive and symmetric as well. 
  Finally $\sim_I$ is transitive as a consequence of \Cref{IntervalFiberSizeAtMost2}.
\end{proof}
\begin{definition}
  We define $\I:\Chaus$ as $\I= 2^\N/\sim_I$. 
\end{definition}
%------------------
%\subsection{The topology of the interval}
We have defined the interval as a certain quotient of Cantor space, 
in \Cref{IntervalvsCauchyInterval}, a proof is provided for the following theorem:
\begin{theorem}
  Let $I'$ denote the interval of Cauchy real numbers.
  Then the map $2^\N\to I'$ given by 
  \begin{equation}
    \alpha \mapsto \bigsum\limits_{i\in \N} \frac{\alpha(i)} {2^{i+1}}
  \end{equation}
  is well-defined and induces an equivalence $I\simeq I'$. 
\end{theorem}



%\begin{lemma}
  $b$ sends $\sim_n$ equivalent binary sequences to $\sim_C$ equivalent Cauchy sequences. 
\end{lemma}
\begin{proof}
  Let $\alpha, \beta$ be binary sequences.
  We claim that $|b(\alpha)_n - b(\beta)_n| \leq (\frac12)^{n+1}$ 
  whenever $\text{near}_n(\alpha, \beta)$. 
  It will follow that if $\alpha\sim_n \beta$, then 
  $b(\alpha)\sim_C b(\beta)$. 

  Let $n:\N$ and assume $m:\N$ with $m\leq n$ and 
  let $z$ be a sequence of length $m$ such that 
  $\alpha|_n = z\cdot 1 \cdot \overline 0|_n$ and $\beta|_n = z \cdot 0 \cdot \overline q |_n$. 
  then $b(\alpha)_n = \sum_{i\leq m} \frac{z(i)}{2^{i+1}} + (\frac12)^{m+2}$ and 
  $b(\beta)_n = \sum_{i\leq m} \frac{z(i)}{2^{i+1}} + \sum\limits_{m+2 \leq i \leq n}(\frac12)^{i+1}$. 
  Thus 
  $b(\alpha)_n - b(\beta)_n = (\frac12)^{m+2} - \sum\limits_{m+2 \leq i \leq n}(\frac12)^{i+1} = 
  (\frac12)^{n+1}$, 
  which is smaller than required. 
\end{proof}  

\begin{lemma}
  Whenever $b(\alpha) \sim_C b(\beta)$, 
  we have $\alpha \sim_n \beta$. 
\end{lemma}
\begin{proof}
  Assume $b(\alpha) \sim_Cb (\beta)$. 
  Let $n:\N$. 
  We shall show that $\text{near}_n(\alpha , \beta)$. 

  As we're only checking finitely many entries, 
  we either have $\alpha|_n = \beta|_n$, 
  or there exists a smallest $m\leq n$ with 
  $\alpha(m) \neq \beta(m)$. 

  If $\alpha|_n = \beta|_n$, we have $\text{near}_n(\alpha,\beta)$ and are done. 
  WLOG assume $\alpha(m) = 1, \beta(m) = 0$ for $m$ minimal. 

  Now note that 
  \begin{equation} 
    b(\alpha)_{k+1} - b(\beta)_{k+1} = 
    b(\alpha)_{k} - b(\beta)_{k} + 
    \frac{\alpha(k+1) - \beta(k+1)}{2^{k+2}}.
  \end{equation}

  For $k>m$, we have that 
  \begin{equation}
  |b(\alpha)_k - b(\beta)_k |= 
  |(\frac12)^{m+1} + \sum\limits_{i=m+1}^k \frac{ \alpha(i) -\beta(i)}{2^{i+1}}|. 
  \end{equation}
  Note that the right summand is always $\leq (\frac12)^{m+1}$. 
  Therefore, we can leave out the absolute value function. 

  We claim that for every $k\geq m+1$, we have $\alpha(k) = 0, \beta(k) = 1$. 
  We will use induction. 
  Suppose that for every $m <i<j$, we have $\alpha(i) = 0$, and $\beta(i) = 1$. 
  Then 
  \begin{equation}
    b(\alpha)_{j-1} - b(\beta)_{j-1} = 
    (\frac12)^{m+1} + 
    \sum\limits_{i=m+1}^{j-1} \frac{ -1}{2^{i+1}} 
    = (\frac12)^{j}
  \end{equation}
   
  \begin{itemize}
    \item 
      we claim that $\alpha(j) = 0$ 
      Suppose $\alpha(j) = 1$. 
      Then $\alpha(j) -\beta(j) \geq 0$. 
      And for $j + 2$, we have that 
  \begin{align}
    &b(\alpha)_{j+2} - b(\beta)_{j+2}
    \\
    =  
    &(b(\alpha)_{j-1} - b(\beta)_{j-1}) + 
    &\frac{\alpha(j)-\beta(j)}{2^{j+1}} +  
    &\frac{\alpha(j+1) - \beta_(j+1)}{2^{j+2}}
    +
    &\frac{\alpha(j+2) - \beta_(j+2)}{2^{j+3}}
    \\
    \geq  
      & (\frac12)^j + &0 
    + &\frac{-1}{2^{j+2}} 
    + &\frac{-1}{2^{j+3}} 
    \\
      > &(\frac12)^{j+1}
  \end{align}
  which contradicts $b(\alpha) \sim_Cb(\beta)$, 
  which would require that $|b(\alpha)_{j+2} - b(\beta_{j+2} | \leq (\frac{12})^{j+2}+ (\frac12)^{j+2} = (\frac12)^{j+1}$. 
  Therefore $\alpha(j) \neq 1$, and thus $\alpha(j) = 0$. 
    \item 
      We also claim that $\beta(i) = 1$. 
      If $\beta(i) = 0$, we also have 
      $\alpha(j) -\beta(j) \geq 0$, and the rest of the proof is similar as above. 
  \end{itemize}
\end{proof}


%\begin{lemma}
  The map $b: 2^\N \to [0,1]$ is surjective. 
\end{lemma}
\begin{proof}
  First, suppose we have a function 
  $d:\Pi_{x:\mathbb R} \Pi_{q: \mathbb Q} (x \leq q + x \geq q)$
  Then we could recursively define 
  $$\alpha(n) = \begin{cases}
    0 \text{ if } d(x - \sum\limits_{i<n} \frac{\alpha(i)}{2^{i+1}} , \frac{1}{2^{n+1}}) = inl(\cdot) \\
    1 \text{ otherwise}
  \end{cases}
  $$
%  Recall that inequality between rational numbers is decidable, therefore we can define
%  $$\alpha(n) = \begin{cases}
%    0 \text{ if } |x_n - \sum\limits_{i<n} \frac{\alpha(i)}{2^{i+1}}| \leq  \frac{1}{2^{n+1}} \\
%    1 \text{ otherwise}
%  \end{cases}
%  $$
  Note that 
  $$\alpha(n) = \begin{cases}
    0 \text{ if } d(x - b(\alpha)_{n-1} , \frac{1}{2^{n+1}}) = inl(\cdot) \\
    1 \text{ otherwise}
  \end{cases}
  $$
  We'll show by induction that $b(\alpha)_n \leq x$ for every $n:\N$. 
  First $b(\alpha)_0 = 0 \leq x$. 
  Assuming, $b(\alpha)_k \leq x$, for $b(\alpha)_{k+1}$, 
  there are two cases:
  \begin{itemize}
    \item 
     if $d(x -  b(\alpha)_k, \frac{1}{2^{n+1}}) = inl(\cdot)$, 
     then $b(\alpha)_{k+1} = b(\alpha)_k$, which is $\leq x$ by induction hypothesis. 
   \item 
     Otherwise, $ x - b(\alpha)_k \geq (\frac12)^{k+1}$
     So $x-b(\alpha)_k - (\frac12)^{k+1} \geq 0$, 
     and $b(\alpha)_{k+1} = b(\alpha)_k + (\frac12)^{k+1}$. 
     So $x-b(\alpha)_{k+1} \geq 0$, and $b(\alpha)_{k+1} \leq x$ as required. 
 \end{itemize}
 So by induction $b(\alpha)_n\leq x$ for every $n:\N$. 
 Therefore, $|x-b(\alpha)_n| = x-b(\alpha)_n$. 
  
  We shall also show by induction that 
  $ x- b(\alpha)_n \leq (\frac12)^{n+1} $
  for every natural number $n:\N$. 
%
  For $n = 0$, this follows from the assumption that $x\leq 1$. 
%
  Suppose that $ x- b(\alpha)_k  \leq (\frac12)^{k+1} $. 
  We make a case distinction on the form of $d(x-b(\alpha)_k, (\frac12)^{k+2})$.
  \begin{itemize}
    \item 
      If $d(x-b(\alpha)_k , (\frac12)^{k+2}) = inl(\cdot)$, 
      then $  x-b(\alpha)_k  \leq (\frac12)^{k+2}$, 
      and $b(\alpha)_{k+1} = b(\alpha)_k$, 
      and $x-b(\alpha)_{k+1}  \leq (\frac12)^{k+2}$ as well, 
      as required. 
    \item 
      Otherwise, we must have
      $ x- b(\alpha)_k  \geq (\frac12)^{k+2}$, 
      and $b(\alpha)_{k+1} = b(\alpha)_k + (\frac12)^{k+1}$.
      By induction hypothesis, we have 
      $x-b(\alpha)_k \leq (\frac12)^{k+1}$. 
      Thus \begin{equation}
        x-b(\alpha)_{k+1} = x - b(\alpha)_k - (\frac12)^{k+1}
        \leq (\frac12)^{k+1} - (\frac12)^{k+2} = (\frac12)^{k+2}
      \end{equation}
      as required. 
  \end{itemize}
  
  By induction, we conclude that 
  $ | b(\alpha)_n - x |  \leq (\frac12)^{n+1} $
  for every $n:\N$. 
  Therefore $b(\alpha)$ converges to $x$. 

  We may conclude that $\Pi_{x:[0,1]} \Pi_{q: \mathbb Q} (x \leq q + x \geq q)$ implies that 
  we can give for each $x: [0,1]$ a binary sequence $\alpha$ with $b(\alpha) = x$. 
  As we have the propositional trunctation of the premise by \Cref{ComparisonLemma}, 
  we may conclude that for each $x:[0,1]$ there merely exists $\alpha$ with $b(\alpha) = x$. 
  Therefore $b$ is surjective. 
\end{proof}



%
%
%
%\begin{theorem}
%  The interval of Cauchy reals is isomorphic to $2^\N / \sim_t$. 
%\end{theorem} 
%\begin{proof}
%  This follows from the fact that $b:2^\N$ is such that $\alpha\sim_n \beta$ iff $b(\alpha)\sim_t b(\beta)$. 
%  and for every Cauchy real, there is a binary sequence being sent to it, so the composition of $b$ and the 
%  quotient from Caucy sequences to Cauchy real is a surjection. 
%\end{proof}
%
%\begin{corollary}
%  The interval is compact Hausdorff. 
%\end{corollary}

%%\printindex
%
%\section{Directed Univalence}
%% !TEX encoding = UTF-8 Unicode
\subsection{Subquotient systems}

\begin{definition}
A subquotient pre-system consists of $X$ a type and $U$ a class of propositions.
\end{definition}

\begin{definition}
For $(X,U)$ a subquotient pre-system, we define:
\[Sub_{X,U} = \sum_{Y:\Type} \exists (P : X\to U).\ Y = \Sigma_XP\]
\[SubQ_{X,U} = \sum_{Y:\Type} \exists (P : X\to U)(R: \Sigma_XP\to \Sigma_XP\to U\ \mathrm{equivalence\ relation}).\ Y = (\Sigma_XP)/R\]
\end{definition}

\begin{definition}
We say a class of types $T$ has local choice if for all $X\in T$ and $P:X\to\Type$ such that:
\[\prod_{x:X}\propTrunc{P(x)}\]
there merely exists $Y\in T$ and a surjection:
\[f:Y\to X\]
such that:
\[\prod_{y:Y}P(f(y))\]
\end{definition}

\begin{proposition}\label{lex-sub-pro}
Assume $(X,U)$ a Subquotient pre-system such that:
\begin{itemize}
\item Identity types in $X$ are in $U$.
\item $U$ is closed by $\sum$ and $\top$.
\item $\propTrunc{X}$ and $X\times X = X$.
\item $Sub_{X,U}$ has local choice.
\end{itemize}
Then we $SubQ_{X,U}$ has the following:
\begin{itemize}
\item Stability under finite limits.
\item Stability by quotient by equivalence relation with value in $U$.
\item Local choice.
\end{itemize}
\end{proposition}

\begin{proof}
\begin{itemize}
\item First we check that $SubQ_{X,U}$ has local choice. Since we assume that $Sub_{X.U}$ has local choice and that any type in $SubQ_{X,U}$ is covered by a type in $Sub_{X,U}$ by definition, it is enough to check that $Sub_{X,U}\subset SubQ_{X,U}$ to conclude. But given $S = \Sigma_XP$ in $Sub_{X,U}$ we have that:
\[\Sigma_XP = (\Sigma_XP)/L\]
where:
\[L((x,_),(y,_))= (x=_Xy)\]
and since $x=_Xy$ is assumed to be in $U$ we conclude.

\item Stability by quotient by equivalence relation with value in $U$ is clear.

\item Now we want to check stability under finite limits.

First we check that $U\subset SubQ_{X,U}$. Indeed assume $P\in U$, then with $L$ the trivial relation we have:
\[(X\times P) / L = \propTrunc{X\times P} = P\]
as $\propTrunc{X} = 1$ so that since $\top\in U$ we conclude $P\in SubQ_{X,U}$.

This means that $SubQ_{X,U}$ is stable by identity type, and that $1\in SubQ_{X,U}$.

All that is left is to check stability under $\Sigma$. Assume $S: SubQ_{X,U}$ and $T:S\to SubQ_{X,U}$. Through the fact that $S$ is covered by a type in $Sub_{X,U}$ and local choice for $Sub_{X,U}$ we merely get $S':Sub_{X,U}$, say $S'=\Sigma_XP$ and a surjective map:
\[f:S'\to S\]
such that for all $x:\sum_XP$ we have:
\[T(f(x)) = (\Sigma_XP_x)/R_x\]
so we get a surjective map:
\[\sum_{x:X}\sum_{P(x)}(\Sigma_X P_x)/R_x  \to \sum_ST\]
Then the identity types in $\sum_ST$ are in $U$ as $U$ is stable by $\Sigma$, so it is enough to check that:
\[\sum_{x:X}\sum_{P(x)}(\Sigma_X P_x)/R_x\]
is in $SubQ_{X,U}$ to conclude, as we can then apply the previous bullet-point. But this type is equivalent to:
\[\left(\sum_{(x,x'):X\times X}\Sigma_{P(x)}P_x(x')\right)/ L\]
where:
\[L((x,x'),(y,y')) =\sum_{x=y} R_y(x',y') \]
which is in $SubQ_{X,U}$ as $U$ is stable by $\Sigma$, $x=y$ in in $U$ and $X\times X = X$.
\end{itemize}
\end{proof}

\begin{proposition}\label{coproducts-sub-quo}
Assume $(X,U)$ a subquotient pre-system such that $\bot\in U$ and $X+X = X$. Then $SubQ_{X,U}$ is stable by finite coproducts.
\end{proposition}

\begin{proof}
We have that:
\[\bot = (X\times \bot) / L\]
where $L$ is the unique such equivalence relation. Since $\bot\in U$ we conclude that $\bot\in SubQ_{X,U}$.

Given $S$ and $S'$ in $SubQ_{X,U}$, say:
\[S = (\sigma_XP)/R\]
\[S' = (\sigma_XP')/R'\]
Then we have that:
\[S+S' = \left(\sum_{X+X}[P,P']\right) L\]
where:
\[[P,P'](l(x)) = P(x)\]
\[[P,P'](r(x)) = P'(x)\]
and:
\[L(l(x),l(y)) = R(x,y)\]
\[L(l(x),r(y)) = \bot\]
\[L(r(x),l(y)) = \bot\]
\[L(r(x),r(y)) = R'(x,y)\]
Since $\bot\in U$ and $X+X=X$ we conclude that $S+S'$ is in $SubQ_{X,U}$.
\end{proof}

\begin{proposition}\label{prop-sub-quo}
Assume $(X,U)$ a subquotient pre-system such that $\top\in U$ and for all $S\in Sub_{X,U}$ we have that $\propTrunc{S}\in U$. Then any proposition in $SubQ_{X,U}$ is in $U$. 
\end{proposition}

\begin{proof}
If we have a proposition $S$ in $SubQ_{X,U}$, say:
\[S = (\Sigma_XP)/R\]
then:
\[S = \propTrunc{S} = \propTrunc{\Sigma_XP}\]
and we can conclude.
\end{proof}

\begin{definition}
A subquotient system is a subquotient pre-system obeying the hypothesis of \cref{lex-sub-pro}, \cref{coproducts-sub-quo} and \cref{prop-sub-quo}.
\end{definition}

We just pack all this up in one theorem:

\begin{theorem}\label{stabitity-sub-quo}
Let $(X,U)$ be a subquotient system, then $SubQ_{X,U}$ has the following:
\begin{itemize}
\item Stability under finite limits.
\item Stability under finite coproducts.
\item Stability under quotient by equivalence relations.
\item Local choice.
\end{itemize}
\end{theorem}

We have two main examples in mind.

\begin{example}
The subquotient pre-system $St = (2^\N,\mathrm{Closed})$ is a quotient system. We have that $Sub_{St}$ is the type of stone spaces, and $CHaus = SubQ_{St}$ the type of compact Haussdorf spaces.

Closed propositions are stable by $\Sigma$. TODO 

We also need that for any stone space $S$ we have that $\propTrunc{S}$ is a closed proposition. TODO
\end{example}

\begin{example}
The subquotient pre-system $Od = (\N,\mathrm{Open})$ is a quotient system. We have that $ODisc = SubQ_{Od}$ the type of so-called overtly discrete types.

A key observation is that open propositions are stable by countable disjunctions.

This means open propositions are stable by $\sum$ because we can assume:
\[P = \Sigma_{n:\N} A(n)\]
with $A(n)$ decidable and:
\[Q:P \to \mathrm{Open}\]
Then we have that:
\[\Sigma_PQ = \exists(n:\N).\ \Sigma_{A(n)} Q(n)\]
which is open as $\Sigma_{A(n)} Q(n)$ is open for all $n$, as $A(n)$ is decidable.

Types in $Sub_{Od}$ even have full choice because both $\N$ and decidable propositions have full choice.
\end{example}

So both $ODisc$ and $CHaus$ enjoys the conclusion of \cref{stabitity-sub-quo}.

\rednote{Active, we should have a remark on CHaus and ODisc being stronger notions 
than they are in general synthetic topology}
\begin{remark}
  In Synthetic topology, the notion of overtness is dual to that of compactness. 
  There, one calls a space $X$ overt iff for every open $U\subseteq X$ we have that 
  $\exists_{x:X} U(x)$ is an open proposition. 
  Note that all countable types are overt, as $\exists_{x:X} U(x)$ is a countable disjunction 
  of open propositions whenever $X$ is countable.
%
  Furthermore, in synthetic topology, discreteness means that equality is open, 
  and as equality in $\mathbb N$ is decidable, it is also open. 
%
  Our notion of overtly discrete is thus stronger than the synthetic notion of being overt and discrete. 
  Similarly, our notion of compact Hausdorff is stronger than it's synthetic variant. 
\end{remark}


\subsection{Tychonov}

THERE IS A MISTAKE IN THE PROOF.

\begin{proposition}\label{tychonov}
Assume $(X,U)$ and $(Y,C)$ two subquotient system such that:
\begin{itemize}
\item $S\to Y$ is in $SubQ_{Y,C}$ for all $S:Sub_{X,U}$.
\item If $P\in U$ and $Q:P\to C$ then $\prod_{p:P}Q(p) \in C$.
\item If $Q:X\to C$ then $\prod_{x:X}Q(x) \in C$.
\end{itemize}
Then we have the following:
\begin{itemize}
\item If $S:SubQ_{X,U}$ and $T:S\to SubQ_{Y,C}$, then:
\[\prod_{s:S}T(s) \in SubQ_{Y,C}\]
\end{itemize}
\end{proposition}

\begin{proof}
Note that for $S':Sub_{X,U}$ and $Q:S'\to C$ we have that:
\[\prod_{S'}Q\]
is in $C$.

Now we use local choice to get $S':Sub_{X,U}$ with a surjective map:
\[f:S'\to S\]
such that for all $s:S'$ we have:
\[T(f(s)) = (\Sigma_YU_s)/R_s\]

Then the map:
\[\prod_ST \to \prod_{s:S'}(\Sigma_YU_s)/R_s\]
is an embedding, its fiber over $\alpha$ is:
\[\prod_{s,t:S'} \prod_{f(s) =_S f(t)} \alpha(s) = \alpha(t)\]
which is in $C$ by the hypothesis. Therefore it is enough to prove that:
\[\prod_{s:S'}(\Sigma_YP_s)/R_s\]
is in $SubQ_{Y,C}$. 

(NOT TRUE.) But this type is the quotient of:
\[\prod_{s:S'}(\Sigma_YP_s)\]
by:
\[L(\alpha,\beta) = \prod_{s:S'} R_s(\alpha(s),\beta(s))\]
which is in $C$, therefore it is enough to to prove that:
\[\prod_{s:S'}(\Sigma_YP_s)\]
is in $SubQ_{Y,C}$.

But this type is equivalent to:
\[\sum_{f:S'\to Y} \prod_{s:S'}P_s(f(s))\]
Since $\prod_{s:S'}P_s(f(s))$ is in $C$, it is enough to prove that:
\[S'\to Y\]
is in $SubQ_{Y,C}$. But this is one of the hypothesis.
\end{proof}

\begin{definition}
Two subquotient systems $A,B$ are called dual if both $(A,B)$ and $(B,A)$ satisfy the hypothesis of \cref{tychonov}.
\end{definition}

\begin{example}
We have that $St = (2^\N,\mathrm{Closed})$ and $Od = (\N,\mathrm{Open})$ are dual quotient systems.

\begin{itemize}
\item We need that if $P$ open and $Q:P\to \mathrm{Closed}$, then $\Sigma_PQ$ is closed. Assume $Q=\Sigma_{n:\N}A(n)$ with $A(n)$ decidable, since open propositions have choice we can assume for $n:\N$ such that $A(n)$ that $Q(n) = \forall_{k:\N} B_n(k)$ with $B_n(k)$ decidable. Then:
\[\Sigma_PQ = \prod_{n,k:\N} \prod_{A(n)} B_n(k) \]
which is indeed closed.

It is clear that closed propositions are closed by countable products.

$\Sigma_\N P\to 2^\N$ is compact Hausdorff? Yes indeed, it is even Stone because it is equivalent to:
\[\prod_{k,n:\N} 2^{P(n)}\]
and $2^{P(n)}$ is Stone as $P(n)$ is open, indeed:
\[(\Sigma_\N A)\to 2\] 
for $A$ decidable is a countable product of Stone space, as $2^A$ is Stone for $A$ decidable.

\item First we check that given $S$ Stone, we have that:
\[S\to \N\]
is overtly discrete. Indeed identity types in $S\to \N$ are closed and there is a surjection from fundamental systems of idempotent in $2^S$ to $S\to \N$, so it is enough to prove that the type fundamental systems of idempotent in a c.p. algebra is overtly discrete. To have this it is enough to prove that countably presented algebra are overtly discrete. TODO
\end{itemize}
\end{example}

When applying \cref{tychonov} to $ODisc$ and $CHaus$ we get Tychonov theorem and its dual.


\subsection{Factorisation}

\begin{proposition}\label{factorisation-subquotient}
Assume given dual subquotient systems $(X,C)$ and $(Y,U)$. such that
such that:
\begin{itemize}
\item Given $S:Sub_{X,U}$ any map:
\[S\to \N\]
merely factors through a finite type.
\item Given $P\in C$ and $Q\in U$ we have that:
\[(P\to Q) \to (\neg P \lor Q)\]
\end{itemize}
Then for $S:SubQ_{X,C}$ and $T:SubQ_{Y,U}$, any map:
\[S\to T\]
merely factors through a finite type.
\end{proposition}

\begin{proof}
We proceed in three steps:
\begin{enumerate}[(i)]
\item First we show the factorisation of any map:
\[S\to T\]
for $S:Sub_{X,C}$ and $T:Sub_{Y,U}$. Indeed we then we merely have $Q:Y\to U$ such that:
\[T = \Sigma_YQ\]
so the map:
\[S\to \Sigma_YQ\]
induces a map:
\[S\to Y\]
which we factor as:
\[S\to \mathrm{Fin}(n) \to Y\]
This gives a factorisation of the starting map as follows:
\[S\to \Sigma_{\mathrm{Fin}(n)} Q \to T\]
Then for any $k:\mathrm{Fin}(n)$ we define:
\[S_k \subset S\]
\[s\in S_k := (f(s) = k)\]
which is decidable and therefore in $C$, so that $\propTrunc{S_k}$ is in $C$. We have that:
\[\propTrunc{S_k} \to Q(k)\]
therefore we have
\[\neg S_k \lor Q(k)\]
Now we just need to remove the $k$ such that $\neg S_k$ to get a factorisation:
\[S\to \mathrm{Fin}(l) \to \Sigma_{\mathrm{Fin}(n)}Q\]
\item We show how to factor any map:
\[S\to \mathrm{Fin}(n)/R\]
where $S:SubQ_{X,U}$ and $R$ in an equivalence relation in $U$. We do this by induction on $n$, if $n=0,1$ it is clear. Using local choice we get:
\begin{center}
\begin{tikzcd}
S'\ar[d,"p"]\ar[r,"f"] & \mathrm{Fin}(n)\ar[d] \\
S\ar[r,"g"] & \mathrm{Fin}(n)/R
\end{tikzcd}
\end{center}
Then for all $k:\mathrm{Fin}(n)$ we define:
\[S'_k \subset S'\]
\[s'\in S'_k := (f(s) = k)\]
which is a decidable cover of $S'$. Then we use:
\[S_k = p(S'_k)\]
which is a cover of $S$ such that:
\[(S_k\cap S_l) \to R(k,l)\]
since $g$ restricted to an $S_i$ has value $[i]$. But since $\propTrunc{S_k\cap S_l}$ is in $C$, this means that:
\[\neg(S_k\cap S_l) \lor R(k,l)\]
If for all $k\not=l$ we have $\neg(S_k\cap S_l)$ then we can conclude by factoring through $\mathrm{Fin}(n)$, otherwise we have $R(k,l)$ for some $k\not=l$ and we have that $\mathrm{Fin}(n)/R$ is equivalent to $\mathrm{Fin}(n-1)/R$ where we have removed $l$. We conclude by induction on $n$.
\item  Now we show how to factor any map:
\[S\to T\]
for $S:SubQ_{X,C}$ and $T:SubQ_{Y,U}$. There is $Q:Y\to U$ and $R$ an $U$-valued equivalence relation on $\Sigma_YQ$ such that:
\[T = (\Sigma_YQ)/R\]
Using local choice we have that:
\begin{center}
\begin{tikzcd}
S'\ar[d]\ar[r] & \Sigma_YQ\ar[d] \\
S\ar[r] & (\Sigma_YQ)/R
\end{tikzcd}
\end{center}
and by factoring the top map using (i), we can get a factorisation:
\[S\to \mathrm{Fin}(n)/R\to (\Sigma_YQ)/R\]
and we conclude using (ii).
\end{enumerate}
\end{proof}

\begin{definition}
Dual subquotient systems $(X,C)$ and $(Y,U)$ obeying the hypothesis of \cref{factorisation-subquotient} are called factorial.
\end{definition}

\begin{remark}
We have that $St = (2^\N,\mathrm{Closed})$ and $Od = (\N,\mathrm{Open})$ are factorial. We need to check the following:
\begin{itemize}
\item Given a stone space $S$, any map $S\to\N$ merely factors through a finite type. This is known.
\item Given $P$ closed and $Q$ open, we have that:
\[(P\to Q) = \neg P \lor Q\]
We just note that as $\neg P \lor Q$ is open it is $\neg\neg$-stable, and the rest is just intuitionistic logic.
\end{itemize}
From this we know that for $S$ compact Hausdorff and $T$ overtly discrete, any map:
\[S\to T\]
merely factors through a finite type. 
\end{remark}

\subsection{Scott continuity}




%
%\appendix
%\section{Appendix}
%\subsection*{Rank of matrices}

\begin{definition}
A matrix is said to have rank $\leq n$ if all its $n+1$-minors are zero. It is said to have rank $n$ if it has rank $\leq n$ and does not have rank $\leq n-1$.
\end{definition}

Having a rank is a property of matrices, as a rank function defined on all matrices would allow to e.g. decide if an $r:R$ is invertible.

\begin{lemma}\label{rank-bloc-matrix}
Assume given a matrix $M$ of rank $n$ decomposed into blocks:
\[M = \begin{pmatrix}
P & Q  \\
R & S \\
\end{pmatrix}\]
Such that $P$ is square of size $n$ and invertible. Then we have:
\[S = RP^{-1}Q\]
\end{lemma}

\begin{proof}
By columns manipulation the matrix is equivalent to:
\[M = \begin{pmatrix}
P & Q  \\
0 & S - RP^{-1}Q \\
\end{pmatrix}\]
but equivalent matrices have the same rank so $S=RP^{-1}Q$.
\end{proof}

\begin{lemma}\label{rank-equivalent-definitions}
If a linear map $R^m \to R^n$ given by multiplication with $M$
has finite free kernel of rank $k$, then $M$ has rank $m-k$.
\end{lemma}

\begin{proof}
  Let $a_1,\dots,a_{k}$ be a basis for the kernel of $M$ in $R^m$, which we complete into a basis of $R^m$ via $b_{k+1},\dots,b_m$.
  By completing $Mb_{k+1},\dots, Mb_m$ to a basis of $R^n$, we get a basis where $M$ is written as:
\[\begin{pmatrix}
I_{m-k} & 0  \\
0 & 0 \\
\end{pmatrix}\]
so that $M$ has rank $m-k$.
\end{proof}

%\begin{definition}
%Two matrices $M,N$ are said equivalent if there are invertible matrices $P,Q$ such that $M = PNQ$.
%\end{definition}

%It is clear that equivalent matrices have the same rank.

%\begin{lemma}\label{rank-equivalent-definitions}
%Assume given a matrix:
%\[M : R^m\to R^k\]
%Then the following are equivalent:
%\begin{enumerate}[(i)]
%\item $M$ has rank $n$.
%\item The kernel of $M$ is equivalent to $R^{m-n}$.
%\item The image of $M$ is equivalent to $R^n$.
%\item $M$ is equivalent to the bloc matrix:
%\[\begin{pmatrix}
%I_n & (0)  \\
%(0) & (0) \\
%\end{pmatrix}\]
%\end{enumerate}
%\end{lemma}

%\begin{proof}
%\end{proof}

%
\printbibliography
%
\end{document}
