\subsection{Principles of omniscience}
In constructive mathematics, we do not assume the law of excluded middle (LEM).
There are some principles called principles of omniscience that are weaker than LEM, which can be used to describe 
how close a logical system is to satisfying LEM.
References on these principles include \cite{HannesDiener, ReverseMathsBishop}.
In this section, we will show that two of them (MP and LLPO) hold, 
and one (WLPO) fails in our system.

\begin{theorem}[The negation of the weak lesser principle of omniscience ($\neg$WLPO)]\label{NotWLPO}
  \[
    \neg \forall_{\alpha:2^\N} 
    ((\forall_{n:\N} \alpha_n = 0 ) \vee \neg (\forall_{n:\N} \alpha_n = 0))
  \]
%  We cannot decide for general $\alpha:2^\N$, whether $\forall_{n:\mathbb N} \alpha(n) = 0$.
%  It is not the case that the statement %There is no method which given $\alpha:2^\mathbb N$ decides whether 
%  $\forall_{n:\mathbb N} \alpha(n) = 0$ is decidable for general $\alpha:2^\mathbb N$. 
\end{theorem}
\begin{proof}
%  Such a decision method is a function 
  Assume $f:2^\mathbb N \to 2$ such that 
  $f(\alpha) = 0$ if and only if $\forall_{n:\mathbb N} \alpha_n= 0$. 
  By \Cref{AxStoneDuality}, there is some $c:2[\N]$ with 
  $f(\alpha) = 0 \leftrightarrow \alpha(c) = 0$. 
  There exists $k:\N$ such that $c$ is expressed the generators $(g_n)_{n\leq k}$. 
  Now consider $\beta,\gamma:2^\N$ given by 
  $\beta(g_n) = 0$ for all $n:\mathbb N$ and
  $\gamma(g_n) = 0$ if and only if $n\leq k$. 
  As $\beta, \gamma$ are equal on $(g_n)_{n\leq k}$, we have $\beta(c) = \gamma(c)$. 
  However, $f(\beta) = 0$ and $f(\gamma) = 1$, giving a contradiction. 
%  We thus have a contradiction, thus a decision method as required doesn't exist. 
\end{proof}

\begin{theorem}
  For $\alpha:\Noo$, we have that 
  \[
    (\neg (\forall_{n:\mathbb N} \alpha_n= 0)) \to \Sigma_{n:\mathbb N} \alpha_n= 1
  \]
\end{theorem}
\begin{proof}
  By \Cref{ClosedPropAsSpectrum}, we have that $\neg(\forall_{n:\N} \alpha_n = 0)$ implies that 
  $Sp(2/(\alpha_n)_{n:\N}$ is empty. 
%  We will show that the spectrum of $2/(\alpha(n))_{n:\N}$ is empty. 
%  Suppose $x:2\to 2$, if  $x(\alpha(n)) = 0$, we get $\alpha(n) \neq 1$. 
%  Thus if $\neg (\forall_{n:\N} \alpha(n) = 0$, we have $\neg Sp(2/(\alpha(n))_{n:\N})$.
  Hence $2/(\alpha_n)_{n:\N}$ is trivial by \Cref{SpectrumEmptyIff01Equal}. 
  Then there exists $k:\N$ such that $\bigvee_{i\leq k} \alpha_i = 1$. 
  As $\alpha_i = 1$ for at most one $i:\N$, 
  there exists an unique $n:\mathbb N$ with $\alpha_n = 1$. 
%  Assume $\neg (\forall_{n:\mathbb N} \alpha (n)= 0)$.
%  It is sufficient to show that $2/\{\alpha(n)|n\in\N\}$ is the trivial Boolean algebra. 
%  It will then follow that there is a finite subset $N_0\subseteq \N$ 
%  with $\bigvee_{i:N_0} \alpha(i) = 1$.
%  As $\alpha(i) \in \{0,1\}$ and $\alpha(i) = 1$ for at most one $i$, it then follows that 
%  there exists an unique $n\in\mathbb N$ with $\alpha(n) = 1$. 
%%
%  To show that $2/\{\alpha(n)|n\in\N\}$ is trivial, we will show it has an empty spectrum. 
%  Suppose $x: 2 \to 2$ is such that $x(\alpha(n)) = 0$ for every $n:\N$. 
%  As $x(1) = 1$, we must have for every $n:\N$ that $\alpha(n) \neq 1$. 
%  But then $\alpha(n) = 0$, contradicting our assumption. 
%  We get a contradicition and there thus there are no points in the spectrum of $2/\{\alpha(n)|n\in\N\}$ as required. 
\end{proof}

\begin{corollary}[Markov's principle (MP)]\label{MarkovPrinciple}
  For $\alpha:2^\mathbb N$, we have that 
  \[
    (\neg (\forall_{n:\mathbb N} \alpha_n= 0)) \to \Sigma_{n:\mathbb N} \alpha_n= 1
  \]
\end{corollary}
\begin{proof}
  Given $\alpha:2^\mathbb N$, consider the sequence $\alpha':\Noo$ satisfying $\alpha'_n = 1$ if and only if
  $n$ is minimal with $\alpha_n = 1$. Then apply the above theorem.
\end{proof}

\begin{theorem}[The lesser limited principle of omniscience (LLPO)]\label{LLPO}
  For $\alpha:\N_\infty$, 
  we have: 
  \[\label{eqnLLPO}
    \forall_{k:\N} \alpha_{2k} = 0  \vee \forall_{k:\N} \alpha_{2k+1} = 0
  \]
\end{theorem}
\begin{proof}
%
%  We first will define a map $f:B_\infty \to B_\infty \times B_\infty$. 
%  Because of \Cref{rmkMorphismsOutOfQuotient}, it is sufficient to define $f$ on $(p_n)_{n:\N}$ with 
%  $f(p_n) \wedge f(p_m) = (0,0)$ for $n\neq m$. 
%  To define $f(p_n)$, we use a case distinction on whether $n$ is odd or even. 
  Define $f:B_\infty \to B_\infty \times B_\infty$ on generators as follows:
  \[\label{eqnLLPOProofMap}
    f(g_n) =\begin{cases}
      (g_k,0) \text{ if } n = 2k\\
      (0,g_k) \text{ if } n = 2k+1\\
    \end{cases}
  \]
  Note that $f$ is well-defined as map in $\Boole$ as 
  $f(g_n) \wedge f(g_m) = 0$ whenever $m\neq n$. 
 % , can make a case distinction on parity. 
%  By making a case distinction on $n,m$ being odd or even, 
%  we can see that 
%  $f(p_n) \wedge f(p_m) = (0,0)$ when $n\neq m$, thus $f$ is well-defined. 
  We claim that $f$ is injective. 
  If $I\subseteq \N$, write 
  $ I_0 =\{k\ |\ 2k \in I\}, 
    I_1 =\{k\ |\ 2k+1 \in I\}
  $.
  Recall that any $x:B_\infty$ is of the form 
  $\bigvee_{i\in I} g_i$ or $\bigwedge_{i\in I} \neg g_i$ for some finite set $I$. 
  \begin{itemize}
    \item If $x = \bigvee_{i\in I} g_i$, then 
      $f(x) = (\bigvee_{i\in I_0}g_{i}, \bigvee_{i\in I_1}g_i)$. 
      So if $f(x) = 0$, then $I_0=I_1 = I = \emptyset$ and $x = 0$. 
    \item Suppose 
      $x = \bigwedge_{i\in I} \neg g_i$.
      Then $f(x) = (\bigwedge_{i\in I_0} \neg g_i, \bigwedge_{i\in I_1} \neg g_i)$, 
      so $f(x) \neq 0$. 
  \end{itemize}
%  In both cases, we conclude $x=0$, thus $f$ is injective. 
  By \Cref{SurjectionsAreFormalSurjections},
%  \Cref{FormalSurjectionsAreSurjections}, 
  $f$ corresponds to a surjection 
  $s:\Noo + \Noo \to \Noo$.
  Thus for $\alpha : \Noo$, 
  there exists some $x:\Noo + \Noo$ such that $s(x) = \alpha$. 
  If $x = inl(\beta)$, 
  for any $k:\N$, we have that 
\[\alpha_{2k+1} = s(x)_{2k+1} = x(f(g_{2k+1})) = inl(\beta) (0,g_k)  = \beta(0) = 0.\]
  Similarly, if $x = inr(\beta)$, we have $\alpha_{2k} = 0$ for all $k:\N$. 
  %Thus \Cref{eqnLLPO} holds for $\alpha$.% as required. 
\end{proof}
%As the following shows, our use of \Cref{SurjectionsAreFormalSurjections} was non-trivial: 
%The use of \Cref{FormalSurjectionsAreSurjections}, and hence of propositional completeness, 
%was helpful in the above proof, as the following shows:
The surjection $s:\Noo + \Noo \to \Noo$ as above does not have a section 
as the following shows:
\begin{lemma}
  The function $f$ defined above does not have a retraction. 
\end{lemma}
\begin{proof}
  Suppose $r:B_\infty \times B_\infty \to B_\infty$ is a retraction of $f$. 
  Then $r(0,g_k) = g_{2k+1}$ and $r(g_k,0) = g_{2k}$. 
  %Note that $r(0,1):B_\infty$ is expressible using only finitely many generators $(g_n)_{n\leq N}$.
  Note that $r(0,1) \geq r(0,g_k) = g_{2k+1}$ for all $k:\N$. 
  As a consequence, $r(0,1)$ is of the form $\bigwedge_{i\in I} \neg g_i$ for some finite set $I$.
  %, and by \Cref{BinftyTermsWriting}, 
  %$r(0,1)$ corresponds to a cofinite subset of $\N$. % = \bigwedge_{i:I_0} \neg p_i$, where $i\leq N$ for $i\in I_0$. 
  By similar reasoning so is $r(1,0)$. % corresponds to a cofinite subset of $\N$. 
%  But the intersection of cofinite subsets is cofinite, while 
  But this contradicts:
  \[r(0,1) \wedge r(1,0) = r( (1,0) \wedge (0,1)) = r(0,0) = 0.\]
%  which gives a contradiction. 
  Thus no retraction exists. 
\end{proof}


%We finish with an equivalent formulation of LLPO:
%
%
%\begin{lemma}\label{corAlternativeLLPO}
%  Let $(\phi_n)_{n:\N}, (\psi_m)_{m:\N}$ be families of decidable propositions indexed over $\N$.
%  We then have 
%  \begin{equation}
%    (\forall_{n:\N} \forall_{m:\N} (\phi_n \vee \psi_m) )
%    \leftrightarrow
%    ((\forall_{n:\N} \phi_n) \vee (\forall_{m:\N} \psi_m) )
%  \end{equation}
%\end{lemma}
%\begin{proof}
%  See \cite{HannesDiener, ReverseMathsBishop}
%\end{proof}
%\begin{proof}
%  Note that the implication from right to left in the above equation always holds.
%  Assume that for all $m,n:\mathbb N$ we have $\phi_n\vee \psi_m$ 
%  Consider the sequence $\alpha:2^\mathbb N$ where $\alpha(2n) = 0$ iff $\phi_n$ and 
%  $\alpha(2m+1) = 0$ iff $\psi_m$. 
%  Let $\beta:\Noo$ be such that $\beta(i) = 1$ iff $i$ is minimal with $\alpha(i) = 1$
%  By LLPO, we have that 
%  $\beta$ is $0$ on all odd entries or on all even entries. 
%  Suppose that $\beta$ hits $0$ on all odd entries. 
%  We will show $\psi_m$ for all $m:\N$. 
%  As $\beta(2m+1) = 0$, there are two options:
%  \begin{itemize}
%    	\item If $\alpha(l)=0$ for all $l\leq 2m+1$. Then in particular $\alpha(2m+1)=0$ and $\psi_m$ holds.
%	\item Otherwise there is some $l<2m+1$ with $\beta(l) = 1$. 
%  As $\beta$ hits $0$ on odd entries, $l$ is even. 
%  So $\alpha(2n) = 1$ for $n = \frac{l}2$, meaning that $\neg \phi_n$. 
%  By assumption, $\phi_n \vee \psi_m$ holds, hence $\psi_m$ must hold. 
%  Thus for all $m:\N$, we have $\psi_m$ if $\beta$ hits $0$ on all odd entries. 
%  By a symmetric argument, if $\beta$ hits $0$ on all even entries, we have $\phi_n$ for all $n:\N$. 
%  We conclude that 
%  $((\forall_{n:\N} \phi_n) \vee (\forall_{m:\N} \psi_m) )$ 
%  as required. 
%  \end{itemize}
%\end{proof}
%
%\begin{remark}
%Note that the above statement implies LLPO as $\alpha(2n) =0 \vee \alpha(2m+1) =0$ for all $n,m:\mathbb N$ if $\alpha:\Noo$. 
%\end{remark}
