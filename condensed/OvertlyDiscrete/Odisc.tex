%\subsection{Countably presented algebras as sequential colimits}\label{secBooleAsColimits}
\begin{definition}
  We define a type $E$ to be Overtly Discrete iff it is the colimit of an $\N$-indexed sequence of finite sets. 
\end{definition} 
%
%\begin{definition}
%  A sequence in a category is a diagram of shape $\N$, 
%  where $\N$ carries the natural structure of a poset. 
%\end{definition}
\begin{lemma}
  Every countably presented Boolean algebra is overtly discrete.
\end{lemma}
%\begin{lemma}\label{lemProFinitePresentation}
%  For every countably presented Boolean algebra $B$
%  there merely exists a sequence of finitely presented Boolean algebras 
%  whose colimit in the category of Boolean algebras is $B$. 
%\end{lemma}
\begin{proof}
  Consider $\langle G \rangle \langle\langle R \rangle\rangle$ a countable presentation of a Boolean algebra $B$. 
  We will show there exists a diagram of shape $\N$ taking values in Boolean algebras 
  with $\langle G\rangle / R$ as the colimit.
  \paragraph{The diagram}
  Let $R_n$ be the first $n$ terms in $R$. 
  Note that each of these finitely many terms uses only finitely many symbols from $G$.
  Let $G_n$ be the finite set of terms used in $R_n$, unioned with the finite set of the first $n$ elements of $G$. 
  Define for each $n\in\N$ the finitely presented Boolean algebra $B_n = \langle G_n \rangle  \langle R_n \rangle$. 
  If $n\leq m$, then \Cref{rmkMorphismsOutOfQuotient} gives us a map $B_n \to B_m$ 
  as $G_n \subseteq G_{n+1}$ and $R_n \subseteq R_{n+1}$. 
  Thus $(B_n)_{n\in \N}$ gives us a diagram of shape $\N$
  with values in finitely presented algebras. 

  \paragraph{The colimit}
  As $G_n\subseteq G$ and $R_n \subseteq R$, 
  \Cref{rmkMorphismsOutOfQuotient} also gives us a map $B_n\to \langle G \rangle \langle R \rangle$. 
  We claim the resulting cocone is a colimit. 

  Suppose we have a cocone $C$ on the diagram $(B_n)_{n\in\N}$. 
  We need to show that there exists a map $\langle G \rangle / R\to C$ and
  we need to show this map is unique as map between cocones. 
  \begin{itemize}
    \item To show there exists a map $\langle G \rangle / R \to C$, 
      we use remark \Cref{rmkMorphismsOutOfQuotient} again. 
      Let $g\in G$ be the $n$'th element of $G$, 
      note that $g\in G_n$, and consider the image of $g$ under the map $B_n \to C$. 
      This procedure defines a function from $G$ to the underlying set of $C$. 
      Let $\phi \in R$ be the $n$'th element of $R$, 
      note that $\phi \in R_n$, and the map $B_n \to C$ must send $\phi$ to $0$. 
      Thus the function from $G$ to the underlying set of $C$ also sends $\phi$ to $0$. 
      This thus defines a map $\langle G \rangle / R \to C$. 
    \item To show uniqueness, consider that any map of cocones $\langle G \rangle / \langle R \rangle \to C$ 
      must take the same values on all $g\in G_n$ for all $n\in\N$. 
      Now all $g\in G$ occur in some $G_n$, so any map of cocones $\langle G \rangle /  \langle R \rangle \to C$ 
      takes the same values for all $g\in G$. 
      \Cref{rmkMorphismsOutOfQuotient} now tell us that these values uniquely determine the map. 
  \end{itemize}
\end{proof}
\begin{remark}
  Conversely, any colimit of a sequence of finite Boolean algebras 
  is a countably presented Boolean algebra with 
  as underlying sets of generators and relations the countable union of the finite sets of 
  generators and relations, which are both countable. 
\end{remark}
\begin{lemma}\label{lemFinitelyPresentedBACompact}
  For any finitely presented Boolean algebra $A$,
  and any sequence $(B_n)_{n:\N}$ of Boolean algebras with colimit $B$
  we have that the set $B^A$ is the colimit of the sequence of sets $(B_n^A)_{n:\N}$. 
\end{lemma}  
\begin{proof}
  First note that $B^A$ forms a cocone on $(B_n^A)_{n:\N}$ 
  because any map $A \to B_n$ induces a map $A \to B$. 
  Let $C$ be a cocone on $(B_n^A)_{n:\N}$. 
  We shall show there is an unique morphism of cocones $B^A \to C$. 
  \begin{itemize}
    \item For existence, let $f:B^A$. 
      As $A$ is finitely presented, we write $A = \langle G \rangle / \langle R \rangle$ with $G$ finite.
      By \Cref{rmkMorphismsOutOfQuotient}, $f$ is uniquely determined by it's values on $g\in G$. 
      As $G$ is finite, so is it's image $f(G)\subseteq B$. 
      But any finite subset of $B$ already occurs in $B_n$ for some $n\in\N$. 
      Consequently, the image of $f$ is already contained in some $B_n$. 
      Thus there is some $f_n:(B_n^A)$ such that postcomposing 
      $f_n$ with the map $B_n \to B$ gives back $f$. 
      The image of $f_n$ under the map $(B_n^A) \to C$ is how we define the image of $f$. 
      This is well-defined by the cocone conditions on $C$. 
    \item 
      For uniqueness, by function extensionality maps $B^A \to C$ are uniquely determined by their values on 
      $f:B^A$. By the above, the value of $f$ is uniquely determined by it's value on $B_n$ for 
      any $n$ with the image of $f$ in $B_n$. Thus there is at most one morphism of cocones $B^A \to C$. 
  \end{itemize}
\end{proof}
\begin{remark}\label{rmkEqualityColimit}
  In the above proof, we used that any element $b\in B$ already occurs in some $B_n$. 
  However, please note that it is not necessarily the case that it occurs uniquely in $B_n$, 
  there might be multiple elements in $B_n$ which can all be sent to $b$ in the end. 

  In case our sequence comes from the construction in \Cref{lemProFinitePresentation}, 
  we can see that whenever there are two elements in 
  $B_n$ corresponding to $b\in B$, they will become equal in $B_m$ for some $m\geq n$. 
  The reason is that if $b \sim_{\langle R \rangle} c$, there is a finite subset $R_0 \subseteq R$ such that 
  $b\sim_{\langle R_0 \rangle} c$, which will occur in some $R_m$. 

  One could wonder whether this property holds for general colimits of sequences. 
  In general, if we assume $B$ is the colimit of an arbitrary sequence $(B_n)_{n:\N}$, 
  and there exist some $B_n$ with two elements corresponding to the same element in $B$, 
  Theorem 7.4 from \cite{SequentialColimitHoTT} says that there merely exists some $m\geq n$
  such that they are already equal in $B_m$. 
\end{remark}

%For our next lemma on this presentation of sequences we need the axiom of dependent choice. 
%\begin{axiomNum}[Dependent choice]\label{axDependentChoice}
%  Given a family of types $(E_n)_{n:\N}$ and 
%  a relation 
%  $R_n:E_n\rightarrow E_{n+1}\rightarrow {\mathcal U}$ such that
%  for all $n$ and $x:E_n$ there exists $y:E_{n+1}$ with $p:R_n~x~y$ 
%  then given $x_0:E_0$ there exists
%  $u:\Pi_{n:\N}E_n$ and $v:\Pi_{n:\N}R_n~(u~n)~(u~(n+1))$ and $u~0 = x_0$.
%\end{axiomNum}
\begin{lemma}[Using dependent choice]\label{lemDecompositionOfColimitMorphisms}
  Let $B,C$ be countably presented Boolean algebras, 
  and suppose we have a morphism $f:B\to C$.
  There exists sequences of finitely presented Boolean algebras 
  $(B_n)_{n:\N}, (C_n)_{n:\N}$ with colimits $B,C$ respectively
  and compatible maps of Boolean algebras $f_n:B_n \to C_n$, 
  such that $f$ is the induced morphism $B\to C$.
\end{lemma}
\begin{proof}
  Let $(B_n)_{n:\N}, (C_n)_{n:\N}$ be 
  sequences of finitely presented Boolean algebras with colimits $B$ and $C$. 
  We will take a subsequence of $(C_n)_{n:\N}$, using the axiom of dependent choice above. 

  Our family of types $E_k$ as in \Cref{axDependentChoice} 
  will be strictly increasing sequences $(n_i)_{i\leq k}$ of natural numbers together with a finite family of maps 
  $(f_i: B_{i} \to C_{n_i})_{i\leq k}$ such that
  for all $0\leq i<k$ the following diagram commutes:
  \begin{equation}\label{eqnDecompositionOfColimitMorphisms}
    \begin{tikzcd}
      B_{i} \arrow[r] \arrow[d, "f_i"]& B_{{i+1}} \arrow[r] \arrow[d,"f_{i+1}"]& B \arrow[d,"f"] \\
      C_{n_i} \arrow[r] & C_{n_{i+1}} \arrow[r] & C 
    \end{tikzcd}
  \end{equation}
  Our relation $R_k$ will tell whether the second sequence extends the first one. 
%
  By \Cref{lemFinitelyPresentedBACompact} 
  there exists some $n_0:\N$ 
  such that $B_0 \to B \to C$ factors as 
  \begin{equation}
    \begin{tikzcd}
      B_{0} \arrow[r] \arrow[d, "f_0"]& B \arrow[d,"f"] \\
      C_{n_0} \arrow[r] & C 
    \end{tikzcd}
  \end{equation}
  Because our goal is a proposition, we can untracate this existence to data. 
  This data will form our $x_0:E_0$. %from \Cref{axDependentChoice}. 
%
  Now suppose we have $(f_i: B_{i} \to C_{n_i})_{i\leq k}$ for some $k\geq 0$ 
  such that
  for all $0\leq i<k$ the diagram of \Cref{eqnDecompositionOfColimitMorphisms} commutes.
  We shall show that in this case there exists an $n_{k+1}, f_{k+1}$ 
  making the same diagram commute for $i = k$. 
  Consider $B_{{k}+1}\to B \to C$. By the same argument as for $B_0$, we have a factorization 
  \begin{equation}
    \begin{tikzcd}
    B_{k+1} \arrow[r]  \arrow[d,"f'_{k+1}"]& B \arrow[d,"f"]\\
    C_{n'_{k+1}} \arrow[r] & C
    \end{tikzcd}
  \end{equation}
  Note that we may assume $n'_{k+1} > n_k$.
  Note that it is not necessarily the case that 
  $f'_{k+1}$ is compatibly with $f_k$, meaning the left square in the following diagram needn't commute:
  \begin{equation}
    \begin{tikzcd}
      B_{k} \arrow[r] \arrow[d, "f_k"]& B_{{k+1}}  \arrow[r] \arrow[d,"f'_{k+1}"] & B \arrow[d,"f"] \\
      C_{n_k} \arrow[r] & C_{n'_{k+1}} \arrow[r]  & C 
    \end{tikzcd}
  \end{equation}
  However, both $f'_{k+1}, f_k$ induce the same map $B_{k} \to C$. 
  Recall by \Cref{rmkMorphismsOutOfQuotient} this map is induced by it's value on finitely many elements. 
  By \Cref{rmkEqualityColimit}, it follows there is an $n_{k+1} \geq {n'_{k+1}}$ 
  such that for $f_{k+1}$ the composition of $f'_{k+1}:B_{k+1} \to C_{n'_{k+1}}$ and 
  the map $C_{n'_{k+1}} \to C_{n_{k+1}}$, the following diagram does commute:
  \begin{equation}
    \begin{tikzcd}
      B_{k} \arrow[d,"f_k"]\arrow[r] & B_{{k+1}} \arrow[rd, "f_{k+1}"] \arrow[rr] & & B \arrow[d,"f"] \\
      C_{n_k} \arrow[r] & C_{n'_{k+1}} \arrow[r] & C_{n_{k+1}} \arrow[r] & C 
    \end{tikzcd}
  \end{equation}
  Now by dependent choice for the above $x_0, R_n, E_n$, we get a sequence $(f_i:B_i \to C_{n_i})$  for some 
  strictly increasing sequence $n_i$ of natural numbers. 
  Note that for such a sequence $(n_i)_{i:\N}$, 
  $(C_{n_i})_{i:\N}$ converges to $C$. Also $(B_i)_{i:\N}$ still converges to $B$. 
  Futhermore, by construction the map that sequence $f_i$ induces from $B \to C$ shares all values with $f$
  and thus is equal to $f$. 
  Thus our sequence $f_i$ is as required. 
\end{proof}
\begin{remark}\label{rmkEpiMonoFactorizationCommutes}
  For $f,(f_i)_{i:\N}$ as above, whenever $f_n(x) = 0$, we have $f_{n+1}(x \circ \iota_{n,n+1}) = 0$
  for $\iota_{n,n+1}$ the map $A_n \to A_{n+1}$. 
  By \Cref{rmkMorphismsOutOfQuotient}, $\iota_{n,n+1}$ induces a map $A_n/Ker(f_n)\to A_{n+1}/Ker(f_{n+1})$. 
  This induced map is such that the following diagram commutes:
  \begin{equation}\begin{tikzcd}
    A_n \arrow[d, two heads] \arrow[r, "\iota_{n,n+1}"] & A_{n+1} \arrow[d,two heads]\\
    A_n /Ker(f_n) \arrow[d,hook] \arrow[r] & A_{n+1} /Ker(f_{n+1}) \arrow[d,hook] \\
    B_n \arrow[r] & B_{n+1}
  \end{tikzcd}\end{equation}  
  As the induced maps be epi's / mono's  is epi /mono, the colimit of the sequence 
  $A_n / Ker(f_n)$ will fit into an epi-mono factorization of $f$ and thus be iso to $A/Ker(f)$. 
  Thus the epi-mono factorization of the colimit is the colimit of the epi-mono factorizations. 
\end{remark}
\begin{remark}\label{rmkIsoEpiMonoMapColimit}
  Whenever $f:B \to C$ is an iso, any sequence with $B$ as colimit, also has $C$ as colimit. 
  Thus any iso can be represented this way as sequence of iso's. 
  Conversely, any sequence of isomorphisms induces an isomorphism of their colimits. 

  It follows from \Cref{rmkEpiMonoFactorizationCommutes} that when $f$ is epi/mono, 
  we can say that $f$ can be induced by a sequence 
  $(f_i)_{i\in \N}$ with all $f_i$ epi/mono. 
\end{remark}




\begin{definition}
  We call a type overtly discrete iff it can be described as 
  the colimit of an $(\N,\leq)$-indexed sequence of finite sets. 
\end{definition} 
\begin{remark}
  We will denote $\ODisc$ for an universe of overtly discrete types. 
  If $B:\ODisc$, we will denote $B_n$ for the objects of the underlying sequence and 
  $\iota_n^m: B_n \to B_m, \iota_n:B_n \to B$ for the obvious maps. 
%  $\iota_n^m:B_n \to B_m$ for the maps and objects in the underlying sequence and 
%  $\iota_n:B_n \to B$ for the colimit inclusion map. 
%  If $B$ is overtly disc
%  If we denote an overtly discrete type by $B$, we will denote the objects of the underlying sequence as 
%  $B_n,~n:\N$, and for $n\leq m$, we denote the maps $B_n \to B_m$ with 
%  Greek letters with lower index $n$ and upper index $m$. 
%  So for example $\iota_n^m:B_n \to B_m$. 
%  The maps $B_n \to B$ will in this case be denoted $\iota_n$. 
%  If convenient, given a sequence $B_n$, we will denote $B_\infty$ for the colimit $B$.
\end{remark}
%\begin{definition}
%A type $X$ is countable if there merely exists a decidable subset of $\N$ equal to $X$.
%\end{definition}
%


\subsection{Maps of overtly discrete types}
\begin{lemma}[Compactness of finite sets] \label{colimitCompact}
  For any finite set $A$ and $(\N,\leq)$-indexed sequence of finite sets $B_n$ with colimit $B$, 
  the colimit of $B_n^A$ is $B^A$. 
\end{lemma}  
\begin{proof}
  \rednote{Should there be a reference here?}
%  First note that $B^A$ forms a cocone on $(B_n^A)_{n:\N}$ 
  Any map $A \to B_n$ induces a map $A \to B$, hence $B^A$ is a cocone on $(B_n^A)_{n:\N}$.
  Let $C$ form a cocone on $(B_n^A)_{n:\N}$. %with maps $F_n:B_n^A \to C$.
  For any $f:A \to B$, the finite image $f(A)$ must already occur in some $B_n$, 
  thus there is some $f':A\to B_n$ with $\iota_n\circ f' = f$.% occurs as some map $A\to B_n$, 
  As $C$ is a cocone, $f'$ corresponds to some $c$ in $C$, and this term does not depend on $n$. 
  Also any map $B^A \to C$ respecting the cocone conditions must send $f$ to $c$, 
  hence $B^A$ is indeed the colimit. 
%%  We shall show there is an unique morphism of cocones $B^A \to C$. 
%%  Denote $\iota_n:B_n \to B, F_n:B_n^A\to C$ for the cocone maps. 
%  \begin{itemize}
%    \item 
%      If $f:A\to B$, we have that $f(A)$ is a finite subset of $B$, and thus occurs already in some $B_n$. 
%      This induces a map $f'_n:A\to B_n$ with $\iota_n\circ f'_n = f$. 
%      As $C$ is a cocone, we have that $F_n(f'_n)$ does not depend on $n:\N$. 
%%      Thus $(F_n)_{n:\N}$ induces a map 
%      Thus we get a map $B^A\to C$. 
%    \item 
%      For uniqueness, by function extensionality maps $B^A \to C$ are uniquely determined by their values on 
%      $f:B^A$. By the above, the value of $f$ is uniquely determined by it's value on $B_n$ for 
%      any $n$ with the image of $f$ in $B_n$. Thus there is at most one morphism of cocones $B^A \to C$. 
%  \end{itemize}
\end{proof}
\begin{remark}\label{rmkEqualityColimit}
  In the above proof, we used that any element $b\in B$ already occurs in some $B_n$. 
  However, it does not necessarily occur uniquely in $B_n$.
  In general, $B$ is overtly discrete 
  and there exist some $B_n$ with two elements corresponding to the same element in $B$, 
  Theorem 7.4 from \cite{SequentialColimitHoTT} says that there merely exists some $m\geq n$
  such that these elements become equal in $B_m$. 
\end{remark}
\begin{lemma}\label{lemDecompositionOfColimitMorphisms}
  Let $B,C$ be overtly discrete, 
  and let $f:B\to C$.
  There exists $(\N,\leq)$-indexed sequences of finite sets 
  $(B_n)_{n:\N}, (C_n)_{n:\N}$ with colimits $B,C$ respectively
  and compatible maps $f_n:B_n \to C_n$, 
  such that $f$ is the induced morphism $B\to C$.
\end{lemma}
\begin{proof}
  Let $(B_n)_{n:\N}, (C_n)_{n:\N}$ be 
  sequences of finite sets with colimits $B$ and $C$. 
  Using \Cref{axDependentChoice}, we will construct an increasing sequence of natural numbers $n_i$ 
  with a family of maps $f_i:B_i \to C_{n_i}$ such that the following diagram commutes for all $i>0$. :
%  We will construct a subsequence of $(C_n)_{n:\N}$, using \Cref{axDependentChoice}.
%choiceintro%%
%choiceintro%  For $k:\N$, let $E_k$ consist of 
%choiceintro%  strictly increasing sequences $(n_i)_{i<k}$ of natural numbers together with a finite family of maps 
%choiceintro%  $(f_i: B_{i} \to C_{n_i})_{i<k}$ such that
%choiceintro%  for all $0<i<k$ the following diagram commutes:
  \begin{equation}\label{eqnDecompositionOfColimitMorphisms}
    \begin{tikzcd}
      B_{i-1} \arrow[r,"\iota_{i-1}^i"] \arrow[d, "f_{i-1}"]& B_{{i}} \arrow[r, "\iota_i"] 
      \arrow[d,"f_{i}"]& B \arrow[d,"f"] \\
      C_{n_{i-1}} \arrow[r,"\kappa_{i-1}^i"] & C_{n_{i}} \arrow[r,"\kappa_i"] & C 
    \end{tikzcd}
  \end{equation}
%choiceintro%  For $e:E_k, e':E_{k+1}$, we let 
%choiceintro%  $R_k(e,e')$ denote  whether the underlying sequences of $e'$ extends that of $e$. 
%choiceintro%  The empty sequence inhabits $E_0$. We will show that if $e:E_k$, there exists some $e':E_{k+1}$ with 
%choiceintro%  $R_k(e,e')$. Then \Cref{axDependentChoice} will give the required sequence $(f_i:B_i\to C_{n_i})$.
%choiceintro%
%  Suppose we have $(f_i: B_{i} \to C_{n_i})_{i<k}$ for some $k\geq 0$ 
%  such that for all $0<i<k$ the diagram of \Cref{eqnDecompositionOfColimitMorphisms} commutes.
  Suppose we have an initial segment $(n_i)_{i<k}$ of such a sequence with maps $(f_i)_{i<k}$ making 
  \Cref{eqnDecompositionOfColimitMorphisms} commute for $i<k$. 
  We shall show that in this case there exist $n_{k}:\N, f_{k}:B_k \to C_{n_k}$ extending it. 
%  making the same diagram commute for $i = k$. 
  Consider the map $f\circ \iota_k: B_{k}\to C$. 
  As $B_k$ is finite, \Cref{colimitCompact} gives some $n_k':\N $ such that %, f_k':B_k \to C_{n_k'}$ such that 
  it factors over some $C_{n_k'}$.
%  \begin{equation}
%    \begin{tikzcd}
%    B_{k} \arrow[r,"\iota_k"]  \arrow[d,"f'_{k}"]& B \arrow[d,"f"]\\
%    C_{n'_{k}} \arrow[r, "\kappa_{n'_k}"'] & C
%    \end{tikzcd}
%  \end{equation}
%ExtraExplanationWhichReadercanNote%  We may assume $n'_{k+1} > n_k$.
%ExtraExplanationWhichReadercanNote%  Note that it is not necessarily the case that 
%ExtraExplanationWhichReadercanNote%  $f'_{k} \circ \iota_{k-1}^k = \kappa_{n_{k-1}}^{n'_k}\circ f_{k-1}$. 
%  $f'_{k+1}$ is compatible with $f_k$, meaning the left square in the following diagram needn't commute:
%  \begin{equation}
%    \begin{tikzcd}
%      B_{k-1} \arrow[r] \arrow[d, "f_{k-1}"]& B_{{k}}  \arrow[r] \arrow[d,"f'_{k}"] & B \arrow[d,"f"] \\
%      C_{n_{k-1}} \arrow[r] & C_{n'_{k}} \arrow[r]  & C 
%    \end{tikzcd}
%  \end{equation}
%ExtraExplanationWhichReadercanNote%  However, 
  Both $f'_{k}, f_{k-1}$ induce the same map $B_{k-1} \to C$. 
%  Recall by \Cref{rmkMorphismsOutOfQuotient} this map is induced by it's value on finitely many elements. 
  As $B_{k-1}$ is finite, from \Cref{rmkEqualityColimit} there is some $n_{k} \geq {n'_{k}}$ 
  such that they become equal in $C_{n_k}$, and we have $f_k:B_k \to C_{n_k}$ such that the following does commute;
  by \Cref{axDependentChoice} we then get compatible maps as required. 
%choiceintro% and we're done:

%  such that for $f_{k}$ the composition of $f'_{k+1}:B_{k+1} \to C_{n'_{k+1}}$ and 
%  the map $C_{n'_{k+1}} \to C_{n_{k+1}}$, the following diagram does commute:
  \begin{equation}
    \begin{tikzcd}
      B_{k-1} \arrow[d,"f_{k-1}"]\arrow[r] & B_{{k}} \arrow[rd, "f_{k}"] \arrow[rr,"\iota_k"] & & B \arrow[d,"f"] \\
      C_{n_{k-1}} \arrow[r] & C_{n'_{k}} \arrow[r] & C_{n_{k}} \arrow[r] & C 
    \end{tikzcd}
  \end{equation}
%  Now by dependent choice for the above $x_0, R_n, E_n$, we get a sequence $(f_i:B_i \to C_{n_i})$  for some 
%  strictly increasing sequence $n_i$ of natural numbers. 
%  Note that for such a sequence $(n_i)_{i:\N}$, 
%  $(C_{n_i})_{i:\N}$ converges to $C$. Also $(B_i)_{i:\N}$ still converges to $B$. 
%  Furthermore, by construction the map that sequence $f_i$ induces from $B \to C$ shares all values with $f$
%  and thus is equal to $f$. 
%  Thus our sequence $f_i$ is as required. 
\end{proof}

\begin{lemma}\label{lemDecompositionOfEpiMonoFactorization}
  \rednote{Can this be a shorter remark?}
  Let $f:A_\infty\to B_\infty$ be a map between overtly discrete types, and suppose we have $f_n:A_n\to B_n$ such that 
  the following diagram commutes:
  \begin{equation}
    \begin{tikzcd}
      A_n \arrow[d,"f_n"]\arrow[r, "\iota_n^m"]  & A_m \arrow[d,"f_m"] \arrow[r,"\iota_m^\infty"]  & A_\infty \arrow[d,"f"] 
      \\
      B_n \arrow[r, "\kappa_n^m"'] & B_m \arrow[r,"\kappa_m^\infty"'] & B_\infty
    \end{tikzcd}
  \end{equation}
  Then $f(A)$ is the colimit of $f_n(A_n)$, 
  and the maps $A\twoheadrightarrow f(A)$ and $f(A) \hookrightarrow B$ 
  are induced by the maps $A_n\twoheadrightarrow f_n(A_n)$ and $f_n(A_n) \hookrightarrow B_n$ respectively. 
\end{lemma}
\begin{proof}
  For $n\leq m$, we have that $\kappa_n^m(f_n(A_n)) = f_m(\iota_n^m(A_n))\subseteq f_m(A_m)$, 
  hence we can take the corestriction of the map $f_n(A_n) \to B_m$ to $f_m(A_m)$ to get 
  maps $\lambda_n^m :f_n(A_n) \to f_m(A_m)$ making the following diagram commute:
  % https://q.uiver.app/#q=WzAsOSxbMSwwLCJBX20iXSxbMiwwLCJBX1xcaW5mdHkiXSxbMSwyLCJCX20iXSxbMiwyLCJCX1xcaW5mdHkiXSxbMiwxLCJmKEEpIl0sWzEsMSwiZl9tKEFfbSkiXSxbMCwwLCJBX24iXSxbMCwyLCJCX24iXSxbMCwxLCJmX24oQV9uKSJdLFswLDFdLFsyLDNdLFsxLDQsIiIsMCx7InN0eWxlIjp7ImhlYWQiOnsibmFtZSI6ImVwaSJ9fX1dLFs0LDMsIiIsMSx7InN0eWxlIjp7InRhaWwiOnsibmFtZSI6Imhvb2siLCJzaWRlIjoidG9wIn19fV0sWzAsNSwiIiwyLHsic3R5bGUiOnsiaGVhZCI6eyJuYW1lIjoiZXBpIn19fV0sWzUsMiwiIiwxLHsic3R5bGUiOnsidGFpbCI6eyJuYW1lIjoiaG9vayIsInNpZGUiOiJ0b3AifX19XSxbNywyLCJcXGthcHBhX25ebSIsMl0sWzYsMCwiXFxpb3RhX25ebSJdLFs2LDgsIiIsMix7InN0eWxlIjp7ImhlYWQiOnsibmFtZSI6ImVwaSJ9fX1dLFs4LDcsIiIsMSx7InN0eWxlIjp7InRhaWwiOnsibmFtZSI6Imhvb2siLCJzaWRlIjoidG9wIn19fV0sWzgsNSwiIiwxLHsic3R5bGUiOnsiYm9keSI6eyJuYW1lIjoiZGFzaGVkIn19fV0sWzUsNCwiIiwxLHsic3R5bGUiOnsiYm9keSI6eyJuYW1lIjoiZG90dGVkIn19fV1d
  \begin{equation}\label{eqnEpiMonoFactorizationDecomposition}
    \begin{tikzcd}
    {A_n} & {A_m} & {A_\infty} \\
    {f_n(A_n)} & {f_m(A_m)} & {f(A_\infty)} \\
    {B_n} & {B_m} & {B_\infty}
    \arrow["{\iota_n^m}", from=1-1, to=1-2]
    \arrow[two heads, from=1-1, to=2-1,"e_n"]
    \arrow[from=1-2, to=1-3,"\iota_m^\infty"]
    \arrow[two heads, from=1-2, to=2-2,"e_m"]
    \arrow[two heads, from=1-3, to=2-3,"e_\infty"]
    \arrow[dashed, from=2-1, to=2-2, "\lambda_n^m"]
    \arrow[hook, from=2-1, to=3-1, "i_n"]
    \arrow[dashed, from=2-2, to=2-3, "\lambda_m^\infty"]
    \arrow[hook, from=2-2, to=3-2,"i_m"]
    \arrow[hook, from=2-3, to=3-3,"i_\infty"]
    \arrow["{\kappa_n^m}"', from=3-1, to=3-2]
    \arrow[from=3-2, to=3-3,"\kappa_m^\infty"']
  \end{tikzcd}
\end{equation}
  Also it is clear that any $b:f(A_\infty)$ already occurs in some $f_n(A_n)$, hence $f(A_\infty)$ is colimiting. 
\end{proof}
\begin{corollary}\label{decompositionInjectionSurjectionOfOdisc}
  In \Cref{lemDecompositionOfColimitMorphisms}, when $f$ is injective or surjective, 
  we can choose presentations such that each $f_n$ is also injective or surjective respectively. 
\end{corollary}
\begin{proof}
  Using \Cref{lemDecompositionOfColimitMorphisms} and \Cref{lemDecompositionOfEpiMonoFactorization}, 
  we get a factorization as in \Cref{eqnEpiMonoFactorizationDecomposition}. 
  If $f$ is injective, then $e$ is an isomorphism. 
  Hence $A$ is the colimit of $f_n(A_n)$, and we can take $f_n' = i_n$.
  Similarly, if $f$ is surjective $i$ is an isomorphism and we consider $B$ as colimit of $f_n(A_n)$ and 
  take $f_n' = e_n$.
\end{proof}
\subsection{Closure properties of $\ODisc$}
\begin{remark}\label{ODiscFiniteColim}
  As sequential colimits commute with finite colimits, 
  and finite sets are closed under finite colimits,
  $\ODisc$ is closed under finite colimits as well.
\end{remark}
\begin{lemma}\label{ODiscClosedUnderSequentialColimits}
  The colimit an $(\N,\leq)$-indexed sequence overtly discrete types is overtly discrete. 
\end{lemma}
\begin{proof}
  By applying \Cref{axDependentChoice} to \Cref{lemDecompositionOfColimitMorphisms}, 
  given a colimit of the sequence $A_i$, we can find a quarterplane of the form 
  \begin{equation}
    \begin{tikzcd}
    A_{0,0}\ar[d]\ar[r] & A_{0,1}\ar[d]\ar[r] & \cdots \\
    A_{1,0}\ar[d]\ar[r] & A_{1,1}\ar[d]\ar[r] & \cdots \\
    \vdots & \vdots & \ddots\\
    \end{tikzcd}
  \end{equation}
  where all the $A_{i,j}$ are finite sets, and $A_i$ is the colimit in $j$ of $A_{i,j}$ and 
  the maps $A_i \to A_k$ are induced by maps $A_{i,j}\to A_{k,j}$. 
  The colimit of the above quarter-plane is also the colimit of the induced $(\N,\leq)$-indexed sequence $A_{j,j}$, 
  which is overtly discrete by definition. 
\end{proof}
\begin{corollary}\label{OdiscSigma}
  Overtly discrete types are closed under $\Sigma$. 
\end{corollary}
\begin{proof}
  Let $B$ be overtly discrete and $X:B \to \mathcal U$ be a 
  $B$-indexed family of overtly discrete types. 
  For any $i:\N$, we have a finite coproduct of overtly discrete types 
  $\Sigma_{b:B_i} (X\circ \iota_i(b))$.  
  As colimits commute with finite coproducts, this is overtly discrete. 
  By Theorem 5.1 of \cite{SequentialColimitHoTT}, 
  taking the colimit in $i$, we get $\Sigma_{b:B} X(b)$. 
  By the above Lemma, this is overtly discrete. 
\end{proof}
\begin{remark}
  Note that the sequential colimit commutes with the propositional truncation, thus for $B:\ODisc$, we have 
  $||B||:\ODisc$. 
\end{remark}



%\begin{theorem}
%We have the following:
%\begin{enumerate}[(i)]
%\item Overtly dicrete types are stable under identity types and and sigma types.
%\item Overly discrete types are stable under quotients by equivalence relation with value in overtly discrete types.
%\item Overtly discrete type are stable under sequential colimits.
%\item Overtly discrete types have local choice.
%\end{enumerate}
%\end{theorem}
%
%\begin{proof}
%\begin{enumerate}[(i)]
%\item For stability under identity types, we use that sequential colimits commutes with identity types. 
%
%For stability under sigma types, sequential colimits commutes with sigma so that by (iii) it is enough to show that overtly discrete types are stable under finite coproduct. But sequential colimits commute with finite coproducts.
%
%\item Clear from the alternative description in \cref{overtly-discrete-colimit-finite}.
%
%\item Assume given a tower of sequential colimits of finite types. By using dependent choice with \cref{presentation-maps-overtly-discrete} repeatedly, we get a quarter plane of finite types:
%\begin{center}
%\begin{tikzcd}
%F_{0,0}\ar[d]\ar[r] & F_{0,1}\ar[d]\ar[r] & \cdots \\
%F_{1,0}\ar[d]\ar[r] & F_{1,1}\ar[d]\ar[r] & \cdots \\
%\vdots & \vdots & \ddots\\
%\end{tikzcd}
%\end{center}
%which colimit is the colimit of the assumed tower. Then we just use \cref{colimit-quarter-diagonal} to conclude that this colimit is overtly discrete.
%
%\item By \cref{overtly-discrete-colimit-finite}, we have a cover of any overtly discrete type by a countable type, which is an overtly discrete type that has choice.
%\end{enumerate}
%\end{proof}
%
%\begin{remark}
%(ii) implies that the propositional truncation of an overtly discrete type is open.
%
%(iii) implies that overtly discrete types are closed under countable coproducts.
%\end{remark}
%
%\begin{lemma}\label{equivalence-induced-by-open-is-open}
%Let $I$ be overtly discrete, and let $R$ be an open relation on $I$. Then the equivalence relation induced by $R$ is open.
%\end{lemma}
%
%\begin{proof}
%The equivalence relation $L(x,y)$ induced by $R$ is:
%\[\exists(k:\N). \exists(i_0,\cdots,i_k:I). i_0=x\land i_k=y \land (\forall l<k.\ R(i_l,i_{l+1})) \]
%which is open.
%\end{proof}
%
%\begin{lemma}
%Assume given a pushout square:
%\begin{center}
%\begin{tikzcd}
%I\ar[d]\ar[r] & K\ar[d] \\
%J\ar[r] & L  \\
%\end{tikzcd}
%\end{center}
%such that $I\to J$ is an embedding and $I,J,K$ are overtly discrete. Then $L$ is overtly discrete.
%\end{lemma}
%
%\begin{proof}
%The situation means we have an open $I\subset J$ and a map $f:I\to K$. Then $L$ is equivalent to the quotient of $J+K$ by the  equivalence relation generated by:
%\[i_0(x) \sim i_1(y)\ \mathrm{if\ we\ have\ that}\ (x\in I)\land f(x) = y\]
%It is overtly discrete by \cref{equivalence-induced-by-open-is-open}.
%\end{proof}
%
%
%


%DecompositionStone%
%DecompositionStone%\begin{proof}
%DecompositionStone%  By \Cref{FormalSurjectionsAreSurjections}, 
%DecompositionStone%  $g$ corresponds to an injective map $f:2^T \to 2^S$ of Boolean algebras, 
%DecompositionStone%  which by \Cref{decompositionInjectionSurjectionOfOdisc}
%DecompositionStone%  can be decomposed into maps $f_n:(2^T)_n \to (2^S)_n$ which are all injective, 
%DecompositionStone%  and by the duality described in \Cref{StoneDualToOdisc}, correspond to maps of finite Stone types
%DecompositionStone%  $g_n : T_n \to S_n$ with $T_n,S_n$ $(\N,\geq)$-indexed sequences with limits $T,S$, and $g_N$ inducing $g:T\to S$.
%DecompositionStone%  As all $f_n$ were injective, by \Cref{FormalSurjectionsAreSurjections}, all $g_n$ are surjective as required. 
%DecompositionStone%\end{proof}
