
%\begin{lemma}\label{overtly-discrete-colimit-finite}
%Let $X$ be a type, the following are equivalent:
%\begin{enumerate}[(i)]
%\item $X$ is overtly discrete.
%\item $X$ is a quotient of a countable type by an open equivalence relation.
%\end{enumerate}
%\end{lemma}
%
%\begin{proof}
%\begin{itemize}
%\item (i) implies (ii). Assume $X$ is of the form
%\[X  = (\Sigma_\N D)/R\]
%with $D$ decidable and $R$ open. Using choice for $\Sigma_\N D$ we get:
%\[\alpha : (\Sigma_\N D) \to (\Sigma_\N D)\to 2^\N\]
%such that:
%\[R(x,y) = \exists_{k:\N} \alpha(x,y,k) = 1\]
%Then we define:
%\[X_n = (\Sigma_{\mathrm{Fin}(n)} D) / L\]
%\[L(x,y) = \exists_{k:\mathrm{Fin}(n)} \alpha(x,y,k) = 1\]
%We have that $X_n$ is a finite type as it is a decidable quotient of a decidable subset of a finite type. Moreover:
%\[\mathrm{colim}_n X_n = X\]
%as sequential colimits commute with quotients by equivalence relations.
%\item (ii) implies (i). Indeed consider a sequential colimit of:
%\[f_k : \mathrm{Fin}(l_k) \to \mathrm{Fin}(l_{k+1})\]
%Then:
%\[\mathrm{colim}_k \mathrm{Fin}(l_k)  =  \left(\sum_{k:\N} \mathrm{Fin}(l_k)\right) / L\]
%where $L$ is the equivalence relation generated by $(k,x) \sim (k+1,f_k(x))$. But $\sum_{k:\N} \mathrm{Fin}(l_k)$ is countable and the equivalence relation generated by a decidable relation on such a type is open.
%\end{itemize}
%\end{proof}

%\begin{remark}
%A proposition is overtly discrete if and only if it is open.
%\end{remark}


%
%\begin{lemma}
%  Let $A,B,C$ be overtly discrete, represented by sequences $(A_n)_{n:\N}, (B_n)_{n:\N},(C_n)_{n:\N}$. 
%  Let $u,v,(u_n)_{n:\N},(v_n)_{n:\N}$ be as in the following diagram of Boolean algebras:
%  \begin{equation}
%    \begin{tikzcd}
%      A_\infty \arrow[r,"u_\infty"] & B_\infty  \arrow[r,"v_\infty"] & C_\infty
%      \\
%      A_n \arrow[u,"i_n^\infty"] \arrow[r,"u_n"] & B_n \arrow[u,"j_n^\infty"] \arrow[r,"v_n"] & C_n \arrow[u,"k^\infty_n"]
%      \\
%      A_m \arrow[u,"i_m^n"] \arrow[r,"u_m"] & B_m \arrow[u,"j_m^n"] \arrow[r,"v_m"] & C_m \arrow[u,"k_m^n"]
%    \end{tikzcd} 
%  \end{equation} 
%  Furthermore, assume that $v\circ u = 0$ and $v_n \circ u_n = 0$ for all $n:\N$.
%  Then $Ker(v)/Im(u)$ is the colimit of the sequence $Ker(v_n)/Im(u_n)$. 
%\end{lemma}
%\begin{proof}
%  First, we will note what the maps in this sequence are, which by some abuse of notation also gives
%  the cocone maps. 
%\paragraph{If $n\leq m$, there are maps $Ker(v_n)/Im(u_n)\to Ker(v_m) / Im(u_m)$.}
%Let $x,y\in B_n, a \in A_n$ be such that $x - y  = u_n(a)$, 
%then $j_n^m(x) - j_n^m(y) = j_n^m(u_n(a)) = u_m(i_n^m(a))$.
%Thus whenever $x,y\in B_n$ are such that $x \sim_{Im(u_n)} y$, we have that 
%$j_n^m(x) \sim_{Im(u_m)} j_n^m(y)$. 
%
%%
%Furthermore, if $x\in Ker(v_n)$, then $v_n(x) = 0$, thus 
%\begin{equation}
%  v_m(j_n^m(x)) = k_n^m(v_n(x)) = k_n^m(0) = 0
%\end{equation} 
%and hence $j_n^m(x) \in Ker(v_m)$. 
%Thus $j_n^m$ induces a map $\iota_n^m:Ker(v_n)/Im(u_n) \to Ker(v_m)/Im(u_m)$, 
%with $\iota^n_m([x]) = [j_n^m(x)]$ for $x\in Ker(v_n)$. 
%
%\paragraph{These maps are compatible (in particular, the maps $j_n^\infty$ form a cocone).}. 
%If $k\leq n \leq m$, we have that $j_n^m \circ j_k^n = j_k^m$.
%We thus have that $\iota_n^m \circ \iota_k^n = \iota_k^m$.
%
%\paragraph{Given any cocone $\kappa_n : Ker(v_n)/Im(u_n)\to K$, 
%  there exists an extension $\kappa_\infty(v_\infty)/Im(u_\infty)\to K$}
%      If $\kappa_n$ forms a cocone, this means that for $n\leq m$ we have 
%      $\kappa_m = \kappa_n \circ \iota_m^n$.
%
%      We shall give a map $\kappa_\infty:Ker(v_\infty)/Im(u_\infty) \to K$ satisfying 
%      $\kappa_\infty \circ \iota_n^\infty= \kappa_n$ for all $n:\N$.
%      We're going to define a map $k:Ker(v_\infty) \to K$.
%%%
%      Let $x\in Ker(v_\infty)$. Then $x\in B_\infty$ and $v\infty(x) = 0$. 
%      As $B_\infty$ is the colimit of the sequence $B_n$, 
%      there is some $n:\N$ and some $x':B_n$ with $v_n(x') = 0$. 
%      We'd like to define $k(x) = \kappa_n([x'])$. We need to check this definition doesn't depend on $n$. 
%      \begin{itemize}
%        \item \textbf{$k$ is well-defined} 
%      Assume $n\leq m$ are such that we have $x':B_n, x'':B_m$ with $v_n(x') = 0, v_m(x'') = 0$ 
%      and $j_n(x') = j_m(x'') = x$. 
%      Then there exists some $l\geq m\geq n$ with 
%      $j_n^l (x') = j_m^l(x'')$, hence 
%      $$\iota_n^l[x'] = \iota_m^l[x'']$$ and thus 
%      $$\kappa_l(\iota_n^l[x']) = \kappa_l(\iota_m^l[x''])$$
%      But now $\kappa_l\circ \iota_n^l = \kappa_n$ and $\kappa_l\circ \iota_m^l = \kappa_m$, hence 
%      $$\kappa_n([x']) = \kappa_m([x''])$$
%      Thus $k$ is well-defined. 
%      \item \textbf{$k$ respects $Im(u)$}
%        Let $x,y:B$ and let $a:A$ be such that are such that $x-y = u(a)$ and $v(x) = v(y) = 0$.
%        Then there is some $n:\N$, and some $x',y':B_n, a':A_n$ with $x'-y'= u_n(a'), v_n(x') = v_n(y') = 0$ 
%        and $j_n(x') = x, j_n(y') = y, i_n(a') = a$. 
%        Hence $[x'] = [y']$, hence $\kappa_n([x']) = \kappa_n([y'])$.
%        Thus $k(x) = k(y)$. 
%      \end{itemize}
%      We conclude that $k$ induces a map $\kappa_\infty:Ker(v)/Im(u) \to K$. 
%    \paragraph{$\kappa$ is colimiting}
%    Let $\kappa_n$ be as above, and suppose that 
%    $\lambda \circ \iota_n^\infty = \kappa_n$. 
%    Then for all $n:\N$, $x':B_n$ such that $j_n(x) = x$, we have 
%    $$\lambda ([x]) = \lambda \circ \iota_n([x']) = \kappa_n([x']) = k(x)$$
%    As such $n,x'$ always exist, it follows that 
%    $\lambda([x]) = k(x)$, hence $\lambda[x] = \kappa[x]$, so $\lambda = \kappa$ as required. 
%\end{proof} 
%
%
%
%







%%\begin{remark}
%%  For any overtly discrete type $B$, we can choose a presentation of the underlying sequence 
%%  $B_n' = \iota_n(B_n)$ such that $B$ is the colimit of $B_n'$, and the inclusion maps $\iota'_n$ are injective. 
%%  Applying the above lemma to some $B$ of this form, we see that if $f$ is injective, so are all the $f_n$. 
%%%  
%%%
%%%  For the above lemma, if $f$ is injective, so are all the $f_n$. This isn't clear at all. Consider 
%%%  the sequence of $\{x,y\} \to \{x\} \to \{x,y\} \to \{x \} \to \cdots$ with colimit $\{x\}$. 
%%\end{remark}
%\begin{remark}
%  Let $B:\Boole$. 
%  Recall that $B$ can be seen as the colimit of $(B_n)_{n:\N}$ with $B_n$ finite Boolean algebras. 
%  Now in the category of Boolean algebras, we have $(B\to 2) \simeq lim(B_n\to 2)$.
%  As $B_n\to 2$ is finite whenever $B_n$ is, it follows that Stone spaces are limits of $\N$-indexed diagrams of finite sets. 
%\end{remark}
%
%
%\begin{remark}\label{rmkEpiMonoFactorizationCommutes}
%  For $f,(f_i)_{i:\N}$ as above, whenever $f_n(x) = 0$, we have $f_{n+1}(x \circ \iota_{n,n+1}) = 0$
%  for $\iota_{n,n+1}$ the map $A_n \to A_{n+1}$. 
%  By \Cref{rmkMorphismsOutOfQuotient}, $\iota_{n,n+1}$ induces a map $A_n/Ker(f_n)\to A_{n+1}/Ker(f_{n+1})$. 
%  This induced map is such that the following diagram commutes:
%  \begin{equation}\begin{tikzcd}
%    A_n \arrow[d, two heads] \arrow[r, "\iota_{n,n+1}"] & A_{n+1} \arrow[d,two heads]\\
%    A_n /Ker(f_n) \arrow[d,hook] \arrow[r] & A_{n+1} /Ker(f_{n+1}) \arrow[d,hook] \\
%    B_n \arrow[r] & B_{n+1}
%  \end{tikzcd}\end{equation}  
%  As the induced maps be epi's / mono's  is epi /mono, the colimit of the sequence 
%  $A_n / Ker(f_n)$ will fit into an epi-mono factorization of $f$ and thus be iso to $A/Ker(f)$. 
%  Thus the epi-mono factorization of the colimit is the colimit of the epi-mono factorizations. 
%\end{remark}
%\begin{remark}\label{rmkIsoEpiMonoMapColimit}
%  Whenever $f:B \to C$ is an iso, any sequence with $B$ as colimit, also has $C$ as colimit. 
%  Thus any iso can be represented this way as sequence of iso's. 
%  Conversely, any sequence of isomorphisms induces an isomorphism of their colimits. 
%
%  It follows from \Cref{rmkEpiMonoFactorizationCommutes} that when $f$ is epi/mono, 
%  we can say that $f$ can be induced by a sequence 
%  $(f_i)_{i\in \N}$ with all $f_i$ epi/mono. 
%\end{remark}
%
%
%
