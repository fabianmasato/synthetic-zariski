\subsection{Order on the interval}
\begin{definition}
  For $n:\N$ we define 
  $cs_n:2^n \to \mathbb Q$ by 
  \begin{equation}
    cs_n(a) = \sum\limits_{i=0}^{n-1} \frac{a(i)} {2^{i+1}}
  \end{equation}
  And for $\alpha:2^\N$, we define the sequence $cs(\alpha) : \N \to \mathbb Q$ by 
  \begin{equation}
    cs(\alpha)_n = cs_n(\alpha|_n)
  \end{equation}
\end{definition}
\begin{remark}\label{rmkPropertiesCSn}
  $cs_n$ gives a bijection between $2^n$ and it's image 
  $\{\frac{k}{2^n}|0\leq k \leq 2^{n}-1\}\subseteq \mathbb Q$.
%  of rational numbers of the form  
%  $\frac{k}{2^n}$ for $0\leq k \leq 2^n-1$. 
  This observation has some corollaries: 
  \begin{itemize}
    \item In particular, each $cs_n$ is injective. 
    \item Furthermore, whenever $a\neq b:2^n$, we must have that 
      \begin{equation} 
        |cs_n(a)-cs_n(b)|\geq \frac{1}{2^n}.
      \end{equation}
    \item It is known that $\bigcup_{n:\N} \{\frac{k}{2^n}|0\leq k \leq 2^{n}-1\}$ 
      lies dense in the interval of Cauchy reals $[0,1]$. 
      It follows that $cs$ induces a surjection from Cantor space to $[0,1]$. %the interval of Cauchy reals. 
      We claim without proof it in fact induces an equivalence between $I$ and $[0,1]$.
%      between $I$ and the interval of Cauchy reals. 
  \end{itemize}
  Finally, let us repeat a well-known identity for all $m<n$ on such sums, which we'll make some use of 
  \begin{equation}
   \sum\limits_{i = m}^{n-1} \frac{1}{2^{i+1}} = \frac{1}{2^{m}} - \frac{1}{2^n}
  \end{equation}
\end{remark}
\begin{lemma}\label{CauchyApproxLemma}
  Let $n:\N$ and  $s,t:2^n$. Assume there is some $ m \leq n$ with $cs_m(s|_m) = cs_m(t|_m) + \frac{1}{2^m}$, and 
  at the same time $cs_n(s) -cs_n(t)\leq \frac{1}{2^n}$. 
  Then there is some $k< m$ and some $u:2^k$ such that 
  \begin{equation}
    (s = u \cdot 1 \cdot \overline 0|_n)
    \wedge 
    (t = u \cdot 0 \cdot \overline 1|_n)
  \end{equation}
\end{lemma}
\begin{proof}
  By injectivity of $cs_m$, we have that $s|_m \neq t|_m$. 
  Then there must be some smallest number $k<m$ such that 
  $s(k) \neq t(k)$. As $k$ is minimal, we have $s|_k = t|_k = : u$. 
%  WLOG, we assume that $s(m) = 1, t(m) = 0$. 
  It follows for all $l\leq n$ that 
%  We thus have for all $k<l\leq n$ that 
%  \begin{align}
%    cs_l(s|_l) &= 
%    cs_k(u|_k) + \sum\limits_{i = k}^{l-1} \frac{s(i)}{2^{i+1}}\\
%    cs_l(t|_l) &= 
%    cs_k(u|_k) + \sum\limits_{i = k}^{l-1} \frac{t(i)}{2^{i+1}}
%  \end{align}
%  And thus 
  \begin{align}
    cs_l(s|_l)-cs_l(t|_l) 
    = \sum\limits_{i = k}^{l-1} \frac{s(i)-t(i)}{2^{i+1}}
%    =\frac{s(k)-t(k)}{2^{k+1}} + \sum\limits_{i = k+1}^{l-1} \frac{s(i)-t(i)}{2^{i+1}}
  \end{align}
  Note that as $s(i),t(i) \in \{0,1\}$, we must have %that $s(i) -t(i) \in \{-1,0,1\}$. 
  $|s(i)-t(i)| \leq 1$. 
  Hence for any $k'<l$, we have 
  \begin{equation}
    \left|\sum\limits_{i = k'}^{l-1} \frac{s(i)-t(i)}{2^{i+1}}\right|
    \leq 
    \sum\limits_{i = k'}^{l-1} \frac{1}{2^{i+1}}
    = 
    \frac{1}{2^{k'}} - \frac{1}{2^{l}}
  \end{equation}
  Note that using the two equations above for $l=m$ and $k'=k+1$ we have:
  \begin{align}
    cs_m(s|_m) -cs_m(t|_m) 
    =&
    \frac{s(k)-t(k)}{2^{k+1}} + \sum\limits_{i = k+1}^{m-1} \frac{s(i)-t(i)}{2^{i+1}} \\
    \leq& 
    \frac{s(k)-t(k)}{2^{k+1}} + \left(\frac{1}{2^{k+1}} - \frac{1}{2^{m}}\right)
  \end{align}
  As the left hand side should equal $\frac{1}{2^m}$,
  we must have that $s(k)-t(k) \neq -1$. 
  As $s(k) \neq t(k)$ it follows that $s(k) = 1, t(k) = 0$.
  But now 
  \begin{equation}
    cs_n(s) -cs_n(t) 
    =
    \frac{1}{2^{k+1}} + \sum\limits_{i = k+1}^{n-1} \frac{s(i)-t(i)}{2^{i+1}}
    \geq 
    \frac{1}{2^{k+1}} - \left(\frac{1}{2^{k+1}} - \frac{1}{2^n} \right)
    =
    \frac{1}{2^{n}}
  \end{equation}
  And as $cs_n(s)-cs_n(t) \leq \frac{1}{2^n}$ as well, we get that 
  $cs_n(s)-cs_n(t) = \frac{1}{2^n}$. 
  Note that this lower bound is only reached if $s(i)-t(i) = -1$ for all $k<i<n$. 
  Hence for those $i$, we have $s(i) = 0, t(i) = 1$. 
  Thus 
  \begin{equation}
    s = (u \cdot 1\cdot \overline 0) |_n \wedge 
    t = (u \cdot 0\cdot \overline 1) |_n.
  \end{equation}
\end{proof}

 
\begin{corollary}\label{alternativeSimByCauchyDistance}
  Let $n:\N$ and let $s,t:2^n$. Then 
  \begin{equation}
    s\sim_n t \leftrightarrow |cs_n(s) - cs_n(t)| \leq \frac{1}{2^{n}}.
  \end{equation} 
\end{corollary}

\begin{proof}
  \item  
    Assume $ s \sim_n t$. If $s=t$, we have $cs_n(s) - cs_n(t) = 0$, 
    otherwise, we may without loss of generality assume there is some $m<n$ and some $u:2^m$ such that 
  \begin{equation}
    (s = u \cdot 0 \cdot \overline 1|_n) \wedge ( t = u \cdot 1 \cdot \overline 0 |_n) . 
  \end{equation}
  Then 
  \begin{align}
    cs_n(s) &= 
    cs_m(u) + 0 + \sum\limits_{i = m+1}^{n-1} \frac{1}{2^{i+1}}\\
    cs_n(t) &= 
    cs_m(u) + \frac{1}{2^{m+1}} + 0  
  \end{align}
  And hence 
  \begin{equation}
    cs_n(t) - cs_n(s) = \frac{1}{2^{m+1}} - \sum\limits_{i = m+1}^{n-1} \frac{1}{2^{i+1}} = \frac{1}{2^n}
  \end{equation}
  Thus in all cases, from $s\sim_n t$, we can conclude that 
  \begin{equation}
    |cs_n(s) -cs_n(t) |\leq \frac{1}{2^n}
  \end{equation}
  \item 
  Conversely, assume that $|cs_n(s) - cs_n(t)| \leq \frac{1}{2^n}$. 
  If $s = t$, it is clear that $s \sim_n t$.
  If $s\neq t$, there must be some smallest number $m<n$ such that 
  $s(m) \neq t(m)$. As $m$ is minimal, we have $s|_m = t|_m = : u$. 
  WLOG, we assume that $s(m) = 1, t(m) = 0$. 
  Then $cs_m(s|_{m+1})  = cs_{m+1}(t|_{m+1}) + \frac{1}{2^{m+1}}$
  and by \Cref{CauchyApproxLemma} it follows that 
  \begin{equation}
    s = (u \cdot 1\cdot \overline 0) |_n \wedge 
    t = (u \cdot 0\cdot \overline 1) |_n.
  \end{equation}
  and thus we can conclude $s\sim_n t$ as required. 
\end{proof}


Inspired by Definitions 2.7 and 2.10 \Cite{Bishop}, 
we define inequality on $I$ as follows:
\begin{definition}
  Let $\alpha,\beta:2^\N$. 
  We define $\alpha\leq_I \beta$ and $\alpha<_I\beta$ as follows:
  \begin{align}
  \alpha\leq_I\beta : = \forall_{n:\N} \left( cs(\alpha)_n \leq cs(\beta)_n + \frac {1} {2^n}\right)\\ 
    \alpha   <_I \beta : = \exists_{n:\N} \left( cs(\alpha)_n < cs(\beta)_n - \frac {1} {2^n}\right)
%    \\\rednote{Can become n\pm1, \leq ,<, +\frac1{2^n+2} }
\end{align}
\end{definition}
\begin{lemma}
  $\leq_I$ respects $\sim_I$. 
\end{lemma}
\begin{proof}
  We will show that whenever $\alpha\leq_I \gamma$ and $\alpha\sim_I\beta$, we have $\beta\leq_I\gamma$. 
  The other proof obligation goes similarly. 
%  The proof is similar to $\alpha'\leq_I\gamma'$ and $\gamma'\sim_I\beta'$, we have $\alpha'\leq_I\beta'$.


  As $\beta\leq_I\gamma$ is closed, so by \Cref{rmkOpenClosedNegation} it is double negation stable. 
  By \Cref{MarkovPrinciple}, the negation is that there is some 
  $N:\N$ with 
  $cs(\beta)_N > cs(\gamma)_N + \frac{1}{2^N}.$
  As $\alpha\leq_I\gamma$, we have 
  $cs(\gamma)_N + \frac{1}{2^N}\geq cs(\alpha)_N $. 
  Thus $cs(\beta)_N > cs(\alpha)_N$ and therefore $cs(\beta)_N = cs(\alpha)_N+\frac{1}{2^N}$ using  $\alpha\sim_I\beta$.
%  Yet as $\alpha\sim_I\beta$, from \Cref{alternativeSimByCauchyDistance}
%  we have $cs(\beta)_n \leq cs(\alpha)_n + \frac{1}{2^n}$ for all $n:\N$. 
%  Therefore, by \Cref{CauchyApproxLemma}, for $n\geq N$, we may conclude that 
  It follows that 
  $$
  cs(\alpha)_N+\frac{1}{2^N} > cs(\gamma)_N + \frac{1}{2^N} \geq cs(\alpha)_N
  $$
  From \Cref{rmkPropertiesCSn}, we must have
  $cs(\gamma)_N  + \frac{1}{2^N} = cs(\alpha)_N$, otherwise the distance 
  between $cs(\gamma)_N$ and $cs(\alpha)_N$ 
  would be smaller than $\frac{1}{2^N}$.
%  Using again $\alpha\sim_I\beta$ and \Cref{CauchyApproxLemma}, 
%  for $n\geq N$ we get 
%  $cs(\beta)_n = cs(\alpha)_n + \frac{1}{2^n}$.
  As $cs(\alpha)_n \leq cs(\gamma)_n + \frac{1}{2^n}$ for all $n\geq N$, 
  \Cref{CauchyApproxLemma} gives that 
  $\alpha\sim_I\gamma$. But also $\beta\sim_I\gamma$. 
  But now $\alpha,\beta,\gamma$ are all distinct yet related by $\sim_I$, contradicting 
  \Cref{IntervalFiberSizeAtMost2}. 
\end{proof}

\begin{remark}
  By \Cref{MarkovPrinciple}, we have that $\neg (\alpha \leq \beta) \leftrightarrow (\beta <_I \alpha)$. 
  It follows immediately that $<_I$ also respects $I$. 
  Therefore, $\leq_I, <_I$ induce relations $\leq,<$ on $I$.
  As the order in $\mathbb Q$ is decidable, $\leq, <$ are closed and open respectively. 
\end{remark} 


%
%\begin{itemize}
%  \item As a consequence of \Cref{alternativeSimByCauchyDistance}, we have for any $x,y:I$ that 
%    $x\leq y \wedge y \leq x \to x = y$. 
%  \item TODO $x\leq y \wedge x \neq y \to x < y$. 
%  \item TODO $x\leq y \vee y \leq x$. 
%\end{itemize}
%


\subsection{The topology of the interval}


\begin{definition}
  Let $a,b:I$. 
  Following standard notation, we denote
  \begin{equation}
    [a,b]:= \Sigma_{x:I} (a\leq x \wedge x \leq b),
  \end{equation}
  we call subsets of $I$ of this form closed intervals. 
%
  We also denote 
  \begin{align}
    (-\infty,a) &:= \Sigma_{x:I} (x < a)   \\
    (a,\infty) &:= \Sigma_{x:I} (a < x)  \\
    (a,b) &:= \Sigma_{x:I} (a < x \wedge x < b),
  \end{align}
  We call subsets of $I$ of these forms open intervals. 
\end{definition}
\begin{remark}
  Note that closed intervals and open intervals are closed and open respectively. 
\end{remark}


\begin{lemma}\label{IntervalQuotientMapIntersectionCommute}
  Let $D_n:2^\N \to 2$ be a sequence of decidable subsets with $D_{n+1}\subseteq D_n$.
  For $p$ the quotient map $2^\N \to I$, we have that 
  $p(\bigcap_{n:\N} D_n) = p(\bigcap_{n:\N} D_n)$
\end{lemma}
\begin{proof}
  It is always the case that $$p(\bigcap_{n:\N} D_n) \subseteq \bigcap_{n:\N} p(D_n).$$
  For the converse direction, let $(\bigcap_{n:\N} p(D_n))(x)$. 
  We will show that $ \neg \neg (p(\bigcap D_n)) (x)$, which is sufficient by \Cref{rmkOpenClosedNegation}. 
%
  As $(\bigcap_{n:\N} p(D_n))(x)$, there exists some $y\in D_0$ with $p(y) = x$. 
%
  If $x\notin p(\bigcap_{n:\N} D_n)$, we cannot have for all $n:\N$ that $y_0 \in  D_n$. 
  By Markov, there must exist some $k:\N$ with $\neg D_k(y_0)$. 
  As $D_{n+1}\subseteq D_n$ for all $n:\N$, it follows that $y_0\notin D_n$ for all $n\geq k$. 
%
  As $x\in \bigcap_{n:\N}p(D_n)$, there is however some $y_k\in D_k$ with $p(y_k) = x$. 
  By a similar argument, we have some $l>k$ with $y_k\notin D_l$, and some $y_l$ with $p(y_l) = x, y_l \in D_l$. 
  But now we have that $y_0, y_k, y_l:2^\N$ are all distinct, but $p(y_0) = p(y_k) = p(y_l) = x$. 
  This contradicts \Cref{IntervalFiberSizeAtMost2}, and we're done. 
\end{proof}


\begin{lemma}\label{ImageDecidableClosedInterval}
  For $p:2^\N \to I$ the quotient map and $D\subseteq 2^\N$ decidable, we have $p(D)$ a finite union of closed intervals. 
\end{lemma}
\begin{proof}
  We will show the above if there exists some $n:\N, u:2^n$ such that $D(\alpha) \leftrightarrow \alpha|_n = u$.
  This is sufficient, as any decidable subset of $2^\N$ can be written as finite union of such decidable subsets. 
  We claim that $p(D) = [p(u\cdot \overline 0) , p(u \cdot \overline 1)]$. 
\item 
  We will first show that $p(D) \subseteq [p(u\cdot \overline 0) , p(u \cdot \overline 1)$. 
  Suppose $D(\alpha)$. Then 
  Then $\alpha|_n = u$ and hence 
%  for $m\leq n$ we have 
%  \begin{equation}
%    cs(\alpha)_m = cs_m(u|_m) = cs(u\cdot \overline 0)_m= cs(u\cdot \overline 1)_m
%  \end{equation}
%  For $m>n$, we have that 
%  \begin{align}
%    cs(u\cdot \overline 1)_m =
%    cs_n(u) +\sum_{i = n} ^{m-1} \frac{1}{2^{i+1}}
%    \\
%    cs(\alpha)_m =
%    cs_n(u) +\sum_{i = n} ^{m-1} \frac{\alpha(i)}{2^{i+1}}
%    \\
%    cs(u\cdot \overline 0)_m = 
%    cs_n(u) +\sum_{i = n} ^{m-1} \frac{0}{2^{i+1}}
%  \end{align} 
%  Hence for all $m:\N$, we have 
  \begin{equation}
    cs(u\cdot \overline 1)_m \geq 
    cs(\alpha)_m \geq 
    cs(u\cdot\overline 0)_m
  \end{equation}
 which implies that $p(u\cdot \overline 1) \geq_I p(\alpha) \geq_I p(u\cdot\overline 0)$, as required. 
\item 
  To show that $[p(u\cdot \overline 0) , p(u \cdot \overline 1)]\subseteq p(D)$, 
  Suppose
  $(u\cdot \overline 0) \leq_I \alpha \leq_I (u \cdot \overline 1)$. 
  It is sufficient to show that 
  $$(\alpha|_n = u )\vee (\alpha \sim_I u \cdot \overline 0 )\vee (\alpha \sim_I u \cdot \overline 1).$$
  As this is a disjunction of closed propositions, by \Cref{ClosedFiniteDisjunction} it's closed, and by 
  \Cref{rmkOpenClosedNegation}, we can instead show the double negation. 
  So suppose that none of the disjoints hold. 
  As $\alpha|_n \neq u$, there is some minimal $m$ with $\alpha(m) \neq u(m)$. 
  We assume that $\alpha(m) = 1, u(m) = 0$, the other case goes similarly. 
  Then for all $k:\N$, we have 
  $cs(\alpha)_k \geq cs(u \cdot \overline 1)|_k$. 
  As also 
  $(u\cdot \overline 1)\geq_I \alpha$, we have 
  $$cs(u \cdot \overline 1)|_k + \frac{1}{2^k} \geq cs(\alpha)_k \geq cs(u\cdot \overline 1)_k,$$
  From which it follows that $|cs(a\cdot\overline 1)_k - cs(x)_k|\leq \frac{1}{2^k}$. 
  Hence $(u\cdot \overline 1)|_k \sim_k \alpha|_k$ by \Cref{alternativeSimByCauchyDistance}. 
  Hence $x\sim_I (a\cdot\overline 1)$, contradicting our assumption as required. 
\end{proof}
\begin{lemma}
  A finite intersection of open intervals is a
  a finite union of open intervals. 
\end{lemma}
\begin{proof}
  It suffices to show that the intersection of two open intervals is    
\end{proof}
%
\begin{lemma}
  Every open $U\subseteq I$ can be written as countable union of open intervals.
\end{lemma} 
\begin{proof}
  Let $U\subseteq I$ open, then $\neg U$ is closed and $U = \neg \neg U$ by \Cref{rmkOpenClosedNegation}. 
  By \Cref{StoneClosedSubsets}, \Cref{CompactHausdorffClosed} and \Cref{IntervalQuotientMapIntersectionCommute}, 
  we have that there is some sequence $D_n\subseteq 2^\N$ with 
  $\neg U = \bigcap_{n:\N} p(D_n)$. Thus $U = \neg \bigcap_{n:\N} p(D_n)$. 
  By \Cref{ClosedMarkov}, it follows that $U = \bigcup_{n:\N} \neg p(D_n)$. 
\end{proof}
%
%%  $\neg U$ is a countable intersection of finite unions of closed intervals. 
%%  Thus $\neg\neg U$ is a countable union of finite intersections of complements of closed intervals. 
%%  As complements of closed intervals are finite unions of open intervals (TODO), 
%%  and finite intersections of such things are still finite unions of open intervals, 
%%  it follows that $\neg\neg U$ is a countable union of open intervals. 
%%  By \Cref{rmkOpenClosedNegation}, $\neg \neg U = U$ and we're done. 
%%  \rednote{Lotta handwaving here, definitely not finished} 
%\end{proof}
%

\begin{remark}
  It follows that the topology of $I$ is generated by open intervals, 
  which corresponds to the standard topology on $I$. 
  Hence our notion of continuity corresponds with the $\epsilon,\delta$-definition of continuity one would expect. 
  Thus every function $f:I\to I$ in the system we presented is continuous in the $\epsilon,\delta$-sense. 
\end{remark}
