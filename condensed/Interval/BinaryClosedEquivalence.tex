We want to show that the interval of Cauchy reals is Compact Hausdorff. 
Informally, to any binary sequence $\alpha : \N \to 2$, 
we can associate a Cauchy sequence 
$cs(\alpha)$, given by 
\begin{equation}\label{eqnBinaryEncode}
  (cs(\alpha))_n = \sum\limits_{i = 0 }^n \frac {\alpha(i)}{2^{i+1}}
\end{equation}
and we are going to give a closed relation on Cantor space such that 
two binary sequences are equivalent iff they correspond to the same Cauchy reals. 
\begin{example}
  Let $n:\N$, we denote $C_n = 2[n]$ for the free Boolean algebra on $n$ generators 
  and no relations. 
  Note that $Sp(C_n) = 2^n$ corresponds to the space of finite binary sequences. 
\end{example}
Now we introduce some notation:
\begin{definition}
  \item Given an infinite binary sequence $\alpha:2^\N$ and a natural number $n : \N$  
    we denote $\alpha|_n: 2^n$ for the 
    restriction of $\alpha$ to a finite sequence of length $n$. 
  \item We denote $\overline 0, \overline 1$ 
    for the binary sequences which are constantly $0$ and $1$ respectively. 
  \item We denote $0,1$ for the sequences of length $1$ hitting $0,1$ respectively. 
  \item If $x$ is a finite sequence and $y$ is any sequence, 
    denote $x\cdot y$ for their concatenation. 
\end{definition} 
Now we'll give a definition for when two finite binary sequences of length $n$ correspond 
to real numbers whose distance is $\leq (\frac12)^n$.
Informally, we want for every finite sequence $s$ that 
$(s \cdot 0 \cdot \overline 1)$ and  $(s \cdot 1 \cdot \overline 0)$ are equivalent. 

\begin{definition}
  Let $n:\N$ and let $s,t : 2^n$. 
  We say $s,t$ are $n$-near, and write $s\sim_n t$ if 
  there merely exists some $m:\N$ with $m$ and some $u:2^m$, such that 
 \begin{equation}\label{EqnNearness}
   \big(
     (s = (u\cdot 0\cdot \overline 1)|_n) \vee (s = (u \cdot 1 \cdot \overline 0) |_n)
   \big)
    \wedge 
   \big(
     (t = (u\cdot 0\cdot \overline 1)|_n) \vee (t = (u \cdot 1 \cdot \overline 0) |_n)
   \big)
  \end{equation} 
\end{definition}
\begin{remark}\label{nearnessProperties}
\item As we're dealing with finite sequences, $s\sim_n t$ is decidable. 
\item Given any $s:2^n$, using $m=n, u = s$ above, we can show that $s\sim_n s$. 
  So $n$-nearness is reflexive. 
\item \Cref{EqnNearness} is symmetric in $s$ and $t$. Hence $n$-nearness is symmetric.
\item Note that $0\cdot 0\sim_2 0\cdot 1 \sim_2 1\cdot 0 \sim_2 1\cdot 1$, 
  but $0\cdot 0\nsim_2 1\cdot 1$. %is not $2$-near to $1\cdot 1$. 
  Thus $n$-nearness is not in general transitive. 
\end{remark}
\begin{definition}
  Let $\alpha, \beta: 2^\N$, we define $a\sim_I\beta$ as 
  $\forall_{n:\N} (\alpha|_n \sim_n \beta|_n)$. 
\end{definition}
\begin{lemma}\label{IntervalFiberSizeAtMost2}
  Whenever $\alpha,\beta,\gamma:2^\N$, are such that 
  $\alpha\sim_I \beta, \beta\sim_I \gamma$, 
  at least two of $\alpha,\beta,\gamma$ are equal. 
\end{lemma}
\begin{proof}
  We will show that $\beta = \gamma \vee \alpha = \gamma \vee \alpha = \beta$. 
  By \Cref{StoneEqualityClosed} and \Cref{ClosedFiniteDisjunction}, this is a closed proposition. 
  By \Cref{rmkOpenClosedNegation}, we can instead show the double negation. 
  To this end, assume that none of $\alpha,\beta,\gamma$ are equal. 
  By \Cref{MarkovPrinciple}, there exist indices $i,j,k\in \N$ with 
  \begin{equation}
    \beta(i) \neq \gamma(i), \alpha(j) \neq \gamma(j), \alpha(k) \neq \beta(k)
  \end{equation}
  Let $n:=\max(i,j,k) + 2$. 
  As $\alpha\sim_I \beta$, we have $\alpha|_n\sim_n\beta|_n$. 
  By assumption $\alpha|_n \neq \beta|_n$, so WLOG we may assume that 
  we have some $m: \N, u:2^m$ with 
  \begin{equation}
    \alpha|_n = (u\cdot 0 \cdot \overline 1) |_n , \beta|_n = (u \cdot 1 \cdot \overline 0)|_n.
  \end{equation}
  As $\alpha(k) \neq \beta(k) $ and $n\geq k+2$, 
  it follows in particular that $m\leq n-2$ and hence 
  $\beta(n-1) = 0$.% and $\beta(m) = 1$. 

  As also $\beta\sim_I \gamma$, we have $\beta|_n \sim_n \gamma|_n$.
  So there exists some $m':\N, u':2^m$ with 
 \begin{equation} %\label{EqnNearness}
   \big(
     (\beta|_{n} = (u'\cdot 0\cdot \overline 1)|_n) \vee (\beta|_{n} = (u' \cdot 1 \cdot \overline 0) |_n)
   \big)
    \wedge 
   \big(
     (\gamma|_{n} = (u'\cdot 0\cdot \overline 1)|_n) \vee (\gamma|_{n} = (u' \cdot 1 \cdot \overline 0) |_n)
   \big).
  \end{equation} 
  Similarly as above, we have that $m'\leq n-2$, and as $\beta(n-1) = 0$, it follows that 
  $\beta|_{n} = (u' \cdot 1 \cdot \overline 0) |_n$. 
  Now as $\beta(i)\neq \gamma(i)$ with $i<n$, we have that $\beta|_n \neq \gamma|_n$, hence 
  $\gamma|_{n} = (u'\cdot 0\cdot \overline 1)|_n$. 
  Now we have $m,m'\leq n-2$ and $u:2^m, u':2^{m'}$ such that 
  \begin{equation}
    (u\cdot 1 \cdot \overline 0)|_n = \beta|_n = (u'\cdot 1 \cdot \overline 0)|_n
  \end{equation}
  Note that $\beta(m') = 1$. 
  But also $\beta(l) = 0$ for all $l$ with $m<l<n$
  Therefore $m'\leq m$. 
  By similar reasoning, $m\leq m'$. We conclude $m=m'$. 
  As a consequence, $u = u'$, but then 
  $\gamma|_n = \alpha|_n$, contradicting that $\alpha(j)\neq \gamma(j) $ for $j<n$. 
  Hence we arrive at a contradiction, as required. 
\end{proof}


\begin{corollary}
  $\sim_I$ is a closed equivalence relation on $2^\N$. 
\end{corollary}
\begin{proof}
  By \Cref{nearnessProperties}, $\sim_I$ is a countable conjunction of decidable propositions. 
  Also by \Cref{nearnessProperties}, $\sim_n$ is reflexive and symmetric for all $n:\N$, thus
  $\sim_I$ is reflexive and symmetric as well. 
  Finally $\sim_I$ is transitive as a consequence of \Cref{IntervalFiberSizeAtMost2}.
\end{proof}
\begin{definition}
  We define $I:\Chaus$ as $I= 2^\N/\sim_I$. 
\end{definition}
