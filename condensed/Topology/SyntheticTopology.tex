\subsection{Types as spaces}
\begin{definition}
%  \index{$\subseteq$}
  Let $X$ be a type.
  A {subtype} of $X$ is a function $U:X\to\Prop$ to the type of propositions.
  We write $U\subseteq X$ to indicate that $U$ is as above.
  If $X$ is a set, a subtype may be called {subset} for emphasis.
  For subtypes $A,B\subseteq X$, 
  we write $A\subseteq B$ as a shorthand for pointwise implication.
  We will freely switch between subtypes $U:X\to\Prop$ and the corresponding embeddings
\[
  \begin{tikzcd}
    \sum_{x:X}U(x) \ar[r,hook] & X
  \end{tikzcd}
  \rlap{.}
\]
In particular, if we write $x:U$ for a subtype $U:X\to\Prop$, 
we mean that $x:\sum_{x:X}U(x)$ -- but we might silently project $x$ to $X$.
We will also denote $x\in U$ or $U(x)$ if we know that $x:X$. 
\end{definition}


The subobject $\Open$ of the type of propositions induces a topology on every type. 
This is the viewpoint taken in synthetic topology. 
We will follow the terminology of \cite{SyntheticTopologyEscardo, SyntheticTopologyLesnik}. 
%other references include \cite{SyntheticTopologyEscardo}%, TODOSortOutTaylorsReferences}.
%Defining a topology in this way has some benefits, which we summarize in this section. 

\begin{definition}
  Let $T$ be a type, and let $A\subseteq T$ be a subtype. 
  We call $A\subseteq T$ open or closed iff $A(t)$ is open or closed respectively for all $t:T$.
\end{definition}

\begin{remark}
  It follows immediately that the pre-image of an open by any map of types sends is open, so that any map is continuous. 
%  This is only relevant for a space if the topology we defined above matches the topology one would expect. 
  In \Cref{StoneClosedSubsets}, we shall see that the resulting topology is as expected for second countable Stone spaces.
  In \Cref{IntervalTopologyStandard}, we shall see that the same for the unit interval. 
\end{remark}



%\begin{remark}
%  Phao's principle is a special case of directed univalence. 
%\end{remark}
%\begin{proof}
%  \rednote{TODO}
%\end{proof}
