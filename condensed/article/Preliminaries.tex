\subsection{Preliminaries}
%
%In this section, we introduce the type of countably presented Boolean algebras $\Boole$ and of Stone spaces $\Stone$. 
%Both of these types carry a natural category structure. 
%In later sections, we will axiomatize an anti-equivalence between these categories, 
%which is classically valid and called Stone duality. 
\begin{definition}
  A type is countable iff it is 
  %By countable in the previous definition we mean sets that are 
  merely equal to some decidable subset of $\N$. 
\end{definition}
\begin{definition}
  A countably presented Boolean algebra $B$ is a Boolean algebra such that there merely are 
  countable sets $I,J$, 
  a set of generators $g_i,~{i\in I}$ and a set $f_j,~{j\in J}$ 
  of Boolean expressions over these generators 
  such that $B$ is equivalent to the quotient of the free Boolean 
  algebra over the generators by the relations
  $f_j=0$. We denote this algebra by $2[I]/(f_j)_{j:J}$.
\end{definition} 
\begin{remark}\label{BooleAsCQuotient}
Any countably presented algebra is also merely of the form 
$2[\N]/ (r_n)_{n:\N}$.
%, if we add dummy variables that we equate to $0$, and dummy relations that equate $0$ to itself.
% 0 is not special here right? 
\end{remark}


%We will call the family $(f_j)_{j\in J}$ as above a set of relations. 
%If $I,J$ are finite, we call $B$ a finitely presented Boolean algebra. 
%Once we have postulated the axiom of dependent choice, 
%in \Cref{secBooleAsColimits}
%we will be able to show that every countably presented algebra 
%is actually a colimit of a sequence of finitely presented Boolean algebras.
%They are therefore dual to pro-finite objects, which are used 
%in the theory of light condensed sets \cite{Scholze,Dagur,TODO}.

\begin{remark}
  We denote the type of countably presented Boolean algebras $\Boole$. 
  This type does not depend on a choice of universe. 
  Also note $\Boole$ has a natural category structure. 
\end{remark}

\begin{example}
  If both the set of generators and relations are empty, we have the Boolean algebra $2$.
  The underlying set is $\{0,1\}$ and $0\neq_2 1$.
  $2$ is initial in $\Boole$. 
\end{example}
%\begin{remark}
%Note that any Boolean algebra must contain the elements $0,1$. 
%Therefore, $2$ is initial in $\Boole$. 
%\end{remark} 
%We can therefore use it to define points of objects in the category dual to that of countably presented Boolean algebras. 
%
%\subsection{Stone spaces}
\begin{definition}
  For $B$ a countably presented Boolean algebra, 
  we define $Sp(B)$ as the set of Boolean morphisms from $B$ to $2$.
  Any type which is merely equivalent to a type of the form $Sp(B)$ is called a Stone space.
\end{definition}
%\begin{definition}
%  We define the predicate on types $\isSt$ by 
%  \begin{equation}
%    \isSt(X) := ||\sum\limits_{B : Boole} X = Sp(B)||
%  \end{equation} 
%  Given some universe $\mathcal U$, we denote 
%  $\Stone = \Sigma_{S:\mathcal U} \isSt(S)$. 
%%  A type $X$ is called \textit{Stone} if $\isSt(X)$ is inhabited.
%\end{definition}

%\subsection{Examples}
\begin{example}
  \label{boolean-algebra-examples}
  \item 
  \begin{enumerate}[(i)]
  \item There is only one Boolean map $2\to 2$, thus $Sp(2)$ is the singleton type $\top$. 
  \item   
    The tivial Boolean algebra is given by $2/(1)$. 
    We have $0=1$ in the trivial Boolean algebra. 
    As there cannot be a map from the trivial Boolean algebra into $2$ preserving both $0,1$, 
    the corresponding Stone space is the empty type $\bot$.
  \item\label{ExampleBAunderCantor}   
    We denote by $C$ the Boolean algebra $2[\N]$.
    In this case $Sp(C)$ is Cantor space: $2^\N$, the set of binary sequences. 
    If $\alpha:2^\N$ and $n:\N$ we write $\alpha(n) $ for $\alpha(g_n)$. 
%    %given by $\N$ as a set of generators and no relations. We write $p_n$ for the generator corresponding to $n$.
%    A morphism $C\to 2$ corresponds to a function $\mathbb N\to 2$, 
%    which is a binary sequence. 
%    So $Sp(C) = 2^\N$, 
%%    The Stone space $Sp(C)$ of these binary sequences is denoted 
%    $2^{\N}$ and called \notion{Cantor space}.
  \item\label{ExampleBAunderNinfty}
    We denote $B_\infty$ for the Boolean algebra generated by 
    $(g_n)_{n:\N}$ quotiented by the relations $g_m \wedge g_n = 0$ for ${n\neq m}$.
%    $C/(g_m\wedge g_n)_{m\neq n}$.
    A morphism $B_\infty\to 2$ corresponds to a function 
    $\mathbb N \to 2$ that hits $1$ at most once. 
    We denote $Sp(B_\infty) = \Noo$. 
    For $\alpha:\Noo$ and $n:\N$ we denote $\alpha(n)$ for $\alpha(g_n)$. 
%    The corresponding Stone space is denoted by $\N_\infty$.
  By conjunctive normal form, 
  any element of $B_\infty$ can be written uniquely as 
  $\bigvee_{i\in I} g_n$ or as $\bigwedge_{i\in I} \neg g_n$ for some finite $I\subseteq \N$. 
  If $I=\emptyset$, then $\vee_{i\in I} g_i = 0, \bigwedge_{i\in I} \neg g_i = 1$. 
  \end{enumerate}
\end{example}
%  In \Cref{N-co-fin-cp}, we will show 
%  It can be shown that $B_\infty$ is equivalent to the Boolean algebra on 
%  subsets of $\N$ which are finite or co-finite. 
%  Under this equivalence, the generator $g_n$ is sent to the singleton $\{n\}$. 
%  Because of this, we have that any $b:B_\infty$ can be written 

\begin{lemma}\label{ClosedPropAsSpectrum}
  For $\alpha:2^\N$, we have an equivalence of propositions: 
  $$
    (\forall_{n:\N} \alpha(n) = 0 )\leftrightarrow Sp(2/(\alpha(n))_{n:\N}).
  $$
\end{lemma}
\begin{proof}
  There is at most one $x:2\to 2$, and it can only satisfy 
  $x(\alpha(n)) = 0$ for all $n:\N$ iff 
  $\alpha(n) \neq_2 1$ for all $n:\N$. 
  As $2$ has underlying set $\{0,1\}$, 
  we have $\alpha(n) =_2 0 $ for all $n:\N$. 
  %we have $(\alpha(n) \neq_2 1) \to (\alpha(n) =_2 0)$. 
\end{proof}


%\begin{remark}
%  As Boolean algebras are rings, any relation of the form $f=g$ with both $f,g$ Boolean expressions 
%  can be written as $h=0$ with $h=f-g$ a Boolean expression. 
%\end{remark} 



