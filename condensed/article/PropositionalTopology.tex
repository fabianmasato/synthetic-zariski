%In this section, we will define the types of open and closed propositions. 
%These will allow us to define a (synthetic) topology  \cite{SyntheticTopologyLesnik} on any type.
%We will study this topology on Stone types in particular.
%
\subsection{Open and closed propositions}
In this section we will introduce a topology on the type of propositions, and 
study their logical properties.
We think of open and closed propositions respectively as countable disjunctions and conjunctions of decidable propositions.
Such a definition is universe-independent, and can be made internally.

\begin{definition}
A proposition $P$ is open (resp. closed) if there exists some $\alpha:2^\N$ such that $P \leftrightarrow \exists_{n:\mathbb N} \alpha_n = 0$ (resp. $P \leftrightarrow \forall_{n:\mathbb N} \alpha_n = 0$). We denote by $\Open$ and $\Closed$ the types of open and closed propositions.
\end{definition}

\begin{remark}\label{rmkOpenClosedNegation}
  The negation of an open proposition is closed, 
  and by MP (\Cref{MarkovPrinciple}), the negation of a closed proposition is open %. 
%  Also by MP, we have 
  and both open, closed propositions are $\neg\neg$-stable. 
%  and $\neg \neg P \to P$ whenever $P$ is open or closed. 
%  By the negation of WLPO (\Cref{NotWLPO}), 
  By $\neg$WLPO (\Cref{NotWLPO}), 
  not every closed proposition is decidable. 
  Therefore, not every open proposition is decidable. 
  % Both therefore and similarly can be used here, by a similar proof we can show it, or we can use that 
  % if $P$ is closed and $\neg P$ is decidable, so is $\neg \neg P = P$. 
  Every decidable proposition is both open and closed.
%  and in \Cref{ClopenDecidable} we shall see the converse. 
\end{remark}
\begin{lemma}
  We have the following results on open and closed propositions:
  \begin{itemize}
    \item Closed propositions are closed under finite disjunctions. 
    \item Closed propositions are closed under countable conjunctions. 
    \item Open propositions are closed under finite conjunctions. 
    \item Open propositions are closed under countable disjunctions. 
  \end{itemize}
\end{lemma}
\begin{proof}
  By Proposition 1.4.1 of \cite{HannesDiener}, LLPO (\Cref{LLPO}) is equivalent to the statement that 
  the disjunction of two closed propositions are closed. 
  The other statements have similar proofs, and we only present the proof that closed propositions are closed under 
  countable conjunctions. 
  Let $(P_n)_{n:\N}$ be a countable family of closed propositions. 
  By countable choice, for each 
  $n:\N$ we have an $\alpha_n:2^\N $ 
  such that $P_n \leftrightarrow \forall_{m:\N} \alpha_{n,m} =0$. 
  Consider a surjection $s:\N \twoheadrightarrow \N \times \N$, and let 
%  Let 
%  $$\beta_k = \alpha_{s(k)}.$$
  $\beta_k = \alpha_{s(k)}.$
  Note that $\forall_{k:\N} \beta_k = 0$ if and only if 
%  $\forall_{m,n:\N}\alpha_{m,n} = 0$, which happens if and only if 
  $\forall_{n:\N} P_n$. 
%  Hence the countable conjunction of closed propositions is closed. 
\end{proof}
We will use the above properties silently from now on. 
%OneBigLemma#
%OneBigLemma#\rednote{Phrase the following lemmas as one big lemma, 
%OneBigLemma#and use them silently without reference, also we should just state $\neg\neg$-stability instead of referring to the above all the time. }
%OneBigLemma#
%OneBigLemma#\begin{lemma}\label{ClosedCountableConjunction}
%OneBigLemma#  Closed propositions are closed under countable conjunctions.
%OneBigLemma#\end{lemma}
%OneBigLemma#\begin{proof}
%OneBigLemma#  Let $(P_n)_{n:\N}$ be a countable family of closed propositions. 
%OneBigLemma#  By countable choice, for each 
%OneBigLemma#  $n:\N$ we have an $\alpha_n:2^\N $ 
%OneBigLemma#  such that $P_n \leftrightarrow \forall_{m:\N} \alpha_{n,m} =0$. 
%OneBigLemma#  Consider a surjection $s:\N \twoheadrightarrow \N \times \N$, and let 
%OneBigLemma#%  Let 
%OneBigLemma#%  $$\beta_k = \alpha_{s(k)}.$$
%OneBigLemma#  $\beta_k = \alpha_{s(k)}.$
%OneBigLemma#  Note that $\forall_{k:\N} \beta_k = 0$ if and only if 
%OneBigLemma#%  $\forall_{m,n:\N}\alpha_{m,n} = 0$, which happens if and only if 
%OneBigLemma#  $\forall_{n:\N} P_n$. 
%OneBigLemma#  Hence the countable conjunction of closed propositions is closed. 
%OneBigLemma#\end{proof} 
%OneBigLemma#Using similar arguments, we can show the following two lemmas:
%OneBigLemma#\begin{lemma}\label{OpenCountableDisjunction}
%OneBigLemma#  Open propositions are closed under countable disjunctions. 
%OneBigLemma#\end{lemma}
%OneBigLemma#\begin{lemma}\label{OpenFiniteConjunction}
%OneBigLemma#Open propositions are closed under finite conjunctions. 
%OneBigLemma#\end{lemma}
%OneBigLemma#%\begin{proof}
%OneBigLemma#%We use \Cref{ClosedFiniteDisjunction} and the fact that $\neg(P\lor Q) \leftrightarrow \neg P \land \neg Q$.
%OneBigLemma#%\end{proof}
%OneBigLemma#%\begin{proof}
%OneBigLemma#%  Similar to the previous lemma. 
%OneBigLemma#%\end{proof}
%OneBigLemma#\begin{lemma}\label{ClosedFiniteDisjunction} 
%OneBigLemma#  Closed propositions are closed under finite disjunctions. 
%OneBigLemma#\end{lemma}
%OneBigLemma#\begin{proof}
%OneBigLemma#  This statement is equivalent to LLPO (\Cref{LLPO}) by  
%OneBigLemma#  Proposition 1.4.1 of \cite{HannesDiener}. 
%OneBigLemma#%  , LLPO is equivalent to the statement that 
%OneBigLemma#%  for $(\phi_n)_{n:\N}, (\psi_m)_{m:\N}$ families of decidable propositions indexed over $\N$, we have
%OneBigLemma#%  \begin{equation}
%OneBigLemma#%    (\forall_{n:\N} \forall_{m:\N} (\phi_n \vee \psi_m) )
%OneBigLemma#%    \leftrightarrow
%OneBigLemma#%    ((\forall_{n:\N} \phi_n) \vee (\forall_{m:\N} \psi_m) )
%OneBigLemma#%  \end{equation}
%OneBigLemma#%%  $(\forall_{n:\N} \alpha(n) = 0 )\vee (\forall_{n:\N} \beta(n) = 0 )$ is closed for any $\alpha,\beta:2^\N$.
%OneBigLemma#%%  By \Cref{corAlternativeLLPO}, the statement is equivalent to 
%OneBigLemma#%%  $ \forall_{n:\N}  \forall_{m:\N}  (\alpha(n) = 0 \vee \beta(m) = 0)$, 
%OneBigLemma#%  The latter which is a countable conjunction of decidable propositions, 
%OneBigLemma#%  hence closed by \Cref{ClosedCountableConjunction}.
%OneBigLemma#\end{proof}
%OneBigLemma#
\begin{corollary}\label{ClopenDecidable}
  If a proposition is both open and closed, it is decidable. 
\end{corollary}
\begin{proof}
  If $P$ is open and closed, %$\neg P$, and hence 
  $P\vee \neg P$ is open, 
  hence $\neg\neg$-stable and provable. 
%  and we conclude by $\neg\neg$-stability of open propositions. 
%  but open propositions are $\neg\neg$-stable by \Cref{rmkOpenClosedNegation} so we can conclude.
%  hence 
 % equivalent to $\neg \neg (P \vee \neg P)$ by \Cref{rmkOpenClosedNegation}.
 % As the latter proposition is provable, we may conclude $P$ is decidable. 
%  
%  If $P$ is open and closed, $P\vee \neg P$ is open, hence
%  equivalent to $\neg \neg (P \vee \neg P)$, which is provable. 
\end{proof}


%\begin{lemma}\label{OpenFiniteConjunction}
%  Open propositions are closed under finite conjunctions. 
%\end{lemma}
%\begin{proof}
%  We need to show that for any $\alpha,\beta:2^\N$, the following proposition is open:
%  \begin{equation}\label{eqnConjunctionOpen}
%    (\exists_{n:\N} \alpha(n) = 0 )\wedge(\exists_{n:\N} \beta(n) = 0 )
%  \end{equation}
%  Consider $\gamma:2^\N$ given by 
%  $\gamma(l) = 1$ iff there exist some $k,k'\leq l$ with 
%  $\alpha(k) = \beta(k') = 0$. 
%  As we only need to check finitely many combinations 
%  of $k,k'$, this is a decidable property for each $l:\N$ and $\gamma$ is well-defined. 
%  Then $\exists_{k:\N}\gamma(k)=0$ if and only if \Cref{eqnConjunctionOpen} holds.
%\end{proof}

\begin{lemma}\label{ClosedMarkov}
  For $(P_n)_{n:\N}$ a sequence of closed propositions, we have 
  $\neg \forall_{n:\N} P_n \leftrightarrow  \exists_{n:\N} \neg P_n$. 
\end{lemma}
\begin{proof}
  Both $\neg \forall_{n:\N} P_n$ and $\exists_{n:\N} \neg P_n$ are open, hence $\neg\neg$-stable.
  The equivalence follows. 
%  and the equivalence follows. 
%  from which the equivalence follows. 
%  We have that $\forall_{n:\N}P_n$ is closed and $\exists_{n:\N} \neg P_n$ is open by \Cref{OpenCountableDisjunction}, therefore both are $\neg\neg$-stable by \Cref{rmkOpenClosedNegation} and we can conclude.
%It is always the case that $\exists_{n:\N}\neg P_n \to \neg \forall_{n:\N} P_n$. 
  %For the converse direction,
  %note that $\neg \exists_{n:\N} \neg P_n(x) \to \forall_{n:\N} \neg \neg P_n(x).$
  %By \Cref{rmkOpenClosedNegation}, $\neg \neg  P_n(x)\leftrightarrow P_n(x)$ for all $n:\N$. 
  %It follows that 
  %$\neg \forall_{n:\N} P_n(x)\to 
  %\neg \neg \exists_{n:\N} \neg P_n(x).$
  %As $\exists_{n:\N}\neg P_n(x)$ is a countable disjunction of open propositions, 
  %it is open by \Cref{OpenCountableDisjunction} and thus equivalent to 
  %$\neg\neg\exists_{n:\N} \neg P_n(x)$ by \Cref{rmkOpenClosedNegation}.
  %We conclude that $\neg \forall_{n:\N} P_n \to \exists_{n:\N} \neg P_n$ as required. 
\end{proof} 

%\begin{lemma}\label{OpenDependentSums}
%  Open propositions are closed under dependent sums.
%\end{lemma}
%\begin{proof}
%  \rednote{If we show that Open propositions are exactly the overtly discrete ones, this is implied by $\Sigma$-closure}
%  First note that for $D$ a decidable proposition, and $X:D \to \Open$,
%  by case splitting on $D$, we can see 
%  $\Sigma_{d:D} X(d)$ is open.
%%
%  Then note that for $P$ an open proposition, 
%  there exists a sequence of decidable propositions $A_n$ with 
%  $P = \exists_{n:\N} A_n $.
%%
%  So for $Y : P \to Open $, the dependent sum $\Sigma_P Y$ is given by 
%  $\exists_{n:\N} (\Sigma_{a:A_n} Y(n,a))$,
%  which is a countable disjunction of open propositions, 
%  hence open by \Cref{OpenCountableDisjunction}.
%\end{proof}
%
%We will see the same holds for closed propositions in \Cref{ClosedDependentSums}.
%
%\begin{remark}\label{ImplicationOpenClosed}
%  If $P$ is open, $P \to \bot$ is only open if $P$ is decidable, which is not in general the case. 
%  Thus $\Open$ is not closed under dependent products. Neither is $\Closed$. 
%  However, as $(P\to Q)  \to \neg \neg (\neg P \vee Q)$,
%  we have that if $P$ is open and $Q$ is closed, then $P\to Q$ is closed, and similarly $Q\to P$ is open.
%\end{remark}
\begin{lemma}\label{ImplicationOpenClosed}
  If $P$ is open %(resp. closed) 
  and $Q$ is closed % (resp. open) 
  then $P\to Q$ is closed. % (resp. open). 
  If $P$ is closed and $Q$ open, then $P\to Q$ is open. 
\end{lemma}
\begin{proof}
  Note that $\neg P \vee Q$ is closed. Using $\neg\neg$-stability
  we can show $(P\to Q) \leftrightarrow (\neg P \vee Q)$. 
  The other proof is similar. 
%  and we conclude by 
%  Assume $P$ open and $Q$ closed, the other proof is similar. 
%  Note that $(\neg P \vee Q) \to (P \to Q)$ and 
%  $(P\to Q)\to \neg\neg(\neg P \vee Q)$. 
%  By \Cref{rmkOpenClosedNegation} it follows that 
%  $(\neg P \vee Q)\leftrightarrow (P \to Q)$, and using \Cref{ClosedFiniteDisjunction}, 
%  we can conclude that $P\to Q$ is closed. 
\end{proof}
%
%The following question was asked by Bas Spitters at TYPES 2024:

