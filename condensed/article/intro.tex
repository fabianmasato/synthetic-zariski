The language of homotopy type theory consists of dependent type theory enriched with the univalence axiom and higher inductive types. It has proven exceptionnally well-suited to
a synthetic development of homotopy theory \cite{hott}. It also provides
a framework precise enough to analyze categorical models of type theory \cite{vanderweide2024}.
Moreover, arguments in this language can be represented in proof assistants rather directly. In this article we use homotopy type theory to give a synthetic development of topology, analogous to the synthetic development of algebraic geometry in \cite{draft}. 

We introduce four axioms inspired by the light condensed sets introduced in \cite{Scholze}.
Interestingly, our axioms have strong connections with constructive mathematics \cite{Bishop},
in particular constructive reverse mathematics \cite{ReverseMathsBishop,HannesDiener}. Indeed they imply several of Brouwer's principles (e.g. any real function on the unit interval is continuous, the celebrated fan theorem), as well as not intuitionistically valid principles (Markov's Principle, the so-called Lesser Limited Principle of Omniscience).

Our axioms also closely align with the program of Synthetic Topology \cite{SyntheticTopologyEscardo,SyntheticTopologyLesnik,abstractstone,faissole2017synthetic,vickers2007locales}.
Indeed we have a dominance of open propositions, so that any type comes with an induced topology. Using this induced topology, we manage to capture synthetically the notion of second-countable compact Hausdorff spaces. While working on our axioms, we learnt about \cite{bc24}, which provides a similar axiomatisation in extensional type theory. We show that some of their axioms are consequences of ours. For example\footnote{We can actually prove all of their axioms, from which their \emph{directed univalence} follows. This will be presented in a following paper.}, we can define in our setting the notion of overtly discrete types, which is dual to the notion of compact Hausdorff spaces.

A central theme of homotopy type theory is that the notion of {\em type} is more general than the notion of {\em set}. We illustrate this theme in this work. Indeed we can form the types of Stone spaces and of compact Hausdorff spaces, which are not sets but rather a groupoids. Moreover these spaces are closed under $\Sigma$-type types, which would be impossible to formulate in the traditional setting.
Additionally, we can leverage higher types by using the elegant definition of cohomology groups in homotopy type theory \cite{hott}. We then prove a special case of a theorem of Dyckhoff \cite{dyckhoff76} describing
the cohomology of compact Hausdorff spaces. As an application, we give a synthetic proof of Brouwer's fixed point theorem, similar to the proof of an approximated form in \cite{shulman-Brouwer-fixed-point}.

We expect our axioms to be validated by the interpretation of homotopy type theory into the higher topos of light condensed anima \cite{shulman2019all}, although checking this rigorously is still work in progress. We even expect this to be valid in a constructive metatheory, using \cite{CRS21}. It is important to stress that our axioms only capture the properties of light condensed anima that are {\em internally} valid. Since David W\"arn \cite{warn2024} has proved that an important property of condensed abelian groups is {\em not} valid internally, this means that we cannot prove it in our setting. We also conjecture that the present axiom system is {\em complete} for the properties that are internally valid.
