The language of dependent type theory, with the axiom of univalence, has proved to be extremely well-adapted to
develop in a synthetic way homotopy theory \cite{hott}, but
also to analyse, with the required precisions, categorical models of type theory \cite{vanderweide24}. One
further important feature is that
the arguments can be rather directly represented in proof assistants. The work \cite{draft} shows how one can use
this language for a synthetic development of basic notions of algebraic geometry. The goal of the present paper
is to present a axiom system which seems sufficient for expressing and proving basic notions of light condensed
sets, introduced in \cite{Scholze}.

This relatively simple system is based on four axioms, similar to the one used
in \cite{SAG}. Maybe surprisingly, this development has strong connections with constructive mathematics \cite{Bishop},
and especially constructive reverse mathematics \cite{Ishihara,Diener}. Several of Brouwer's principles, such that
any real function on the unit interval, is continuous, or the celebrated fan theorem, are consequences of this system
of axioms. However, we can also prove principles that are not intuitionistically valid, such as Markov's Principle,
or even the so-called Less Limited Principle of Omniscience, a principle well studied in constructive reverse mathematics,
which is {\em not} valid effectively.  This development has also strong connections with the program of Synthetic
Topology \cite{SyntheticTopologyEscardo,SyntheticTopologyLesnik}:
we can show that there is a dominance of open propositions, providing any type with an intrinsic
topology, and we capture in this way synthetically the notion of (second-countable) compact Hausdorff spaces.
While working on this axiom system, we learnt about the related work \cite{bc24}, which provides a different axiomatisation
at the set level, and we can show that some of their axioms are consequences of our axiom system. In particular, we can introduce
in our setting a notion of ``Overtly Discrete'' spaces, dual in some way to the notion of compact Hausforff spaces, like
in Synthetic Topology\footnote{We actually have a
derivation of their ``directed univalence'', but this will be presented in a following paper.}.

One important theme of homotopy type theory is that the notion of {\em type} is more general than the notion of {\em set}. We illustrate
this theme here as well: we can form in our setting the types of Stone spaces and of compact Hausdorff spaces
(types which don't form a set but a groupoid),
and show these types are
closed by dependent sum types. It would be impossible to formulate such properties in the setting of simple type or set theory.
We can also use the elegant definition of cohomology group in homotopy type theory \cite{HoTT}, which makes essential
use of higher types that are not sets, and prove, in a purely axiomatic way,
a special case of a theorem of Dyckhoff \cite{dyckhoff76}, which provides
a description of the cohomology of compact Hausdorff spaces. We use this characterisation to prove Brouwer's fixed point theorem
in a type theoretic way (similar to the proof in \cite{shulman-Brouwer-fixed-point}, but
in our setting the theorem can be formulated in the usual
way, and not in an approximated form).

It is important to stress that what we capture in this axiomatic way are the properties of light condensed
sets that are {\em internally} valid. David W\"arn \cite{warn24} has proved that an important property of abelian
groups in the setting of light condensed sets, is {\em not} valid internally and thus cannot be formulated in this context.
We believe however that our axiom system can be convenient for proving the results that are internally valid, as we hope
is illustrated by that the present paper. We also conjecture that the present axiom system is actually {\em complete}
for the properties that are internally valid. Finally, we think that this system can be justified in a constructive meta theory
using the work \cite{CRS21}.
