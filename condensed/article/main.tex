% latexmk -pdf -pvc main-fancy.tex
\documentclass[a4paper,UKenglish,cleveref, autoref, thm-restate, numberwithinsect]{lipics-v2021}
\RequirePackage{amsmath,amssymb,mathtools,amsxtra,stmaryrd}
\RequirePackage{tikz}
\RequirePackage{cleveref}
\usetikzlibrary{arrows, cd, babel}
% names for types
\newcommand{\R}{\mathbb{R}}
\newcommand{\Z}{\mathbb{Z}}
\newcommand{\N}{\mathbb{N}}
\newcommand{\C}{\mathbb{C}}
\newcommand{\Bool}{\mathrm{Bool}}
\DeclareMathOperator{\Fin}{Fin}
\newcommand{\unit}{\mathbf{1}}
\newcommand{\two}{\mathbf{2}}
\newcommand{\isContr}{\mathrm{isContr}}
\newcommand{\isProp}{\mathrm{isProp}}
\newcommand{\isSet}{\mathrm{isSet}}
\newcommand{\isEquiv}{\mathrm{isEquiv}}
\newcommand{\qinv}{\mathrm{qinv}}
\newcommand{\mU}{\mathcal{U}}
\newcommand{\Eq}[1]{\mathrm{Eq}_{#1}}
\newcommand{\isNType}[1]{\mathrm{is}\mbox{-}{#1}\mbox{-}\mathrm{type}}
\newcommand{\nType}[1]{#1\mbox{-}\mathrm{Type}}
\newcommand{\Type}{\mathrm{Type}}
\newcommand{\Prop}{\mathrm{Prop}}
\newcommand{\Open}{\mathsf{Open}}
\newcommand{\Susp}{\mathrm{Susp}}
\newcommand{\propTrunc}[1]{\lVert #1 \rVert}
\newcommand{\norm}[1]{\lVert #1 \rVert}
\newcommand{\Um}{\mathrm{Um}}
\newcommand{\Boole}{\mathsf{Boole}}
\newcommand{\Stone}{\mathsf{Stone}}
\newcommand{\CHaus}{\mathsf{CHaus}}
\newcommand{\Chaus}{\mathsf{CHaus}}
\newcommand{\ODisc}{\mathsf{ODisc}}
\newcommand{\Noo}{\N_{\infty}}
\newcommand{\Closed}{\mathsf{Closed}}
\newcommand{\ints}{\mathbb{Z}}
%\newcommand{\KR}{K(R^{\times},1)}
\newcommand{\KR}{\mathsf{Line}}
\newcommand{\Line}{\mathsf{Line}}
\newcommand{\B}{\mathrm{B}}

% names for terms
\newcommand{\id}{\mathrm{id}}
\newcommand{\refl}{\mathrm{refl}}
\newcommand{\pair}{\mathrm{pair}}
\newcommand{\FunExt}{\mathrm{FunExt}}
\newcommand{\transp}{\mathrm{tr}}
\newcommand{\transpconst}{\mathrm{tconst}}
\newcommand{\ua}{\mathrm{ua}}
\newcommand{\fib}{\mathrm{fib}}
\newcommand{\pt}{\mathrm{pt}}

% category theory/algebra
\newcommand{\Hom}{\mathrm{Hom}}
\newcommand{\Ext}{\mathrm{Ext}}
\newcommand{\HomBun}{\mathit{Hom}}
\newcommand{\ExtBun}{\mathit{Ext}}
\newcommand{\Sh}{\mathrm{Sh}}
\newcommand{\yo}{\mathrm{y}}

% algebra
\newcommand{\inv}{\mathrm{inv}}
\newcommand{\divides}{\mathbin{|}}
\DeclareMathOperator{\AbGroup}{Ab}
\DeclareMathOperator{\im}{im}
\DeclareMathOperator{\coker}{coker}
\newcommand{\Tors}[1]{#1\text{-}\mathrm{Tors}}
\newcommand{\Mod}[1]{#1\text{-}\mathrm{Mod}}
\newcommand{\Vect}[2]{#1\text{-}\mathrm{Vect}_{#2}}
\newcommand{\fpMod}[1]{#1\text{-}\mathrm{Mod}_{\text{fp}}}
\newcommand{\Alg}[1]{#1\text{-}\mathrm{Alg}}
\newcommand{\Sym}{\mathrm{Sym}}

% algebraic geometry
\DeclareMathOperator{\Sp}{Sp}
\DeclareMathOperator{\Spec}{Spec}
\DeclareMathOperator{\Sch}{\mathrm{Sch}_{qc}}
\newcommand{\A}{\mathbb{A}}
\newcommand{\D}{\mathbb{D}}
\newcommand{\bP}{\mathbb{P}}
\newcommand{\Gm}{\mathbb{G}_m}
\newcommand{\OO}{\mathcal{O}}
\newcommand{\mm}{\mathfrak{m}}
\newcommand{\Bl}{\mathrm{Bl}}
\DeclareMathOperator{\PGL}{PGL}
\DeclareMathOperator{\GL}{GL}
\newcommand{\Pic}{\mathrm{Pic}}
\newcommand{\Gr}{\mathrm{Gr}}
\newcommand{\Aut}{\mathrm{Aut}}


% misc
\newcommand{\I}{\mathbb{I}}
\newcommand{\bD}{\mathbb{D}}
\newcommand{\notion}[1]{\emph{#1}\index{#1}}
\providecommand*\colonequiv{\vcentcolon\mspace{-1.2mu}\equiv}
\newcommand{\ignore}[1]{}
\newcommand{\rednote}[1]{{\color{red}(#1)}}

% cohesion
\DeclareMathOperator{\shape}{\textrm{\textesh}}

% condensed
\newcommand{\isSt}{\mathrm{isStone}}
\newcommand{\bS}{\mathbb{S}}

\newtheorem{axiom}{Axiom}

%This is a template for producing LIPIcs articles. 
%See lipics-v2021-authors-guidelines.pdf for further information.
%for A4 paper format use option "a4paper", for US-letter use option "letterpaper"
%for british hyphenation rules use option "UKenglish", for american hyphenation rules use option "USenglish"
%for section-numbered lemmas etc., use "numberwithinsect"
%for enabling cleveref support, use "cleveref"
%for enabling autoref support, use "autoref"
%for anonymousing the authors (e.g. for double-blind review), add "anonymous"
%for enabling thm-restate support, use "thm-restate"
%for enabling a two-column layout for the author/affilation part (only applicable for > 6 authors), use "authorcolumns"
%for producing a PDF according the PDF/A standard, add "pdfa"

%\pdfoutput=1 %uncomment to ensure pdflatex processing (mandatatory e.g. to submit to arXiv)
%\hideLIPIcs  %uncomment to remove references to LIPIcs series (logo, DOI, ...), e.g. when preparing a pre-final version to be uploaded to arXiv or another public repository

%\graphicspath{{./graphics/}}%helpful if your graphic files are in another directory

\bibliographystyle{plainurl}% the mandatory bibstyle

\title{A Foundation for Synthetic Stone Duality}

%\titlerunning{Dummy short title} %TODO optional, please use if title is longer than one line
%,  and 
\author{Felix {Cherubini}}{University of Gothenburg and Chalmers University of Technology, Sweden \and \url{felix-cherubini.de}}{felix.cherubini@posteo.de}{https://orcid.org/0000-0002-6589-1874}{}
\author{Thierry {Coquand}}{University of Gothenburg and Chalmers University of Technology, Sweden \and \url{https://www.cse.chalmers.se/~coquand/}}{Thierry.Coquand@cse.gu.se}{https://orcid.org/0000-0002-5429-5153}{}
\author{Freek {Geerligs}}{University of Gothenburg and Chalmers University of Technology, Sweden \and \url{https://www.gu.se/en/about/find-staff/freekgeerligs}}{geerligs@chalmers.se}{https://orcid.org/0009-0003-6938-4807}{}
\author{Hugo {Moeneclaey}}{University of Gothenburg and Chalmers University of Technology, Sweden \and \url{https://www.hugomoeneclaey.com/}}{hugomo@chalmers.se}{}{}

\authorrunning{F.\ Cherubini, T.\ Coquand, F.\ Geerligs and H.\ Moeneclaey}

\Copyright{Felix Cherubini, Thierry Coquand, Freek Geerligs and Hugo Moeneclaey}

\begin{CCSXML}
  <ccs2012>
  <concept>
  <concept_id>10003752.10003790.10011740</concept_id>
  <concept_desc>Theory of computation~Type theory</concept_desc>
  <concept_significance>500</concept_significance>
  </concept>
  </ccs2012>
\end{CCSXML}

\ccsdesc[100]{Theory of computation~Type theory}

\keywords{Homotopy Type Theory, Synthetic Topology, Cohomology}

\category{} %optional, e.g. invited paper

\relatedversion{} %optional, e.g. full version hosted on arXiv, HAL, or other respository/website
%\relatedversiondetails[linktext={opt. text shown instead of the URL}, cite=DBLP:books/mk/GrayR93]{Classification (e.g. Full Version, Extended Version, Previous Version}{URL to related version} %linktext and cite are optional

%\supplement{}%optional, e.g. related research data, source code, ... hosted on a repository like zenodo, figshare, GitHub, ...
%\supplementdetails[linktext={opt. text shown instead of the URL}, cite=DBLP:books/mk/GrayR93, subcategory={Description, Subcategory}, swhid={Software Heritage Identifier}]{General Classification (e.g. Software, Dataset, Model, ...)}{URL to related version} %linktext, cite, and subcategory are optional

%\funding{(Optional) general funding statement \dots}%optional, to capture a funding statement, which applies to all authors. Please enter author specific funding statements as fifth argument of the \author macro.

\acknowledgements{The idea to use the topological characterization of stone spaces as totally disconnected, compact Hausdorff spaces to prove \Cref{stone-sigma-closed} was explained to us by Martín Escardó.
We profited a lot from a discussion with Reid Barton and Johann Commelin. 
David Wärn noticed that Markov's principle (\Cref{MarkovPrinciple}) holds. 
At TYPES 2024, we had an interesting discussion with Bas Spitters on the topic of the article.
}%optional

%\nolinenumbers %uncomment to disable line numbering



%Editor-only macros:: begin (do not touch as author)%%%%%%%%%%%%%%%%%%%%%%%%%%%%%%%%%%
\EventEditors{John Q. Open and Joan R. Access}
\EventNoEds{2}
\EventLongTitle{42nd Conference on Very Important Topics (CVIT 2016)}
\EventShortTitle{CVIT 2016}
\EventAcronym{CVIT}
\EventYear{2016}
\EventDate{December 24--27, 2016}
\EventLocation{Little Whinging, United Kingdom}
\EventLogo{}
\SeriesVolume{42}
\ArticleNo{23}
%%%%%%%%%%%%%%%%%%%%%%%%%%%%%%%%%%%%%%%%%%%%%%%%%%%%%%

\begin{document}

\maketitle

%TODO mandatory: add short abstract of the document
\begin{abstract}
  The language of homotopy type theory has proved to be appropriate as an internal language for various higher toposes, 
for example with Synthetic Algebraic Geometry for the Zariski topos.
In this paper we apply such techniques to the higher topos corresponding to the light condensed sets 
of Dustin Clausen and Peter Scholze.
This seems to be an appropriate setting to develop synthetic topology, similar to the work of 
Martín Escardó.
To reason internally about light condensed sets, we use homotopy type theory extended with 4 axioms.
Our axioms are strong enough to prove Markov's principle, LLPO and the negation of WLPO. 
We also define a type of open propositions, inducing a topology on any type. 
This leads to a (synthetic) topological study of Stone and compact Hausdorff spaces. 
Indeed all functions are continuous in the sense that they respect this induced topology, 
and this topology is as expected for these class of types.
For example, any map from the unit interval to itself is continuous in the usual epsilon-delta sense.
We also use the synthetic homotopy theory 
given by the higher types of homotopy type theory to define and 
work with cohomology.
As an application, we compute the cohomology of the interval and use this to prove Brouwer's fixed point theorem
internally. 

\end{abstract}

\section*{Introduction}
The language of homotopy type theory is a  dependent type theory enriched with the univalence axiom and higher inductive types. It has proven exceptionnally well-suited to
develop homotopy theory in a synthetic way \cite{hott}. It also provides
the precision needed to analyze categorical models of type theory \cite{vanderweide2024}.
Moreover, the arguments in this language can be rather directly represented in proof assistants. We use homotopy type theory to give a synthetic development of topology, which is analogous to the work on synthetic algebraic geometry \cite{draft}. 

We introduce 
four axioms which seem sufficient for expressing and proving basic notions of topology, based on the light condensed
sets approach, introduced in \cite{Scholze}.
Interestingly, this development establishes strong connections with constructive mathematics \cite{Bishop},
particularly constructive reverse mathematics \cite{ReverseMathsBishop,HannesDiener}. Several of Brouwer's principles, such that
any real function on the unit interval is continuous, or the celebrated fan theorem, are consequences of this system
of axioms. Furthermore, we can also prove principles that are not intuitionistically valid, such as Markov's Principle,
or even the so-called Lesser Limited Principle of Omniscience, a principle well studied in constructive reverse mathematics,
which is {\em not} valid effectively.

This development also aligns closely with the program of Synthetic
Topology \cite{SyntheticTopologyEscardo,SyntheticTopologyLesnik,abstractstone}:
there is a dominance of open propositions, providing any type with an intrinsic
topology, and we capture in this way synthetically the notion of (second-countable) compact Hausdorff spaces.
While working on this axiom system, we learnt about the related work \cite{bc24}, which provides a different axiomatisation
at the set level. We show that some of their axioms are consequences of our axiom system. In particular, we can introduce
in our setting a notion of ``Overtly Discrete'' spaces, dual in some way to the notion of compact Hausforff spaces, like
in Synthetic Topology\footnote{We actually have a
derivation of their ``directed univalence'', but this will be presented in a following paper.}.

A central theme of homotopy type theory is that the notion of {\em type} is more general than the notion of {\em set}. We illustrate
this theme here as well: we can form in our setting the types of Stone spaces and of compact Hausdorff spaces
(types which don't form a set but a groupoid),
and show these types are
closed under sigma types. It would be impossible to formulate such properties in the setting of simple type or set theory.
Additionally, leveraging the elegant definition of cohomology groups in homotopy type theory \cite{hott}, which relies
on higher types that are not sets, we prove, in a purely axiomatic way,
a special case of a theorem of Dyckhoff \cite{dyckhoff76}, describing
the cohomology of compact Hausdorff spaces. This characterisation also supports a type-theoretic proof of
Brouwer's fixed point theorem, similar to the proof in \cite{shulman-Brouwer-fixed-point}. In our setting the theorem can be formulated in the usual way, and not in an approximated form.

We expect our axioms to be validated by the interpretation of homotopy type theory into the higher topos of light condensed sets \cite{shulman2019all}, although checking this rigorously is still work in progress. We even expect this to be valid in a constructive metatheory
using the work \cite{CRS21}. It is important to stress that what we capture in this axiomatic way are the properties of light condensed
sets that are {\em internally} valid. David W\"arn \cite{warn2024} has proved that an important property of abelian
groups in the setting of light condensed sets, is {\em not} valid internally and thus cannot be proved in this axiomatic context.
We believe however that our axiom system can be convenient for proving the results that are internally valid, as we hope
is illustrated by the present paper. We also conjecture that the present axiom system is actually {\em complete}
for the properties that are internally valid.


\section{Stone duality}
\subsection{Preliminaries}
%
%In this section, we introduce the type of countably presented Boolean algebras $\Boole$ and of Stone spaces $\Stone$. 
%Both of these types carry a natural category structure. 
%In later sections, we will axiomatize an anti-equivalence between these categories, 
%which is classically valid and called Stone duality. 
\begin{definition}
  A countably presented Boolean algebra $B$ is a Boolean algebra such that there merely are 
  countable sets $I,J$, 
  a set of generators $g_i,~{i\in I}$ and a set $f_j,~{j\in J}$ 
  of Boolean expressions over these generators 
  such that $B$ is equivalent to the quotient of the free Boolean 
  algebra over the generators by the relations
  $f_j=0$. We denote this algebra by $2[I]/(f_j)_{j:J}$.
\end{definition} 
\begin{remark}\label{BooleAsCQuotient}
By countable in the previous definition we mean sets that are 
merely equal to some decidable in $\N$. 
Note that any countably presented algebra is also merely of the form 
$2[\N]\rangle / (r_n)_{n:\N}$.
%, if we add dummy variables that we equate to $0$, and dummy relations that equate $0$ to itself.
% 0 is not special here right? 
\end{remark}


%We will call the family $(f_j)_{j\in J}$ as above a set of relations. 
%If $I,J$ are finite, we call $B$ a finitely presented Boolean algebra. 
%Once we have postulated the axiom of dependent choice, 
%in \Cref{secBooleAsColimits}
%we will be able to show that every countably presented algebra 
%is actually a colimit of a sequence of finitely presented Boolean algebras.
%They are therefore dual to pro-finite objects, which are used 
%in the theory of light condensed sets \cite{Scholze,Dagur,TODO}.

\begin{remark}
  We denote the type of countably presented Boolean algebras $\Boole$. 
  Note that this type does not depend on a choice of universe. 
  Also note $\Boole$ has a natural category structure. 
\end{remark}

\begin{example}
  If both the set of generators and relations are empty, we have the Boolean algebra $2$.
  The underlying set is $\{0,1\}$ and $0\neq_2 1$.
  Remark that $2$ is initial in $\Boole$. 
\end{example}
%\begin{remark}
%Note that any Boolean algebra must contain the elements $0,1$. 
%Therefore, $2$ is initial in $\Boole$. 
%\end{remark} 
%We can therefore use it to define points of objects in the category dual to that of countably presented Boolean algebras. 
%
%\subsection{Stone spaces}
\begin{definition}
  For $B$ a countably presented Boolean algebra, 
  we define $Sp(B)$ as the set of Boolean morphisms from $B$ to $2$. 
\end{definition}
\begin{definition}
  We define the predicate on types $\isSt$ by 
  \begin{equation}
    \isSt(X) := \sum\limits_{B : Boole} X = Sp(B)
  \end{equation} 
  A type $X$ is called \textit{Stone} if $\isSt(X)$ is inhabited.
\end{definition}

%\subsection{Examples}
\begin{example}
  \label{boolean-algebra-examples}
  \begin{enumerate}[(i)]
  \item There is only one Boolean map $2\to 2$, thus $Sp(2)$ is the singleton type $\top$. 
  \item   
    The tivial Boolean algebra is given by $2/(1)$. 
    We have $0=1$ in the trivial Boolean algebra. 
    As there cannot be a map from the trivial Boolean algebra into $2$ preserving both $0,1$, 
    the corresponding Stone space is the empty type $\bot$, 
  \item\label{ExampleBAunderCantor}   
    We denote by $C$ the Boolean algebra $2[\N]$.
    %given by $\N$ as a set of generators and no relations. We write $p_n$ for the generator corresponding to $n$.
    A morphism $C\to 2$ corresponds to a function $\mathbb N\to 2$, 
    which is a binary sequence. 
    The Stone space $Sp(C)$ of these binary sequences is denoted 
    $2^{\N}$ and called \notion{Cantor space}.
  \item\label{ExampleBAunderNinfty}
    We denote by $B_\infty$ for $C/(g_m\wedge g_n)_{m\neq n}$.
    A morphism $B_\infty\to 2$ corresponds to a function 
    $\mathbb N \to 2$ that hits $1$ at most once. 
    The corresponding Stone space is denoted by $\N_\infty$.
  \end{enumerate}
\end{example}

\begin{lemma}\label{ClosedPropAsSpectrum}
  For $\alpha:2^\N$, we have an equivalence of propositions: 
  $$
    (\forall_{n:\N} \alpha(n) = 0 )\leftrightarrow Sp(2/(\alpha(n))_{n:\N}).
  $$
\end{lemma}
\begin{proof}
  There is at most one $x:2\to 2$, and it can only satisfy 
  $x(\alpha(n)) = 0$ for all $n:\N$ iff 
  $\alpha(n) \neq_2 1$ for all $n:\N$. 
  As $2$ has underlying set $\{0,1\}$, we have $(\alpha(n) \neq_2 1) \to (\alpha(n) =_2 0)$. 
\end{proof}

\begin{remark}\label{BinftyTermsWriting}
%  In \Cref{N-co-fin-cp}, we will show 
  It can be shown that $B_\infty$ is equivalent to the Boolean algebra on 
  subsets of $\N$ which are finite or co-finite. 
  Under this equivalence, the generator $g_n$ is sent to the singleton $\{n\}$. 
  Because of this, we have that any $b:B_\infty$ can be written 
  either as $\bigvee_{i\in I_0} p_i$ or as $\bigwedge_{i\in I_0} \neg p_i$ for some finite $I_0\subseteq \N$. 
\end{remark}


%\begin{remark}
%  As Boolean algebras are rings, any relation of the form $f=g$ with both $f,g$ Boolean expressions 
%  can be written as $h=0$ with $h=f-g$ a Boolean expression. 
%\end{remark} 




%In this section, we will introduce the basic rules we'll use in this paper. 
%We will first state our axioms, and then draw out some first consequences.
%Most notably, 
%we will see that Markov's principle (\Cref{MarkovPrinciple}) and the 
%lesser limited principle of omniscience (\Cref{LLPO}) can be shown. 
%%There are equivalent axiom systems we could have stated instead, which we discuss in \Cref{NotesOnAxioms}.
\subsection{Axioms}\label{Axioms}
%In this section, we will state our axioms. 
%In \Cref{NotesOnAxioms}, we will discuss alternative versions of our axiom system. 
\begin{axiom}[Stone duality]\label{AxStoneDuality}
  For any $B:\Boole$, 
  the evaluation map $B\rightarrow  2^{Sp(B)}$ is an isomorphism.
\end{axiom} 

%\begin{axiom}[Propositional completeness]
%  For $S:\Stone$, we have that $\neg \neg S \leftrightarrow || S ||$
%\end{axiom}

\begin{axiom}[Surjections are formal surjections]\label{SurjectionsAreFormalSurjections}
  For $g:B\to C$ a map in $\Boole$, $g$ is injective if and only if
  $(-)\circ g: Sp(C) \to Sp(B)$ is surjective. 
%  A map $f:Sp(B')\to Sp(B)$ is surjective iff the corresponding map $B \to B'$ is injective.
\end{axiom} 
%
%\begin{lemma}\label{LemSurjectionsFormalToCompleteness}
% For $S:\Stone$, we have that $\neg \neg S \to || S ||$
%\end{lemma}
%\begin{proof}
%  First, assume that surjections are formal surjections. 
%  Let $B:\Boole$ and suppose $\neg \neg Sp(B)$. 
%  %Note that if $0=1$ in $B$, then $Sp(B) =\emptyset$, meaning $\neg Sp(B)$. 
%  %Therefore, we have $0\neq 1$ in $B$. 
%  We will show that the map $f:2\to B$ is injective. 
%  Let $f:2 \to B$, note that if $f(0) = f(1)$ then $0=1$ in $B$, 
%  If $0=1$ in $B$, there are no maps $B\to 2$ preserving $0$ and $1$, thus $\neg Sp(B)$. 
%  This is a contradiction with $\neg \neg Sp(B)$. Thus we may conclude that $f(0)\neq f(1)$. 
%  Hence by case distinction on $2$ we can show $f$ we have that $f x = f y$ implies $ x= y$. Thus 
%  $f$ is injective thus the map $Sp(B) \to Sp(2) = 1$ is surjective, thus $Sp(B)$ is merely inhabited. 
%\end{proof} 
%Actually, we will see in \Cref{CorDoubleNegToAx2} that the converse is also true. 

%\begin{axiom}[Local choice]
%  Whenever $S$ Stone and $E\twoheadrightarrow S$ surjective, then there is some $T$ Stone,
%    a surjection $T \twoheadrightarrow S$ and a map $T\to E$ 
%    such that the following diagram commutes:
%    \begin{equation}\begin{tikzcd}
%      E \arrow[d,""',two heads]\\
%      S & \arrow[l, "", two heads, dashed] T\arrow[lu, ""',dashed ]
%    \end{tikzcd}\end{equation}  
%\end{axiom} 
%\begin{axiom}[Local choice]\label{AxLocalChoice}
%  Whenever we have $S:\Stone$, $E,F$ arbitrary types, a map $f:S \to F$ and a 
%  surjection $e:E \twoheadrightarrow F$, 
%  there exists a Stone space $T$, a surjective map 
%  $T\twoheadrightarrow S$ and an arrow $T\to E$ making the following diagram commute:
%    \begin{equation}\begin{tikzcd}
%      T \arrow[d,dashed, two heads ] \arrow[r,dashed]&  E \arrow[d,""',two heads, "e"]\\
%      S  \arrow[r, "f"] & F
%    \end{tikzcd}\end{equation}  
%\end{axiom}

%\begin{axiom}[Local choice]\label{AxLocalChoice}
%  Whenever we have $S:\Stone$, $X$ an arbitrary type and a predicate $P:S \times X \to \Prop$, 
%  such that $\forall_{s:S} \exists_{x:X} P(s,x)$, then there merely exists some $T:\Stone$ with surjection 
%  $q:T\twoheadrightarrow S$ and a function $\Pi_{t:T} \Sigma_{x:X} P(q(t), x)$. 
%\end{axiom}
%\begin{axiom}[Local choice]\label{AxLocalChoice}
%  Whenever we have $S:\Stone$, and some type family $P:S\to\Type$ such that 
%  $\Pi_{s:S} ||P s||$, then there 
%  merely exists some $T:\Stone$ and surjection $q:T\to S$ with 
%$  \Pi_{t:T} P(q(t))$.
%\end{axiom}
\begin{axiom}[Local choice]\label{AxLocalChoice}
  Whenever we have $B:\Boole$, and some type family $P$ over $Sp(B)$ with 
  $\Pi_{s:Sp(B)} \propTrunc{P(s)}$, then there 
  merely exists some $C:\Boole$ and surjection $q:Sp(C)\to Sp(B)$ with 
$  \Pi_{t:Sp(C)} P(q(t))$.
\end{axiom}

\begin{axiom}[Dependent choice]\label{axDependentChoice}
Given types $(E_n)_{n:\N}$ with for all $n:\N$ a surjection $E_{n+1}\twoheadrightarrow E_n$, the projection from the sequential limit $\lim_kE_k$ to $E_0$ is surjective.
\end{axiom}
%\begin{remark}
%  Local choice can also be formulated as follows:
%  whenever we have $S:\Stone$, $E,F$ arbitrary types, a map $f:S \to F$ and a 
%  surjection $e:E \twoheadrightarrow F$, 
%  there exists a Stone space $T$, a surjective map 
%  $T\twoheadrightarrow S$ and an arrow $T\to E$ making the following diagram commute:
%    \begin{equation}\begin{tikzcd}
%      T \arrow[d,dashed, two heads ] \arrow[r,dashed]&  E \arrow[d,""',two heads, "e"]\\
%      S  \arrow[r, "f"] & F
%    \end{tikzcd}\end{equation}  
%\end{remark}

\subsection{Anti-equivalence of $\Boole$ and $\Stone$}

\begin{remark}\label{SpIsAntiEquivalence}
Stone types will take over the role of affine scheme from \cite{draft}, 
and we repeat some results here. 
Analogously to Lemma 3.1.2 of \cite{draft}, 
for $X$ Stone, Stone duality tells us that $X = Sp(2^X)$. 
%
Proposition 2.2.1 of \cite{draft} now says that 
$Sp$ gives a natural equivalence 
\begin{equation}
   Hom_{\Boole} (A, B) = (Sp(B) \to Sp(A))
\end{equation}
Therefore $Sp$ is an embedding from $\Boole$ to any universe of types, and $\isSt$ is a proposition.

Its image, $\Stone$ also has a natural category structure.
By the above and Lemma 9.4.5 of \cite{hott}, the map $Sp$ defines a dual equivalence of categories between $\Boole$ and $\Stone$.
\end{remark}

\begin{lemma}\label{SpectrumEmptyIff01Equal}
  For $B:\Boole$, we have $0=_B1$ iff $\neg Sp(B)$.
\end{lemma}
\begin{proof}
  If $0=_B1$, there is no map $B\to 2$ respecting both $0$ and $1$, thus $\neg Sp(B)$. 
  Conversely, if $\neg Sp(B)$, then 
  $Sp(B)$ equals $\bot$, the spectrum of the trivial Boolean algebra. 
  As $Sp$ is an embedding, $B$ is equivalent to the trivial Boolean algebra, hence $0=_B1$. 
\end{proof}
\begin{corollary}\label{LemSurjectionsFormalToCompleteness}
 For $S:\Stone$, we have that $\neg \neg S \to || S ||$
\end{corollary}
\begin{proof}
  Let $B:\Boole$ and suppose $\neg \neg Sp(B)$. 
  Let $f:2 \to B$. If $f(0) = f(1)$ then $0=1$ in $B$, thus $\neg Sp(B)$, 
  contradicting our assumption. Hence $f(0)\neq f(1)$. 
  Hence by case distinction on $2$ we can show that $f$ is injective. 
%  $f x = f y$ implies $ x= y$. 
%  Thus $f$ is injective and 
  By \Cref{SurjectionsAreFormalSurjections} the map $Sp(B) \to Sp(2)$ is surjective, 
  thus $Sp(B)$ is merely inhabited. 
\end{proof} 

%\begin{corollary}\label{MoreConcreteCompleteness}
%  By the above and propositional completeness, we have that $||Sp(B)||$ iff $0\neq_B1$. 
%\end{corollary}



%SurjectionsFormalSurjections%We conclude this section on the anti-equivalence of Stone and $\Boole$ by a relating surjections to injections. 
%SurjectionsFormalSurjections%This theorem is actually equivalent to completeness of propositional logic, which we'll discuss in 
%SurjectionsFormalSurjections%\Cref{NotesOnAxioms}. 
%SurjectionsFormalSurjections%
%SurjectionsFormalSurjections%\begin{theorem}\label{FormalSurjectionsAreSurjections}
%SurjectionsFormalSurjections%  Let $f:A\to B$ be a map of countably presented Boolean algebras. 
%SurjectionsFormalSurjections%  If $f$ is injective, then the corresponding map $(\cdot) \circ f : Sp(B) \to Sp(A)$ is surjective. 
%SurjectionsFormalSurjections%\end{theorem}
%SurjectionsFormalSurjections%
%SurjectionsFormalSurjections%\begin{proof}
%SurjectionsFormalSurjections%  Assume $f$ injective and let $x:Sp(A)$.
%SurjectionsFormalSurjections%  By \Cref{FiberConstruction}, we have that $\left(\sum\limits_{y:Sp(B)} y\circ f = x \right) = Sp(B/R) $
%SurjectionsFormalSurjections%  for $R=f(G)$ for some countable $G\subseteq A$ with $x(g) = 0$ for all $g\in G$. 
%SurjectionsFormalSurjections%  By propositional completeness and \Cref{SpectrumEmptyIff01Equal}, 
%SurjectionsFormalSurjections%  it's sufficient to show that $0\neq_{B/R}1$. 
%SurjectionsFormalSurjections%  Note that $0=_{B/R} 1$ iff 
%SurjectionsFormalSurjections%  $1 =_B \bigvee R_0$ for some $R_0\subseteq R$ finite. 
%SurjectionsFormalSurjections%  But then $$1 = \bigvee f(G_0) = f(\bigvee  G_0)$$ for some $G_0\subseteq G$ finite. 
%SurjectionsFormalSurjections%  And as $f$ is injective, $\bigvee G_0 = 1$. 
%SurjectionsFormalSurjections%  However, 
%SurjectionsFormalSurjections%  $$
%SurjectionsFormalSurjections%  x(\bigvee G_0) = 
%SurjectionsFormalSurjections%  x(\bigvee_{g\in G_0} g ) = \bigvee_{g \in G_0} x(g) = \bigvee_{g\in G_0} 0 = 0$$
%SurjectionsFormalSurjections%  And as $x(1) = 1$, we get a contradiction. Therefore $0\neq_{B/R} 1$ as required. 
%SurjectionsFormalSurjections%\end{proof}  
%SurjectionsFormalSurjections%The converse to the above theorem is true as well, regardless of propositional completeness:
%SurjectionsFormalSurjections%\begin{lemma}\label{SurjectionsAreFormalSurjections}
%SurjectionsFormalSurjections%If $f:A\to B$ is a map in $\Boole$ and $(\cdot) \circ f :Sp(B) \to Sp(A)$ is surjective, 
%SurjectionsFormalSurjections%$f$ is injective. 
%SurjectionsFormalSurjections%\end{lemma}
%SurjectionsFormalSurjections%\begin{proof}
%SurjectionsFormalSurjections%  Suppose precomposition with $f$ is surjective. 
%SurjectionsFormalSurjections%  Let $a:A$ be such that $f(a)= 0$. 
%SurjectionsFormalSurjections%  By assumption, for every $x:A\to 2$, there is a $y:B\to 2$ with $y\circ f = x$. 
%SurjectionsFormalSurjections%  Consequentely $x(a) = y(f(a)) = y(0) = 0$. 
%SurjectionsFormalSurjections%  So $x(a) = 0$ for every $x:Sp(A)$. 
%SurjectionsFormalSurjections%  Thus $Sp(A) = Sp(A/\{a\})$, and as $Sp$ is an embedding, 
%SurjectionsFormalSurjections%  $A \simeq A/\{a\}$, and $a = 0$ in $A$. 
%SurjectionsFormalSurjections%  So whenever $f(a) = 0$, we have $a=0$. Thus $f$ is injective. 
%SurjectionsFormalSurjections%\end{proof}

\subsection{Principles of omniscience}
In constructive mathematics, we do not assume the law of excluded middle (LEM).
There are some principles called principles of omniscience that are weaker than LEM, which can be used to describe 
how close a logical system is to satisfying LEM.
References on these principles include \cite{HannesDiener, ReverseMathsBishop}.
In this section, we will show that two of them (MP and LLPO) hold, 
and one (WLPO) fails in our system.

\begin{theorem}[The negation of the weak lesser principle of omniscience ($\neg$WLPO)]\label{NotWLPO}
  \begin{equation}
    \neg \forall_{\alpha:2^\N} 
    ((\forall_{n:\N} \alpha(n) = 0 ) \vee \neg (\forall_{n:\N} \alpha(n) = 0))
  \end{equation}
%  We cannot decide for general $\alpha:2^\N$, whether $\forall_{n:\mathbb N} \alpha(n) = 0$.
%  It is not the case that the statement %There is no method which given $\alpha:2^\mathbb N$ decides whether 
%  $\forall_{n:\mathbb N} \alpha(n) = 0$ is decidable for general $\alpha:2^\mathbb N$. 
\end{theorem}
\begin{proof}
%  Such a decision method is a function 
  Let $f:2^\mathbb N \to 2$ such that 
  $f(\alpha) = 0$ iff $\forall_{n:\mathbb N} \alpha (n)= 0$. 
  By \Cref{AxStoneDuality}, there is some $c:C$ with 
  $f(\alpha) = 0 \leftrightarrow \alpha(c) = 0$. 
  We can express $c$ using finitely many generators $(g_n)_{n\leq N}$. 
  Now consider $\beta,\gamma:2^\N$ given by 
  $\beta(g_n) = 0$ for all $n:\mathbb N$ and
  $\gamma(g_n) = 0$ iff $n\leq N$. 
  As $\beta, \gamma$ are equal on $(g_n)_{n\leq N}$, we have $\beta(c) = \gamma(c)$. 
  However, $f(\beta) = 0$ and $f(\gamma) = 1$, giving a contradiction as required. 
%  We thus have a contradiction, thus a decision method as required doesn't exist. 
\end{proof}

The following result is due to David W\"arn:
\begin{theorem}[Markov's principle (MP)]\label{MarkovPrinciple}
  For $\alpha:\Noo$, we have that 
  \begin{equation}
    (\neg (\forall_{n:\mathbb N} \alpha (n)= 0)) \to \Sigma_{n:\mathbb N} \alpha (n)= 1
  \end{equation}
\end{theorem}
\begin{proof}
  By \Cref{ClosedPropAsSpectrum}, we have that $\neg(\forall_{n:\N} \alpha(n) = 0)$ implies that 
  $Sp(2/(\alpha(n))_{n:\N}$ is empty. 
%  We will show that the spectrum of $2/(\alpha(n))_{n:\N}$ is empty. 
%  Suppose $x:2\to 2$, if  $x(\alpha(n)) = 0$, we get $\alpha(n) \neq 1$. 
%  Thus if $\neg (\forall_{n:\N} \alpha(n) = 0$, we have $\neg Sp(2/(\alpha(n))_{n:\N})$.
  Hence $2/(\alpha(n))_{n:\N}$ is trivial by \Cref{SpectrumEmptyIff01Equal}. 
  Then there is a finite subset $N_0\subseteq \N$ with $\bigvee_{i:N_0} \alpha(i) = 1$. 
  As $\alpha(i) \in \{0,1\}$ and $\alpha(i) = 1$ for at most one $i:\N$, 
  there exists an unique $n\in\mathbb N$ with $\alpha(n) = 1$. 
%  Assume $\neg (\forall_{n:\mathbb N} \alpha (n)= 0)$.
%  It is sufficient to show that $2/\{\alpha(n)|n\in\N\}$ is the trivial Boolean algebra. 
%  It will then follow that there is a finite subset $N_0\subseteq \N$ 
%  with $\bigvee_{i:N_0} \alpha(i) = 1$.
%  As $\alpha(i) \in \{0,1\}$ and $\alpha(i) = 1$ for at most one $i$, it then follows that 
%  there exists an unique $n\in\mathbb N$ with $\alpha(n) = 1$. 
%%
%  To show that $2/\{\alpha(n)|n\in\N\}$ is trivial, we will show it has an empty spectrum. 
%  Suppose $x: 2 \to 2$ is such that $x(\alpha(n)) = 0$ for every $n:\N$. 
%  As $x(1) = 1$, we must have for every $n:\N$ that $\alpha(n) \neq 1$. 
%  But then $\alpha(n) = 0$, contradicting our assumption. 
%  We get a contradicition and there thus there are no points in the spectrum of $2/\{\alpha(n)|n\in\N\}$ as required. 
\end{proof}

\begin{corollary}
  For $\alpha:2^\mathbb N$, we have that 
  \begin{equation}
    (\neg (\forall_{n:\mathbb N} \alpha (n)= 0)) \to \Sigma_{n:\mathbb N} \alpha (n)= 1
  \end{equation}
\end{corollary}
\begin{proof}
  Given $\alpha:2^\mathbb N$, consider the sequence $\alpha':\Noo$ satisfying $\alpha'(n) = 1$ iff 
  $n$ is minimal with $\alpha(n) = 1$. Then apply the above theorem.
\end{proof}

\begin{theorem}[The lesser limited principle of omniscience (LLPO)]\label{LLPO}
  For $\alpha:\N_\infty$, 
  we have that 
  \begin{equation}\label{eqnLLPO}
    \forall_{k:\N} \alpha(2k) = 0  \vee \forall_{k:\N} \alpha(2k+1) = 0
  \end{equation}
\end{theorem}
\begin{proof}
%
%  We first will define a map $f:B_\infty \to B_\infty \times B_\infty$. 
%  Because of \Cref{rmkMorphismsOutOfQuotient}, it is sufficient to define $f$ on $(p_n)_{n:\N}$ with 
%  $f(p_n) \wedge f(p_m) = (0,0)$ for $n\neq m$. 
%  To define $f(p_n)$, we use a case distinction on whether $n$ is odd or even. 
  Define $f:B_\infty \to B_\infty \times B_\infty$ as follows:
  \begin{equation}\label{eqnLLPOProofMap}
    f(p_n) =\begin{cases}
      (p_k,0) \text{ if } n = 2k\\
      (0,p_k) \text{ if } n = 2k+1\\
    \end{cases}
  \end{equation}
  Note that $f$ is well-defined as map in $\Boole$. 
 % , can make a case distinction on parity. 
%  By making a case distinction on $n,m$ being odd or even, 
%  we can see that 
%  $f(p_n) \wedge f(p_m) = (0,0)$ when $n\neq m$, thus $f$ is well-defined. 
  We claim $f$ is injective. Assume $f(x) = 0$, 
  to see that $x=0$, we make a case distinction on whether $x$ corresponds to a finite or a cofinite set as in \Cref{BinftyTermsWriting}.
%
%  We also claim it is injective.
%  Now let $x:B_\infty$ with $f(x) = 0$. 
  We denote $E,O\subseteq \N$ for the even and odd numbers respectively. 
%  and we make a case distincition based on \Cref{BinftyTermsWriting}.
  \begin{itemize}
    \item Suppose 
      $x = \bigvee_{i\in I_0} g_i$ with $I_0$ finite. 
      Then 
      $$f(x) = (\bigvee_{i\in I_0 \cap E } g_{\frac i2} , \bigvee_{i\in I_0 \cap O } g_{\frac {i-1}2} ) = (0,0)$$
      As $g_j\neq 0$ for all $j\in\N$, we must have $I_0 \cap E = \emptyset = I_0 \cap O$. 
      Thus $I_0= \emptyset$, and $x = 0$. 
    \item Suppose 
%      Let $x$ correspond to a cofinite subset of $\N$. Write 
      $x = \bigwedge_{j\in J} \neg g_j$ with $J$ finite. % for $J$ finite. 
      We will derive a contradiction. %, from which we can conclude that $x=0$ after all. 
      Note that   
      $$f(x) = (\bigwedge_{j\in J \cap E } \neg g_j , \bigwedge_{j\in J \cap O } \neg g_j ) = (0,0)$$
%      As $f(x) = (0,0)$, we have that 
%      $\bigwedge_{j\in J \cap E } \neg p_j =0$ and
%      $\bigwedge_{j\in J \cap O } \neg p_j  = 0$.
      However, any finite meet of negations corresponds to a cofinite set, hence is nonzero. 
      We get a contradiction and conclude $x=0$. 
%      However, any finite meet of negations will correspond to a cofinite set,
%      in particular it will not correspond to the empty set, and thus not be $0$.
%      Thus $f(x)\neq 0$, contradicting the assumption that $f(x) = 0$, hence $x=0$ ex falso. 
  \end{itemize}
%  In both cases, we conclude $x=0$, thus $f$ is injective. 
  By \Cref{SurjectionsAreFormalSurjections},
%  \Cref{FormalSurjectionsAreSurjections}, 
  $f$ corresponds to a surjection 
  $s:\Noo + \Noo \to \Noo$.
  Thus for $\alpha : \Noo$, 
  there exists some $x:\Noo + \Noo$ such that $s x = \alpha$. 
  If $x = inl(\beta)$, 
  for any $k:\N$, we have that 
  $$\alpha (g_{2k+1}) = s(x) (g_{2k+1}) = x(f(g_{2k+1})) = inl(\beta) (0,g_k)  = \beta(0) = 0.$$
  Similarly, if $x = inr(\beta)$, we have $\alpha(g_{2k}) = 0$ for all $k:\N$. 
  Thus \Cref{eqnLLPO} holds for $\alpha$ as required. 
\end{proof}
As the following shows, our use of \Cref{SurjectionsAreFormalSurjections} was non-trivial: 
%The use of \Cref{FormalSurjectionsAreSurjections}, and hence of propositional completeness, 
%was helpful in the above proof, as the following shows:
\begin{lemma}
  The function $f$  as in \Cref{eqnLLPOProofMap} does not have a retraction. 
\end{lemma}
\begin{proof}
  Suppose $r:B_\infty \times B_\infty \to B_\infty$ is a retraction of $f$. 
  Note that $r(0,1):B_\infty$ is expressable using only finitely many generators $(g_n)_{n\leq N}$
  Note that $r(0,1) \geq r(0,g_k) = g_{2k+1}$ for all $k:\N$. 
  As a consequence, $r(0,1)$ cannot be of the form $\bigvee_{i\in I_0} g_i$, and by \Cref{BinftyTermsWriting}, 
  $r(0,1)$ corresponds to a cofinite subset of $\N$. % = \bigwedge_{i:I_0} \neg p_i$, where $i\leq N$ for $i\in I_0$. 
  By similar reasoning so does $r(1,0)$.% corresponds to a cofinite subset of $\N$. 
  But the intersection of cofinite subsets is cofinite, while 
  $$r(0,1) \wedge r(1,0) = r( (1,0) \wedge (0,1)) = r(0,0) = 0$$
  which gives a contradiction. Thus no retraction exists. 
\end{proof}


%We finish with an equivalent formulation of LLPO:
%
%
%\begin{lemma}\label{corAlternativeLLPO}
%  Let $(\phi_n)_{n:\N}, (\psi_m)_{m:\N}$ be families of decidable propositions indexed over $\N$.
%  We then have 
%  \begin{equation}
%    (\forall_{n:\N} \forall_{m:\N} (\phi_n \vee \psi_m) )
%    \leftrightarrow
%    ((\forall_{n:\N} \phi_n) \vee (\forall_{m:\N} \psi_m) )
%  \end{equation}
%\end{lemma}
%\begin{proof}
%  See \cite{HannesDiener, ReverseMathsBishop}
%\end{proof}
%\begin{proof}
%  Note that the implication from right to left in the above equation always holds.
%  Assume that for all $m,n:\mathbb N$ we have $\phi_n\vee \psi_m$ 
%  Consider the sequence $\alpha:2^\mathbb N$ where $\alpha(2n) = 0$ iff $\phi_n$ and 
%  $\alpha(2m+1) = 0$ iff $\psi_m$. 
%  Let $\beta:\Noo$ be such that $\beta(i) = 1$ iff $i$ is minimal with $\alpha(i) = 1$
%  By LLPO, we have that 
%  $\beta$ is $0$ on all odd entries or on all even entries. 
%  Suppose that $\beta$ hits $0$ on all odd entries. 
%  We will show $\psi_m$ for all $m:\N$. 
%  As $\beta(2m+1) = 0$, there are two options:
%  \begin{itemize}
%    	\item If $\alpha(l)=0$ for all $l\leq 2m+1$. Then in particular $\alpha(2m+1)=0$ and $\psi_m$ holds.
%	\item Otherwise there is some $l<2m+1$ with $\beta(l) = 1$. 
%  As $\beta$ hits $0$ on odd entries, $l$ is even. 
%  So $\alpha(2n) = 1$ for $n = \frac{l}2$, meaning that $\neg \phi_n$. 
%  By assumption, $\phi_n \vee \psi_m$ holds, hence $\psi_m$ must hold. 
%  Thus for all $m:\N$, we have $\psi_m$ if $\beta$ hits $0$ on all odd entries. 
%  By a symmetric argument, if $\beta$ hits $0$ on all even entries, we have $\phi_n$ for all $n:\N$. 
%  We conclude that 
%  $((\forall_{n:\N} \phi_n) \vee (\forall_{m:\N} \psi_m) )$ 
%  as required. 
%  \end{itemize}
%\end{proof}
%
%\begin{remark}
%Note that the above statement implies LLPO as $\alpha(2n) =0 \vee \alpha(2m+1) =0$ for all $n,m:\mathbb N$ if $\alpha:\Noo$. 
%\end{remark}

%In this section, we will define the types of open and closed propositions. 
%These will allow us to define a (synthetic) topology  \cite{SyntheticTopologyLesnik} on any type.
%We will study this topology on Stone types in particular.
%
\subsection{Open and closed propositions}
In this section we will introduce a topology on the type of propositions, and 
study their logical properties.
We think of open and closed propositions respectively as countable disjunctions and conjunctions of decidable propositions.
Such a definition is universe-independent, and can be made internally.

\begin{definition}
A proposition $P$ is open (resp. closed) if there exists some $\alpha:2^\N$ such that $P \leftrightarrow \exists_{n:\mathbb N} \alpha_n = 0$ (resp. $P \leftrightarrow \forall_{n:\mathbb N} \alpha_n = 0$). We denote by $\Open$ and $\Closed$ the types of open and closed propositions.
\end{definition}

\begin{remark}\label{rmkOpenClosedNegation}
  The negation of an open proposition is closed, 
  and by MP (\Cref{MarkovPrinciple}), the negation of a closed proposition is open %. 
%  Also by MP, we have 
  and both open, closed propositions are $\neg\neg$-stable. 
%  and $\neg \neg P \to P$ whenever $P$ is open or closed. 
%  By the negation of WLPO (\Cref{NotWLPO}), 
  By $\neg$WLPO (\Cref{NotWLPO}), 
  not every closed proposition is decidable. 
  Therefore, not every open proposition is decidable. 
  % Both therefore and similarly can be used here, by a similar proof we can show it, or we can use that 
  % if $P$ is closed and $\neg P$ is decidable, so is $\neg \neg P = P$. 
  Every decidable proposition is both open and closed.
%  and in \Cref{ClopenDecidable} we shall see the converse. 
\end{remark}
\begin{lemma}
  We have the following results on open and closed propositions:
  \begin{itemize}
%    \item Closed propositions are closed under finite disjunctions. 
    \item Closed propositions are closed under countable conjunctions. 
    \item Open propositions are closed under finite conjunctions. 
    \item Open propositions are closed under countable disjunctions. 
  \end{itemize}
\end{lemma}
\begin{proof}
%  By Proposition 1.4.1 of \cite{HannesDiener}, LLPO (\Cref{LLPO}) is equivalent to the statement that 
%  the disjunction of two closed propositions are closed. 
  The statements have similar proofs, and we only present the proof that closed propositions are closed under 
  countable conjunctions. 
  Let $(P_n)_{n:\N}$ be a countable family of closed propositions. 
  By countable choice, for each 
  $n:\N$ we have an $\alpha_n:2^\N $ 
  such that $P_n \leftrightarrow \forall_{m:\N} \alpha_{n,m} =0$. 
  Consider a surjection $s:\N \twoheadrightarrow \N \times \N$, and let 
%  Let 
%  $$\beta_k = \alpha_{s(k)}.$$
  $\beta_k = \alpha_{s(k)}.$
  Note that $\forall_{k:\N} \beta_k = 0$ if and only if 
%  $\forall_{m,n:\N}\alpha_{m,n} = 0$, which happens if and only if 
  $\forall_{n:\N} P_n$. 
%  Hence the countable conjunction of closed propositions is closed. 
\end{proof}
\begin{remark}
  LLPO (\Cref{LLPO}) is equivalent to the statement that for $P,Q$ open, we have 
  $(\neg P \vee \neg Q) \leftrightarrow \neg (P\wedge Q)$. 
\end{remark}
%\begin{proof}
%  Assuming the above statement, let $\alpha:\Noo$, and consider 
%  the open propositions 
%  $\exists_{k:\mathbb N}\alpha_{2k+1} = 1, \exists_{k:\N} \alpha_{2k} = 1$. 
%  As $\alpha:\Noo$ there's at most one $n:\mathbb N$ with $\alpha_n =1$, so their conjunction is false. 
%  By the above statement, it follows one of them is false, leading to the statement of LLPO. 
%  
%  Now suppose LLPO holds, and assume $\neg (\exists_{n:\mathbb N} \alpha_n = 1\wedge \exists_{n:\mathbb N} \beta_n = 1$
%  for $\alpha,\beta :\Noo$. Then the sequence $\gamma:2^\N$ given by $\gamma_{2n} = \alpha_n, \gamma_{2n+1} = \beta_n$
%  can hit $1$ at most once and thus induces a sequence in $\Noo$. LLPO then shows that 
%  $\neg \forall_{n:\mathbb N} \alpha_n = 1\vee \neg \forall_{n:\mathbb N} \beta_n=1$ as required. 
%\end{proof}
\begin{corollary}
  Closed propositions are closed under finite disjunctions.
\end{corollary}
\begin{proof}
  Closed propositions are negations of open propositions. 
  As the conjunction of two open propositions is open, LLPO gives that 
  the disjunction of two closed propositions is closed. 
\end{proof}
We will use the above properties silently from now on. 
%OneBigLemma#
%OneBigLemma#\rednote{Phrase the following lemmas as one big lemma, 
%OneBigLemma#and use them silently without reference, also we should just state $\neg\neg$-stability instead of referring to the above all the time. }
%OneBigLemma#
%OneBigLemma#\begin{lemma}\label{ClosedCountableConjunction}
%OneBigLemma#  Closed propositions are closed under countable conjunctions.
%OneBigLemma#\end{lemma}
%OneBigLemma#\begin{proof}
%OneBigLemma#  Let $(P_n)_{n:\N}$ be a countable family of closed propositions. 
%OneBigLemma#  By countable choice, for each 
%OneBigLemma#  $n:\N$ we have an $\alpha_n:2^\N $ 
%OneBigLemma#  such that $P_n \leftrightarrow \forall_{m:\N} \alpha_{n,m} =0$. 
%OneBigLemma#  Consider a surjection $s:\N \twoheadrightarrow \N \times \N$, and let 
%OneBigLemma#%  Let 
%OneBigLemma#%  $$\beta_k = \alpha_{s(k)}.$$
%OneBigLemma#  $\beta_k = \alpha_{s(k)}.$
%OneBigLemma#  Note that $\forall_{k:\N} \beta_k = 0$ if and only if 
%OneBigLemma#%  $\forall_{m,n:\N}\alpha_{m,n} = 0$, which happens if and only if 
%OneBigLemma#  $\forall_{n:\N} P_n$. 
%OneBigLemma#  Hence the countable conjunction of closed propositions is closed. 
%OneBigLemma#\end{proof} 
%OneBigLemma#Using similar arguments, we can show the following two lemmas:
%OneBigLemma#\begin{lemma}\label{OpenCountableDisjunction}
%OneBigLemma#  Open propositions are closed under countable disjunctions. 
%OneBigLemma#\end{lemma}
%OneBigLemma#\begin{lemma}\label{OpenFiniteConjunction}
%OneBigLemma#Open propositions are closed under finite conjunctions. 
%OneBigLemma#\end{lemma}
%OneBigLemma#%\begin{proof}
%OneBigLemma#%We use \Cref{ClosedFiniteDisjunction} and the fact that $\neg(P\lor Q) \leftrightarrow \neg P \land \neg Q$.
%OneBigLemma#%\end{proof}
%OneBigLemma#%\begin{proof}
%OneBigLemma#%  Similar to the previous lemma. 
%OneBigLemma#%\end{proof}
%OneBigLemma#\begin{lemma}\label{ClosedFiniteDisjunction} 
%OneBigLemma#  Closed propositions are closed under finite disjunctions. 
%OneBigLemma#\end{lemma}
%OneBigLemma#\begin{proof}
%OneBigLemma#  This statement is equivalent to LLPO (\Cref{LLPO}) by  
%OneBigLemma#  Proposition 1.4.1 of \cite{HannesDiener}. 
%OneBigLemma#%  , LLPO is equivalent to the statement that 
%OneBigLemma#%  for $(\phi_n)_{n:\N}, (\psi_m)_{m:\N}$ families of decidable propositions indexed over $\N$, we have
%OneBigLemma#%  \begin{equation}
%OneBigLemma#%    (\forall_{n:\N} \forall_{m:\N} (\phi_n \vee \psi_m) )
%OneBigLemma#%    \leftrightarrow
%OneBigLemma#%    ((\forall_{n:\N} \phi_n) \vee (\forall_{m:\N} \psi_m) )
%OneBigLemma#%  \end{equation}
%OneBigLemma#%%  $(\forall_{n:\N} \alpha(n) = 0 )\vee (\forall_{n:\N} \beta(n) = 0 )$ is closed for any $\alpha,\beta:2^\N$.
%OneBigLemma#%%  By \Cref{corAlternativeLLPO}, the statement is equivalent to 
%OneBigLemma#%%  $ \forall_{n:\N}  \forall_{m:\N}  (\alpha(n) = 0 \vee \beta(m) = 0)$, 
%OneBigLemma#%  The latter which is a countable conjunction of decidable propositions, 
%OneBigLemma#%  hence closed by \Cref{ClosedCountableConjunction}.
%OneBigLemma#\end{proof}
%OneBigLemma#
\begin{corollary}\label{ClopenDecidable}
  If a proposition is both open and closed, it is decidable. 
\end{corollary}
\begin{proof}
  If $P$ is open and closed, %$\neg P$, and hence 
  $P\vee \neg P$ is open, 
  hence $\neg\neg$-stable and provable. 
%  and we conclude by $\neg\neg$-stability of open propositions. 
%  but open propositions are $\neg\neg$-stable by \Cref{rmkOpenClosedNegation} so we can conclude.
%  hence 
 % equivalent to $\neg \neg (P \vee \neg P)$ by \Cref{rmkOpenClosedNegation}.
 % As the latter proposition is provable, we may conclude $P$ is decidable. 
%  
%  If $P$ is open and closed, $P\vee \neg P$ is open, hence
%  equivalent to $\neg \neg (P \vee \neg P)$, which is provable. 
\end{proof}


%\begin{lemma}\label{OpenFiniteConjunction}
%  Open propositions are closed under finite conjunctions. 
%\end{lemma}
%\begin{proof}
%  We need to show that for any $\alpha,\beta:2^\N$, the following proposition is open:
%  \begin{equation}\label{eqnConjunctionOpen}
%    (\exists_{n:\N} \alpha(n) = 0 )\wedge(\exists_{n:\N} \beta(n) = 0 )
%  \end{equation}
%  Consider $\gamma:2^\N$ given by 
%  $\gamma(l) = 1$ iff there exist some $k,k'\leq l$ with 
%  $\alpha(k) = \beta(k') = 0$. 
%  As we only need to check finitely many combinations 
%  of $k,k'$, this is a decidable property for each $l:\N$ and $\gamma$ is well-defined. 
%  Then $\exists_{k:\N}\gamma(k)=0$ if and only if \Cref{eqnConjunctionOpen} holds.
%\end{proof}

\begin{lemma}\label{ClosedMarkov}
  For $(P_n)_{n:\N}$ a sequence of closed propositions, we have 
  $\neg \forall_{n:\N} P_n \leftrightarrow  \exists_{n:\N} \neg P_n$. 
\end{lemma}
\begin{proof}
  Both $\neg \forall_{n:\N} P_n$ and $\exists_{n:\N} \neg P_n$ are open, hence $\neg\neg$-stable.
  The equivalence follows. 
%  and the equivalence follows. 
%  from which the equivalence follows. 
%  We have that $\forall_{n:\N}P_n$ is closed and $\exists_{n:\N} \neg P_n$ is open by \Cref{OpenCountableDisjunction}, therefore both are $\neg\neg$-stable by \Cref{rmkOpenClosedNegation} and we can conclude.
%It is always the case that $\exists_{n:\N}\neg P_n \to \neg \forall_{n:\N} P_n$. 
  %For the converse direction,
  %note that $\neg \exists_{n:\N} \neg P_n(x) \to \forall_{n:\N} \neg \neg P_n(x).$
  %By \Cref{rmkOpenClosedNegation}, $\neg \neg  P_n(x)\leftrightarrow P_n(x)$ for all $n:\N$. 
  %It follows that 
  %$\neg \forall_{n:\N} P_n(x)\to 
  %\neg \neg \exists_{n:\N} \neg P_n(x).$
  %As $\exists_{n:\N}\neg P_n(x)$ is a countable disjunction of open propositions, 
  %it is open by \Cref{OpenCountableDisjunction} and thus equivalent to 
  %$\neg\neg\exists_{n:\N} \neg P_n(x)$ by \Cref{rmkOpenClosedNegation}.
  %We conclude that $\neg \forall_{n:\N} P_n \to \exists_{n:\N} \neg P_n$ as required. 
\end{proof} 

%\begin{lemma}\label{OpenDependentSums}
%  Open propositions are closed under dependent sums.
%\end{lemma}
%\begin{proof}
%  \rednote{If we show that Open propositions are exactly the overtly discrete ones, this is implied by $\Sigma$-closure}
%  First note that for $D$ a decidable proposition, and $X:D \to \Open$,
%  by case splitting on $D$, we can see 
%  $\Sigma_{d:D} X(d)$ is open.
%%
%  Then note that for $P$ an open proposition, 
%  there exists a sequence of decidable propositions $A_n$ with 
%  $P = \exists_{n:\N} A_n $.
%%
%  So for $Y : P \to Open $, the dependent sum $\Sigma_P Y$ is given by 
%  $\exists_{n:\N} (\Sigma_{a:A_n} Y(n,a))$,
%  which is a countable disjunction of open propositions, 
%  hence open by \Cref{OpenCountableDisjunction}.
%\end{proof}
%
%We will see the same holds for closed propositions in \Cref{ClosedDependentSums}.
%
%\begin{remark}\label{ImplicationOpenClosed}
%  If $P$ is open, $P \to \bot$ is only open if $P$ is decidable, which is not in general the case. 
%  Thus $\Open$ is not closed under dependent products. Neither is $\Closed$. 
%  However, as $(P\to Q)  \to \neg \neg (\neg P \vee Q)$,
%  we have that if $P$ is open and $Q$ is closed, then $P\to Q$ is closed, and similarly $Q\to P$ is open.
%\end{remark}
\begin{lemma}\label{ImplicationOpenClosed}
  If $P$ is open %(resp. closed) 
  and $Q$ is closed % (resp. open) 
  then $P\to Q$ is closed. % (resp. open). 
  If $P$ is closed and $Q$ open, then $P\to Q$ is open. 
\end{lemma}
\begin{proof}
  Note that $\neg P \vee Q$ is closed. Using $\neg\neg$-stability
  we can show $(P\to Q) \leftrightarrow (\neg P \vee Q)$. 
  The other proof is similar. 
%  and we conclude by 
%  Assume $P$ open and $Q$ closed, the other proof is similar. 
%  Note that $(\neg P \vee Q) \to (P \to Q)$ and 
%  $(P\to Q)\to \neg\neg(\neg P \vee Q)$. 
%  By \Cref{rmkOpenClosedNegation} it follows that 
%  $(\neg P \vee Q)\leftrightarrow (P \to Q)$, and using \Cref{ClosedFiniteDisjunction}, 
%  we can conclude that $P\to Q$ is closed. 
\end{proof}
%
%The following question was asked by Bas Spitters at TYPES 2024:


\subsection{Types as spaces}
The subobject $\Open$ of the type of propositions induces a topology on every type. 
This is the viewpoint taken in synthetic topology. 
We will follow the terminology of \cite{SyntheticTopologyLesnik}, 
other references include \cite{SyntheticTopologyEscardo}%, TODOSortOutTaylorsReferences}.
%Defining a topology in this way has some benefits, which we summarize in this section. 

\begin{definition}
  Let $T$ be a type, and let $A\subseteq T$ be a subtype. 
  We call $A\subseteq T$ open or closed iff $A(t)$ is open or closed respectively for all $t:T$.
\end{definition}

\begin{remark}
  It follows immediately that the pre-image of an open by any map of types sends is open, so that any map is continuous. 
%  This is only relevant for a space if the topology we defined above matches the topology one would expect. 
  In \Cref{StoneClosedSubsets}, we shall see that the resulting topology is as expected for second countable Stone spaces.
  In \Cref{IntervalTopologyStandard}, we shall see that the same for the unit interval. 
\end{remark}



%\begin{remark}
%  Phao's principle is a special case of directed univalence. 
%\end{remark}
%\begin{proof}
%  \rednote{TODO}
%\end{proof}


\section{Overtly discrete spaces}
%\subsection{Countably presented algebras as sequential colimits}\label{secBooleAsColimits}
\begin{definition}
  We define a type $E$ to be Overtly Discrete iff it is the colimit of an $\N$-indexed sequence of finite sets. 
\end{definition} 
%
%\begin{definition}
%  A sequence in a category is a diagram of shape $\N$, 
%  where $\N$ carries the natural structure of a poset. 
%\end{definition}
\begin{lemma}
  Every countably presented Boolean algebra is overtly discrete.
\end{lemma}
%\begin{lemma}\label{lemProFinitePresentation}
%  For every countably presented Boolean algebra $B$
%  there merely exists a sequence of finitely presented Boolean algebras 
%  whose colimit in the category of Boolean algebras is $B$. 
%\end{lemma}
\begin{proof}
  Consider $\langle G \rangle \langle\langle R \rangle\rangle$ a countable presentation of a Boolean algebra $B$. 
  We will show there exists a diagram of shape $\N$ taking values in Boolean algebras 
  with $\langle G\rangle / R$ as the colimit.
  \paragraph{The diagram}
  Let $R_n$ be the first $n$ terms in $R$. 
  Note that each of these finitely many terms uses only finitely many symbols from $G$.
  Let $G_n$ be the finite set of terms used in $R_n$, unioned with the finite set of the first $n$ elements of $G$. 
  Define for each $n\in\N$ the finitely presented Boolean algebra $B_n = \langle G_n \rangle  \langle R_n \rangle$. 
  If $n\leq m$, then \Cref{rmkMorphismsOutOfQuotient} gives us a map $B_n \to B_m$ 
  as $G_n \subseteq G_{n+1}$ and $R_n \subseteq R_{n+1}$. 
  Thus $(B_n)_{n\in \N}$ gives us a diagram of shape $\N$
  with values in finitely presented algebras. 

  \paragraph{The colimit}
  As $G_n\subseteq G$ and $R_n \subseteq R$, 
  \Cref{rmkMorphismsOutOfQuotient} also gives us a map $B_n\to \langle G \rangle \langle R \rangle$. 
  We claim the resulting cocone is a colimit. 

  Suppose we have a cocone $C$ on the diagram $(B_n)_{n\in\N}$. 
  We need to show that there exists a map $\langle G \rangle / R\to C$ and
  we need to show this map is unique as map between cocones. 
  \begin{itemize}
    \item To show there exists a map $\langle G \rangle / R \to C$, 
      we use remark \Cref{rmkMorphismsOutOfQuotient} again. 
      Let $g\in G$ be the $n$'th element of $G$, 
      note that $g\in G_n$, and consider the image of $g$ under the map $B_n \to C$. 
      This procedure defines a function from $G$ to the underlying set of $C$. 
      Let $\phi \in R$ be the $n$'th element of $R$, 
      note that $\phi \in R_n$, and the map $B_n \to C$ must send $\phi$ to $0$. 
      Thus the function from $G$ to the underlying set of $C$ also sends $\phi$ to $0$. 
      This thus defines a map $\langle G \rangle / R \to C$. 
    \item To show uniqueness, consider that any map of cocones $\langle G \rangle / \langle R \rangle \to C$ 
      must take the same values on all $g\in G_n$ for all $n\in\N$. 
      Now all $g\in G$ occur in some $G_n$, so any map of cocones $\langle G \rangle /  \langle R \rangle \to C$ 
      takes the same values for all $g\in G$. 
      \Cref{rmkMorphismsOutOfQuotient} now tell us that these values uniquely determine the map. 
  \end{itemize}
\end{proof}
\begin{remark}
  Conversely, any colimit of a sequence of finite Boolean algebras 
  is a countably presented Boolean algebra with 
  as underlying sets of generators and relations the countable union of the finite sets of 
  generators and relations, which are both countable. 
\end{remark}
\begin{lemma}\label{lemFinitelyPresentedBACompact}
  For any finitely presented Boolean algebra $A$,
  and any sequence $(B_n)_{n:\N}$ of Boolean algebras with colimit $B$
  we have that the set $B^A$ is the colimit of the sequence of sets $(B_n^A)_{n:\N}$. 
\end{lemma}  
\begin{proof}
  First note that $B^A$ forms a cocone on $(B_n^A)_{n:\N}$ 
  because any map $A \to B_n$ induces a map $A \to B$. 
  Let $C$ be a cocone on $(B_n^A)_{n:\N}$. 
  We shall show there is an unique morphism of cocones $B^A \to C$. 
  \begin{itemize}
    \item For existence, let $f:B^A$. 
      As $A$ is finitely presented, we write $A = \langle G \rangle / \langle R \rangle$ with $G$ finite.
      By \Cref{rmkMorphismsOutOfQuotient}, $f$ is uniquely determined by it's values on $g\in G$. 
      As $G$ is finite, so is it's image $f(G)\subseteq B$. 
      But any finite subset of $B$ already occurs in $B_n$ for some $n\in\N$. 
      Consequently, the image of $f$ is already contained in some $B_n$. 
      Thus there is some $f_n:(B_n^A)$ such that postcomposing 
      $f_n$ with the map $B_n \to B$ gives back $f$. 
      The image of $f_n$ under the map $(B_n^A) \to C$ is how we define the image of $f$. 
      This is well-defined by the cocone conditions on $C$. 
    \item 
      For uniqueness, by function extensionality maps $B^A \to C$ are uniquely determined by their values on 
      $f:B^A$. By the above, the value of $f$ is uniquely determined by it's value on $B_n$ for 
      any $n$ with the image of $f$ in $B_n$. Thus there is at most one morphism of cocones $B^A \to C$. 
  \end{itemize}
\end{proof}
\begin{remark}\label{rmkEqualityColimit}
  In the above proof, we used that any element $b\in B$ already occurs in some $B_n$. 
  However, please note that it is not necessarily the case that it occurs uniquely in $B_n$, 
  there might be multiple elements in $B_n$ which can all be sent to $b$ in the end. 

  In case our sequence comes from the construction in \Cref{lemProFinitePresentation}, 
  we can see that whenever there are two elements in 
  $B_n$ corresponding to $b\in B$, they will become equal in $B_m$ for some $m\geq n$. 
  The reason is that if $b \sim_{\langle R \rangle} c$, there is a finite subset $R_0 \subseteq R$ such that 
  $b\sim_{\langle R_0 \rangle} c$, which will occur in some $R_m$. 

  One could wonder whether this property holds for general colimits of sequences. 
  In general, if we assume $B$ is the colimit of an arbitrary sequence $(B_n)_{n:\N}$, 
  and there exist some $B_n$ with two elements corresponding to the same element in $B$, 
  Theorem 7.4 from \cite{SequentialColimitHoTT} says that there merely exists some $m\geq n$
  such that they are already equal in $B_m$. 
\end{remark}

%For our next lemma on this presentation of sequences we need the axiom of dependent choice. 
%\begin{axiomNum}[Dependent choice]\label{axDependentChoice}
%  Given a family of types $(E_n)_{n:\N}$ and 
%  a relation 
%  $R_n:E_n\rightarrow E_{n+1}\rightarrow {\mathcal U}$ such that
%  for all $n$ and $x:E_n$ there exists $y:E_{n+1}$ with $p:R_n~x~y$ 
%  then given $x_0:E_0$ there exists
%  $u:\Pi_{n:\N}E_n$ and $v:\Pi_{n:\N}R_n~(u~n)~(u~(n+1))$ and $u~0 = x_0$.
%\end{axiomNum}
\begin{lemma}[Using dependent choice]\label{lemDecompositionOfColimitMorphisms}
  Let $B,C$ be countably presented Boolean algebras, 
  and suppose we have a morphism $f:B\to C$.
  There exists sequences of finitely presented Boolean algebras 
  $(B_n)_{n:\N}, (C_n)_{n:\N}$ with colimits $B,C$ respectively
  and compatible maps of Boolean algebras $f_n:B_n \to C_n$, 
  such that $f$ is the induced morphism $B\to C$.
\end{lemma}
\begin{proof}
  Let $(B_n)_{n:\N}, (C_n)_{n:\N}$ be 
  sequences of finitely presented Boolean algebras with colimits $B$ and $C$. 
  We will take a subsequence of $(C_n)_{n:\N}$, using the axiom of dependent choice above. 

  Our family of types $E_k$ as in \Cref{axDependentChoice} 
  will be strictly increasing sequences $(n_i)_{i\leq k}$ of natural numbers together with a finite family of maps 
  $(f_i: B_{i} \to C_{n_i})_{i\leq k}$ such that
  for all $0\leq i<k$ the following diagram commutes:
  \begin{equation}\label{eqnDecompositionOfColimitMorphisms}
    \begin{tikzcd}
      B_{i} \arrow[r] \arrow[d, "f_i"]& B_{{i+1}} \arrow[r] \arrow[d,"f_{i+1}"]& B \arrow[d,"f"] \\
      C_{n_i} \arrow[r] & C_{n_{i+1}} \arrow[r] & C 
    \end{tikzcd}
  \end{equation}
  Our relation $R_k$ will tell whether the second sequence extends the first one. 
%
  By \Cref{lemFinitelyPresentedBACompact} 
  there exists some $n_0:\N$ 
  such that $B_0 \to B \to C$ factors as 
  \begin{equation}
    \begin{tikzcd}
      B_{0} \arrow[r] \arrow[d, "f_0"]& B \arrow[d,"f"] \\
      C_{n_0} \arrow[r] & C 
    \end{tikzcd}
  \end{equation}
  Because our goal is a proposition, we can untracate this existence to data. 
  This data will form our $x_0:E_0$. %from \Cref{axDependentChoice}. 
%
  Now suppose we have $(f_i: B_{i} \to C_{n_i})_{i\leq k}$ for some $k\geq 0$ 
  such that
  for all $0\leq i<k$ the diagram of \Cref{eqnDecompositionOfColimitMorphisms} commutes.
  We shall show that in this case there exists an $n_{k+1}, f_{k+1}$ 
  making the same diagram commute for $i = k$. 
  Consider $B_{{k}+1}\to B \to C$. By the same argument as for $B_0$, we have a factorization 
  \begin{equation}
    \begin{tikzcd}
    B_{k+1} \arrow[r]  \arrow[d,"f'_{k+1}"]& B \arrow[d,"f"]\\
    C_{n'_{k+1}} \arrow[r] & C
    \end{tikzcd}
  \end{equation}
  Note that we may assume $n'_{k+1} > n_k$.
  Note that it is not necessarily the case that 
  $f'_{k+1}$ is compatibly with $f_k$, meaning the left square in the following diagram needn't commute:
  \begin{equation}
    \begin{tikzcd}
      B_{k} \arrow[r] \arrow[d, "f_k"]& B_{{k+1}}  \arrow[r] \arrow[d,"f'_{k+1}"] & B \arrow[d,"f"] \\
      C_{n_k} \arrow[r] & C_{n'_{k+1}} \arrow[r]  & C 
    \end{tikzcd}
  \end{equation}
  However, both $f'_{k+1}, f_k$ induce the same map $B_{k} \to C$. 
  Recall by \Cref{rmkMorphismsOutOfQuotient} this map is induced by it's value on finitely many elements. 
  By \Cref{rmkEqualityColimit}, it follows there is an $n_{k+1} \geq {n'_{k+1}}$ 
  such that for $f_{k+1}$ the composition of $f'_{k+1}:B_{k+1} \to C_{n'_{k+1}}$ and 
  the map $C_{n'_{k+1}} \to C_{n_{k+1}}$, the following diagram does commute:
  \begin{equation}
    \begin{tikzcd}
      B_{k} \arrow[d,"f_k"]\arrow[r] & B_{{k+1}} \arrow[rd, "f_{k+1}"] \arrow[rr] & & B \arrow[d,"f"] \\
      C_{n_k} \arrow[r] & C_{n'_{k+1}} \arrow[r] & C_{n_{k+1}} \arrow[r] & C 
    \end{tikzcd}
  \end{equation}
  Now by dependent choice for the above $x_0, R_n, E_n$, we get a sequence $(f_i:B_i \to C_{n_i})$  for some 
  strictly increasing sequence $n_i$ of natural numbers. 
  Note that for such a sequence $(n_i)_{i:\N}$, 
  $(C_{n_i})_{i:\N}$ converges to $C$. Also $(B_i)_{i:\N}$ still converges to $B$. 
  Futhermore, by construction the map that sequence $f_i$ induces from $B \to C$ shares all values with $f$
  and thus is equal to $f$. 
  Thus our sequence $f_i$ is as required. 
\end{proof}
\begin{remark}\label{rmkEpiMonoFactorizationCommutes}
  For $f,(f_i)_{i:\N}$ as above, whenever $f_n(x) = 0$, we have $f_{n+1}(x \circ \iota_{n,n+1}) = 0$
  for $\iota_{n,n+1}$ the map $A_n \to A_{n+1}$. 
  By \Cref{rmkMorphismsOutOfQuotient}, $\iota_{n,n+1}$ induces a map $A_n/Ker(f_n)\to A_{n+1}/Ker(f_{n+1})$. 
  This induced map is such that the following diagram commutes:
  \begin{equation}\begin{tikzcd}
    A_n \arrow[d, two heads] \arrow[r, "\iota_{n,n+1}"] & A_{n+1} \arrow[d,two heads]\\
    A_n /Ker(f_n) \arrow[d,hook] \arrow[r] & A_{n+1} /Ker(f_{n+1}) \arrow[d,hook] \\
    B_n \arrow[r] & B_{n+1}
  \end{tikzcd}\end{equation}  
  As the induced maps be epi's / mono's  is epi /mono, the colimit of the sequence 
  $A_n / Ker(f_n)$ will fit into an epi-mono factorization of $f$ and thus be iso to $A/Ker(f)$. 
  Thus the epi-mono factorization of the colimit is the colimit of the epi-mono factorizations. 
\end{remark}
\begin{remark}\label{rmkIsoEpiMonoMapColimit}
  Whenever $f:B \to C$ is an iso, any sequence with $B$ as colimit, also has $C$ as colimit. 
  Thus any iso can be represented this way as sequence of iso's. 
  Conversely, any sequence of isomorphisms induces an isomorphism of their colimits. 

  It follows from \Cref{rmkEpiMonoFactorizationCommutes} that when $f$ is epi/mono, 
  we can say that $f$ can be induced by a sequence 
  $(f_i)_{i\in \N}$ with all $f_i$ epi/mono. 
\end{remark}




\rednote{
  Change the $\iota$ notation to have the domain on top, 
and $\pi$ have codomain at bottom So $\pi_n^m \circ \pi_m = \pi_n, 
\iota_m^n \circ \iota_n = \iota_m$ $\iota_m^n : B_n \to B_m$. } 
\rednote{Discussion on what the colimit is exactly, refer to definition in\cite{SequentialColimitHoTT}}



\begin{definition}
  We call a type overtly discrete iff 
  it is a sequential colimit of finite sets. 
%  it can be described as 
%  the colimit of an $(\N,\leq)$-indexed sequence of finite sets. 
\end{definition} 
\begin{remark}
  It follows from Corollary 7.7 of \cite{SequentialColimitHoTT} that 
  overtly discrete types are sets. 
  Note that the type of overtly discrete types is independent on choice of universe, 
  so we can write $\ODisc$ for this type. 
%
  If $B:\ODisc$, we will denote $B_n$ for the objects of the underlying sequence and 
  $\iota^n)m: B_n \to B_m, \iota_n:B_n \to B$ for the obvious maps. 
%  $\iota_n^m:B_n \to B_m$ for the maps and objects in the underlying sequence and 
%  $\iota_n:B_n \to B$ for the colimit inclusion map. 
%  If $B$ is overtly disc
%  If we denote an overtly discrete type by $B$, we will denote the objects of the underlying sequence as 
%  $B_n,~n:\N$, and for $n\leq m$, we denote the maps $B_n \to B_m$ with 
%  Greek letters with lower index $n$ and upper index $m$. 
%  So for example $\iota_n^m:B_n \to B_m$. 
%  The maps $B_n \to B$ will in this case be denoted $\iota_n$. 
%  If convenient, given a sequence $B_n$, we will denote $B_\infty$ for the colimit $B$.
\end{remark}
%\begin{definition}
%A type $X$ is countable if there merely exists a decidable subset of $\N$ equal to $X$.
%\end{definition}
%


\subsection{Maps of overtly discrete types}
\begin{lemma}[Compactness of finite sets] \label{colimitCompact}
%  For any finite set $A$ and $(\N,\leq)$-indexed sequence of finite sets $B_n$ with colimit $B$, 
%  the colimit of $B_n^A$ is $B^A$. 
  Exponentiation by a finite sets commute with sequential colimit. 
\end{lemma}  
\begin{proof}
  \rednote{Reference to standard proof working here as well.}
%reference%
%reference%  \rednote{Should there be a reference here?}
%reference%%  First note that $B^A$ forms a cocone on $(B_n^A)_{n:\N}$ 
%reference%  Any map $A \to B_n$ induces a map $A \to B$, hence $B^A$ is a cocone on $(B_n^A)_{n:\N}$.
%reference%  Let $C$ form a cocone on $(B_n^A)_{n:\N}$. %with maps $F_n:B_n^A \to C$.
%reference%  For any $f:A \to B$, the finite image $f(A)$ must already occur in some $B_n$, 
%reference%  thus there is some $f':A\to B_n$ with $\iota_n\circ f' = f$.% occurs as some map $A\to B_n$, 
%reference%  As $C$ is a cocone, $f'$ corresponds to some $c$ in $C$, and this term does not depend on $n$. 
%reference%  Also any map $B^A \to C$ respecting the cocone conditions must send $f$ to $c$, 
%reference%  hence $B^A$ is indeed the colimit. 
%reference%%%  We shall show there is an unique morphism of cocones $B^A \to C$. 
%reference%%%  Denote $\iota_n:B_n \to B, F_n:B_n^A\to C$ for the cocone maps. 
%reference%%  \begin{itemize}
%reference%%    \item 
%reference%%      If $f:A\to B$, we have that $f(A)$ is a finite subset of $B$, and thus occurs already in some $B_n$. 
%reference%%      This induces a map $f'_n:A\to B_n$ with $\iota_n\circ f'_n = f$. 
%reference%%      As $C$ is a cocone, we have that $F_n(f'_n)$ does not depend on $n:\N$. 
%reference%%%      Thus $(F_n)_{n:\N}$ induces a map 
%reference%%      Thus we get a map $B^A\to C$. 
%reference%%    \item 
%reference%%      For uniqueness, by function extensionality maps $B^A \to C$ are uniquely determined by their values on 
%reference%%      $f:B^A$. By the above, the value of $f$ is uniquely determined by it's value on $B_n$ for 
%reference%%      any $n$ with the image of $f$ in $B_n$. Thus there is at most one morphism of cocones $B^A \to C$. 
%reference%%  \end{itemize}
\end{proof}
\begin{remark}\label{rmkEqualityColimit}
  In the above proof, we used that any element $b\in B$ already occurs in some $B_n$. 
  However, it does not necessarily occur uniquely in $B_n$.
  In general, $B$ is overtly discrete 
  and there exist some $B_n$ with two elements corresponding to the same element in $B$, 
  Theorem 7.4 from \cite{SequentialColimitHoTT} says that there merely exists some $m\geq n$
  such that these elements become equal in $B_m$. 
\end{remark}
\begin{lemma}\label{lemDecompositionOfColimitMorphisms}
  \rednote{
  Any map between overtly discrete sets is a sequential colimit of maps between finite sets.
  }


  Let $B,C$ be overtly discrete, 
  and let $f:B\to C$.
  There exists $(\N,\leq)$-indexed sequences of finite sets 
  $(B_n)_{n:\N}, (C_n)_{n:\N}$ with colimits $B,C$ respectively
  and compatible maps $f_n:B_n \to C_n$, 
  such that $f$ is the induced morphism $B\to C$.
\end{lemma}
\begin{proof}
  Let $(B_n)_{n:\N}, (C_n)_{n:\N}$ be 
  sequences of finite sets with colimits $B$ and $C$. 
  Using \Cref{axDependentChoice}, we will construct an increasing sequence of natural numbers $n_i$ 
  with a family of maps $f_i:B_i \to C_{n_i}$ such that the following diagram commutes for all $i>0$. :
%  We will construct a subsequence of $(C_n)_{n:\N}$, using \Cref{axDependentChoice}.
%choiceintro%%
%choiceintro%  For $k:\N$, let $E_k$ consist of 
%choiceintro%  strictly increasing sequences $(n_i)_{i<k}$ of natural numbers together with a finite family of maps 
%choiceintro%  $(f_i: B_{i} \to C_{n_i})_{i<k}$ such that
%choiceintro%  for all $0<i<k$ the following diagram commutes:
  \begin{equation}\label{eqnDecompositionOfColimitMorphisms}
    \begin{tikzcd}
      B_{i-1} \arrow[r,"\iota_{i-1}^i"] \arrow[d, "f_{i-1}"]& B_{{i}} \arrow[r, "\iota_i"] 
      \arrow[d,"f_{i}"]& B \arrow[d,"f"] \\
      C_{n_{i-1}} \arrow[r,"\kappa_{i-1}^i"] & C_{n_{i}} \arrow[r,"\kappa_i"] & C 
    \end{tikzcd}
  \end{equation}
%choiceintro%  For $e:E_k, e':E_{k+1}$, we let 
%choiceintro%  $R_k(e,e')$ denote  whether the underlying sequences of $e'$ extends that of $e$. 
%choiceintro%  The empty sequence inhabits $E_0$. We will show that if $e:E_k$, there exists some $e':E_{k+1}$ with 
%choiceintro%  $R_k(e,e')$. Then \Cref{axDependentChoice} will give the required sequence $(f_i:B_i\to C_{n_i})$.
%choiceintro%
%  Suppose we have $(f_i: B_{i} \to C_{n_i})_{i<k}$ for some $k\geq 0$ 
%  such that for all $0<i<k$ the diagram of \Cref{eqnDecompositionOfColimitMorphisms} commutes.
  Suppose we have an initial segment $(n_i)_{i<k}$ of such a sequence with maps $(f_i)_{i<k}$ making 
  \Cref{eqnDecompositionOfColimitMorphisms} commute for $i<k$. 
  We shall show that in this case there exist $n_{k}:\N, f_{k}:B_k \to C_{n_k}$ extending it. 
%  making the same diagram commute for $i = k$. 
  Consider the map $f\circ \iota_k: B_{k}\to C$. 
  As $B_k$ is finite, \Cref{colimitCompact} gives some $n_k':\N $ such that %, f_k':B_k \to C_{n_k'}$ such that 
  it factors over some $C_{n_k'}$.
%  \begin{equation}
%    \begin{tikzcd}
%    B_{k} \arrow[r,"\iota_k"]  \arrow[d,"f'_{k}"]& B \arrow[d,"f"]\\
%    C_{n'_{k}} \arrow[r, "\kappa_{n'_k}"'] & C
%    \end{tikzcd}
%  \end{equation}
%ExtraExplanationWhichReadercanNote%  We may assume $n'_{k+1} > n_k$.
%ExtraExplanationWhichReadercanNote%  Note that it is not necessarily the case that 
%ExtraExplanationWhichReadercanNote%  $f'_{k} \circ \iota_{k-1}^k = \kappa_{n_{k-1}}^{n'_k}\circ f_{k-1}$. 
%  $f'_{k+1}$ is compatible with $f_k$, meaning the left square in the following diagram needn't commute:
%  \begin{equation}
%    \begin{tikzcd}
%      B_{k-1} \arrow[r] \arrow[d, "f_{k-1}"]& B_{{k}}  \arrow[r] \arrow[d,"f'_{k}"] & B \arrow[d,"f"] \\
%      C_{n_{k-1}} \arrow[r] & C_{n'_{k}} \arrow[r]  & C 
%    \end{tikzcd}
%  \end{equation}
%ExtraExplanationWhichReadercanNote%  However, 
  Both $f'_{k}, f_{k-1}$ induce the same map $B_{k-1} \to C$. 
%  Recall by \Cref{rmkMorphismsOutOfQuotient} this map is induced by it's value on finitely many elements. 
  As $B_{k-1}$ is finite, from \Cref{rmkEqualityColimit} there is some $n_{k} \geq {n'_{k}}$ 
  such that they become equal in $C_{n_k}$, and we have $f_k:B_k \to C_{n_k}$ such that the following does commute;
  by \Cref{axDependentChoice} we then get compatible maps as required. 
%choiceintro% and we're done:

%  such that for $f_{k}$ the composition of $f'_{k+1}:B_{k+1} \to C_{n'_{k+1}}$ and 
%  the map $C_{n'_{k+1}} \to C_{n_{k+1}}$, the following diagram does commute:
  \begin{equation}
    \begin{tikzcd}
      B_{k-1} \arrow[d,"f_{k-1}"]\arrow[r] & B_{{k}} \arrow[rd, "f_{k}"] \arrow[rr,"\iota_k"] & & B \arrow[d,"f"] \\
      C_{n_{k-1}} \arrow[r] & C_{n'_{k}} \arrow[r] & C_{n_{k}} \arrow[r] & C 
    \end{tikzcd}
  \end{equation}
%  Now by dependent choice for the above $x_0, R_n, E_n$, we get a sequence $(f_i:B_i \to C_{n_i})$  for some 
%  strictly increasing sequence $n_i$ of natural numbers. 
%  Note that for such a sequence $(n_i)_{i:\N}$, 
%  $(C_{n_i})_{i:\N}$ converges to $C$. Also $(B_i)_{i:\N}$ still converges to $B$. 
%  Furthermore, by construction the map that sequence $f_i$ induces from $B \to C$ shares all values with $f$
%  and thus is equal to $f$. 
%  Thus our sequence $f_i$ is as required. 
\end{proof}

\begin{lemma}\label{lemDecompositionOfEpiMonoFactorization}
  \rednote{Denote $\Im(f)$ instead of $f(A)$.} 
  \rednote{Can this be a shorter remark?}
  Let $f:A_\infty\to B_\infty$ be a map between overtly discrete types, and suppose we have $f_n:A_n\to B_n$ such that 
  the following diagram commutes:
  \begin{equation}
    \begin{tikzcd}
      A_n \arrow[d,"f_n"]\arrow[r, "\iota_n^m"]  & A_m \arrow[d,"f_m"] \arrow[r,"\iota_m^\infty"]  & A_\infty \arrow[d,"f"] 
      \\
      B_n \arrow[r, "\kappa_n^m"'] & B_m \arrow[r,"\kappa_m^\infty"'] & B_\infty
    \end{tikzcd}
  \end{equation}
  Then $f(A)$ is the colimit of $f_n(A_n)$, 
  and the maps $A\twoheadrightarrow f(A)$ and $f(A) \hookrightarrow B$ 
  are induced by the maps $A_n\twoheadrightarrow f_n(A_n)$ and $f_n(A_n) \hookrightarrow B_n$ respectively. 
\end{lemma}
\begin{proof}
  For $n\leq m$, we have that $\kappa_n^m(f_n(A_n)) = f_m(\iota_n^m(A_n))\subseteq f_m(A_m)$, 
  hence we can take the corestriction of the map $f_n(A_n) \to B_m$ to $f_m(A_m)$ to get 
  maps $\lambda_n^m :f_n(A_n) \to f_m(A_m)$ making the following diagram commute:
  % https://q.uiver.app/#q=WzAsOSxbMSwwLCJBX20iXSxbMiwwLCJBX1xcaW5mdHkiXSxbMSwyLCJCX20iXSxbMiwyLCJCX1xcaW5mdHkiXSxbMiwxLCJmKEEpIl0sWzEsMSwiZl9tKEFfbSkiXSxbMCwwLCJBX24iXSxbMCwyLCJCX24iXSxbMCwxLCJmX24oQV9uKSJdLFswLDFdLFsyLDNdLFsxLDQsIiIsMCx7InN0eWxlIjp7ImhlYWQiOnsibmFtZSI6ImVwaSJ9fX1dLFs0LDMsIiIsMSx7InN0eWxlIjp7InRhaWwiOnsibmFtZSI6Imhvb2siLCJzaWRlIjoidG9wIn19fV0sWzAsNSwiIiwyLHsic3R5bGUiOnsiaGVhZCI6eyJuYW1lIjoiZXBpIn19fV0sWzUsMiwiIiwxLHsic3R5bGUiOnsidGFpbCI6eyJuYW1lIjoiaG9vayIsInNpZGUiOiJ0b3AifX19XSxbNywyLCJcXGthcHBhX25ebSIsMl0sWzYsMCwiXFxpb3RhX25ebSJdLFs2LDgsIiIsMix7InN0eWxlIjp7ImhlYWQiOnsibmFtZSI6ImVwaSJ9fX1dLFs4LDcsIiIsMSx7InN0eWxlIjp7InRhaWwiOnsibmFtZSI6Imhvb2siLCJzaWRlIjoidG9wIn19fV0sWzgsNSwiIiwxLHsic3R5bGUiOnsiYm9keSI6eyJuYW1lIjoiZGFzaGVkIn19fV0sWzUsNCwiIiwxLHsic3R5bGUiOnsiYm9keSI6eyJuYW1lIjoiZG90dGVkIn19fV1d
  \begin{equation}\label{eqnEpiMonoFactorizationDecomposition}
    \begin{tikzcd}
    {A_n} & {A_m} & {A_\infty} \\
    {f_n(A_n)} & {f_m(A_m)} & {f(A_\infty)} \\
    {B_n} & {B_m} & {B_\infty}
    \arrow["{\iota_n^m}", from=1-1, to=1-2]
    \arrow[two heads, from=1-1, to=2-1,"e_n"]
    \arrow[from=1-2, to=1-3,"\iota_m^\infty"]
    \arrow[two heads, from=1-2, to=2-2,"e_m"]
    \arrow[two heads, from=1-3, to=2-3,"e_\infty"]
    \arrow[dashed, from=2-1, to=2-2, "\lambda_n^m"]
    \arrow[hook, from=2-1, to=3-1, "i_n"]
    \arrow[dashed, from=2-2, to=2-3, "\lambda_m^\infty"]
    \arrow[hook, from=2-2, to=3-2,"i_m"]
    \arrow[hook, from=2-3, to=3-3,"i_\infty"]
    \arrow["{\kappa_n^m}"', from=3-1, to=3-2]
    \arrow[from=3-2, to=3-3,"\kappa_m^\infty"']
  \end{tikzcd}
\end{equation}
  Also it is clear that any $b:f(A_\infty)$ already occurs in some $f_n(A_n)$, hence $f(A_\infty)$ is colimiting. 
\end{proof}
\begin{corollary}\label{decompositionInjectionSurjectionOfOdisc}
  In \Cref{lemDecompositionOfColimitMorphisms}, when $f$ is injective or surjective, 
  we can choose presentations such that each $f_n$ is also injective or surjective respectively. 
\end{corollary}
\begin{proof}
  Using \Cref{lemDecompositionOfColimitMorphisms} and \Cref{lemDecompositionOfEpiMonoFactorization}, 
  we get a factorization as in \Cref{eqnEpiMonoFactorizationDecomposition}. 
  If $f$ is injective, then $e$ is an isomorphism. 
  Hence $A$ is the colimit of $f_n(A_n)$, and we can take $f_n' = i_n$.
  Similarly, if $f$ is surjective $i$ is an isomorphism and we consider $B$ as colimit of $f_n(A_n)$ and 
  take $f_n' = e_n$.
\end{proof}
\subsection{Closure properties of $\ODisc$}
\begin{remark}\label{ODiscFiniteColim}
  As sequential colimits commute with finite colimits, 
  and finite sets are closed under finite colimits,
  $\ODisc$ is closed under finite colimits as well.
\end{remark}
\begin{lemma}\label{ODiscClosedUnderSequentialColimits}
  The colimit an $(\N,\leq)$-indexed sequence overtly discrete types is overtly discrete. 
\end{lemma}
\begin{proof}
  By applying \Cref{axDependentChoice} to \Cref{lemDecompositionOfColimitMorphisms}, 
  given a colimit of the sequence $A_i$, we can find a quarterplane of the form 
  \begin{equation}
    \begin{tikzcd}
    A_{0,0}\ar[d]\ar[r] & A_{0,1}\ar[d]\ar[r] & \cdots \\
    A_{1,0}\ar[d]\ar[r] & A_{1,1}\ar[d]\ar[r] & \cdots \\
    \vdots & \vdots & \ddots\\
    \end{tikzcd}
  \end{equation}
  where all the $A_{i,j}$ are finite sets, and $A_i$ is the colimit in $j$ of $A_{i,j}$ and 
  the maps $A_i \to A_k$ are induced by maps $A_{i,j}\to A_{k,j}$. 
  The colimit of the above quarter-plane is also the colimit of the induced $(\N,\leq)$-indexed sequence $A_{j,j}$, 
  which is overtly discrete by definition. 
\end{proof}
\begin{corollary}\label{OdiscSigma}
  Overtly discrete types are closed under $\Sigma$. 
\end{corollary}
\begin{proof}
  Let $B$ be overtly discrete and $X:B \to \mathcal U$ be a 
  $B$-indexed family of overtly discrete types. 
  For any $i:\N$, we have a finite coproduct of overtly discrete types 
  $\Sigma_{b:B_i} (X\circ \iota_i(b))$.  
  As colimits commute with finite coproducts, this is overtly discrete. 
  By Theorem 5.1 of \cite{SequentialColimitHoTT}, 
  taking the colimit in $i$, we get $\Sigma_{b:B} X(b)$. 
  By the above Lemma, this is overtly discrete. 
\end{proof}
\begin{remark}
  Note that the sequential colimit commutes with the propositional truncation, thus for $B:\ODisc$, we have 
  $||B||:\ODisc$. 
\end{remark}



%\begin{theorem}
%We have the following:
%\begin{enumerate}[(i)]
%\item Overtly dicrete types are stable under identity types and and sigma types.
%\item Overly discrete types are stable under quotients by equivalence relation with value in overtly discrete types.
%\item Overtly discrete type are stable under sequential colimits.
%\item Overtly discrete types have local choice.
%\end{enumerate}
%\end{theorem}
%
%\begin{proof}
%\begin{enumerate}[(i)]
%\item For stability under identity types, we use that sequential colimits commutes with identity types. 
%
%For stability under sigma types, sequential colimits commutes with sigma so that by (iii) it is enough to show that overtly discrete types are stable under finite coproduct. But sequential colimits commute with finite coproducts.
%
%\item Clear from the alternative description in \cref{overtly-discrete-colimit-finite}.
%
%\item Assume given a tower of sequential colimits of finite types. By using dependent choice with \cref{presentation-maps-overtly-discrete} repeatedly, we get a quarter plane of finite types:
%\begin{center}
%\begin{tikzcd}
%F_{0,0}\ar[d]\ar[r] & F_{0,1}\ar[d]\ar[r] & \cdots \\
%F_{1,0}\ar[d]\ar[r] & F_{1,1}\ar[d]\ar[r] & \cdots \\
%\vdots & \vdots & \ddots\\
%\end{tikzcd}
%\end{center}
%which colimit is the colimit of the assumed tower. Then we just use \cref{colimit-quarter-diagonal} to conclude that this colimit is overtly discrete.
%
%\item By \cref{overtly-discrete-colimit-finite}, we have a cover of any overtly discrete type by a countable type, which is an overtly discrete type that has choice.
%\end{enumerate}
%\end{proof}
%
%\begin{remark}
%(ii) implies that the propositional truncation of an overtly discrete type is open.
%
%(iii) implies that overtly discrete types are closed under countable coproducts.
%\end{remark}
%
%\begin{lemma}\label{equivalence-induced-by-open-is-open}
%Let $I$ be overtly discrete, and let $R$ be an open relation on $I$. Then the equivalence relation induced by $R$ is open.
%\end{lemma}
%
%\begin{proof}
%The equivalence relation $L(x,y)$ induced by $R$ is:
%\[\exists(k:\N). \exists(i_0,\cdots,i_k:I). i_0=x\land i_k=y \land (\forall l<k.\ R(i_l,i_{l+1})) \]
%which is open.
%\end{proof}
%
%\begin{lemma}
%Assume given a pushout square:
%\begin{center}
%\begin{tikzcd}
%I\ar[d]\ar[r] & K\ar[d] \\
%J\ar[r] & L  \\
%\end{tikzcd}
%\end{center}
%such that $I\to J$ is an embedding and $I,J,K$ are overtly discrete. Then $L$ is overtly discrete.
%\end{lemma}
%
%\begin{proof}
%The situation means we have an open $I\subset J$ and a map $f:I\to K$. Then $L$ is equivalent to the quotient of $J+K$ by the  equivalence relation generated by:
%\[i_0(x) \sim i_1(y)\ \mathrm{if\ we\ have\ that}\ (x\in I)\land f(x) = y\]
%It is overtly discrete by \cref{equivalence-induced-by-open-is-open}.
%\end{proof}
%
%
%


%DecompositionStone%
%DecompositionStone%\begin{proof}
%DecompositionStone%  By \Cref{FormalSurjectionsAreSurjections}, 
%DecompositionStone%  $g$ corresponds to an injective map $f:2^T \to 2^S$ of Boolean algebras, 
%DecompositionStone%  which by \Cref{decompositionInjectionSurjectionOfOdisc}
%DecompositionStone%  can be decomposed into maps $f_n:(2^T)_n \to (2^S)_n$ which are all injective, 
%DecompositionStone%  and by the duality described in \Cref{StoneDualToOdisc}, correspond to maps of finite Stone types
%DecompositionStone%  $g_n : T_n \to S_n$ with $T_n,S_n$ $(\N,\geq)$-indexed sequences with limits $T,S$, and $g_N$ inducing $g:T\to S$.
%DecompositionStone%  As all $f_n$ were injective, by \Cref{FormalSurjectionsAreSurjections}, all $g_n$ are surjective as required. 
%DecompositionStone%\end{proof}


\subsection{$\Open$ and $\ODisc$} %Open propositions are Overtly discrete and Stone propositions.}
%\begin{lemma}
%  Whenever $P$ is a proposition and overtly discrete, $P$ is open. 
%\end{lemma}
%\begin{proof}
%\end{proof}
%\begin{lemma}
%  Whenever $P$ is a an open proposition, it is overtly discrete.
%\end{lemma}
%\begin{proof}
%  Suppose $P\leftrightarrow \exists_{n:\N} \alpha_n = 1$. 
%  Let $P_n = \exists_{k\leq n} (\alpha_k = 1)$, which is a decidable proposition, hence a finite set. 
%  Then the colimit of $P_n$ is $P$. 
%\end{proof} 
\begin{lemma}\label{PropOpenIffOdisc}
  A proposition is open if and only if it is overtly discrete.
\end{lemma}
\begin{proof}
  If $P$ is overtly discrete, then $P\leftrightarrow \exists_{n:\N} \propTrunc{F_n}$ with $F_n$ finite sets. 
  But a finite set being merely inhabited is decidable, hence $P$ is a countable disjunction of decidable propositions, hence open.
  Suppose $P\leftrightarrow \exists_{n:\N} \alpha_n = 1$. 
  Let $P_n = \exists_{n\leq k} (\alpha_n = 1)$, which is a decidable proposition, hence a finite set. 
  Then the colimit of $P_n$ is $P$. 
\end{proof}

\begin{corollary}\label{OpenDependentSums}
  Open propositions are closed under dependent sums. 
\end{corollary}
%\begin{proof}
%  Immediate from \Cref{OdiscSigma} and \Cref{PropOpenIffOdisc}.
%\end{proof}
\begin{corollary}[transitivity of openness]\label{OpenTransitive}
  Let $T$ be a type, let $V\subseteq T$ open and let $W\subseteq V$ open. 
  Then $W\subseteq T$ is open as well. 
\end{corollary}
%\begin{proof}
%  Denote $W'\subseteq T$ for the composite. 
%  Note that $W'(t) = \Sigma_{v:V(t)} W(t,v)$. 
%  As open propositions are closed under dependent sums (\Cref{OpenDependentSums}), 
%  $W'(t)$ is an open proposition, as required. 
%\end{proof}

\begin{remark}\label{OpenDominance}
  It follows from  Proposition 2.25 of \cite{SyntheticTopologyLesnik} that 
  $\Open$ is a dominance in the setting of synthetic topology. 
\end{remark}

\begin{lemma}\label{OdiscQuotientCountableByOpen}\label{ODiscEqualityOpen}
  A type $B$ is overtly discrete if and only if it merely is the quotient of a countable set by an open equivalence relation. 
\end{lemma}
\begin{proof}
  If $B:\ODisc$ is the sequential colimit of finite sets $B_n$, 
  then $B$ is an open quotient of $ (\Sigma_{n:\N} B_n)$.
  %/\sim_B$ where $\sim_B$ is the reflexive closure of  
%  $(n,b)\sim(m,\iota^n_m b)$ for $n\leq m$. 
%
  Conversely, assume $B= D/R$ with $D\subseteq \N$ decidable and $R$ open. 
  By dependent choice we get $\alpha:D \to D \to 2^\N$ such that 
  $R(x,y)\leftrightarrow \exists_{k:\N}\alpha_{x,y}(k) = 1$. 
  Define $D_n = (D \cap \N_{\leq n})$, and $R_n : D_n \to D_n \to 2$ so that 
  $R_n(x,y)$ is the equivalence relation generated by the relation 
  $\exists_{k\leq n} \alpha_{x,y}(k) =1$. 
  Then the $B_n = D_n/R_n$ are finite sets, and have colimit $B$. 
\end{proof}
%\rednote{Not sure whether we need all of these:}
%\begin{lemma}\label{OpenInNAreDecidableInN}
%For any open $U\subseteq \N$, there merely exists a decidable set $D$ in $\N$ such that 
%$\Sigma_{n:\N} D(n) \simeq \Sigma_{n:\N} U(n)$.
%\end{lemma}
%\begin{proof}
%  Using countable choice, we get a map $\alpha_{(\cdot)}: \N \to \Noo$ such that 
%  $U(n) \leftrightarrow \Sigma_{k:\N} \alpha_n(k) = 1$. Hence 
%  $\Sigma_{n:\N}U(n) \simeq \Sigma_{n,k:\N}(\alpha_n(k)=0)$
%  using $\N=\N\times\N$, we can conclude. 
%\end{proof}
%Needed?\begin{corollary}
%Needed?  Any open subset of a countable set is countable. 
%Needed?\end{corollary}
%Needed?\begin{proof}
%Needed?  An open subset of a decidable subset of $\N$ is an open subset of $\N$, 
%Needed?  which by the above is isomorphic to a decidable subset of $\N$. 
%Needed?\end{proof}
%\begin{corollary}
%  Open propositions are closed under countable disjunctions. 
%\end{corollary}
%\begin{proof}
%  Clearly $\N:\ODisc$, and for $P:\N \to \Open$, and by the above 
%  $||\Sigma_{n:\N} P_n||:\ODisc$. 
%\end{proof} 

%\begin{corollary}\label{OpenFiniteConjunction}
%  Open propositions are closed under finite conjunctions. 
%\end{corollary}
%\begin{proof}
%  A conjunction of propositions is a product, which is a dependent sum. 
%\end{proof}
%FollowsAlsoFromNextLemma%\begin{lemma}\label{ODiscEqualityOpen}
%FollowsAlsoFromNextLemma%  Whenever $B$ is overtly discrete and $a,b:B$, the proposition $a=_B b$ is open. 
%FollowsAlsoFromNextLemma%\end{lemma}
%FollowsAlsoFromNextLemma%\begin{proof}
%FollowsAlsoFromNextLemma%  For $a,b:B$ there is some $n:\N$ with $a',b':B_n$ and $\iota_n(a') = a,\iota_n(b') = b$.
%FollowsAlsoFromNextLemma%  By \Cref{rmkEqualityColimit}, we have that $a=_B b$ iff 
%FollowsAlsoFromNextLemma%  there is some $m\geq n$ with $\iota_n^m (a') = \iota_n^m(b')$. 
%FollowsAlsoFromNextLemma%  As equality in finite sets is decidable, this is a countable disjunction of decidable propositions, hence open. 
%FollowsAlsoFromNextLemma%%  \rednote{That reference can also be a direct cite}.
%FollowsAlsoFromNextLemma%\end{proof}

\subsection{Relating $\ODisc$ and $\Boole$}
\begin{lemma}\label{BooleIsODisc}
  Every countably presented Boolean algebra is merely a sequential colimit of finite Boolean algebras. 
\end{lemma}
\begin{proof}
  Consider a countably presented Boolean algebra of the form $B = 2[\N]/(r_n)_{n:\N}$. 
%  We will show there exists a diagram of shape $\N$ taking values in Boolean algebras 
%  with $B$ as colimit.
%  \paragraph{The diagram}
  For each $n:\N$, let $G_n$ be the union of $\{g_i\ |\ {i\leq n}\}$ and 
  the finite set of generators occurring in $r_i$ for some $i\leq n$. 
  Denote $B_n = 2[G_n]/(r_i)_{i\leq n}$. 
  Each $B_n$ is a finite Boolean algebra, and there are canonical maps $B_n \to B_{n+1}$.
%  The inclusion $G_n \hookrightarrow G_{n+1}$ induces maps $B_n \to B_{n+1}$.
%  Hence $B_n,~n:\N$ is an $(\N,\leq)$-indexed sequence of finite sets. 
  Then $B$ is the colimit of this sequence. 
%
%  \paragraph{The colimit}
%  As $G_n\subseteq G$ and $R_n \subseteq R$, 
%  \Cref{rmkMorphismsOutOfQuotient} also gives us a map $B_n\to \langle G \rangle \langle R \rangle$. 
%  We claim the resulting cocone is a colimit. 
%
%  Suppose we have a cocone $C$ on the diagram $(B_n)_{n\in\N}$. 
%  We need to show that there exists a map $\langle G \rangle / R\to C$ and
%  we need to show this map is unique as map between cocones. 
%  \begin{itemize}
%    \item To show there exists a map $\langle G \rangle / R \to C$, 
%      we use remark \Cref{rmkMorphismsOutOfQuotient} again. 
%      Let $g\in G$ be the $n$'th element of $G$, 
%      note that $g\in G_n$, and consider the image of $g$ under the map $B_n \to C$. 
%      This procedure defines a function from $G$ to the underlying set of $C$. 
%      Let $\phi \in R$ be the $n$'th element of $R$, 
%      note that $\phi \in R_n$, and the map $B_n \to C$ must send $\phi$ to $0$. 
%      Thus the function from $G$ to the underlying set of $C$ also sends $\phi$ to $0$. 
%      This thus defines a map $\langle G \rangle / R \to C$. 
%    \item To show uniqueness, consider that any map of cocones $\langle G \rangle / \langle R \rangle \to C$ 
%      must take the same values on all $g\in G_n$ for all $n\in\N$. 
%      Now all $g\in G$ occur in some $G_n$, so any map of cocones $\langle G \rangle /  \langle R \rangle \to C$ 
%      takes the same values for all $g\in G$. 
%      \Cref{rmkMorphismsOutOfQuotient} now tell us that these values uniquely determine the map. 
%  \end{itemize}
\end{proof}




\begin{corollary}\label{ODiscBAareBoole}
  A Boolean algebra $B$ is overtly discrete if and only if it is countably presented. 
\end{corollary}
\begin{proof}
  Assume $B:\ODisc$. 
  By \Cref{OdiscQuotientCountableByOpen}, we get a surjection $\N\twoheadrightarrow B$ and that $B$ has open equality. 
  Consider the induced surjective morphism $f:2[\N]\twoheadrightarrow B$.
  By countable choice, we get for each $b:2[\N]$
  a sequence $\alpha_{b}:2^\N$ such that 
  $(f(b) = 0)\leftrightarrow \exists_{k:\N} (\alpha_{b}(k) =1)$. 
  Consider 
  $r:2[\N] \to \N \to 2[\N]$ 
  given by 
  \[r(b,k) =\begin{cases}
    b &\text{ if } \alpha_{b}(k) = 1\\
    0   &\text{ if } \alpha_{b}(k) = 0
  \end{cases}
  \] 
  Then $B= 2[\N]/(r(b,k))_{b:2^\N,k:\N}$.
  %By countable choice, we get for each $a,b:2[\N]$
  %a sequence $\alpha_{a,b}:2^\N$ such that 
  %$(f(a) = f(b))\leftrightarrow \exists_{k:\N} (\alpha_{a,b}(k) =1)$. 
  %Consider 
  %$r:2[\N] \times 2[\N] \times \N \to 2[\N]$ 
  %given by 
  %$$r(a,b,k) =\begin{cases}
  %  a-b &\text{ if } \alpha_{(a,b)}(k) = 1\\
  %  0   &\text{ if } \alpha_{(a,b)}(k) = 0
  %\end{cases}
  %$$
  %Then $B= 2[\N]/(r(a,b,k))_{(a,b,n): F \times F \times \N}$. 
  \Cref{BooleIsODisc} gives the converse.
\end{proof}

\begin{remark}\label{BooleEpiMono}
%  In particular equality in overtly discrete types is open. 
  By \Cref{OdiscSigma} and \Cref{ODiscBAareBoole}, 
  it follows that any 
  $g:B\to C$ in $\Boole$ has an overtly discrete kernel.
  As a consequence, the kernel is enumerable and $B/Ker(g)$ is in $\Boole$. 
  By uniqueness of epi-mono factorizations and \Cref{SurjectionsAreFormalSurjections}, 
  the factorization 
  $B\twoheadrightarrow B/Ker(g) \hookrightarrow C$ corresponds to 
  $Sp(C) \twoheadrightarrow Sp(B/Ker(g)) \hookrightarrow Sp(B)$. 
\end{remark}
\begin{remark}\label{decompositionBooleMaps}
  Similar to \Cref{lemDecompositionOfColimitMorphisms} and 
  \Cref{lemDecompositionOfEpiMonoFactorization}  a map (resp. surjection, injection) 
  in $\Boole$ is a sequential colimit of maps (resp. surjections, injections) between 
  finite Boolean algebras. 
\end{remark}

%\begin{lemma}\label{OpenOfOdiscIsOdisc}
  Any open subset of an overtly discrete type is overtly discrete. 
\end{lemma}

\begin{lemma}\label{OdiscEnumeration}
  Any overtly discrete type has an enumeration. 
\end{lemma}
\begin{proof}

\end{proof}

\begin{corollary}
  Whenever $f:B\to C$ a map in $\Boole$, we have that
  $Ker(f)$ is overtly discrete and $B/Ker(f):\Boole$.
\end{corollary}
\begin{proof}
  As equality in Boolean algebras is open, 
  $Ker(f)$ is an open subset of $B$, hence overtly discrete by \Cref{OpenOfOdiscIsOdisc}. 
  By \Cref{OdiscEnumeration}, we thus have an enumeration of $Ker(f)$, and $B/Ker(f):Boole$. 
\end{proof}

\begin{lemma}\label{imageAsQuotientKernel}
Let $f:B \to C$ be a map in $\Boole$. 
Then the image of $(-)\circ f : Sp(C) \to Sp(B)$ is given by 
$Sp(B/Ker(f))$. 
\end{lemma}
\begin{proof}
  By \Cref{SurjectionsAreFormalSurjections}, the epi-mono factorization of $f$ as 
  $B\twoheadrightarrow B/Ker(f) \hookrightarrow C$
  gives an epi-mono factorization of $(-) \circ f$ as $Sp(C) \twoheadrightarrow Sp(B/Ker(f)) \hookrightarrow B$, 
  and by unqiueness of epi-mono factorization, $Sp(B/Ker(f))$ is the image of $(-)\circ f$. 
\end{proof}

%\begin{lemma}
  Whenever $g:B\to C$ a map in $\Boole$, we have 
  $(-)\circ g : Sp(C) \to Sp(B)$ has image 
  $Sp(B/Ker(g))$. 
\end{lemma}
\begin{proof}
  Note that $Sp(B/Ker(g))\subseteq Sp(B)$  is given by those $x:Sp(B)$ such that whenever $g(b) =_C 0$, we have $x(b) =_2 0$. 
  Thus the image of $(-)\circ g$ is a subset of $Sp(B/Ker(g))$.
  Conversely, suppose that $x$ lies in $Sp(B/Ker(g))$. 
  Then $((-)\circ g)^{-1}(x)$ defines a closed subset of $C$. 
\end{proof}

\begin{lemma}
  Let $f:A \to B$ be a map in $\Boole$. 
  Then for any $x:Sp(A)$, we have that $\Sigma_{y:Sp(B)} y\circ f = x$ is equal to 
  $Sp(B/f(Ker(x)))$. 
\end{lemma}
\begin{proof}
  Any $y:Sp(B)$ lies in $Sp(B/f(Ker(x))$ iff whenever $a:A$ is such that $x(a) = 0$, then $y(f(a)) = 0$. 
  Clearly if $y\circ f = x$, then $y(f(a)) = 0$ whenever $x(a) = 0$. 
  Now suppose $y$ lies in $Sp(B/f(Ker(x)))$. 
  We claim that then $y\circ f = x$. 
\end{proof}


\begin{theorem}\label{FormalSurjectionsAreSurjections}
  Let $f:A\to B$ be a map of countably presented Boolean algebras. 
  If $f$ is injective, then the corresponding map $(\cdot) \circ f : Sp(B) \to Sp(A)$ is surjective. 
\end{theorem}

\begin{proof}
  Assume $f$ injective and let $x:Sp(A)$.
  By \Cref{FiberConstruction}, we have that $\left(\sum\limits_{y:Sp(B)} y\circ f = x \right) = Sp(B/R) $
  for $R=f(G)$ for some countable $G\subseteq A$ with $x(g) = 0$ for all $g\in G$. 
  By propositional completeness and \Cref{SpectrumEmptyIff01Equal}, 
  it's sufficient to show that $0\neq_{B/R}1$. 
  Note that $0=_{B/R} 1$ iff 
  $1 =_B \bigvee R_0$ for some $R_0\subseteq R$ finite. 
  But then $$1 = \bigvee f(G_0) = f(\bigvee  G_0)$$ for some $G_0\subseteq G$ finite. 
  And as $f$ is injective, $\bigvee G_0 = 1$. 
  However, 
  $$
  x(\bigvee G_0) = 
  x(\bigvee_{g\in G_0} g ) = \bigvee_{g \in G_0} x(g) = \bigvee_{g\in G_0} 0 = 0$$
  And as $x(1) = 1$, we get a contradiction. Therefore $0\neq_{B/R} 1$ as required. 
\end{proof}  
The converse to the above theorem is true as well, regardless of propositional completeness:
\begin{lemma}\label{SurjectionsAreFormalSurjections}
If $f:A\to B$ is a map in $\Boole$ and $(\cdot) \circ f :Sp(B) \to Sp(A)$ is surjective, 
$f$ is injective. 
\end{lemma}
\begin{proof}
  Suppose precomposition with $f$ is surjective. 
  Let $a:A$ be such that $f(a)= 0$. 
  By assumption, for every $x:A\to 2$, there is a $y:B\to 2$ with $y\circ f = x$. 
  Consequentely $x(a) = y(f(a)) = y(0) = 0$. 
  So $x(a) = 0$ for every $x:Sp(A)$. 
  Thus $Sp(A) = Sp(A/\{a\})$, and as $Sp$ is an embedding, 
  $A \simeq A/\{a\}$, and $a = 0$ in $A$. 
  So whenever $f(a) = 0$, we have $a=0$. Thus $f$ is injective. 
\end{proof}


\begin{lemma}\label{BoolePushouts}
  Countably presented Boolean algebras are closed under pushout. 
\end{lemma} 
\begin{proof}
  Let $A,B,C:\Boole$, and suppose $f:A\to B, g:A \to C$ are Boolean morphisms. 
  Let $G_A, G_B,G_C$ be the underlying countable sets of generators for $B,C$ and 
  let $R_A,R_B,R_C$ be the underlying countable sets of relations. 
  Consider $P$ the Boolean algebra generated by $G_B\sqcup G_C$ under the relations 
  $R_B\cup R_C \cup F$ where $F$ is the set of expressions $f(a)-g(a), a\in G_A$.
  
  Note that as the generators of $B$ are included in those of $P$, 
  and all relations of $B$ are included in those of $P$, there is a map $h:B\to P$. 
  Similarly there is a map $i:C\to P$. 
  We now claim that the following is a pushout square:
  \begin{equation}\begin{tikzcd}
    A \arrow[r,"f"] \arrow[d,"g"] & B \arrow[d,"h"]\\
    C \arrow[r,"i"] & P
  \end{tikzcd}\end{equation}  
  Suppose $\beta:B \to X, \gamma:C\to X$ are such that $\beta\circ f = \gamma \circ h$. 
  $\beta,\gamma$ then induce maps on the generators of $P$. 
  These maps respect $F$ as $\beta\circ f=\gamma\circ h$, and they must respect $R_B,R_C$ as they are maps out of $B,C$. 
  Therefore, $\beta,\gamma$ induce a map $e:P\to X$, such that 
  $e(b) = \beta(b)$ for $b:G_B$ and $e(c)=\gamma(c)$ for $c:G_C$. 
  Furthermore, any map $P\to X$ with this property must agree with $e$ on all the generators of $P$, 
  and therefore equal $e$. Thus $e$ is the unique extension $P\to X$. 
  Thus $P$ the above square is indeed a pushout. 
\end{proof}

For some proofs in this paper, 
\rednote{(right now the counter's at two)}
we'd like a very concrete description of the fiber of a map of Stone spaces. 
The following construction turns out to be particularly useful. 
\begin{lemma}\label{FiberConstruction}
  Let $A,B:\Boole$, let $G$ be an explicit countable set of generators for $A$, and let 
  $f:A \to B, x:A\to 2$. 
  Define the countable set 
  \begin{equation}
    G' = \{a | a\in G, x(a) = 0\} \cup \{\neg a | a \in G, x(a) = 1\}
  \end{equation} 
  For $R = f(G')$,
%  Then we can construct a countable set $R\subseteq B$ such that 
  the pushout of $f$ and $x$ is given by $B/R$. 
\end{lemma}  
\begin{proof}
We consider the following pullback in the category of Stone spaces:
  \begin{equation}\begin{tikzcd}
    \sum\limits_{y:Sp(B)} y\circ f = x \arrow[d] \arrow[r] \arrow["\lrcorner"{pos=0.125}, phantom, dr] 
    & \top \arrow[d,"x"]\\
    Sp(B) \arrow[r,"(\cdot) \circ f"] & Sp(A)
  \end{tikzcd}  \end{equation}
Dual to this square, we have the following pushout in the category of Boolean algebras,
where $Sp(P) \simeq  (\sum\limits_{y:Sp(B)} y \circ f = x)$:
  \begin{equation}\begin{tikzcd}
    A \arrow[d,"x"'] \arrow[r,hook,"f"] \arrow[rd,phantom,"\ulcorner"{pos=0.125}] & B\arrow[d]\\
    2 \arrow[r] & P
  \end{tikzcd}\end{equation} 
  Following \Cref{BoolePushouts}, 
  the pushout $P$ is given by $B/R$ with $R$ the relations $f(a) -x(a)$ 
  where $a$ ranges over the generators of $A$.
  Note that $x(a) \in \{0,1\}$. 
  If $x(a)=0$, then $f(a)-x(a) = f(a)$, 
  and if $x(a) = 1$, then $f(a) -x(a) = \neg f(a) = f(\neg a)$. 
  So we can define the subset $G'\subseteq A$ given by 
  \begin{equation}
    G' = \{a | a\in G, x(a) = 0\} \cup \{\neg a | a \in G, x(a) = 1\}
  \end{equation} 
  $G'$ is in bijection with $G$, hence countable. 
  Furthermore, $x(g) = 0$ for all $g\in G'$. 
  And $R = f(G')$.
\end{proof}

%\section{Topology}
\section{Stone spaces}

\subsection{Stone spaces as profinite sets}
%\begin{remark}\label{StoneClosedUnderPullback}
%  Note that Boolean algebras are closed under finite colimits. 
%  By \Cref{ODiscBAareBoole} and \Cref{ODiscFiniteColim}, $\Boole$ is closed under finite colimits.
%  By \Cref {SpIsAntiEquivalence}  it follows that the category of Stone spaces is closed under finite limits. 
%\end{remark}
Here we present Stone spaces as limits of $(\N,\geq)$-indexed sequences of finite sets. 
This is the perspective taken in Condensed Mathematics \cite{Condensed,Dagur,Scholze}.
Some of the results in this section are specific versions of the axioms used in 
\cite{bc24}. A full generalization is part of future work. 
\begin{lemma}
  Any $S:\Stone$ can be described as the limit of some $(\N,\geq)$-indexed sequence of 
  finite sets. 
\end{lemma}
\begin{proof}
  By \Cref{SpIsAntiEquivalence}, \Cref{BooleIsODisc} and \Cref{ODiscClosedUnderSequentialColimits}, 
  for $B:\Boole$, we have $Sp(B)$ the limit of $Sp(B_n)$, which are finite sets. 
\end{proof}
\begin{lemma}\label{StoneAreProfinite}
  The limit of some $(\N,\geq)$-indexed sequence $S_n$ of finite sets is a Stone space. 
\end{lemma}
\begin{proof}
  For finite sets, we have that $Sp(2^{S_n}) = S_n$, hence each $S_n$ is Stone. 
  By \Cref{SpIsAntiEquivalence}, \Cref{BooleIsODisc} and \Cref{ODiscClosedUnderSequentialColimits}, 
  $\Stone$ is closed under sequential limits.
%  As $\Boole$ is closed under sequential colimits, \Cref{SpIsAntiEquivalence} gives that the 
%  sequential limit of Stone space is Stone, hence $S$ is Stone. 
\end{proof}
\begin{remark}
  Whenever $S:\Stone$, we shall denote $S_n$ for the underlying sequence 
  and whenever $n\leq m$, we denote $\pi_m^n$ for the maps $S_m \to S_n$, 
  and $\pi_n:S\to S_n$. 
\end{remark}
\begin{remark}\label{StoneClosedUnderPullback}\label{ProFiniteMapsFactorization}
  Dually to \Cref{ODiscFiniteColim} and \Cref{ODiscClosedUnderSequentialColimits}, 
  Stone spaces are closed under finite limits and sequential limits.
  By \Cref{decompositionInjectionSurjectionOfOdisc} and 
  \Cref{SurjectionsAreFormalSurjections} when we have a map of Stone spaces $f:S\to T$, 
  we have $(\N,\geq)$-indexed sequences $S_n,T_n$ with limits $S$ and $T$ respectively
  and maps $f_n:S_n\to T_n$ inducing $f$, and if $f$ is surjective or injective, we 
  can choose all $f_n$ to be surjective or injective respectively as well. 
\end{remark}
\begin{lemma}\label{ScottFiniteCodomain}
  For $S:\Stone, k:\N$ we have that $Fin(k)^S$ is the colimit of $Fin(k)^{S_n}$. 
\end{lemma}
\begin{proof}
  By \Cref{SpIsAntiEquivalence} we have $Fin(k)^S = (2^S)^{2^{Fin(k)}}$.
  Note that $2^Fin(k)$ is finite, thus by \Cref{colimitCompact}, the latter
  is the colimit of $(2^{S_n})^{2^Fin(k)}$. 
  By applying \Cref{SpIsAntiEquivalence} again, these types are $Fin(k)^S_n$ as required. 
\end{proof}
\begin{lemma}\label{MapsStoneToNareBounded}
  For $S:\Stone$ and $f:S \to \N$, there merely exists some $N:\N$ with $f(S)\subseteq \N_{\leq N}$. 
\end{lemma}
\begin{proof}
  For each $n:\N$, the fiber of $f$ over $n$ is a decidable subset $f_n:S \to 2$. 
  We must have that $Sp(2^S/(f_n)_{n:\N}) = \bot$, hence there exists some $N:\N$ with 
  $\bigvee_{n\leq N} f_n =_{2^S} 1 $. 
  It follows that $f(s)\leq N$ for all $s:S$ as required. 
\end{proof}

%\begin{lemma}\label{scott-continuity}
%  Let $E:\ODisc$ and $S:\Stone$, then 
%  Then $E^S$ is the colimit of the $(\N,\leq)$-indexed sequence $E^{S_n}$.
%%  $\mathrm{colim}_k(Z^{S_k}) \to \Z^S$
%%  is an equivalence.
%\end{lemma}
%\begin{proof}
%  Let $f:S \to E$. By \Cref{OdiscQuotientCountableByOpen}, 
%  we have an enumeration $\N\twoheadrightarrow 1 + E$. 
%  By \Cref{MapsStoneToNareBounded} and \Cref{AxLocalChoice}, there is some $N:\N$ such that 
%  $f(S)\subseteq e(\N_{\leq N})$. 
%\end{proof} 
\begin{corollary}
  For $S:\Stone$, we have that $\N^S$ is the colimit of $\N^{S_n}$. 
\end{corollary}
\begin{proof}
  By \Cref{MapsStoneToNareBounded} we have that any map $S\to \N$ factors as 
  $S \to Fin(k)\hookrightarrow \N$ for some $k:\N$. 
  By \Cref{ScottFiniteCodomain}, such a map is uniquely determined by 
  compatible maps $S_n \to Fin(k)$, hence by compatible maps $S_n \to \N$, 
  as required. 
\end{proof} 



\subsection{$\Closed$ and $\Stone$}
%\begin{lemma}\label{BooleEqualityOpen}
%  Whenever $B:\Boole$, $a,b:B$ the proposition $a=_Bb$ is open. 
%\end{lemma}
%\begin{proof}
%  Let $G,R$ be the generators and relations of $B$. 
%  Let $a,b$ be represented by $x,y$ in the free Boolean algebra on $G$. 
%  Now let $R_n$ denote the first $n$ elements of $R$. 
%  Note that $a=b$ iff there exists some $n:\N$ with $x-y \leq \bigvee_{r\in R_n} r$. 
%  Furthermore, inequality is decidable in the free Boolean algebra, hence
%  $a=b$ is a countable disjunction of decidable propositions, hence open. 
%\end{proof}

\begin{corollary}\label{TruncationStoneClosed}
  Whenever $S:\Stone$, $||S||$ is closed. 
\end{corollary}
\begin{proof}
  By \Cref{SpectrumEmptyIff01Equal}, $\neg S$ is equivalent to $0=_B 1$, which is open by the above. 
  Hence $\neg \neg S$ is a closed proposition, and by \Cref{LemSurjectionsFormalToCompleteness}, so is $||S||$. 
\end{proof}
%\begin{remark}\label{ExplicitTruncationStoneClosed}
%  \rednote{New check later}
%  The above lemma and corollary actually show that if we have an explicit 
%  presentation of a Stone space as $S = Sp(2[G] / R)$, 
%  we can construct an explicit sequence $\alpha:2^\N$ such that $||S|| \leftrightarrow \forall_{n:\N} \alpha(n) = 0$. 
%\end{remark}


\begin{corollary}\label{PropositionsClosedIffStone}
  A proposition $P$ is closed iff it is a Stone space. 
\end{corollary}
\begin{proof}
  By the above, if $S$ is both a Stone space and a proposition, it is closed. 
  By \Cref{ClosedPropAsSpectrum}, any closed proposition is Stone. 
\end{proof}

\begin{lemma}\label{StoneEqualityClosed}
  Whenever $S:\Stone$, and $s,t:S$, the proposition $s=t$ is closed. 
\end{lemma}
\begin{proof}
  Suppose $S= Sp(B)$ and let $G$ be the generators of $B$. 
  Note that $s=t$ iff $s(g) =_2 t(g)$ for all $g:G$. 
  As $G$ is countable, and equality in $2$ is decidable, 
  $s=t$ is a countable conjunction of decidable propositions, hence 
  closed. 
\end{proof}
%
The following question was asked by Bas Spitters at TYPES 2024:
\begin{corollary}
  For $S:\Stone$ and $x,y,z:S$ 
  \begin{equation}\label{Apartness}
  x \neq y \to (x\neq z \vee y \neq z)
  \end{equation}
\end{corollary}
\begin{proof}
  As $x\neq y$, we can show that $\neg ( x = z \wedge y = z)$. 
  This in turn implies $\neg \neg ( x \neq  z \vee y \neq  z)$. 
  As, $x\neq z$ and $y \neq z$ are both open propositions, by \Cref{OpenCountableDisjunction} so is their disjunction. 
  By \Cref{rmkOpenClosedNegation}, that disjunction is double negation stable and \Cref{Apartness} follows. 
\end{proof}
\begin{remark}
  If \Cref{Apartness} holds in a type, we say that it's inequality is an apartness relation. 
  By a similar proof as above, it can be shown that in our setting inequality is an apartness relation 
  as soon as equality is open or closed. 
\end{remark}

\subsection{The topology on Stone spaces}
\begin{theorem}\label{StoneClosedSubsets}
  Let $A\subseteq S$ be a subset of a Stone space. The following are equivalent:
  \begin{enumerate}[(i)]
    \item There exists a map $\alpha:S \to 2^\N$ such that 
      $A (x) \leftrightarrow \forall_{n:\N} \alpha_{x,n} = 0$ for any $x:S$. 
    \item There exists a family 
      $(D_n)_{n:\N}$ 
      of decidable subsets of $S$ such that $A = \bigcap_{n:\N} D_n$. 
    \item There exists a Stone space $T$ and some embedding $T\to S$ which image is $A$
    \item There exists a Stone space $T$ and some map $T\to S$ which image is $A$. 
    \item $A$ is closed.
  \end{enumerate}
\end{theorem}
\begin{proof}
\item 
  \begin{itemize}
  \item 
    $(i)\leftrightarrow (ii)$. $D_n$ and $\alpha$ can be defined from each other by 
     $D_n(x) \leftrightarrow (\alpha_{x,n} = 0)$. Then observe that
     \begin{equation}
     x\in \bigcap_{n:\N} D_n \leftrightarrow 
      \forall_{n:\mathbb N} (\alpha_{x,n} = 0) 
     \end{equation}
     
   \item $(ii) \to (iii)$. Let $S=Sp(B)$. 
%      By Stone duality, we have $d_n,~n:\N$ terms of $B$ such that $D_n = \{x:S| x(d_n) = 1\}$. 
%      Let $C = B/(\neg d_n)_{n:\N}$.
%      Then the map $Sp(C) \to S$ is as desired because
%      $$Sp(C) = \{x:S| \forall_{n:\N} x(\neg d_n) =0\}  = \bigcap_{n:\N} D_n.$$
      By \Cref{AxStoneDuality}, we have $(d_n)_{n:\N}$ in $B$ such that $D_n = \{x:S\ |\ x(d_n) = 0\}$. 
      Let $C = B/(d_n)_{n:\N}$.
      Then $Sp(C) \to S$ is as desired because:
      $$Sp(C) = \{x:S\ |\ \forall_{n:\N} x(d_n) =0\}  = \bigcap_{n:\N} D_n.$$
%      By \Cref{SurjectionsAreFormalSurjections}, t
%      The quotient map $B \twoheadrightarrow C$
%      corresponds to a map $\iota:Sp(C) \hookrightarrow  S$. 
%      For $s:S$, $s$ lies in the image of this map iff 
%      for all $n:\N$ we have  $s(\neg d_n) = 0$, 
%      \begin{equation}
%        x\in \iota(Sp(C)) \leftrightarrow x(\neg d_n) = 0 \leftrightarrow x(d_n) = 1 \leftrightarrow x\in D_n
%      \end{equation}
%      Thus the image of $\iota$ is given by $\bigcap_{n:\N} D_n$. 
   \item $(iii) \to (iv)$. Immediate.
   \item $(iv) \to (ii)$. Assume $f:T\to S$ corresponds to $g:B\to C$ in $\Boole$. 
     By \Cref{BooleEpiMono}, $f(T) = Sp(B/Ker(g))$, and 
%     by \Cref{OdiscQuotientCountableByOpen}
     there is a surjection $d:\N\to Ker(g)$. Denote by $D_n$ the decidable subset of $S$ corresponding to $d_n$. Then we have that $Sp(B/Ker(g)) = \bigcap_{n:\N} D_n$. 
%     Note that the factorization 
%     $B\twoheadrightarrow B/Ker(g) \hookrightarrow C$ 
%     corresponds to the factorization 
%     $S\twotheadleftarrow f(T) \hookleftarrow T$
%%     We have a surjection $B\twoheadrightarrow B/Ker(g)$ in $\Boole$, 
%%     which corresponds to the inclusion $f(T) = Sp(B/Ker(g)) \subseteq S$. 
%     Ans $f(T) = Sp(B/Ker(g))$
%
%     Therefore $B/Ker(g):\Boole$, 
%     and the quotient map $B\twoheadrightarrow B/Ker(g)$ induces an inclusion 
%     $Sp(B/Ker(g)) \hookrightarrow Sp(B)$ with 
%     $f\in Sp(B/Ker(g))\leftrightarrow \forall_{n:\N}f(d_n) = 0 $. 
%     
%
%
%
%
%
%oldProof%     \rednote{TODO 
%oldProof%       The order of untracating is important in this proof, 
%oldProof%     and I struggle a bit with stressing this in a way this is clear (and concise). 
%oldProof%    Check with fresh eyes later. }
%oldProof%      Let $f:T\to S$ be a map between Stone spaces. 
%oldProof%      Assume $S = Sp(A), T = Sp (B)$. 
%oldProof%%      For this proof, we work with explicit presentations for $A,B$. 
%oldProof%%
%oldProof%      Let $G$ be a countable set of generators of $A$. 
%oldProof%      Assume also we have countable sets of generators and relations for $B$. 
%oldProof%%
%oldProof%      Following \Cref{FiberConstruction}, using $G$, for each $x:S$, we can construct 
%oldProof%      a countable set $I_x\subseteq B$ such that $$Sp(B/I_x) = (\Sigma_{y:T} f(y) = x) .$$
%oldProof%      By \Cref{ExplicitTruncationStoneClosed}, we can construct a sequence 
%oldProof%      $\alpha_x$ such that this type is inhabited iff $\forall_{n:\N} \alpha_x(n) = 0$,
%oldProof%      as required. 
%
%      Recall that the propositional truncation of a Stone is closed, as it is the negation of $0=1$ in the underlying 
%      Boolean algebra, which is open as it f
%
%
%      the core idea of the proof was that the closed proposition corresponds to checking equality in the underlying BA, 
%      which was closed as 
%
%
%
%
%      Note that $x$ in the image of $f$ iff $0\neq_{B/I_x} 1$. 
%      At this point, we have generators and relations of $B/I_x$ as data.
%      Hence using the proof of \Cref{BooleEqualityOpen}, we can construct a sequence 
%      $\alpha_x:2^\N$ such that $0 =_{B/I_x}1\leftrightarrow \exists_{n:\N} \alpha_x(n) = 0$. 
%      And for $\beta_x(n) = 1-\alpha_x(n)$, we conclude that 
%      \begin{equation}
%        x\in f(T) \leftrightarrow \forall_{n:\N} \beta_x(n) = 0
%      \end{equation}
%      Note that we did not use any choice axioms in the proof of this implication,
%      as we untruncated our assumptions before we specified $x$. 
   \item $(i) \to (v)$. By definition.
   \item $(v) \to (iv)$.
     %As $A$ is closed, it corresponds to a map $a:S\to \Closed$. 
     We have a surjection $2^\N\to\Closed$ defined by $\alpha \mapsto \forall_{n:\mathbb N} \alpha_n = 0.$
     \Cref{AxLocalChoice} gives us that there merely exists $T, e, \beta_\cdot$ as follows:
     \begin{equation}
       \begin{tikzcd}
         T \arrow[r,"\beta"] \arrow[d, two heads,swap,"e"] & 2^\mathbb N 
         \arrow[d,two heads] \\
         S \arrow[r,swap,"A"] & \Closed
       \end{tikzcd} 
     \end{equation} 
     Define $B(x) \leftrightarrow \forall_{n:\mathbb N} \beta_{x,n} = 0$. 
     As $(i) \to (iii)$ by the above, $B$ is the image of some Stone space. 
     Note that $A$ is the image of $B$, 
     thus $A$ is the image of some Stone space. 
\end{itemize} 
\end{proof} 

\begin{remark}\label{ClosedInStoneIsStone}
%Using condition $(iii)$, 
The previous result implies that closed subtype of Stone spaces are Stone.
\end{remark}

\begin{corollary}\label{InhabitedClosedSubSpaceClosed}
  For $S:\Stone$ and $A\subseteq S$ closed, we have 
  $\exists_{x:S} A(x)$ is closed. 
\end{corollary}
\begin{proof}
  By \Cref{ClosedInStoneIsStone}, $\Sigma_{x:S}A(x)$ is Stone, 
  so its truncation is closed by \Cref{TruncationStoneClosed}.
\end{proof}

\begin{corollary}\label{ClosedDependentSums}
  Closed propositions are closed under dependent sums. 
\end{corollary}
\begin{proof}
  Let $P:\Closed$ and $Q:P \to \Closed$. 
  Then $\Sigma_{p:P} Q(p) \leftrightarrow \exists_{p:P} Q(p)$.
  As $P$ is Stone by \Cref{PropositionsClosedIffStone}, 
  \Cref{InhabitedClosedSubSpaceClosed} gives that $\Sigma_{p:P} Q(p)$ is closed. 
\end{proof}
\begin{remark}\label{ClosedDominance}\label{ClosedTransitive}
  Analogously to \Cref{OpenTransitive} and \Cref{OpenDominance}, it follows that 
  closedness is transitive and $\Closed$ forms a dominance. 
\end{remark}

%We can get a dual to completeness.


%ImpliedByBooleEpiMono%\begin{lemma}\label{DualCompleteness}
%ImpliedByBooleEpiMono%Let $A$ and $B$ be c.p. boolean algebra with a map:
%ImpliedByBooleEpiMono%\[i:Sp(A)\to Sp(B)\] 
%ImpliedByBooleEpiMono%The following are equivalent:
%ImpliedByBooleEpiMono%\begin{enumerate}[(i)]
%ImpliedByBooleEpiMono%\item The induced map $B\to A$ is surjective.
%ImpliedByBooleEpiMono%\item The map $i$ is an embedding.
%ImpliedByBooleEpiMono%\item The map $i$ is a closed embedding.
%ImpliedByBooleEpiMono%\end{enumerate}
%ImpliedByBooleEpiMono%\end{lemma}
%ImpliedByBooleEpiMono%
%ImpliedByBooleEpiMono%\begin{proof}
%ImpliedByBooleEpiMono%  \item
%ImpliedByBooleEpiMono%\begin{itemize}
%ImpliedByBooleEpiMono%\item[$(i)\to (ii)$] Immediate.
%ImpliedByBooleEpiMono%\item[$(ii)\to (iii)$] By \Cref{StoneEqualityClosed} the fibers of $i$ are closed in $Sp(A)$, so by \Cref{ClosedInStoneIsStone} they are Stone, so they are closed by \Cref{PropositionsClosedIffStone}.
%ImpliedByBooleEpiMono%\item[$(iii)\to (i)$] By we have that $Sp(A) = \cap_{n:\N}D_n$ for $D_n$ decidable in $Sp(B)$. Assuming $D_n$ correponds to $b_n:B$ though duality, we then have that $A=B/ (b_n)_{n:\N}$ and the induced map is the quotient map:
%ImpliedByBooleEpiMono%$$B\to B/(b_n)_{n:\N}$$
%ImpliedByBooleEpiMono%which is surjective.
%ImpliedByBooleEpiMono%\end{itemize}
%ImpliedByBooleEpiMono%\end{proof}

\begin{lemma}\label{StoneSeperated}
  Assume $S:\Stone $ with $F,G:S \to \Closed$ be such that $F\cap G = \emptyset$. 
  Then there exists a decidable subset $D:S \to 2$ such $F\subseteq D, G \subseteq \neg D$. 
\end{lemma}
\begin{proof}
%  \rednote{Too shorten this (and some other proofs), I've removed some negations and pretended  $D:S\to 2$ is given by $\{x:S|x(d) = 0\}$ instead of $\{x:S|x(d) = 1\}$ }
  Assume $S = Sp(B)$. 
  By \Cref{StoneClosedSubsets}, for all $n:\N$ there is $f_n,g_n:B$ such that 
  $x\in F$ if and only if $x(f_n) = 0$ for all $n:\N$ and 
  $y\in G$ if and only if $y(g_m) = 0$ for all $m:\N$. 
%  $x\in F$ iff $x(f_n) = 1$ for all $n:\N$ and 
%  $y\in G$ iff $y(g_m) = 1$ for all $m:\N$. 
%
%  Denote $R\subseteq B$ for $\{\neg f_n|n:\N\} \cup\{\neg g_n|n:\N\}$. 
  Denote $R\subseteq B$ for $\{f_n|n:\N\} \cup\{g_n|n:\N\}$. 
  Note that $Sp(B/R) = F \cap G=\emptyset$, so by \Cref{SpectrumEmptyIff01Equal}
%  Note that any $x:Sp(B/R)$ is such %gives a map $x:B\to 2$ such that
%  $x(g_n)= x(f_n) = 1$ for all $n:\N$, hence $x\in F \cap G$. 
%  As $F\cap G = \emptyset $, it follows that $Sp(B/R)$ is empty.
%
  there exists finite sets $I,J\subseteq \N $ such that 
%  $$1 =_B \left(\left(\bigvee_{i\in I} \neg f_i\right) \vee \left(\bigvee_{j\in J} \neg g_j\right)\right).$$
%  $$1 =_B \left(\left(\bigvee_{i\in I}  f_i\right) \vee \left(\bigvee_{j\in J}  g_j\right)\right).$$
  $1 =_B ((\bigvee_{i\in I}  f_i) \vee (\bigvee_{j\in J}  g_j)).$
%
  Let $y\in F$, then $y(f_i) = 0$ for all $i\in I$, hence
%  $$1 =_2 y(1) = y\left(\bigvee_{j\in J} g_j\right).$$
  $y(\bigvee_{j\in J} g_j) = 1 $
  And if $x\in G$, we have 
%  $x\left(\bigvee_{j\in J} g_j\right) = 0$. 
  $x(\bigvee_{j\in J} g_j) = 0$. 
  Thus we can define the required $D$ by 
  $D(x) \leftrightarrow x(\bigvee_{j\in J} g_j) = 0$.
  %$$D = \{x:S | x\left(\bigvee_{j\in J} g_j \right) = 0\}$$.
  %, we have $F\subseteq D, G\subseteq \neg D$. 
  
%  Let $y\in G$. Then $y(\neg g_j) = 0$ for all $j \in J$. 
%  Let $y\in G$. Then $y(g_j) = 0$ for all $j \in J$. 
%  Hence 
%  $$
%  1 
%%  = y(1) 
%  =_2
%%  y(\bigvee_{i\in I} \neg f_i) = y (\neg (\bigwedge_{i\in I} f_i))
%  y\left(\left(\bigvee_{i\in I}  f_i\right) \vee \left(\bigvee_{j\in J}  g_j\right)\right)
%  = 
%%  \left(y\left(\bigvee_{i\in I}  f_i\right)\right) \vee \left(\bigvee_{j\in J} y(g_j)\right)
%%  = 
%  y\left(\bigvee_{i\in I}  f_i\right)
%  $$
%%  Thus $y(\bigwedge_{i\in I} f_i) = 0$. 
%%  Note that if $x\in F$, we have $x(f_i) = 1$ for all $i\in I$, hence 
%%  $x(\bigwedge_{i\in I} f_i) = 1$. 
%%  Thus for $D$ corresponding to $\bigwedge_{i\in I} f_i$, we have that 
%%  $F\subseteq D, G\subseteq \neg D$ as required. 
\end{proof} 

%RedundantByOpenInCHaus%\begin{corollary}\label{StoneOpenSubsets}
%RedundantByOpenInCHaus%  Let $A\subseteq S$ be a subset of a Stone space, then 
%RedundantByOpenInCHaus%  $A$ is open iff there exists some countable family $D_n,~n:\N$ of decidable subsets of $S$ with 
%RedundantByOpenInCHaus%  $A = \bigcup_{n:\N} D_n$. 
%RedundantByOpenInCHaus%\end{corollary}
%RedundantByOpenInCHaus%\begin{proof}
%RedundantByOpenInCHaus%  By \Cref{rmkOpenClosedNegation}, $A$ is open iff $\neg A$ is closed and $A = \neg \neg A$. 
%RedundantByOpenInCHaus%  By \Cref{StoneClosedSubsets}, $\neg A$ is closed iff 
%RedundantByOpenInCHaus%  $\neg A = \bigcap_{n\in \N} E_n$ for some countably family of decidable subsets $E_n,~n:\N$. 
%RedundantByOpenInCHaus%  Thus $\neg \neg A = \neg (\bigcap_{n\in \N} E_n)$. 
%RedundantByOpenInCHaus%  By MP (\Cref{MarkovPrinciple}), we have that 
%RedundantByOpenInCHaus%  $\neg (\bigcap_{n\in \N} E_n)= \bigcup_{n\in \N} \neg E_n$. 
%RedundantByOpenInCHaus%  Thus $D_n := \neg E_n$ is as required. 
%RedundantByOpenInCHaus%\end{proof}


\section{Compact Hausdorff spaces}
\begin{definition}
  A type $X$ is called a compact Hausdorff space if its identity types are closed propositions and there exists some $S:\Stone$ and a surjection $S\twoheadrightarrow X$.
\end{definition}

%This means that compact Hausdorff spaces are precisely quotients of Stone spaces by closed equivalence relations.

\subsection{Topology on compact Hausdorff spaces}

\begin{lemma}\label{CompactHausdorffClosed}
  Let $X:\CHaus$ with $S:\Stone$ and a surjective map $q:S\twoheadrightarrow X$.
  Then $A\subseteq X$ is closed if and only if it is the image of a closed subset of $S$ by $q$. 
\end{lemma}
\begin{proof}
%  If $A$ is closed, then it's pre-image under any map is also closed. 
%  In particular for $q:S\to X$ the quotient map, $q^{-1}(A)$ is closed. 
  As $q$ is surjective, we have $q(q^{-1}(A)) = A$.
  If $A$ is closed, so is $q^{-1}(A)$ and 
  hence $A$ is the image of a closed subtype of $S$. 
  Conversely, let $B\subseteq S$ be closed. 
%  Then for any $s:S$, the subtype $\{t:S| B(s) \wedge s \sim t\} \subseteq S$ is closed. 
%  Hence by 
  Define $A'\subseteq S$ by 
  $$A'(s) := \exists_{t:S} (B(t) \wedge q(s) = q(t)).$$
  As $B(t)$ and $q(s) = q(t)$ are closed, by \Cref{ClosedCountableConjunction} and \Cref{InhabitedClosedSubSpaceClosed}, 
  we have that $A'$ is closed. 
  Also $A'$ factors through $q$ as a map $A: X\to \Closed$.
  Furthermore, $A'(s) \leftrightarrow (q(s)\in q(B))$. 
  Hence $A=q(B)$. 
%  Therefore $A(x)$ iff $x$ is in the image of $B$. 
\end{proof}

\begin{remark}\label{InhabitedClosedSubSpaceClosedCHaus}
  Let $X:\Chaus$.
  From \Cref{StoneClosedSubsets}, it follows that $A\subseteq X$ is closed if and only if it is the image of a map 
  $T\to X$ for some $T:\Stone$. 
  If $A$ is closed, it follows from \Cref{InhabitedClosedSubSpaceClosed} that $\exists_{x:X} A(x)$ is closed as well. 
\end{remark}
%\begin{corollary}
%  For $X:\CHaus$ a subtype $A\subseteq X$ is closed iff it is the image of 
%  a map $T\to X$ for some $T:\Stone$. 
%\end{corollary}
%\begin{proof}
%  Directly from the above and \Cref{StoneClosedSubsets}.
%\end{proof}
%WhyDidWeNeedThis%\begin{remark}
%WhyDidWeNeedThis%  It is not the case that every closed subset of a compact Hausdorff space can be written 
%WhyDidWeNeedThis%  as countable intersection of decidable subsets. 
%WhyDidWeNeedThis%  In \Cref{UnitInterval}, we shall introduce the unit interval $[0,1]$ as a compact Hausdorff space with many closed 
%WhyDidWeNeedThis%  subsets, but only two decidable subsets. 
%WhyDidWeNeedThis%  In \Cref{ConnectedComponent}, we shall actually see that whenever every singleton of a compact Hausdorff space $X$
%WhyDidWeNeedThis%  can be written as countable intersection of decidable subsets, $X$ is Stone. 
%WhyDidWeNeedThis%  \rednote{Actually, we'll see that $Sp(2^X)$ and $X$ are bijective sets, 
%WhyDidWeNeedThis%    which only implies that $X$ is Stone if $2^X:\Boole$, but this depends on our definition of countable, 
%WhyDidWeNeedThis%see \Cref{CountabilityDiscussion}}
%WhyDidWeNeedThis%\end{remark}


\begin{corollary}\label{AllOpenSubspaceOpen}
  For $U\subseteq X$ an open subset of a compact Hausdorff space, we have that the proposition 
  $\forall_{x:X} U(x)$ is open. 
\end{corollary}
\begin{proof}
  As $U$ is open, $\neg U$ is closed. 
  So $\exists_{x:X} \neg U(x)$ is closed by \Cref{InhabitedClosedSubSpaceClosedCHaus}. 
  Using \Cref{rmkOpenClosedNegation}, it follows that 
  $\neg (\exists_{x:X} \neg U(x))$ is open. 
  Furthermore, it is equivalent to $\forall_{x:X} \neg \neg U(x)$, 
  which is equivalent to $\forall_{x:X} U(x)$ by \Cref{rmkOpenClosedNegation}.
\end{proof}

\begin{lemma}\label{CHausFiniteIntersectionProperty}
  Given $X:\Chaus$ and $C_n:X\to \Closed$ closed subsets such that $\bigcap_{n:\N} C_n =\emptyset$, there is some $k:\N$ 
  with $\bigcap_{n\leq k} C_n  = \emptyset$. 
\end{lemma}
\begin{proof}
  By \Cref{CompactHausdorffClosed} it is enough to prove the result when $X$ is Stone, and by \Cref{StoneClosedSubsets} we can assume $C_n$ decidable.
  So assume 
  $X=Sp(B)$ and $c_n:B$ such that: 
  $$C_n = \{x:B\to 2\ |\ x(c_n) = 0\}.$$ 
  Then the set of maps $B\to 2$ sending all $c_n$ to $0$ is given by: 
  $$Sp(B/(c_n)_{n:\N})%\ |\ n:\N\}) 
  \simeq \bigcap_{n:\N} C_n = \emptyset .$$
  Hence 
%  $0=_{B/(\neg c_n)_{n:\N}}1$ 
  $0=1$ in $B/(c_n)_{n:\N}$ %\ |\ n:\N\}$, 
  and there is some $k:\N$ with 
  $\bigvee_{n\leq k} c_n = 1$, which also means that: 
  $$\emptyset = Sp(B/(c_n)_{n\leq k}) %\ |\ n\leq k\})  
  \simeq \bigcap_{n\leq k} C_n $$
  as required.
\end{proof}

\begin{corollary}\label{ChausMapsPreserveIntersectionOfClosed}
  Let $X,Y:\CHaus$ and $f:X \to Y$. 
  Suppose $(G_n)_{n:\N}$ is a decreasing sequence of closed subsets of $X$. 
  Then $f(\bigcap_{n:\N} G_n) = \bigcap_{n:\N}f(G_n)$. 
\end{corollary}
\begin{proof}
  It is always the case that $f(\bigcap_{n:\N} G_n) \subseteq \bigcap_{n:\N} f(G_n)$. 
  For the converse direction, suppose that $y \in f(G_n)$ for all $n:\N$. 
  We define $F\subseteq X$ closed by $F=f^{-1}(y)$. 
  Then for all $n:\N$ we have that $F\cap G_n$ is merely inhabited and therefore non-empty. 
  By \Cref{CHausFiniteIntersectionProperty} this implies that $\bigcap_{n:\N}(F\cap G_n) \neq \emptyset$. 
  By \Cref{InhabitedClosedSubSpaceClosedCHaus} and \Cref{rmkOpenClosedNegation}, we have that $\bigcap_{n:\N} (F\cap G_n)$ is merely inhabited. Thus $y\in f(\bigcap_{n:\N} G_n)$ as required. 
\end{proof}

\begin{corollary}\label{CompactHausdorffTopology}
Let $A\subseteq X$ be a subset of a compact Hausdorff space and $p:S\twoheadrightarrow X$ be a surjective map with $S:\Stone$. Then $A$ is closed (resp. open) if and only if there exists a sequence $(D_n)_{n:\N}$ of decidable in $S$ such that $A = \bigcap_{n:\N} p(D_n)$ (resp. $A = \bigcup_{n:\N} \neg p(D_n)$).
%\begin{itemize}
%  \item $A$ is closed iff %if and only if 
%    it can be written as $\bigcap_{n:\N} p(D_n)$
%for some $D_n\subseteq S$ decidable. 
%  \item $A$ is open iff %if and only if 
%    it can be written as $\bigcup_{n:\N} \neg p(D_n)$
%for some $D_n\subseteq S$ decidable.
%\end{itemize}  
\end{corollary}
\begin{proof}
  The characterization of closed sets follows from characterization (ii) in \Cref{StoneClosedSubsets}, 
  \Cref{CompactHausdorffClosed} 
  and \Cref{ChausMapsPreserveIntersectionOfClosed}. 
%  The characterization of open sets 
  For open sets we use \Cref{rmkOpenClosedNegation} and
  \Cref{ClosedMarkov}.
\end{proof}
%
\begin{remark}
  For $S:\Stone$, there is a surjection $\N\twoheadrightarrow 2^S$. 
  It follows that for any $X:\CHaus$ there is a surjection from $\N$ to a basis of $X$. 
  Classically this means that $X$ is second countable. 
\end{remark}
%It follows that compact Hausdorff spaces are second countable:
%\begin{corollary}
%  Any $X:\Chaus$ is has a topological basis which is countable.
%\end{corollary}
%\begin{proof}
%  By \Cref{CompactHausdorffTopology}, 
%  a basis is given by the images of the decidable subsets of some $S:\Stone$. 
%  By \cref{ODiscBAareBoole}, $2^S$ is 
%  overtly discrete so we have a surjection $\N\to 2^S$.
%  \end{proof}
%
\begin{lemma}\label{CHausSeperationOfClosedByOpens}
 Assume $X:\CHaus$ and $A,B\subseteq X$ closed such that $A\cap B=\emptyset$. 
  Then there exist $U,V\subseteq X$ open such that $A\subseteq U$, $B\subseteq V$ and $U\cap V=\emptyset$. 
\end{lemma}
\begin{proof}
  Let $q:S\to X$ be a surjective map with $S:\Stone$.
  As $q^{-1}(A)$ and $q^{-1}(B)$ are closed, 
  by \Cref{StoneSeperated}, there is some $D:S \to 2$ such that
  $q^{-1}(A) \subseteq D, q^{-1}(B) \subseteq \neg D$. 
  Note that $q(D)$ and $q(\neg D)$ are closed by \Cref{CompactHausdorffClosed}. 
  As $q^{-1}(A) \cap \neg D  =\emptyset$, we have that 
  $A\subseteq \neg q(\neg D)$. As $A\cap B = \emptyset$, we have that 
  $A\subseteq U:= \neg q(\neg D) \cap \neg B$.
  Similarly, $B\subseteq V:=\neg  q(D) \cap \neg A$. 
  Then $U$ and $V$ are disjoint because $\neg q(D)\cap \neg q(\neg D) \subset \neg (q(D)\cup q(\neg D)) = \neg X = \emptyset$.
\end{proof}


\subsection{Compact Hausdorff spaces are stable under dependent sums}

\begin{lemma}\label{StoneAsClosedSubsetOfCantor}
A type $X$ is Stone iff it is merely a closed in $2^\N$.
\end{lemma}
\begin{proof}
  By \Cref{BooleAsCQuotient}, any $B:\Boole$ is can be written as $C/(r_n)_{n:\N}$.
  By \Cref{BooleEpiMono}, the quotient map induces an embedding $Sp(B)\hookrightarrow Sp(C)= 2^\N$, 
  which is closed by 
  by \Cref{StoneClosedSubsets}.
\end{proof}

%
%\begin{proof}
%Any countably presented boolean algebra $B$ is enumerable, which gives a surjective morphism:
%$$ 2[\N]\to B$$
%so that by \Cref{DualCompleteness} we merely have a closed embedding:
%$$ Sp(B)\to 2^\N$$
%\end{proof}
\rednote{Can we maybe combine the next two Lemmas?}
\begin{lemma}\label{SigmaStoneCompactHausdorff}
Assume $S:\Stone$ and $T:S\to \Stone$. Then $\Sigma_{x:S}T(x)$ is Compact Hausdorff.
\end{lemma}

\begin{proof}
  By \Cref{ClosedDependentSums} and \Cref{StoneEqualityClosed}, the identity types in $\Sigma_{x:S}T(x)$ are closed.
  By \Cref{StoneAsClosedSubsetOfCantor} %, there is a surjection 
%  $\Sigma_{y:2^\N}(\cdot)(y):(2^\N \to \Closed) \twoheadrightarrow \Stone$. 
%  By \Cref{AxLocalChoice}, it follows we have some $S':\Stone$, a surjection $q:S'\twoheadrightarrow S$ and 
%
  we have for each $x:S$ that 
  $\exists_{A:2^\N\to \Closed} T(x) = \Sigma_{y:2^\N}A(y)$. 
  Using \Cref{AxLocalChoice} we get $S':\Stone$ with a surjective map:
  $q:S'\to S$ 
%$$q:S'\to S$$
and:
$ C : S'\to (2^\N\to\Closed)$
%$ C : (S'\times 2^\N)\to\Closed$
  such that for all $x:S'$ we have 
  $T(q(x)) = \Sigma_{y:2^\N}C(x,y).$
This gives a surjective map:
%$$ \Sigma_{x:S'}\Sigma_{y:2^\N} C(x,y)\twoheadrightarrow \Sigma_{x:S}T(x)$$
$$ \Sigma_{c:(S'\times 2^\N)}C(c)\twoheadrightarrow \Sigma_{x:S}T(x)$$
%By \Cref{StoneClosedUnderPullback}, $S'\times 2^\N$ is Stone, 
%as $C(x,y):\Closed$ for any $(x,y):S'\times 2^\N$, by $\Cref{ClosedInStoneIsStone}$, it follows
The source is Stone by \Cref{StoneClosedUnderPullback} and \Cref{ClosedInStoneIsStone} so we can conclude.
\end{proof}

\begin{lemma}
Assume $X:\CHaus$ and $T:X\to \CHaus$. Then $\Sigma_{x:X}T(x)$ is Compact Hausdorff.
\end{lemma}

\begin{proof}
By \Cref{ClosedDependentSums} we have that identity type in $\Sigma_{x:X}T(x)$ are closed.
%
We know that for any $x:X$ we have $\exists_{Y:\Stone} S'\twoheadrightarrow C(x)$. 
Consider the quotient map $q:S \twoheadrightarrow  X$ with $S:\Stone$. 
By \Cref{AxLocalChoice} we get $S':\Stone$ with a surjective map: $e:S'\to S$
such that for all $x:S'$ we have $Y(x):\Stone$ and a surjective map $Y(x)\to T(q(e(x)))$. 
This gives a surjective map:
$$ \Sigma_{x:S'}Y(x)\to \Sigma_{x:X}T(x)$$
By \Cref{SigmaStoneCompactHausdorff} we have a surjective map from a Stone space to the source so we can conclude.
\end{proof}

\subsection{Stone spaces are stable under dependent sums}
We will show that Stone spaces are precisely totally disconnected compact Hausdorff spaces. 
We will use this to prove that a dependent sum of Stone spaces is Stone.

\begin{lemma}\label{AlgebraCompactHausdorffCountablyPresented}
Assume $X:\Chaus$, then $2^X$ is countably presented.
\end{lemma}

\begin{proof}
  Consider some quotient map $q:S\twoheadrightarrow X$ with $S:\Stone$. 
%First we choose $S\to X$ surjective with $S$ Stone and prove that $2^X$ is an open subalgebra of $2^S$.
%
  This induces an injection of Boolean algebras $2^X \hookrightarrow 2^S$.
  Note that $a:S\to 2$ lies in $2^X$ iff %for all $s,t:S$, 
  $$\forall_{s,t:S}\ q(s) =_X q(t) \to a(s) =_2a(t).$$
  As equality in $X$ is closed and equality in $2$ is decidable, so \Cref{ImplicationOpenClosed}
  tells us that the implication is open for every $s,t:S$. 
  By \Cref{AllOpenSubspaceOpen}, we conclude that 
%  $\forall_{s:S} \forall_{t:S} ((q(s) =_X q(t)) \to (a(s) =_2 a(t)))$ is open. 
%  Hence 
  $2^X$ is an open subalgebra of $2^S$. 
  Therefore, it is in $\ODisc$ by
  \Cref{PropOpenIffOdisc} and \Cref{OdiscSigma} 
  and in $\Boole$ by \Cref{ODiscBAareBoole}.
%
%
%  \rednote{It might be nice to show that Boolean algebras are countably presented iff they are overtly discrete}
%Now we prove that open subalgebras of countably presented agebras are countably presented. Assume $U\subset 2[\N] / F$ such a subalgebra. We have that $U$ is equivalent to the algebra generated by the $s:2[\N]$ such that $[s]\in U$ quotiented by the relation $s=t$ for all $s,t:2[\N]$ such that $[s],[t]\in U$ and $[s]=[t]$.
%
%Using that $2[\N]$ is countable and that $[s]=[t]$ is open by \Cref{BooleEqualityOpen}, we see that $U$ is generated by variables and relations each indexed by an open in $\N$. But by \Cref{OpenInNAreDecidableInN} any open in $\N$ is countable, so $U$ is countably presented.
\end{proof}
\begin{definition}
For all $X:\Chaus$ and $x:X$,
  we define $Q_x$ the connected component of $x$
  as the intersection of all decidable subsets of $X$ containing $x$. 
\end{definition}

\begin{lemma}\label{ConnectedComponentClosedInCompactHausdorff}
For all $X:\CHaus$ with $x:X$, we have that $Q_x$ is a countable intersection of decidables in $X$.
\end{lemma}
\begin{proof}
%  By \Cref{AlgebraCompactHausdorffCountablyPresented} we have that $2^X$ is countably presented, 
%  therefore we can enumerate the elements of $2^X$, say as $(D_n)_{n:\N}$. 
  By \Cref{AlgebraCompactHausdorffCountablyPresented},
  we can enumerate the elements of $2^X$, say as $(D_n)_{n:\N}$. 
  Define $E_n$ for $n:\N$ as $D_n$ if $x\in D_n$ and $X$ otherwise. 
  Then $\cap_{n:\N}E_n = Q_x$.
\end{proof}
%

\begin{lemma}\label{ConnectedComponentSubOpenHasDecidableInbetween}
  Assume $X:\Chaus$ with $x:X$ and suppose $U\subseteq X$ open with $Q_x\subseteq U$. 
  Then we have some decidable $E\subseteq X$ with $x\in E$ and $E\subseteq U$. 
\end{lemma}
\begin{proof}
  By \Cref{ConnectedComponentClosedInCompactHausdorff}, 
  we have $Q_x = \bigcap_{n:\N}D_n$ with $D_n\subseteq X$ decidable. 
  If $Q_x \subseteq U$, we have that 
  $$Q_x\cap \neg U = \bigcap_{n:\N} (D_n \cap \neg U) = \emptyset.$$
  By \Cref{CHausFiniteIntersectionProperty} there is some $k:\N$ with 
  $$(\bigcap_{n\leq k} D_n )\cap \neg U  = \bigcap_{n\leq k} (D_n \cap \neg U) = \emptyset.$$
  Therefore $\bigcap_{n\leq k} D_n \subseteq \neg\neg U$, which equals $U$ by \Cref{rmkOpenClosedNegation}. So $\bigcap_{n\leq k} D_n$ gives us the desired decidable subset.
\end{proof}

\begin{lemma}\label{ConnectedComponentConnected}
Assume $X:\Chaus$ with $x:X$. Then any map $Q_x\to 2$ is constant.
\end{lemma}
\begin{proof}
Assume given a separation $Q_x = A\cup B$ with $A,B$ disjoint and decidable in $Q_x$. Assume $x\in A$. 
%We will show $B=\emptyset$. 
%
By \Cref{ConnectedComponentClosedInCompactHausdorff}, $Q_x\subseteq X$ is closed. 
Using \Cref{ClosedTransitive}, it follows that $A,B\subseteq X$ are closed and disjoint.
By \Cref{CHausSeperationOfClosedByOpens} there exist $U,V$ disjoint open such that $A\subseteq U$ and $B\subseteq V$. 
%
By \Cref{ConnectedComponentSubOpenHasDecidableInbetween} we have a decidable $D$ such that $Q_x\subseteq D\subseteq U\cup V$. 
Note that $D\cap U = D \cap (\neg V):=E$ is clopen, hence decidable by \Cref{ClopenDecidable}.
Remark that $x\in E$, hence $B\subseteq Q_x \subseteq E$ but $B \cap E = \emptyset$, hence $B=\emptyset$. 
%Then we define $E = D\cap U$. 
%We have that $E$ is open, it is closed as $E=D\cap \neg V$, therefore it is decidable by \Cref{ClopenDecidable}.
%
%Then $Q_x\subset E$ with $E$ decidable and $B\cap E = \emptyset$. 
%But then $Q_x\cap B = \emptyset$ and $B=\emptyset$.
\end{proof}

\begin{lemma}\label{StoneCompactHausdorffTotallyDisconnected}
Let $X:\CHaus$, then $X$ is Stone if and only $\forall_{x:X}\ Q_x=\{x\}$.
\end{lemma}

\begin{proof}
  By \Cref{AxStoneDuality}, it is clear that for all $x:S$ with $S:\Stone$ we have that $Q_x=\{x\}$.
%
  Conversely, assume $X:\CHaus$ such that $\forall_{x:X}\ Q_x = \{x\}$.
  We claim that the evaluation map $e:X \to Sp(2^X)$ is both injective and surjective, hence an equivalence. 
%  \item 
    Let $x,y:X$. If $f(x) = f(y)$ for all $f:2^X$, then $y \in Q_x$, hence $x=y$ by assumption. 
    Thus $e$ is injective. 
%  \item 
    Let $q:S\twoheadrightarrow X$ be a surjective map. 
    It induces an injection $2^X \hookrightarrow 2^S$, which by \Cref{SurjectionsAreFormalSurjections}
    induces a surjection $Sp(2^S) \twoheadrightarrow Sp(2^X)$. 
    Note that $e\circ q$ factors as $S\simeq Sp(2^S) \twoheadrightarrow Sp(2^X)$. 
    It follows that $e$ is surjective. 
%
%For the converse, we show that the map:
%\[X\to Sp(2^X)\]
%is an equivalence and conclude by \Cref{AlgebraCompactHausdorffCountablyPresented}. 
%
%Surjectivity always hold, indeed considering $q:S\to X$ surjective with $S$ Stone, we have that $2^X\subset 2^S$ as so that the by \Cref{FormalSurjectionsAreSurjections} the map:
%$$S = Sp(2^S)\to Sp(2^X)$$
%is surjective and it factors though $X$.
%
%Now let us prove injectivity. Assume $x,y:X$ having the same image in $Sp(2^X)$. This means that any map in $X\to 2$ has the same value on $x$ and $y$, so $x\in Q_y$ and by hypothesis $x=y$.
\end{proof}

\begin{theorem}
  \label{stone-sigma-closed}
Assume $S:\Stone$ and $T:S\to\Stone$. Then $\Sigma_{x:S}T(x)$ is Stone.
\end{theorem}

\begin{proof}
By \Cref{SigmaCompactHausdorff} we have that $\Sigma_{x:S}T(x)$ is compact Hausdorff. 
By \Cref{StoneCompactHausdorffTotallyDisconnected} 
it is enough to show that for all $x:S$ and $y:T(x)$ 
we have that $Q_{(x,y)}$ is a singleton.
%
Assume $(x',y')\in Q_{(x,y)}$, then for any map $f:S\to 2$ we have that:
$$ f(x) = f\circ \pi_1(x,y) = f\circ \pi_1(x',y') = f(x')$$
so that $x'\in Q_x$ and since $S$ is Stone by \Cref{StoneCompactHausdorffTotallyDisconnected} we have that $x=x'$.
%
Therefore we have $Q_{(x,y)}\subseteq \{x\}\times T(x)$. Assume $z,z':Q_{(x,y)}$, then for any map $g:T(x)\to 2$ we have that $g(z)=g(z')$ by \Cref{ConnectedComponentConnected}. Since $T(x):\Stone$, we conclude $z=z'$ by \Cref{StoneCompactHausdorffTotallyDisconnected}.
%Therefore we have $Q_{(x,y)}\subseteq \{x\}\times T_{x}$. 
%By \Cref{ConnectedComponentConnected} we have an inhabited 
%connected subtype of a Stone space. 
%Then any map $T_x\to 2$ is constant on $Q_{(x,y)}$ and by 
%\Cref{StoneCompactHausdorffTotallyDisconnected} we conclude that it is a singleton.
\end{proof}




%\section{The Unit interval}
\subsection{The unit interval as a compact Hausdorff space}
Since we have dependent choice, the unit interval $\mathbb I = [0,1]$ can be defined using 
Cauchy reals or Dedekind reals. 
We can freely use results from constructive analysis \cite{Bishop}. 
As we have $\neg$WLPO, MP and LLPO, we can use the results from 
constructive reverse mathematics that follow from these principles \cite{ReverseMathsBishop, HannesDiener}. 
%\begin{definition}
%  We define $cs:2^\N \to \I$ as 
%  $cs(\alpha) = \sum_{n:\N} \frac{\alpha(i)}{2^{i+1}}$. 
%\end{definition}
\begin{definition}
  \label{def-cs-Interval}
  We define for each $n:\N$ the Stone space $2^n$ of binary sequences of length $n$.
  % = \Sp(2[\Fin(n)])$.
  And we define $cs_n:2^n \to \mathbb Q$ by 
  $cs_n(\alpha) = \sum_{i < n } \frac{\alpha(i)}{2^{i+1}}.$
  Finally we write $\sim_n$ for the binary relation on $2^n$ given by 
  $\alpha\sim_n \beta 
  \leftrightarrow \left|cs_n(\alpha) - cs_n(\beta)\right|\leq\frac{1}{2^n}$.
\end{definition}
\begin{remark}
  The inclusion $Fin(n) \hookrightarrow \N$ induces a restriction 
  $\_|_n : 2^\N \to 2^n$ for each $n:\N$. 
\end{remark}
\begin{definition}
  We define $cs:2^\N \to \I$ as 
  $cs(\alpha) = 
  \sum_{i :\N } \frac{\alpha(i)}{2^{i+1}}.
  $
%  \lim_{n\to\infty} cs_n(\alpha|_n)$. 
\end{definition}

\begin{theorem}\label{IntervalIsCHaus}
  $\I$ is compact Hausdorff.
\end{theorem}
\begin{proof}
  By LLPO, we have that $cs$ is surjective.   
  Note that $cs(\alpha) = cs(\beta)$ if and only if 
  for all $n:\N$ we have $\alpha|_n \sim_n \beta|_n$. 
  %$\left|cs_n(\alpha)-cs_n(\beta)\right|\leq \frac{1}{2^n}$
%  $$|\sum_{n=0}^{n-1} \frac{\alpha(i)}{2^{i+1}}-
%  \sum_{n=0}^{n-1} \frac{\beta(i)}{2^{i+1}}|\leq \frac{1}{2^n}$$
%  for all $n:\N$, 
  This is a countable conjunction of decidable propositions.
%  as inequality in $\mathbb Q$ is decidable. 
\end{proof}


%In this section we will introduce the unit interval $I$ as compact Hausdorff space. 
The definition is based on \cite{Bishop}. 
%We will then calculate the cohomology of $I$. 
%For a proof that the unit interval corresponds to the definition using Cauchy sequences, 
%we refer to the appendix. 


%\subsection{The Cauchy reals}
The goal of this section is to introduce the real numbers in a constructive setting, 
following the definition given in \cite{Bishop} with some small adaptations. 
We will later use this definition to show that the interval $[0,1]$ is compact Hausdorff in the sense 
of \Cref{dfnCompactHausdorff}. 

We will assume we are given natural and rational numbers, with decidable (in)equalities
working as expected. 

\begin{definition}
  A \textbf{Cauchy sequence} is a sequence $x : \N \to \mathbb Q$ such that
  for any $n,m:\N$, we have %$0\leq x_n \leq 1$ and 
$|x_n-x_m| \leq (\frac12)^n + (\frac12)^m$. 
\end{definition}
\begin{remark}
  If $x$ is a cauchy sequence and $q$ a rational number, the 
  sequence $(x-q)_n = (x_n-q)$ is also Cauchy.
\end{remark}

Following \cite{Bishop}, we define inequality relations between Cauchy sequences and
rational numbers. 
\begin{definition}
  For $x$ a Cauchy sequence and $q$ a rational number, we define 
  \begin{itemize}
%    \item $x> q = \Sigma(n:\N) x_n > q + {\frac12}^n$. %for some $n:\N$. 
%    \item $x< q = \Sigma(n:\N) x_n < q - {\frac12}^n$. %for some $n:\N$. 
    \item $x\leq  q = \Pi_{n:\N} x_n \leq q+(\frac12)^n$. 
    \item $x\geq  q = \Pi_{n:\N} x_n \geq q-(\frac12)^n$. 
  \end{itemize}
\end{definition}
%\begin{lemma}
%  For $x$ Cauchy and $q$ rational, we have that 
%  $x\leq q$ iff for each $n:\N$, we have a $N_n:\N$ with 
%  $x_m> q-(\frac12)^n$ for all $m \geq N_n$. 
%\end{lemma}
\begin{lemma}\label{ComparisonLemma}
  For $x$ a Cauchy sequence and $q$ a rational number, we have
  $x \leq q \vee x \geq q$. 
\end{lemma}
\begin{proof}
  For rational numbers, we have decidable inequalities, 
  therefore $\geq 0 \vee q \leq 0$. 
  It follows that 
  $ \forall (n:\N) \forall (m:\N) q \geq -(\frac12)^n \vee q \leq (\frac12)^m$. 
  Now by \Cref{TODO}, we may conclude 
  $ (\forall (n:\N) q \geq -(\frac12)^n ) \vee (\forall (m:\N) q \leq (\frac12)^m)$
  as required.
\end{proof}


%%%\begin{definition}
%%%  A Cauchy sequence $x$ is \textbf{nonnegative} if $x_n \geq -(\frac12)^n$
%%%  for every $n:\N$. 
%%%  $x$ is \textbf{nonpositive} if $x_n \leq (\frac12)^n$
%%%  for every $n:\N$. 
%%%\end{definition} 
%%%%\begin{lemma}
%%%%  A Cauchy sequence is nonnegative iff there exists an $N$ such that $x_n \geq -(\frac12)^N$
%%%%  for all $n\geq N$. 
%%%%  A Cauchy sequence is nonpositive iff there exists an $N$ such that $x_n \leq (\frac12)^N$
%%%%  for all $n\geq N$. 
%%%%\end{lemma}
%%%%\begin{proof}
%%%%  Assume $x$ is nonnegative. Thus for every $n:\N$, we have $x_n\geq -(\frac12)^n \geq -(\frac12)^0$. 
%%%%  Thus $N$ can taken to be $0$. 
%%%%%
%%%%  Conversely, as $x$ is Cauchy, we have
%%%%  for all $m :\N$ that  
%%%%%  \begin{equation}- (\frac12)^m -(\frac12)^n \leq    x_m-x_n \leq (\frac12)^m + (\frac12)^n \end{equation}
%%%%  \begin{equation}- (\frac12)^m -(\frac12)^n \leq    x_n-x_m \leq (\frac12)^m + (\frac12)^n \end{equation}
%%%%  If in addition there is an $N$ such that whenever $m\geq N$, we have 
%%%%  $x_m \geq -(\frac12)^N$, so $-x_m \leq (\frac12)^N$, 
%%%%  so $x_n -x_m \leq x_n - (\frac12)^N$. 
%%%%  Therefore, 
%%%%  \begin{equation}- (\frac12)^m -(\frac12)^n \leq    x_n-x_m \leq x_n-(\frac12)^N \end{equation}
%%%%  Thus 
%%%%  \begin{equation}- (\frac12)^m -(\frac12)^n  + (\frac12)^N \leq x_n \end{equation}
%%%%  As $N \geq N$, we have in particular 
%%%%  \begin{equation}- (\frac12)^N -(\frac12)^n  + (\frac12)^N \leq x_n \end{equation}
%%%%  \begin{equation} - (\frac12)^n  \leq x_n \end{equation}
%%%%  thus $x$ is nonnegative. 
%%%%
%%%%  The nonpositive case goes similar. 
%%%%\end{proof}   
%%% 
%%%
%%%\begin{lemma}
%%%  A Caucy sequence is nonnegative or nonpositive. 
%%%\end{lemma}

%\begin{lemma}
%  For any Cauchy sequence $p$, we have 
%  $(\forall (n:\N) p_n \leq (\frac12)^n) \vee (\forall (n:\N) p_n \geq -(\frac12)^n)$. 
%\end{lemma}
%\begin{proof}
%We 
%\end{proof}

\begin{definition}
Given two Cauchy sequences $p = (p_n)_{n\in\N}, q=(q_n)_{n\in\N}$, 
we define the proposition $p \sim_C  q$ as 
\begin{equation}
  p \sim_C q : = \forall (n,m : \N) ((| p_n - q_m| \leq  (\frac12)^n + (\frac12)^m))
\end{equation}
\end{definition}

%\begin{remark}
%  Note that $p\sim_C q$ is equivalent to 
%\begin{equation}
%  \forall (n : \N) | p_n - q_n| \leq  (\frac12)^{n-1}
%\end{equation}
%The equivalence doesn't hold, unless you cut off initial segments (which shouldn't matter). 
%\end{remark} 

\begin{definition}
  The type of \textbf{Cauchy reals} is given by 
  the type of Cauchy sequences modulo $\sim_C$.
\end{definition}

We claim that the inequality in \Cref{TODO} extends to a well-defined 
inequality between Cauchy reals and rational numbers. 

Furthermore, we claim that 
$\Pi_{x:\mathbb R} \Pi_{q:\mathbb Q} x \leq q \vee x \geq q$. 

%\begin{lemma}
%  For any Cauchy real $r$ any Cauchy sequence $p$ representing $r$, 
%  we have 
%  \begin{equation}
%    (\forall (n:\N) p_n \leq (\frac12)^n) \vee (\forall (n:\N) p_n \geq (\frac12)^n)
%  \end{equation}
%
%\end{lemma}

\begin{definition}
  A Cauchy sequence in the interval is a Cauchy sequence $x$ such that 
  for any $n:\N$, we have $0\leq x_n \leq 1$. 
 % 
  The interval of Cauchy reals is given by the type of Cauchy sequences in the interval 
  modulo $\sim_C$. We denote it by $[0,1]$. 
\end{definition}  


%------------------content of file BinaryClosedEquivalence.tex...
%We want to show that the interval of Cauchy reals is Compact Hausdorff. 
%Informally, to any binary sequence $\alpha : \N \to 2$, 
%we can associate a Cauchy sequence 
%$cs(\alpha)$, given by 
%\begin{equation}\label{eqnBinaryEncode}
%  (cs(\alpha))_n = \sum\limits_{i = 0 }^n \frac {\alpha(i)}{2^{i+1}}
%\end{equation}
%and we are going to give a closed relation on Cantor space such that 
%two binary sequences are equivalent iff they correspond to the same Cauchy reals. 
\begin{example}
  Let $n:\N$, we denote $C_n = 2[n]$ for the free Boolean algebra on $n$ generators 
  and no relations. 
  Note that $Sp(C_n) = 2^n$ corresponds to the space of finite binary sequences. 
\end{example}
Now we introduce some notation:
\begin{definition}
  \item Given an infinite binary sequence $\alpha:2^\N$ and a natural number $n : \N$  
    we denote $\alpha|_n: 2^n$ for the 
    restriction of $\alpha$ to a finite sequence of length $n$. 
  \item We denote $\overline 0, \overline 1$ 
    for the binary sequences which are constantly $0$ and $1$ respectively. 
  \item We denote $0,1$ for the sequences of length $1$ hitting $0,1$ respectively. 
  \item If $x$ is a finite sequence and $y$ is any sequence, 
    denote $x\cdot y$ for their concatenation. 
\end{definition} 
Now we'll give a definition for when two finite binary sequences of length $n$ correspond 
to real numbers whose distance is $\leq (\frac12)^n$.
Informally, we want for every finite sequence $s$ that 
$(s \cdot 0 \cdot \overline 1)$ and  $(s \cdot 1 \cdot \overline 0)$ are equivalent. 

\begin{definition}
  Let $n:\N$ and let $s,t : 2^n$. 
  We say $s,t$ are $n$-near, and write $s\sim_n t$ if 
  there merely exists some $m:\N$ and some $u:2^m$, such that 
 \begin{equation}\label{EqnNearness}
   \big(
     (s = (u\cdot 0\cdot \overline 1)|_n) \vee (s = (u \cdot 1 \cdot \overline 0) |_n)
   \big)
    \wedge 
   \big(
     (t = (u\cdot 0\cdot \overline 1)|_n) \vee (t = (u \cdot 1 \cdot \overline 0) |_n)
   \big)
  \end{equation} 
\end{definition}
\begin{remark}\label{nearnessProperties}
\item As we're dealing with finite sequences, $s\sim_n t$ is decidable. 
\item Given any $s:2^n$, using $m=n, u = s$ above, we can show that $s\sim_n s$. 
  So $n$-nearness is reflexive. 
\item \Cref{EqnNearness} is symmetric in $s$ and $t$. Hence $n$-nearness is symmetric.
\item Note that $0\cdot 0\sim_2 0\cdot 1 \sim_2 1\cdot 0 \sim_2 1\cdot 1$, 
  but $0\cdot 0\nsim_2 1\cdot 1$. %is not $2$-near to $1\cdot 1$. 
  Thus $n$-nearness is not in general transitive. 
\end{remark}
\begin{definition}
  Let $\alpha, \beta: 2^\N$, we define $a\sim_I\beta$ as 
  $\forall_{n:\N} (\alpha|_n \sim_n \beta|_n)$. 
\end{definition}
\begin{lemma}\label{IntervalFiberSizeAtMost2}
  Whenever $\alpha,\beta,\gamma:2^\N$, are such that 
  $\alpha\sim_I \beta, \beta\sim_I \gamma$, 
  at least two of $\alpha,\beta,\gamma$ are equal. 
\end{lemma}
\begin{proof}
  We will show that $\beta = \gamma \vee \alpha = \gamma \vee \alpha = \beta$. 
  By \Cref{StoneEqualityClosed} and \Cref{ClosedFiniteDisjunction}, this is a closed proposition. 
  By \Cref{rmkOpenClosedNegation}, we can instead show the double negation. 
  To this end, assume that none of $\alpha,\beta,\gamma$ are equal. 
  By \Cref{MarkovPrinciple}, there exist indices $i,j,k\in \N$ with 
  \begin{equation}
    \beta(i) \neq \gamma(i), \alpha(j) \neq \gamma(j), \alpha(k) \neq \beta(k)
  \end{equation}
  Let $n:=\max(i,j,k) + 2$. 
  As $\alpha\sim_I \beta$, we have $\alpha|_n\sim_n\beta|_n$. 
  By assumption $\alpha|_n \neq \beta|_n$, so WLOG we may assume that 
  we have some $m: \N, u:2^m$ with 
  \begin{equation}
    \alpha|_n = (u\cdot 0 \cdot \overline 1) |_n , \beta|_n = (u \cdot 1 \cdot \overline 0)|_n.
  \end{equation}
  As $\alpha(k) \neq \beta(k) $ and $n\geq k+2$, 
  it follows in particular that $m\leq n-2$ and hence 
  $\beta(n-1) = 0$.% and $\beta(m) = 1$. 

  As also $\beta\sim_I \gamma$, we have $\beta|_n \sim_n \gamma|_n$.
  So there exists some $m':\N, u':2^m$ with 
 \begin{equation} %\label{EqnNearness}
   \big(
     (\beta|_{n} = (u'\cdot 0\cdot \overline 1)|_n) \vee (\beta|_{n} = (u' \cdot 1 \cdot \overline 0) |_n)
   \big)
    \wedge 
   \big(
     (\gamma|_{n} = (u'\cdot 0\cdot \overline 1)|_n) \vee (\gamma|_{n} = (u' \cdot 1 \cdot \overline 0) |_n)
   \big).
  \end{equation} 
  Similarly as above, we have that $m'\leq n-2$, and as $\beta(n-1) = 0$, it follows that 
  $\beta|_{n} = (u' \cdot 1 \cdot \overline 0) |_n$. 
  Now as $\beta(i)\neq \gamma(i)$ with $i<n$, we have that $\beta|_n \neq \gamma|_n$, hence 
  $\gamma|_{n} = (u'\cdot 0\cdot \overline 1)|_n$. 
  Now we have $m,m'\leq n-2$ and $u:2^m, u':2^{m'}$ such that 
  \begin{equation}
    (u\cdot 1 \cdot \overline 0)|_n = \beta|_n = (u'\cdot 1 \cdot \overline 0)|_n
  \end{equation}
  Note that $\beta(m') = 1$. 
  But also $\beta(l) = 0$ for all $l$ with $m<l<n$
  Therefore $m'\leq m$. 
  By similar reasoning, $m\leq m'$. We conclude $m=m'$. 
  As a consequence, $u = u'$, but then 
  $\gamma|_n = \alpha|_n$, contradicting that $\alpha(j)\neq \gamma(j) $ for $j<n$. 
  Hence we arrive at a contradiction, as required. 
\end{proof}


\begin{corollary}
  $\sim_I$ is a closed equivalence relation on $2^\N$. 
\end{corollary}
\begin{proof}
  By \Cref{nearnessProperties}, $\sim_I$ is a countable conjunction of decidable propositions. 
  Also by \Cref{nearnessProperties}, $\sim_n$ is reflexive and symmetric for all $n:\N$, thus
  $\sim_I$ is reflexive and symmetric as well. 
  Finally $\sim_I$ is transitive as a consequence of \Cref{IntervalFiberSizeAtMost2}.
\end{proof}
\begin{definition}
  We define $\I:\Chaus$ as $\I= 2^\N/\sim_I$. 
\end{definition}
%------------------
%\subsection{The topology of the interval}
We have defined the interval as a certain quotient of Cantor space, 
in \Cref{IntervalvsCauchyInterval}, a proof is provided for the following theorem:
\begin{theorem}
  Let $I'$ denote the interval of Cauchy real numbers.
  Then the map $2^\N\to I'$ given by 
  \begin{equation}
    \alpha \mapsto \bigsum\limits_{i\in \N} \frac{\alpha(i)} {2^{i+1}}
  \end{equation}
  is well-defined and induces an equivalence $I\simeq I'$. 
\end{theorem}



%\begin{lemma}
  $b$ sends $\sim_n$ equivalent binary sequences to $\sim_C$ equivalent Cauchy sequences. 
\end{lemma}
\begin{proof}
  Let $\alpha, \beta$ be binary sequences.
  We claim that $|b(\alpha)_n - b(\beta)_n| \leq (\frac12)^{n+1}$ 
  whenever $\text{near}_n(\alpha, \beta)$. 
  It will follow that if $\alpha\sim_n \beta$, then 
  $b(\alpha)\sim_C b(\beta)$. 

  Let $n:\N$ and assume $m:\N$ with $m\leq n$ and 
  let $z$ be a sequence of length $m$ such that 
  $\alpha|_n = z\cdot 1 \cdot \overline 0|_n$ and $\beta|_n = z \cdot 0 \cdot \overline q |_n$. 
  then $b(\alpha)_n = \sum_{i\leq m} \frac{z(i)}{2^{i+1}} + (\frac12)^{m+2}$ and 
  $b(\beta)_n = \sum_{i\leq m} \frac{z(i)}{2^{i+1}} + \sum\limits_{m+2 \leq i \leq n}(\frac12)^{i+1}$. 
  Thus 
  $b(\alpha)_n - b(\beta)_n = (\frac12)^{m+2} - \sum\limits_{m+2 \leq i \leq n}(\frac12)^{i+1} = 
  (\frac12)^{n+1}$, 
  which is smaller than required. 
\end{proof}  

\begin{lemma}
  Whenever $b(\alpha) \sim_C b(\beta)$, 
  we have $\alpha \sim_n \beta$. 
\end{lemma}
\begin{proof}
  Assume $b(\alpha) \sim_Cb (\beta)$. 
  Let $n:\N$. 
  We shall show that $\text{near}_n(\alpha , \beta)$. 

  As we're only checking finitely many entries, 
  we either have $\alpha|_n = \beta|_n$, 
  or there exists a smallest $m\leq n$ with 
  $\alpha(m) \neq \beta(m)$. 

  If $\alpha|_n = \beta|_n$, we have $\text{near}_n(\alpha,\beta)$ and are done. 
  WLOG assume $\alpha(m) = 1, \beta(m) = 0$ for $m$ minimal. 

  Now note that 
  \begin{equation} 
    b(\alpha)_{k+1} - b(\beta)_{k+1} = 
    b(\alpha)_{k} - b(\beta)_{k} + 
    \frac{\alpha(k+1) - \beta(k+1)}{2^{k+2}}.
  \end{equation}

  For $k>m$, we have that 
  \begin{equation}
  |b(\alpha)_k - b(\beta)_k |= 
  |(\frac12)^{m+1} + \sum\limits_{i=m+1}^k \frac{ \alpha(i) -\beta(i)}{2^{i+1}}|. 
  \end{equation}
  Note that the right summand is always $\leq (\frac12)^{m+1}$. 
  Therefore, we can leave out the absolute value function. 

  We claim that for every $k\geq m+1$, we have $\alpha(k) = 0, \beta(k) = 1$. 
  We will use induction. 
  Suppose that for every $m <i<j$, we have $\alpha(i) = 0$, and $\beta(i) = 1$. 
  Then 
  \begin{equation}
    b(\alpha)_{j-1} - b(\beta)_{j-1} = 
    (\frac12)^{m+1} + 
    \sum\limits_{i=m+1}^{j-1} \frac{ -1}{2^{i+1}} 
    = (\frac12)^{j}
  \end{equation}
   
  \begin{itemize}
    \item 
      we claim that $\alpha(j) = 0$ 
      Suppose $\alpha(j) = 1$. 
      Then $\alpha(j) -\beta(j) \geq 0$. 
      And for $j + 2$, we have that 
  \begin{align}
    &b(\alpha)_{j+2} - b(\beta)_{j+2}
    \\
    =  
    &(b(\alpha)_{j-1} - b(\beta)_{j-1}) + 
    &\frac{\alpha(j)-\beta(j)}{2^{j+1}} +  
    &\frac{\alpha(j+1) - \beta_(j+1)}{2^{j+2}}
    +
    &\frac{\alpha(j+2) - \beta_(j+2)}{2^{j+3}}
    \\
    \geq  
      & (\frac12)^j + &0 
    + &\frac{-1}{2^{j+2}} 
    + &\frac{-1}{2^{j+3}} 
    \\
      > &(\frac12)^{j+1}
  \end{align}
  which contradicts $b(\alpha) \sim_Cb(\beta)$, 
  which would require that $|b(\alpha)_{j+2} - b(\beta_{j+2} | \leq (\frac{12})^{j+2}+ (\frac12)^{j+2} = (\frac12)^{j+1}$. 
  Therefore $\alpha(j) \neq 1$, and thus $\alpha(j) = 0$. 
    \item 
      We also claim that $\beta(i) = 1$. 
      If $\beta(i) = 0$, we also have 
      $\alpha(j) -\beta(j) \geq 0$, and the rest of the proof is similar as above. 
  \end{itemize}
\end{proof}


%\begin{lemma}
  The map $b: 2^\N \to [0,1]$ is surjective. 
\end{lemma}
\begin{proof}
  First, suppose we have a function 
  $d:\Pi_{x:\mathbb R} \Pi_{q: \mathbb Q} (x \leq q + x \geq q)$
  Then we could recursively define 
  $$\alpha(n) = \begin{cases}
    0 \text{ if } d(x - \sum\limits_{i<n} \frac{\alpha(i)}{2^{i+1}} , \frac{1}{2^{n+1}}) = inl(\cdot) \\
    1 \text{ otherwise}
  \end{cases}
  $$
%  Recall that inequality between rational numbers is decidable, therefore we can define
%  $$\alpha(n) = \begin{cases}
%    0 \text{ if } |x_n - \sum\limits_{i<n} \frac{\alpha(i)}{2^{i+1}}| \leq  \frac{1}{2^{n+1}} \\
%    1 \text{ otherwise}
%  \end{cases}
%  $$
  Note that 
  $$\alpha(n) = \begin{cases}
    0 \text{ if } d(x - b(\alpha)_{n-1} , \frac{1}{2^{n+1}}) = inl(\cdot) \\
    1 \text{ otherwise}
  \end{cases}
  $$
  We'll show by induction that $b(\alpha)_n \leq x$ for every $n:\N$. 
  First $b(\alpha)_0 = 0 \leq x$. 
  Assuming, $b(\alpha)_k \leq x$, for $b(\alpha)_{k+1}$, 
  there are two cases:
  \begin{itemize}
    \item 
     if $d(x -  b(\alpha)_k, \frac{1}{2^{n+1}}) = inl(\cdot)$, 
     then $b(\alpha)_{k+1} = b(\alpha)_k$, which is $\leq x$ by induction hypothesis. 
   \item 
     Otherwise, $ x - b(\alpha)_k \geq (\frac12)^{k+1}$
     So $x-b(\alpha)_k - (\frac12)^{k+1} \geq 0$, 
     and $b(\alpha)_{k+1} = b(\alpha)_k + (\frac12)^{k+1}$. 
     So $x-b(\alpha)_{k+1} \geq 0$, and $b(\alpha)_{k+1} \leq x$ as required. 
 \end{itemize}
 So by induction $b(\alpha)_n\leq x$ for every $n:\N$. 
 Therefore, $|x-b(\alpha)_n| = x-b(\alpha)_n$. 
  
  We shall also show by induction that 
  $ x- b(\alpha)_n \leq (\frac12)^{n+1} $
  for every natural number $n:\N$. 
%
  For $n = 0$, this follows from the assumption that $x\leq 1$. 
%
  Suppose that $ x- b(\alpha)_k  \leq (\frac12)^{k+1} $. 
  We make a case distinction on the form of $d(x-b(\alpha)_k, (\frac12)^{k+2})$.
  \begin{itemize}
    \item 
      If $d(x-b(\alpha)_k , (\frac12)^{k+2}) = inl(\cdot)$, 
      then $  x-b(\alpha)_k  \leq (\frac12)^{k+2}$, 
      and $b(\alpha)_{k+1} = b(\alpha)_k$, 
      and $x-b(\alpha)_{k+1}  \leq (\frac12)^{k+2}$ as well, 
      as required. 
    \item 
      Otherwise, we must have
      $ x- b(\alpha)_k  \geq (\frac12)^{k+2}$, 
      and $b(\alpha)_{k+1} = b(\alpha)_k + (\frac12)^{k+1}$.
      By induction hypothesis, we have 
      $x-b(\alpha)_k \leq (\frac12)^{k+1}$. 
      Thus \begin{equation}
        x-b(\alpha)_{k+1} = x - b(\alpha)_k - (\frac12)^{k+1}
        \leq (\frac12)^{k+1} - (\frac12)^{k+2} = (\frac12)^{k+2}
      \end{equation}
      as required. 
  \end{itemize}
  
  By induction, we conclude that 
  $ | b(\alpha)_n - x |  \leq (\frac12)^{n+1} $
  for every $n:\N$. 
  Therefore $b(\alpha)$ converges to $x$. 

  We may conclude that $\Pi_{x:[0,1]} \Pi_{q: \mathbb Q} (x \leq q + x \geq q)$ implies that 
  we can give for each $x: [0,1]$ a binary sequence $\alpha$ with $b(\alpha) = x$. 
  As we have the propositional trunctation of the premise by \Cref{ComparisonLemma}, 
  we may conclude that for each $x:[0,1]$ there merely exists $\alpha$ with $b(\alpha) = x$. 
  Therefore $b$ is surjective. 
\end{proof}



%
%
%
%\begin{theorem}
%  The interval of Cauchy reals is isomorphic to $2^\N / \sim_t$. 
%\end{theorem} 
%\begin{proof}
%  This follows from the fact that $b:2^\N$ is such that $\alpha\sim_n \beta$ iff $b(\alpha)\sim_t b(\beta)$. 
%  and for every Cauchy real, there is a binary sequence being sent to it, so the composition of $b$ and the 
%  quotient from Caucy sequences to Cauchy real is a surjection. 
%\end{proof}
%
%\begin{corollary}
%  The interval is compact Hausdorff. 
%\end{corollary}

\subsection{Order on the interval}
\begin{definition}
  For $n:\N$ we define 
  $cs_n:2^n \to \mathbb Q$ by 
  \begin{equation}
    cs_n(a) = \sum\limits_{i=0}^{n-1} \frac{a(i)} {2^{i+1}}
  \end{equation}
  And for $\alpha:2^\N$, we define the sequence $cs(\alpha) : \N \to \mathbb Q$ by 
  \begin{equation}
    cs(\alpha)_n = cs_n(\alpha|_n)
  \end{equation}
\end{definition}
\begin{remark}\label{rmkPropertiesCSn}
  $cs_n$ gives a bijection between $2^n$ and it's image 
  $\{\frac{k}{2^n}|0\leq k \leq 2^{n}-1\}\subseteq \mathbb Q$.
%  of rational numbers of the form  
%  $\frac{k}{2^n}$ for $0\leq k \leq 2^n-1$. 
  This observation has some corollaries: 
  \begin{itemize}
    \item In particular, each $cs_n$ is injective. 
    \item Furthermore, whenever $a\neq b:2^n$, we must have that 
      \begin{equation} 
        |cs_n(a)-cs_n(b)|\geq \frac{1}{2^n}.
      \end{equation}
    \item It is known that $\bigcup_{n:\N} \{\frac{k}{2^n}|0\leq k \leq 2^{n}-1\}$ 
      lies dense in the interval of Cauchy reals $[0,1]$. 
      It follows that $cs$ induces a surjection from Cantor space to $[0,1]$. %the interval of Cauchy reals. 
      We claim without proof it in fact induces an equivalence between $\I$ and $[0,1]$.
%      between $I$ and the interval of Cauchy reals. 
  \end{itemize}
  Finally, let us repeat a well-known identity for all $m<n$ on such sums, which we'll make some use of 
  \begin{equation}
   \sum\limits_{i = m}^{n-1} \frac{1}{2^{i+1}} = \frac{1}{2^{m}} - \frac{1}{2^n}
  \end{equation}
\end{remark}
\begin{lemma}\label{CauchyApproxLemma}
  Let $n:\N$ and  $s,t:2^n$. Assume there is some $ m \leq n$ with $cs_m(s|_m) = cs_m(t|_m) + \frac{1}{2^m}$, and 
  at the same time $cs_n(s) -cs_n(t)\leq \frac{1}{2^n}$. 
  Then there is some $k< m$ and some $u:2^k$ such that 
  \begin{equation}
    (s = u \cdot 1 \cdot \overline 0|_n)
    \wedge 
    (t = u \cdot 0 \cdot \overline 1|_n)
  \end{equation}
\end{lemma}
\begin{proof}
%  By injectivity of $cs_m$, 
  By assumption, we have that $s|_m \neq t|_m$. 
  Then there must be some smallest number $k<m$ such that 
  $s(k) \neq t(k)$. As $k$ is minimal, we have $s|_k = t|_k = : u$. 
%  WLOG, we assume that $s(m) = 1, t(m) = 0$. 
  It follows for all $l\leq n$ that 
%  We thus have for all $k<l\leq n$ that 
%  \begin{align}
%    cs_l(s|_l) &= 
%    cs_k(u|_k) + \sum\limits_{i = k}^{l-1} \frac{s(i)}{2^{i+1}}\\
%    cs_l(t|_l) &= 
%    cs_k(u|_k) + \sum\limits_{i = k}^{l-1} \frac{t(i)}{2^{i+1}}
%  \end{align}
%  And thus 
  \begin{align}
    cs_l(s|_l)-cs_l(t|_l) 
    = \sum\limits_{i = k}^{l-1} \frac{s(i)-t(i)}{2^{i+1}}
%    =\frac{s(k)-t(k)}{2^{k+1}} + \sum\limits_{i = k+1}^{l-1} \frac{s(i)-t(i)}{2^{i+1}}
  \end{align}
  Note that as $s(i),t(i) \in \{0,1\}$, we must have %that $s(i) -t(i) \in \{-1,0,1\}$. 
  $|s(i)-t(i)| \leq 1$. 
  Hence for any $k'<l$, we have 
  \begin{equation}
    \left|\sum\limits_{i = k'}^{l-1} \frac{s(i)-t(i)}{2^{i+1}}\right|
    \leq 
    \sum\limits_{i = k'}^{l-1} \frac{1}{2^{i+1}}
    = 
    \frac{1}{2^{k'}} - \frac{1}{2^{l}}
  \end{equation}
  Note that using the two equations above for $l=m$ and $k'=k+1$ we have:
  \begin{align}
    cs_m(s|_m) -cs_m(t|_m) 
    =&
    \frac{s(k)-t(k)}{2^{k+1}} + \sum\limits_{i = k+1}^{m-1} \frac{s(i)-t(i)}{2^{i+1}} \\
    \leq& 
    \frac{s(k)-t(k)}{2^{k+1}} + \left(\frac{1}{2^{k+1}} - \frac{1}{2^{m}}\right)
  \end{align}
  As the left hand side should equal $\frac{1}{2^m}$,
  we must have that $s(k)-t(k) \neq -1$. 
  As $s(k) \neq t(k)$ it follows that $s(k) = 1, t(k) = 0$.
  But now 
  \begin{equation}
    cs_n(s) -cs_n(t) 
    =
    \frac{1}{2^{k+1}} + \sum\limits_{i = k+1}^{n-1} \frac{s(i)-t(i)}{2^{i+1}}
    \geq 
    \frac{1}{2^{k+1}} - \left(\frac{1}{2^{k+1}} - \frac{1}{2^n} \right)
    =
    \frac{1}{2^{n}}
  \end{equation}
  And as $cs_n(s)-cs_n(t) \leq \frac{1}{2^n}$ as well, we get that 
  $cs_n(s)-cs_n(t) = \frac{1}{2^n}$. 
  Note that this lower bound is only reached if $s(i)-t(i) = -1$ for all $k<i<n$. 
  Hence for those $i$, we have $s(i) = 0, t(i) = 1$. 
  Thus 
  \begin{equation}
    s = (u \cdot 1\cdot \overline 0) |_n \wedge 
    t = (u \cdot 0\cdot \overline 1) |_n.
  \end{equation}
\end{proof}

 
\begin{corollary}\label{alternativeSimByCauchyDistance}
  Let $n:\N$ and let $s,t:2^n$. Then 
  \begin{equation}
    s\sim_n t \leftrightarrow |cs_n(s) - cs_n(t)| \leq \frac{1}{2^{n}}.
  \end{equation} 
\end{corollary}

\begin{proof}
  \item  
    Assume $ s \sim_n t$. If $s=t$, we have $cs_n(s) - cs_n(t) = 0$, 
    otherwise, we may without loss of generality assume there is some $m<n$ and some $u:2^m$ such that 
  \begin{equation}
    (s = u \cdot 0 \cdot \overline 1|_n) \wedge ( t = u \cdot 1 \cdot \overline 0 |_n) . 
  \end{equation}
  Then 
  \begin{align}
    cs_n(s) &= 
    cs_m(u) + 0 + \sum\limits_{i = m+1}^{n-1} \frac{1}{2^{i+1}}\\
    cs_n(t) &= 
    cs_m(u) + \frac{1}{2^{m+1}} + 0  
  \end{align}
  And hence 
  \begin{equation}
    cs_n(t) - cs_n(s) = \frac{1}{2^{m+1}} - \sum\limits_{i = m+1}^{n-1} \frac{1}{2^{i+1}} = \frac{1}{2^n}
  \end{equation}
  Thus in all cases, from $s\sim_n t$, we can conclude that 
  \begin{equation}
    |cs_n(s) -cs_n(t) |\leq \frac{1}{2^n}
  \end{equation}
  \item 
  Conversely, assume that $|cs_n(s) - cs_n(t)| \leq \frac{1}{2^n}$. 
  If $s = t$, it is clear that $s \sim_n t$.
  If $s\neq t$, there must be some smallest number $m<n$ such that 
  $s(m) \neq t(m)$. As $m$ is minimal, we have $s|_m = t|_m = : u$. 
  WLOG, we assume that $s(m) = 1, t(m) = 0$. 
  Then $cs_m(s|_{m+1})  = cs_{m+1}(t|_{m+1}) + \frac{1}{2^{m+1}}$
  and by \Cref{CauchyApproxLemma} it follows that 
  \begin{equation}
    s = (u \cdot 1\cdot \overline 0) |_n \wedge 
    t = (u \cdot 0\cdot \overline 1) |_n.
  \end{equation}
  and thus we can conclude $s\sim_n t$ as required. 
\end{proof}


Inspired by Definitions 2.7 and 2.10 \Cite{Bishop}, 
we define inequality on $\I$ as follows:
\begin{definition}
  Let $\alpha,\beta:2^\N$. 
  We define $\alpha\leq_\I \beta$ and $\alpha<_\I\beta$ as follows:
  \begin{align}
  \alpha\leq_\I\beta : = \forall_{n:\N} \left( cs(\alpha)_n \leq cs(\beta)_n + \frac {1} {2^n}\right)\\ 
    \alpha   <_\I \beta : = \exists_{n:\N} \left( cs(\alpha)_n < cs(\beta)_n - \frac {1} {2^n}\right)
%    \\\rednote{Can become n\pm1, \leq ,<, +\frac1{2^n+2} }
\end{align}
\end{definition}
\begin{lemma}
  $\leq_\I$ respects $\sim_\I$. 
\end{lemma}
\begin{proof}
  We will show that whenever $\alpha\leq_\I \gamma$ and $\alpha\sim_\I\beta$, we have $\beta\leq_\I\gamma$. 
  The other proof obligation goes similarly. 
%  The proof is similar to $\alpha'\leq_\I\gamma'$ and $\gamma'\sim_\I\beta'$, we have $\alpha'\leq_\I\beta'$.


  As $\beta\leq_\I\gamma$ is closed, by \Cref{rmkOpenClosedNegation} it is double negation stable. 
  By \Cref{MarkovPrinciple}, the negation is that there is some 
  $N:\N$ with 
  $cs(\beta)_N > cs(\gamma)_N + \frac{1}{2^N}.$
  As $\alpha\leq_\I\gamma$, we have 
  $cs(\gamma)_N + \frac{1}{2^N}\geq cs(\alpha)_N $. 
  Thus $cs(\beta)_N > cs(\alpha)_N$ and therefore $cs(\beta)_N = cs(\alpha)_N+\frac{1}{2^N}$ using  $\alpha\sim_\I\beta$.
%  Yet as $\alpha\sim_\I\beta$, from \Cref{alternativeSimByCauchyDistance}
%  we have $cs(\beta)_n \leq cs(\alpha)_n + \frac{1}{2^n}$ for all $n:\N$. 
%  Therefore, by \Cref{CauchyApproxLemma}, for $n\geq N$, we may conclude that 
  It follows that 
  $$
  cs(\alpha)_N+\frac{1}{2^N} > cs(\gamma)_N + \frac{1}{2^N} \geq cs(\alpha)_N
  $$
  From \Cref{rmkPropertiesCSn}, we must have
  $cs(\gamma)_N  + \frac{1}{2^N} = cs(\alpha)_N$, otherwise the distance 
  between $cs(\gamma)_N$ and $cs(\alpha)_N$ 
  would be smaller than $\frac{1}{2^N}$.
%  Using again $\alpha\sim_\I\beta$ and \Cref{CauchyApproxLemma}, 
%  for $n\geq N$ we get 
%  $cs(\beta)_n = cs(\alpha)_n + \frac{1}{2^n}$.
  As $cs(\alpha)_n \leq cs(\gamma)_n + \frac{1}{2^n}$ for all $n\geq N$, 
  \Cref{CauchyApproxLemma} gives that 
  $\alpha\sim_\I\gamma$. But also $\beta\sim_\I\gamma$. 
  But now $\alpha,\beta,\gamma$ are all distinct yet related by $\sim_\I$, contradicting 
  \Cref{IntervalFiberSizeAtMost2}. 
\end{proof}

\begin{remark}\label{NegationOfGeq}
  By \Cref{MarkovPrinciple}, we have that $\neg (\alpha \leq \beta) \leftrightarrow (\beta <_\I \alpha)$. 
  It follows immediately that $<_\I$ also respects $\I$. 
  Therefore, $\leq_\I, <_\I$ induce relations $\leq,<$ on $\I$.
  As the order in $\mathbb Q$ is decidable, $\leq, <$ are closed and open respectively. 
\end{remark} 

\begin{lemma}\label{IntervalOrderLeqOrGeq}
  For any $x,y:\I$, we have $x\leq y \vee y \leq x$. 
\end{lemma}
\begin{proof}
  Note that $x\leq y \vee y \leq x$ is the disjunction of two closed propositions, hence by 
  \Cref{ClosedFiniteDisjunction} and \Cref{rmkOpenClosedNegation} we can show it's double negation instead. 
  By the above remark, the negation implies that $x>y$ and $y<x$. We will show this is a contradiction. 
  Let $\alpha,\beta:2^\N$ correspond to $x,y$ and assume $n,m:\N$ with 
  $cs(\alpha)_n < cs(\beta)_n-\frac{1}{2^n}$ and 
  $cs(\beta)_m < cs(\alpha)_m-\frac{1}{2^m}$. 
  WLOG assume $n<m$. In this case for $\gamma$ any of $\alpha,\beta$, we have
  $$0\leq cs(\gamma)_m - cs(\gamma)_n = \sum_{i = n}^{m-1} \frac{\gamma(i)}{2^{i+1}}\leq \frac{1}{2^n}-\frac{1}{2^m}$$
  While at the same time, we have 
  \begin{align}
    cs(\beta)_m - cs(\beta)_n &\leq cs(\alpha)_m -\frac{1}{2^m} - cs (\beta)_n \\
                              & = (cs(\alpha)_m-cs(\alpha)_n)  +      (cs(\alpha)_n -cs(\beta)_n) - \frac{1}{2^m}\\
                              & \leq (\frac{1}{2^n} - \frac{1}{2^m}) -\frac{1}{2^n}               - \frac{1}{2^m}\\
                              &<0
  \end{align}
  giving a contradiction as required. 
\end{proof}

\begin{remark}\label{rmkMapOutOfLeqGeq}
  From \Cref{alternativeSimByCauchyDistance} we have $((x\leq y) \wedge (y \leq x )) \leftrightarrow (x = y)$. 
  So in order to define a map $(x \leq y) \vee (y \leq x) \to P$, we need to define a map 
  $f:x\leq y \to P$ and a map $g:y \leq x \to P$ such that $f|_{x = y} = g|_{x=y}$. 
\end{remark}
\rednote{These properties are nice but not necessary and paused WIP:}
\rednote{It is no used for Bouwer's fixed point theorem}
\begin{corollary}\label{inequality-lesser-greater-than}
    For $x,y:\I$ we have $(x\leq y \wedge x \neq y) \leftrightarrow (x < y)$. 
    Also $(x\neq y) \leftrightarrow (x < y + x > y)$. 
\end{corollary} 
\begin{proof}
    By $(x<y)\leftrightarrow \neg (y\leq x)$
    It's also immediate from the definitions that $x<y$ implies $x\neq y$. 
    As $((x\leq y) \wedge (y \leq x )) \leftrightarrow (x = y)$, 
    if $x\leq y \wedge x \neq y$, we have $\neg (y \leq x)$, hence $x<y$. 
\end{proof}

%    \item $(\exists_{y:I}(x\leq y \wedge y \leq z ))\leftrightarrow (x \leq z)$. 
%    \item $(\exists_{y:I}(x<y \wedge y < z ))\leftrightarrow (x < z)$. 

\begin{lemma}
  Whenever $x,y:\I$ satisfy $x<y$, there is some $z:\I$ with  $x<z \wedge z< y$. 
\end{lemma} 

%
%\rednote{TODO}
%For any $x,y:I$ we have 
%\begin{itemize}
%  \item TODO $x\leq y \wedge x \neq y \to x < y$. 
%\end{itemize}



\subsection{The topology of the interval}


\begin{definition}
  Let $a,b:\I$. 
  Following standard notation, we denote
  \begin{equation}
    [a,b]:= \Sigma_{x:\I} (a\leq x \wedge x \leq b),
  \end{equation}
  we call subsets of $\I$ of this form closed intervals. 
%
  We also denote 
  \begin{align}
    (-\infty,a) &:= \Sigma_{x:\I} (x < a)   \\
    (a,\infty) &:= \Sigma_{x:\I} (a < x)  \\
    (-\infty,\infty) &:= \I  \\
    (a,b) &:= \Sigma_{x:\I} (a < x \wedge x < b),
  \end{align}
  We call subsets of $\I$ of these forms open intervals. 
\end{definition}
\begin{remark}
  Note that closed intervals and open intervals are closed and open respectively. 
\end{remark}


%\begin{lemma}\label{IntervalQuotientMapIntersectionCommute}
%  Let $D_n:2^\N \to 2$ be a sequence of decidable subsets with $D_{n+1}\subseteq D_n$.
%  For $p$ the quotient map $2^\N \to I$, we have that 
%  $p(\bigcap_{n:\N} D_n) = p(\bigcap_{n:\N} D_n)$
%\end{lemma}
%\begin{proof}
%  It is always the case that $$p(\bigcap_{n:\N} D_n) \subseteq \bigcap_{n:\N} p(D_n).$$
%  For the converse direction, let $(\bigcap_{n:\N} p(D_n))(x)$. 
%  We will show that $ \neg \neg (p(\bigcap D_n)) (x)$, which is sufficient by \Cref{rmkOpenClosedNegation}. 
%%
%  As $(\bigcap_{n:\N} p(D_n))(x)$, there exists some $y\in D_0$ with $p(y) = x$. 
%%
%  If $x\notin p(\bigcap_{n:\N} D_n)$, we cannot have for all $n:\N$ that $y_0 \in  D_n$. 
%  By Markov, there must exist some $k:\N$ with $\neg D_k(y_0)$. 
%  As $D_{n+1}\subseteq D_n$ for all $n:\N$, it follows that $y_0\notin D_n$ for all $n\geq k$. 
%%
%  As $x\in \bigcap_{n:\N}p(D_n)$, there is however some $y_k\in D_k$ with $p(y_k) = x$. 
%  By a similar argument, we have some $l>k$ with $y_k\notin D_l$, and some $y_l$ with $p(y_l) = x, y_l \in D_l$. 
%  But now we have that $y_0, y_k, y_l:2^\N$ are all distinct, but $p(y_0) = p(y_k) = p(y_l) = x$. 
%  This contradicts \Cref{IntervalFiberSizeAtMost2}, and we're done. 
%\end{proof}


\begin{lemma}\label{ImageDecidableClosedInterval}
  For $p:2^\N \to \I$ the quotient map and $D\subseteq 2^\N$ decidable, we have $p(D)$ a finite union of closed intervals. 
\end{lemma}
\begin{proof}
  We will show the above if there exists some $n:\N, u:2^n$ such that $D(\alpha) \leftrightarrow \alpha|_n = u$.
  This is sufficient, as any decidable subset of $2^\N$ can be written as finite union of such decidable subsets. 
  We claim that $p(D) = [p(u\cdot \overline 0) , p(u \cdot \overline 1)]$. 
\item 
  We will first show that $p(D) \subseteq [p(u\cdot \overline 0) , p(u \cdot \overline 1)]$. 
  Suppose $D(\alpha)$. Then 
  Then $\alpha|_n = u$ and hence 
%  for $m\leq n$ we have 
%  \begin{equation}
%    cs(\alpha)_m = cs_m(u|_m) = cs(u\cdot \overline 0)_m= cs(u\cdot \overline 1)_m
%  \end{equation}
%  For $m>n$, we have that 
%  \begin{align}
%    cs(u\cdot \overline 1)_m =
%    cs_n(u) +\sum_{i = n} ^{m-1} \frac{1}{2^{i+1}}
%    \\
%    cs(\alpha)_m =
%    cs_n(u) +\sum_{i = n} ^{m-1} \frac{\alpha(i)}{2^{i+1}}
%    \\
%    cs(u\cdot \overline 0)_m = 
%    cs_n(u) +\sum_{i = n} ^{m-1} \frac{0}{2^{i+1}}
%  \end{align} 
%  Hence for all $m:\N$, we have 
  \begin{equation}
    cs(u\cdot \overline 1)_m \geq 
    cs(\alpha)_m \geq 
    cs(u\cdot\overline 0)_m
  \end{equation}
 which implies that $p(u\cdot \overline 1) \geq_\I p(\alpha) \geq_\I p(u\cdot\overline 0)$, as required. 
\item 
  To show that $[p(u\cdot \overline 0) , p(u \cdot \overline 1)]\subseteq p(D)$, 
  Suppose
  $(u\cdot \overline 0) \leq_\I \alpha \leq_\I (u \cdot \overline 1)$. 
  It is sufficient to show that 
  $$(\alpha|_n = u )\vee (\alpha \sim_\I u \cdot \overline 0 )\vee (\alpha \sim_\I u \cdot \overline 1).$$
  As this is a disjunction of closed propositions, by \Cref{ClosedFiniteDisjunction} it's closed, and by 
  \Cref{rmkOpenClosedNegation}, we can instead show the double negation. 
  So suppose that none of the disjoints hold. 
  As $\alpha|_n \neq u$, there is some minimal $m$ with $\alpha(m) \neq u(m)$. 
  We assume that $\alpha(m) = 1, u(m) = 0$, the other case goes similarly. 
  Then for all $k:\N$, we have 
  $cs(\alpha)_k \geq cs(u \cdot \overline 1)|_k$. 
  As also 
  $(u\cdot \overline 1)\geq_\I \alpha$, we have 
  $$cs(u \cdot \overline 1)|_k + \frac{1}{2^k} \geq cs(\alpha)_k \geq cs(u\cdot \overline 1)_k,$$
  From which it follows that $|cs(u\cdot\overline 1)_k - cs(\alpha)_k|\leq \frac{1}{2^k}$. 
  Hence $(u\cdot \overline 1)|_k \sim_k \alpha|_k$ by \Cref{alternativeSimByCauchyDistance}. 
  Hence $x\sim_\I (a\cdot\overline 1)$, contradicting our assumption as required. 
\end{proof}

\begin{lemma}\label{complementClosedIntervalOpenIntervals}
  The complement of a finite union of closed intervals is 
  a finite union of open intervals. 
\end{lemma}
\begin{proof}
  We'll use induction on the amount of closed intervals. 
  The empty union of closed intervals is empty, and hence it's complement is $\I$, which is an open interval.  
  Let $(C_i)_{i<k}$ be a finite set of closed intervals with $\neg (\bigcup_{i<k}C_i)$ 
  a finite union of open intervals $\bigcup_{j<l} O_i$. 
  Suppose $C_{k}$ is closed. We need to show that 
  $\neg (\bigcup_{i\leq k} C_i)$ is also a finite union of open intervals. 
  First note that in general, 
  $(\neg (A \vee B ))\leftrightarrow (\neg A \wedge \neg B)$
  hence 
  $$
  \neg (\bigcup_{i\leq k} C_i)
  = 
  \neg ((\bigcup_{i<k} C_i) \cup C_k) 
  =
  (\neg (\bigcup_{i<k} C_i) )\cap (\neg C_k) 
  $$
  And by the induction hypothesis and distributivity, this equals 
  $$
  (\bigcup_{j<l} O_i) ) \cap (\neg C_k) 
  =
  \bigcup_{j<l} (O_i \cap (\neg C_k) )
  $$
  So we need to show that the intersection of an open interval and the negation of a closed interval is a 
  finite union of open intervals. We assume or open intervals are of the form $(a,b)$ for $a,b:\I$. 
  The other cases are very similar. 
  So let $a,b,c,d:\I$ and consider 
  $U = (a,b) \cap (\neg [c,d])$. 
  Then 
  \begin{align} 
    U(x) &= \Sigma_{x:\I}  
  (a < x \wedge x < b) \wedge ( x < c \vee d < x)\\
  &= \Sigma_{x:\I}
  (a < x \wedge x < b \wedge x < c ) \vee ( d < x \vee a<x \wedge x < b)\\
  &= 
  \Sigma_{x:\I}
  (a < x \wedge x < b \wedge x < c ) 
  \cup 
  \Sigma_{x:\I}
  ( d < x \vee a<x \wedge x < b)
  \end{align} 
  We will show that 
  $U' = \Sigma_{x:\I}(a < x \wedge x < b \wedge x < c ) $ is an open interval. 
  By a similar argument, the other part will be as well, meaning that $U$ is the union of two open intervals. 
  Consider that $b\leq c \vee c \leq b$. 
  If $b \leq c$, $(x<b \wedge x< c) \leftrightarrow x<b$ and $U' = (a,b)$
  If $c \leq b$, $(x<b \wedge x< c) \leftrightarrow x<c$ and $U' = (a,c)$
  If $b=c$, these open intervals agree, hence from \Cref{rmkMapOutOfLeqGeq} we can conclude that $U'$ is an open interval. 
  We conclude that $U$ is the union of two open intervals as required. 
\end{proof}
%
\begin{lemma}
  Every open $U\subseteq \I$ can be written as countable union of open intervals.
\end{lemma} 
\begin{proof}
%  Let $U\subseteq I$ open, then $\neg U$ is closed and $U = \neg \neg U$ by \Cref{rmkOpenClosedNegation}. 
%  By \Cref{StoneClosedSubsets}, \Cref{CompactHausdorffClosed} and \Cref{ChausMapsPreserveIntersectionOfClosed}
  %\Cref{IntervalQuotientMapIntersectionCommute}, 
  By \Cref{CompactHausdorffTopology}
  there is some sequence of decidable subsets $D_n\subseteq 2^\N$ 
%  with $\neg U = \bigcap_{n:\N} p(D_n)$. 
%  Thus $U = \neg \bigcap_{n:\N} p(D_n)$. 
%  By \Cref{ClosedMarkov}, it follows that 
  with $U = \bigcup_{n:\N} \neg p(D_n)$. 
  By \Cref{ImageDecidableClosedInterval}, each $p(D_n)$ is a finite union of closed intervals, 
  and by \Cref{complementClosedIntervalOpenIntervals} it follows that each $\neg p(D_n)$ is a finite union of open intervals. 
  We conclude that $U$ is a countable union of open intervals as required. 
\end{proof}
%
%%  $\neg U$ is a countable intersection of finite unions of closed intervals. 
%%  Thus $\neg\neg U$ is a countable union of finite intersections of complements of closed intervals. 
%%  As complements of closed intervals are finite unions of open intervals (TODO), 
%%  and finite intersections of such things are still finite unions of open intervals, 
%%  it follows that $\neg\neg U$ is a countable union of open intervals. 
%%  By \Cref{rmkOpenClosedNegation}, $\neg \neg U = U$ and we're done. 
%%  \rednote{Lotta handwaving here, definitely not finished} 
%\end{proof}
%

\begin{remark}\label{IntervalTopologyStandard}
  It follows that the topology of $\I$ is generated by open intervals, 
  which corresponds to the standard topology on $\I$. 
  Hence our notion of continuity corresponds with the $\epsilon,\delta$-definition of continuity one would expect. 
  Thus every function $f:\I\to \I$ in the system we presented is continuous in the $\epsilon,\delta$-sense. 
\end{remark}

 

\section{Cohomology}
In this section we compute $H^1(S,\Z) = 0$ for all $S$ Stone, and show that $H^1(X,\Z)$ for $X$ compact Hausdorff can be computed using \v{C}ech cohomology. We use this to compute $H^1(\I,\Z)=0$. 

\begin{remark}
We only work with the first cohomology group with coefficients in $\Z$ as it is sufficient for the proof of Brouwer's fixed-point theorem, but the results could be extended to $H^n(X,A)$ for $A$ any family of countably presented abelian groups indexed by $X$.
\end{remark}

\begin{remark}
We write $\mathrm{Ab}$ for the type of abelian groups and if $G:\mathrm{Ab}$ we write $\B G$ for the delooping of $G$ \cite{hott,davidw23}. This means that $H^1(X,G)$ is the set truncation of $X \to \B G$. 
\end{remark}

\subsection{\v{C}ech cohomology}

\begin{definition}
Given a type $S$, types $T_x$ for $x:S$ and $A:S\to\mathrm{Ab}$, we define $\check{C}(S,T,A)$ as the chain complex
\[
\begin{tikzcd}
     \prod_{x:S}A_x^{T_x} \ar[r,"d_0"] & \prod_{x:S}A_x^{T_x^2}\ar[r,"d_1"] &  \prod_{x:S}A_x^{T_x^3}
\end{tikzcd}
\]
where the boundary maps are defined as
\begin{align*}
d_0(\alpha)_x(u,v) =&\ \alpha_x(v)-\alpha_x(u)\\
d_1(\beta)_x(u,v,w) =&\ \beta_x(v,w) - \beta_x(u,w) + \beta_x(u,v)
\end{align*}
\end{definition}

\begin{definition}
Given a type $S$, types $T_x$ for $x:S$ and $A:S\to\mathrm{Ab}$, we define its \v{C}ech cohomology groups by
\[
  \check{H}^0(S,T,A) = \mathrm{ker}(d_0)\quad \quad \quad \check{H}^1(S,T,A) = \mathrm{ker}(d_1)/\mathrm{im}(d_0)
\]
We call elements of $\mathrm{ker}(d_1)$ cocycles and elements of $\mathrm{im}(d_0)$ coboundaries.
\end{definition}

This means that $\check{H}^1(S,T,A) = 0$ if and only if $\check{C}(S,T,A)$ is exact at the middle term. Now we give three general lemmas about \v{C}ech complexes.

\begin{lemma}\label{section-exact-cech-complex}
Assume a type $S$, types $T_x$ for $x:S$ and $A:S\to\mathrm{Ab}$ with $t:\prod_{x:S}T_x$. Then $\check{H}^1(S,T,A)=0$.
\end{lemma}

\begin{proof}
Assume given a cocycle, i.e. $\beta:\prod_{x:S}A_x^{T_x^2}$ such that for all $x:S$ and $u,v,w:T_x$ we have that $\beta_x(u,v)+\beta_x(v,w) = \beta_x(u,w)$. We define $\alpha:\prod_{x:S}A_x^{T_x}$ by $\alpha_x(u) = \beta_x(t_x,u)$. Then for all $x:S$ and $u,v:T_x$ we have that $d_0(\alpha)_x(u,v) =  \beta_x(t_x,v) - \beta_x(t_x,u) = \beta_x(u,v)$ so that $\beta$ is a coboundary.
\end{proof}

\begin{lemma}\label{canonical-exact-cech-complex}
Given a type $S$, types $T_x$ for $x:S$ and $A:S\to\mathrm{Ab}$, we have that $\check{H}^1(S,T,\lambda x.A_x^{T_x})=0$.
\end{lemma}

\begin{proof}
Assume given a cocycle, i.e. $\beta:\prod_{x:S}A_x^{T_x^3}$ such that for all $x:S$ and $u,v,w,t:T_x$ we have that $\beta_x(u,v,t)+\beta_x(v,w,t) = \beta_x(u,w,t)$. We define $\alpha:\prod_{x:S}A_x^{T_x^2}$ by $\alpha_x(u,t) = \beta_x(t,u,t)$. Then for all $x:S$ and $u,v,t:T_x$ we have that $d_0(\alpha)_x(u,v,t) = \beta_x(t,v,t) - \beta_x(t,u,t) = \beta_x(u,v,t)$ so that $\beta$ is a coboundary.
\end{proof}

\begin{lemma}\label{exact-cech-complex-vanishing-cohomology}
Assume a type $S$ and types $T_x$ for $x:S$ such that $\prod_{x:S}\propTrunc{T_x}$ and $A:S\to\mathrm{Ab}$ such that $\check{H}^1(S,T,A) = 0$.
Then given $\alpha:\prod_{x:S}\B A_x$ with $\beta:\prod_{x:S} (\alpha(x) = *)^{T_x}$, we can conclude $\alpha = *$.
\end{lemma}

\begin{proof}
We define $g : \prod_{x:S} A_x^{T_x^2}$ by $g_x(u,v) = \beta_x(v) - \beta_x(u)$.
It is a cocycle in the \v{C}ech complex, so that by exactness there is $f:\prod_{x:S}A_x^{T_x}$ such that for all $x:S$ and $u,v:T_x$ we have that $g_x(u,v)= f_x(v) - f_x(u)$.
Then we define $\beta' : \prod_{x:S}(\alpha(x)=*)^{T_x}$ by $\beta'_x(u) = \beta_x(u) - f_x(u)$
so that for all $x:S$ and $u,v:T_x$ we have that $\beta'_x(u) = \beta'_x(v)$ is equivalent to $f_x(v) - f_x(u) = \beta_x(v) - \beta_x(u)$, which holds by definition. So $\beta'$ is constant on each $T_x$ and therefore gives $\prod_{x:S} (\alpha(x)=*)^{\propTrunc{T_x}}$. By $\prod_{x:S}\propTrunc{T_x}$ we conclude $\alpha = *$.
\end{proof}


\subsection{Cohomology of Stone spaces}

%
%%\subsection{Needed results}
%
%%\rednote{Should probably be moved elsewhere}
%
\begin{lemma}\label{finite-approximation-surjection-stone}
Assume given $S:\Stone$ and $T:S\to\Stone$ such that $\prod_{x:S}\propTrunc{T(x)}$.
Then there exists a sequence of finite types $(S_k)_{k:\N}$ with limit $S$ 
%\rednote{Should the maps in the sequence be mentioned? (Maps $p_k$ are mentioned below)}
%\begin{equation}
%\begin{tikzcd}
%S_0 & S_1 \ar[l,"p_0"]& S_2\ar[l,"p_1"] & \cdots\ar[l]\\
%\end{tikzcd}
%\end{equation}
%such that: 
%\[\mathrm{lim}_kS_k = S\]
and a compatible sequence of families of finite types $T_k$ over $S_k$
with $\prod_{x:S_k}\propTrunc{T_k(x)}$ and 
$\mathrm{lim}_k\left(\sum_{x:S_k}T_k(x)\right) = \sum_{x:S}T(x)$. 
%
%Given $S:\Stone$ and $T:S\to\Stone$ such that $\prod_{x:S}\propTrunc{T(x)}$, there exists a sequence of finite types $(S_k)_{k:\N}$
%\rednote{Should the maps in the sequence be mentioned? (Maps $p_k$ are mentioned below)}
%%\begin{equation}
%%\begin{tikzcd}
%%S_0 & S_1 \ar[l,"p_0"]& S_2\ar[l,"p_1"] & \cdots\ar[l]\\
%%\end{tikzcd}
%%\end{equation}
%such that: 
%\[\mathrm{lim}_kS_k = S\]
%and for each $k:\N$ we have a family of finite types $T_k(x)$ for $x:S_k$ such that $\prod_{x:S_k}\propTrunc{T_k(x)}$ with maps $T_{k+1}(x) \to T_k(p_k(x))$ such that:
%\[\mathrm{lim}_k\left(\sum_{x:S_k}T_k(x)\right) = \sum_{x:S}T(x)\]
\end{lemma}

\begin{proof}
By theorem \Cref{stone-sigma-closed} and the usual correspondence between surjections and families of inhabited types, a family of inhabited Stone spaces over $S$ correspond to a Stone space $T$ with a surjection $T\to S$. Then we conclude using \Cref{ProFiniteMapsFactorization}.
%\rednote{ \@ Hugo This follows from \Cref{ProFiniteMapsFactorization} and \Cref{stone-sigma-closed} 
%  and considering the surjection $(\Sigma_{x:S} T(x)) \to S$, but we discussed whether it might be easier to 
%  refactor the proof where you use the above or make a remark after \Cref{stone-sigma-closed}}
\end{proof}

\begin{lemma}\label{cech-complex-vanishing-stone}
Assume given $S:\Stone$ with $T:S\to\Stone$ such that $\prod_{x:S}\propTrunc{T_x}$. Then we have that $\check{H}^1(S,T,\Z) = 0$.
\end{lemma}


\begin{proof}
We apply \cref{finite-approximation-surjection-stone} to get $S_k$ and $T_k$ finite. Then by \cref{scott-continuity} we have that $\check{C}(S,T,\Z)$ is the sequential colimit of the $\check{C}(S_k,T_k,\Z)$. By \cref{section-exact-cech-complex} we have that each of the $\check{C}(S_k,T_k,\Z)$ is exact, and a sequential colimit of exact sequences is exact.
\end{proof}

\begin{lemma}\label{eilenberg-stone-vanish}
Given $S:\Stone$, we have that $H^1(S,\Z) = 0$. 
\end{lemma}

\begin{proof}
Assume given a map $\alpha:S\to \B\Z$. We use local choice to get $T:S\to\Stone$ such that $\prod_{x:S}\propTrunc{T_x}$ with $\beta:\prod_{x:S}(\alpha(x)=*)^{T_x}$. Then we conclude by \cref{cech-complex-vanishing-stone} and \cref{exact-cech-complex-vanishing-cohomology}.
\end{proof}

\begin{corollary}\label{stone-commute-delooping}
For any $S:\Stone$ the canonical map $\B(\Z^S) \to (\B\Z)^S$ is an equivalence.
\end{corollary}

\begin{proof}
This map is always an embedding. To show it is surjective it is enough to prove that $(\B\Z)^S$ is connected, which is precisely \Cref{eilenberg-stone-vanish}.
\end{proof}


\subsection{\v{C}ech cohomology of compact Hausdorff spaces}

\begin{definition}
A \v{C}ech cover consists of $X:\CHaus$ and $S:X\to\Stone$ such that $\prod_{x:X}\propTrunc{S_x}$ and $\sum_{x:X}S_x:\Stone$.
\end{definition}

By definition any compact Hausdorff space $X$ is part of a \v{C}ech cover $(X,S)$.

\begin{lemma}\label{cech-eilenberg-0-agree}
Given a \v{C}ech cover $(X,S)$ and $A:X\to\mathrm{Ab}$, we have an isomorphism $H^0(X,A) = \check{H}^0(X,S,A)$ natural in $A$.
\end{lemma}

\begin{proof}
By definition an element in $\check{H}^0(X,S,A)$ is a map $f:\prod_{x:X}A_x^{S_x}$
such that for all $u,v:S_x$ we have $f(u)=f(v)$. Since $A_x$ is a set and the $S_x$ are merely inhabited, this is equivalent to $\prod_{x:X}A_x$. Naturality in $A$ is immediate.
\end{proof}

\begin{lemma}\label{eilenberg-exact}
Given a \v{C}ech cover $(X,S)$ we have an exact sequence
\[H^0(X,\lambda x.\Z^{S_x}) \to H^0(X,\lambda x.\Z^{S_x}/\Z) \to H^1(X,\Z)\to 0\]
\end{lemma}

\begin{proof}
We use the long exact cohomology sequence associated to
\[0 \to \Z \to \Z^{S_x} \to \Z^{S_x}/\Z\to 0\]
We just need $H^1(X,\lambda x.\Z^{S_x}) = 0$ to conclude. But by \cref{stone-commute-delooping} we have that $H^1(X,\lambda x.\Z^{S_x}) = H^1\left(\sum_{x:X}S_x,\Z\right)$ which vanishes by \cref{eilenberg-stone-vanish}.
\end{proof}

\begin{lemma}\label{cech-exact}
Given a \v{C}ech cover $(X,S)$ we have an exact sequence
\[\check{H}^0(X,S,\lambda x.\Z^{S_x}) \to \check{H}^0(X,S,\lambda x.\Z^{S_x}/\Z) \to \check{H}^1(X,S,\Z)\to 0\]
\end{lemma}

\begin{proof}
For $n=1,2,3$, we have that $\Sigma_{x:X}S_x^n$ is Stone so that  $H^1(\Sigma_{x:X}S_x^n, \Z) = 0$ by \cref{eilenberg-stone-vanish}, giving short exact sequences
\[0\to \Pi_{x:X}\Z^{S_x^n} \to \Pi_{x:X}(\Z^{S_x})^{S_x^n}\to \Pi_{x:X}(\Z^{S_x}/\Z)^{S_x^n}\to 0\]
They fit together in a short exact sequence of complexes
\[0 \to \check{C}(X,S,\Z) \to \check{C}(X,S,\lambda x.\Z^{S_x}) \to \check{C}(X,S,\lambda x.\Z^{S_x}/\Z)\to 0\]
But since $\check{H}^1(X,\lambda x.\Z^{S_x}) = 0$ by \cref{canonical-exact-cech-complex}, we conclude using the associated long exact sequence.
\end{proof}

\begin{theorem}\label{cech-eilenberg-1-agree}
Given a \v{C}ech cover $(X,S)$, we have that $H^1(X,\Z) = \check{H}^1(X,S,\Z)$
\end{theorem}

\begin{proof}
By applying \cref{cech-eilenberg-0-agree}, \cref{eilenberg-exact} and \cref{cech-exact} we get that $H^1(X,\Z)$ and $\check{H}^1(X,S,\Z)$ are cokernels of isomorphic maps, so they are isomorphic.
\end{proof}

This means that \v{C}ech cohomology does not depend on $S$.

\subsection{Cohomology of the interval}
%
%Recall that we denote $C_n=2^n$ with a binary relation $\sim_n$ on $C_n$ such that for all $x,y:2^\N$ we have that:
%\[\left(\forall(n:\N).\ x|_n\sim_n y|_n\right) \leftrightarrow x=_\I y\]
%
%\begin{lemma}\label{description-Cn-simn}
%We have that $(C_n,\sim_n)$ is equivalent to $(\Fin(2^n),\lambda x,y.\ |x-y|\leq 1)$.
%\end{lemma}
\begin{remark}\label{description-Cn-simn}
  Recall from \Cref{def-cs-Interval} that 
  there is a binary relation $\sim_n$ on $2^n=:\I_n$ such that 
  $(2^n,\sim_n)$ is equivalent to  $(\Fin(2^n),\lambda x,y.\ |x-y|\leq 1)$
  and for $\alpha,\beta:2^\N$ we have $(cs(\alpha) = cs(\beta)) \leftrightarrow 
  \left(\forall_{n:\N}\alpha|_n \sim_n \beta|_n\right)$. 
\end{remark}

We define $\I_n^{\sim2} = \Sigma_{x,y:\I_n}x\sim_n y$ and $\I_n^{\sim3} = \Sigma_{x,y,z:\I_n}x\sim_n y \land y\sim_n z\land x\sim_n z$.

\begin{lemma}\label{Cn-exact-sequence}
For any $n:\N$ we have an exact sequence
\[0\to \Z\overset{d_0}{\longrightarrow} \Z^{\I_n} \overset{d_1}{\longrightarrow} \Z^{\I_n^{\sim2}} \overset{d_2}{\longrightarrow} \Z^{\I_n^{\sim3}}\]
where $d_0(k) = (\_\mapsto k)$ and
\begin{eqnarray}
 d_1(\alpha)(u,v) &=& \alpha(v)-\alpha(u)\nonumber\\
 d_2(\beta)(u,v,w) &=& \beta(v,w)-\beta(u,w)+\beta(u,v).\nonumber
\end{eqnarray}
\end{lemma}

\begin{proof}
It is clear that the map $\Z\to \Z^{\I_n}$ is injective as $\I_n$ is inhabited, so the sequence is exact at $\Z$. Assume a cocycle $\alpha:\Z^{\I_n}$, meaning that for all $u,v:\I_n$, if $u\sim_nv$ then $\alpha(u)=\alpha(v)$. Then by \cref{description-Cn-simn} we see that $\alpha$ is constant, so the sequence is exact at $\Z^{\I_n}$.

Assume a cocycle $\beta:\Z^{\I_n^{\sim2}}$, meaning that for all $u,v,w:\I_n$ such that $u\sim_nv$, $v\sim_nw$ and $u\sim_nw$ we have that $\beta(u,v)+\beta(v,w) = \beta(u,w)$. %This is equivalent to asking $\beta(u,u)=0$ and $\beta(u,v) = -\beta(v,u)$.
Using \cref{description-Cn-simn} to pass along the equivalence between $2^n$ and $\Fin(2^n)$, we define $\alpha(k) = \beta(0,1)+\cdots+\beta(k-1,k)$.
We can check that $\beta(k,l) = \alpha(l)-\alpha(k)$, so that $\beta$ is indeed a coboundary and the sequence is exact at $\Z^{\I_n^{\sim2}}$.
\end{proof}

\begin{proposition}\label{cohomology-I}
We have that $H^0(\I,\Z) = \Z$ and $H^1(\I,\Z) = 0$.
\end{proposition}

\begin{proof}
Consider $cs:2^\N\to\I$ and the associated \v{C}ech cover $T$ of $\I$ defined by: 
\[T_x = \Sigma_{y:2^\N} (x=_\I cs(y))\]
Then for $l=2,3$ we have that $\mathrm{lim}_n\I_n^{\sim l} = \sum_{x:\I} T_x^l$. By \cref{Cn-exact-sequence} and stability of exactness under sequential colimit, we have an exact sequence
\[ 0\to \Z\to \mathrm{colim}_n \Z^{\I_n} \to \mathrm{colim}_n \Z^{\I_n^{\sim2}}\to \mathrm{colim}_n \Z^{\I_n^{\sim3}}\]
By \cref{scott-continuity} this sequence is equivalent to
\[ 0\to \Z\to \Pi_{x:\I}\Z^{T_x} \to  \Pi_{x:\mathbb{I}}\Z^{T_x^2} \to  \Pi_{x:\mathbb{I}}\Z^{T_x^3}\]
So it being exact implies that $\check{H}^0(\I,T,\Z) = \Z$ and $\check{H}^1(\I,T,\Z) = 0$.
We conclude by \cref{cech-eilenberg-0-agree} and \cref{cech-eilenberg-1-agree}.
\end{proof}

\begin{remark}
We could carry a similar computation for $\mathbb{S}^1$, by approximating it with $2^n$ with $0^n\sim_n1^n$ added. We would find $H^1(\mathbb{S}^1,\Z)=\Z$. We will give an alternative, more conceptual proof in the next section.
\end{remark}


\subsection{Brouwer's fixed-point theorem}

Here we consider the modality defined by localising at $\I$ as explained in \cite{modalities}. It is denoted by $L_\I$. We say that $X$ is $\I$-local if $L_\I(X) = X$ and that it is $\I$-contractible if $L_\I(X)=1$.

\begin{lemma}\label{Z-I-local}
$\Z$ and $2$ are $\I$-local.
\end{lemma}

\begin{proof}
By \cref{cohomology-I}, from $H^0(\I,\Z)=\Z$ we get that the map $\Z\to \Z^\I$ is an equivalence, so $\Z$ is $\I$-local. We see that $2$ is $\I$-local as it is a retract of $\Z$.
\end{proof}

\begin{remark}
Since $2$ is $\I$-local, we have that any Stone space is $\I$-local.
\end{remark}

\begin{lemma}\label{BZ-I-local}
$\B\Z$ is $\I$-local.
\end{lemma}

\begin{proof}
Any identity type in $\B\Z$ is a $\Z$-torsor, so it is $\I$-local by \cref{Z-I-local}. So the map $\B\Z\to \B\Z^{\I}$ is an embedding. From $H^1(\I,\Z)=0$ we get that it is surjective, hence an equivalence.
\end{proof}

\begin{lemma}\label{continuously-path-connected-contractible}
Assume $X$ a type with $x:X$ such that for all $y:X$ we have $f:\I\to X$ such that $f(0)=x$ and $f(1)=y$. Then $X$ is $\I$-contractible.
\end{lemma}

\begin{proof}
%First we prove that the map:
%\[\eta_X:X\to L_\I(X)\] 
%is surjective. Indeed its fiber are $\I$-contractible, but for any type $F$ we have a map:
%\[L_\I(F) \to L_\mathbb{F}(\propTrunc{F}) = \propTrunc{F}\] 
For all $y:X$ we get a map $g:\I\to L_\I(X)$ such that $g(0) = [x]$ and $g(1)=[y]$. Since $L_\I(X)$ is $\I$-local this means that $\prod_{y:X}[x]=[y]$. We conclude $\prod_{y:L_\I(X)}[x]=y$ by applying the elimination principle for the modality.
\end{proof}

\begin{corollary}\label{R-I-contractible}
We have that $\R$ and $\mathbb{D}^2=\{(x,y):\mathbb R^2\ \vert\ x^2+y^2\leq 1\}$ are $\I$-contractible.
\end{corollary}

\begin{proposition}\label{shape-S1-is-BZ}
$L_\I(\R/\Z) = \B\Z$.
\end{proposition}

\begin{proof}
As for any group quotient, the fibers of the map $\R\to\R/\Z$ are $\Z$-torsors, so we have an induced pullback square
\[
\begin{tikzcd}
\R\ar[r]\ar[d] & 1\ar[d] \\
\R/\Z\ar[r] & \B\Z
\end{tikzcd}
\]
Now we check that the bottom map is an $\I$-localisation. Since $\B\Z$ is $\I$-local by \cref{BZ-I-local}, it is enough to check that its fibers are $\I$-contractible. Since $\B\Z$ is connected it is enough to check that $\R$ is $\I$-contractible. This is \cref{R-I-contractible}.
\end{proof}

\begin{remark}
By \cref{BZ-I-local}, for any $X$ we have that $H^1(X,\Z) = H^1(L_{\I}(X),\Z)$, so that by \cref{shape-S1-is-BZ} we have that $H^1(\R/\Z,\Z) = H^1(\B\Z,\Z) = \Z$.
\end{remark}

We omit the proof that $\mathbb{S}^1=\{(x,y):\R^2\ \vert\ x^2+y^2=1\}$ is equivalent to $\R/\Z$.
The equivalence can be constructed using trigonometric functions, which exist by Proposition 4.12 in \cite{Bishop}.

\begin{proposition}
\label{no-retraction}
The map $\mathbb{S}^1\to \mathbb{D}^2$ has no retraction.
\end{proposition}

\begin{proof}
By \cref{R-I-contractible} and \cref{shape-S1-is-BZ} we would get a retraction of $\B\Z\to 1$, so $\B\Z$ would be contractible.
\end{proof}

\begin{theorem}[Intermediate value theorem]
  \label{ivt}
  For any $f: \I\to \I$ and $y:\I$ such that $f(0)\leq y$ and $y\leq f(1)$,
  there exists $x:\I$ such that $f(x)=y$.
\end{theorem}

\begin{proof}
  By \Cref{InhabitedClosedSubSpaceClosedCHaus}, the proposition $\exists_{x:\I}\, f(x)=y$ is closed and therefore $\neg\neg$-stable, so we can proceed with a proof by contradiction.
  If there is no such $x:\I$, we have $f(x)\neq y$ for all $x:\I$.
  By \cref{LesserOpenPropAndApartness} we have that $a<b$ or $b<a$ for all distinct numbers $a,b:\I$. So the following two sets cover $\I$
  \[
    U_0:= \{x:\I\mid f(x)<y\} \quad\quad
    U_1:= \{x:\I\mid y<f(x)\}
    \]
  Since $U_0$ and $U_1$ are disjoint, we have $\I=U_0+U_1$ which allows us to define a non-constant function $\I\to 2$, which contradicts \Cref{Z-I-local}.
\end{proof}

\begin{theorem}[Brouwer's fixed-point theorem]
  For all $f:\mathbb{D}^2\to \mathbb{D}^2$ there exists $x:\mathbb{D}^2$ such that $f(x)=x$.
\end{theorem}

\begin{proof}
  As above, by \Cref{InhabitedClosedSubSpaceClosedCHaus}, we can proceed with a proof by contradiction,
  so we assume $f(x)\neq x$ for all $x:\mathbb{D}^2$.
  For any $x:\mathbb{D}^2$ we set $d_x= x-f(x)$, so we have that one of the coordinates of $d_x$ is invertible.
  Let $H_x(t) = f(x) + t\cdot d_x $ be the line through $x$ and $f(x)$.
  The intersections of $H_x$ and $\partial\mathbb{D}^2=\mathbb{S}^1$ are given by the solutions of an equation quadratic in $t$. By invertibility of one of the coordinates of $d_x$, there is exactly one solution with $t> 0$.
  We denote this intersection by $r(x)$ and the resulting map $r:\mathbb D^2\to\mathbb S^1$ has the property that it preserves $\mathbb{S}^1$.
  Then $r$ is a retraction from $\mathbb{D}^2$ onto its boundary $\mathbb{S}^1$, which is a contradiction by \Cref{no-retraction}.
\end{proof}

\begin{remark}
In constructive reverse mathematics \cite{HannesDiener}, it is known that both the intermediate value theorem and Brouwer's fixed-point theorem are equivalent to LLPO. But LLPO does not hold in real cohesive homotopy type theory, so \cite{shulman-Brouwer-fixed-point} prove a variant of the statement involving a double negation.
\end{remark}


%%
%% Bibliography
%%

%% Please use bibtex, 

\bibliography{literature.bib}

\appendix

\end{document}
