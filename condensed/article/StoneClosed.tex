\subsection{The topology on Stone spaces}
\begin{theorem}\label{StoneClosedSubsets}
  Let $A\subseteq S$ be a subset of a Stone space. The following are equivalent:
  \begin{enumerate}[(i)]
    \item There exists a map $\alpha:S \to 2^\N$ such that 
      $A (x) \leftrightarrow \forall_{n:\N} \alpha_{x,n} = 0$ for any $x:S$. 
    \item There exists a family 
      $(D_n)_{n:\N}$ 
      of decidable subsets of $S$ such that $A = \bigcap_{n:\N} D_n$. 
    \item There exists a Stone space $T$ and some embedding $T\to S$ which image is $A$
    \item There exists a Stone space $T$ and some map $T\to S$ which image is $A$. 
    \item $A$ is closed.
  \end{enumerate}
\end{theorem}
\begin{proof}
\item 
  \begin{itemize}
  \item 
    $(i)\leftrightarrow (ii)$. $D_n$ and $\alpha$ can be defined from each other by 
     $D_n(x) \leftrightarrow (\alpha_{x,n} = 0)$. Then observe that
     \begin{equation}
     x\in \bigcap_{n:\N} D_n \leftrightarrow 
      \forall_{n:\mathbb N} (\alpha_{x,n} = 0) 
     \end{equation}
     
   \item $(ii) \to (iii)$. Let $S=Sp(B)$. 
%      By Stone duality, we have $d_n,~n:\N$ terms of $B$ such that $D_n = \{x:S| x(d_n) = 1\}$. 
%      Let $C = B/(\neg d_n)_{n:\N}$.
%      Then the map $Sp(C) \to S$ is as desired because
%      $$Sp(C) = \{x:S| \forall_{n:\N} x(\neg d_n) =0\}  = \bigcap_{n:\N} D_n.$$
      By \Cref{AxStoneDuality}, we have $(d_n)_{n:\N}$ in $B$ such that $D_n = \{x:S\ |\ x(d_n) = 0\}$. 
      Let $C = B/(d_n)_{n:\N}$.
      Then $Sp(C) \to S$ is as desired because:
      $$Sp(C) = \{x:S\ |\ \forall_{n:\N} x(d_n) =0\}  = \bigcap_{n:\N} D_n.$$
%      By \Cref{SurjectionsAreFormalSurjections}, t
%      The quotient map $B \twoheadrightarrow C$
%      corresponds to a map $\iota:Sp(C) \hookrightarrow  S$. 
%      For $s:S$, $s$ lies in the image of this map iff 
%      for all $n:\N$ we have  $s(\neg d_n) = 0$, 
%      \begin{equation}
%        x\in \iota(Sp(C)) \leftrightarrow x(\neg d_n) = 0 \leftrightarrow x(d_n) = 1 \leftrightarrow x\in D_n
%      \end{equation}
%      Thus the image of $\iota$ is given by $\bigcap_{n:\N} D_n$. 
   \item $(iii) \to (iv)$. Immediate.
   \item $(iv) \to (ii)$. Assume $f:T\to S$ corresponds to $g:B\to C$ in $\Boole$. 
     By \Cref{BooleEpiMono}, $f(T) = Sp(B/Ker(g))$, and 
%     by \Cref{OdiscQuotientCountableByOpen}
     there is a surjection $d:\N\to Ker(g)$. Denote by $D_n$ the decidable subset of $S$ corresponding to $d_n$. Then we have that $Sp(B/Ker(g)) = \bigcap_{n:\N} D_n$. 
%     Note that the factorization 
%     $B\twoheadrightarrow B/Ker(g) \hookrightarrow C$ 
%     corresponds to the factorization 
%     $S\twotheadleftarrow f(T) \hookleftarrow T$
%%     We have a surjection $B\twoheadrightarrow B/Ker(g)$ in $\Boole$, 
%%     which corresponds to the inclusion $f(T) = Sp(B/Ker(g)) \subseteq S$. 
%     Ans $f(T) = Sp(B/Ker(g))$
%
%     Therefore $B/Ker(g):\Boole$, 
%     and the quotient map $B\twoheadrightarrow B/Ker(g)$ induces an inclusion 
%     $Sp(B/Ker(g)) \hookrightarrow Sp(B)$ with 
%     $f\in Sp(B/Ker(g))\leftrightarrow \forall_{n:\N}f(d_n) = 0 $. 
%     
%
%
%
%
%
%oldProof%     \rednote{TODO 
%oldProof%       The order of untracating is important in this proof, 
%oldProof%     and I struggle a bit with stressing this in a way this is clear (and concise). 
%oldProof%    Check with fresh eyes later. }
%oldProof%      Let $f:T\to S$ be a map between Stone spaces. 
%oldProof%      Assume $S = Sp(A), T = Sp (B)$. 
%oldProof%%      For this proof, we work with explicit presentations for $A,B$. 
%oldProof%%
%oldProof%      Let $G$ be a countable set of generators of $A$. 
%oldProof%      Assume also we have countable sets of generators and relations for $B$. 
%oldProof%%
%oldProof%      Following \Cref{FiberConstruction}, using $G$, for each $x:S$, we can construct 
%oldProof%      a countable set $I_x\subseteq B$ such that $$Sp(B/I_x) = (\Sigma_{y:T} f(y) = x) .$$
%oldProof%      By \Cref{ExplicitTruncationStoneClosed}, we can construct a sequence 
%oldProof%      $\alpha_x$ such that this type is inhabited iff $\forall_{n:\N} \alpha_x(n) = 0$,
%oldProof%      as required. 
%
%      Recall that the propositional truncation of a Stone is closed, as it is the negation of $0=1$ in the underlying 
%      Boolean algebra, which is open as it f
%
%
%      the core idea of the proof was that the closed proposition corresponds to checking equality in the underlying BA, 
%      which was closed as 
%
%
%
%
%      Note that $x$ in the image of $f$ iff $0\neq_{B/I_x} 1$. 
%      At this point, we have generators and relations of $B/I_x$ as data.
%      Hence using the proof of \Cref{BooleEqualityOpen}, we can construct a sequence 
%      $\alpha_x:2^\N$ such that $0 =_{B/I_x}1\leftrightarrow \exists_{n:\N} \alpha_x(n) = 0$. 
%      And for $\beta_x(n) = 1-\alpha_x(n)$, we conclude that 
%      \begin{equation}
%        x\in f(T) \leftrightarrow \forall_{n:\N} \beta_x(n) = 0
%      \end{equation}
%      Note that we did not use any choice axioms in the proof of this implication,
%      as we untruncated our assumptions before we specified $x$. 
   \item $(i) \to (v)$. By definition.
   \item $(v) \to (iv)$.
     %As $A$ is closed, it corresponds to a map $a:S\to \Closed$. 
     We have a surjection $2^\N\to\Closed$ defined by $\alpha \mapsto \forall_{n:\mathbb N} \alpha_n = 0.$
     \Cref{LocalChoiceSurjectionForm} 
     gives us that there merely exists $T, e, \beta_\cdot$ as follows:
     \begin{equation}
       \begin{tikzcd}
         T \arrow[r,"\beta"] \arrow[d, two heads,swap,"e"] & 2^\mathbb N 
         \arrow[d,two heads] \\
         S \arrow[r,swap,"A"] & \Closed
       \end{tikzcd} 
     \end{equation} 
     Define $B(x) \leftrightarrow \forall_{n:\mathbb N} \beta_{x,n} = 0$. 
     As $(i) \to (iii)$ by the above, $B$ is the image of some Stone space. 
     Note that $A$ is the image of $B$, 
     thus $A$ is the image of some Stone space. 
     \end{itemize} 
     \end{proof} 

\begin{remark}\label{ClosedInStoneIsStone}
%Using condition $(iii)$, 
The previous result implies that closed subtype of Stone spaces are Stone.
\end{remark}

\begin{corollary}\label{InhabitedClosedSubSpaceClosed}
  For $S:\Stone$ and $A\subseteq S$ closed, we have 
  $\exists_{x:S} A(x)$ is closed. 
\end{corollary}
\begin{proof}
  By \Cref{ClosedInStoneIsStone}, $\Sigma_{x:S}A(x)$ is Stone, 
  so its truncation is closed by \Cref{TruncationStoneClosed}.
\end{proof}

\begin{corollary}\label{ClosedDependentSums}
  Closed propositions are closed under dependent sums. 
\end{corollary}
\begin{proof}
  Let $P:\Closed$ and $Q:P \to \Closed$. 
  Then $\Sigma_{p:P} Q(p) \leftrightarrow \exists_{p:P} Q(p)$.
  As $P$ is Stone by \Cref{PropositionsClosedIffStone}, 
  \Cref{InhabitedClosedSubSpaceClosed} gives that $\Sigma_{p:P} Q(p)$ is closed. 
\end{proof}
\begin{remark}\label{ClosedDominance}\label{ClosedTransitive}
  Analogously to \Cref{OpenTransitive} and \Cref{OpenDominance}, it follows that 
  closedness is transitive and $\Closed$ forms a dominance. 
\end{remark}

%We can get a dual to completeness.


%ImpliedByBooleEpiMono%\begin{lemma}\label{DualCompleteness}
%ImpliedByBooleEpiMono%Let $A$ and $B$ be c.p. boolean algebra with a map:
%ImpliedByBooleEpiMono%\[i:Sp(A)\to Sp(B)\] 
%ImpliedByBooleEpiMono%The following are equivalent:
%ImpliedByBooleEpiMono%\begin{enumerate}[(i)]
%ImpliedByBooleEpiMono%\item The induced map $B\to A$ is surjective.
%ImpliedByBooleEpiMono%\item The map $i$ is an embedding.
%ImpliedByBooleEpiMono%\item The map $i$ is a closed embedding.
%ImpliedByBooleEpiMono%\end{enumerate}
%ImpliedByBooleEpiMono%\end{lemma}
%ImpliedByBooleEpiMono%
%ImpliedByBooleEpiMono%\begin{proof}
%ImpliedByBooleEpiMono%  \item
%ImpliedByBooleEpiMono%\begin{itemize}
%ImpliedByBooleEpiMono%\item[$(i)\to (ii)$] Immediate.
%ImpliedByBooleEpiMono%\item[$(ii)\to (iii)$] By \Cref{StoneEqualityClosed} the fibers of $i$ are closed in $Sp(A)$, so by \Cref{ClosedInStoneIsStone} they are Stone, so they are closed by \Cref{PropositionsClosedIffStone}.
%ImpliedByBooleEpiMono%\item[$(iii)\to (i)$] By we have that $Sp(A) = \cap_{n:\N}D_n$ for $D_n$ decidable in $Sp(B)$. Assuming $D_n$ correponds to $b_n:B$ though duality, we then have that $A=B/ (b_n)_{n:\N}$ and the induced map is the quotient map:
%ImpliedByBooleEpiMono%$$B\to B/(b_n)_{n:\N}$$
%ImpliedByBooleEpiMono%which is surjective.
%ImpliedByBooleEpiMono%\end{itemize}
%ImpliedByBooleEpiMono%\end{proof}

\begin{lemma}\label{StoneSeperated}
  Assume $S:\Stone $ with $F,G:S \to \Closed$ be such that $F\cap G = \emptyset$. 
  Then there exists a decidable subset $D:S \to 2$ such $F\subseteq D, G \subseteq \neg D$. 
\end{lemma}
\begin{proof}
%  \rednote{Too shorten this (and some other proofs), I've removed some negations and pretended  $D:S\to 2$ is given by $\{x:S|x(d) = 0\}$ instead of $\{x:S|x(d) = 1\}$ }
  Assume $S = Sp(B)$. 
  By \Cref{StoneClosedSubsets}, for all $n:\N$ there is $f_n,g_n:B$ such that 
  $x\in F$ if and only if $\forall_{n:\N}x(f_n) = 0$ and 
  $y\in G$ if and only if $\forall_{n:\N}y(g_n) = 0$.
%  $x\in F$ iff $x(f_n) = 1$ for all $n:\N$ and 
%  $y\in G$ iff $y(g_m) = 1$ for all $m:\N$. 
%
%  Denote $R\subseteq B$ for $\{\neg f_n|n:\N\} \cup\{\neg g_n|n:\N\}$. 
  Denote by $h$ the sequence define by $h_{2k}=f_k$ and $h_{2k+1}=g_k$.
Then $Sp(B/(h_k)_{k:\N}) = F \cap G=\emptyset$, so by \Cref{SpectrumEmptyIff01Equal}
%  Note that any $x:Sp(B/R)$ is such %gives a map $x:B\to 2$ such that
%  $x(g_n)= x(f_n) = 1$ for all $n:\N$, hence $x\in F \cap G$. 
%  As $F\cap G = \emptyset $, it follows that $Sp(B/R)$ is empty.
%
  there exists finite sets $I,J\subseteq \N $ such that 
%  $$1 =_B \left(\left(\bigvee_{i\in I} \neg f_i\right) \vee \left(\bigvee_{j\in J} \neg g_j\right)\right).$$
%  $$1 =_B \left(\left(\bigvee_{i\in I}  f_i\right) \vee \left(\bigvee_{j\in J}  g_j\right)\right).$$
  $1 =_B ((\bigvee_{i\in I}  f_i) \vee (\bigvee_{j\in J}  g_j)).$
%
  If $y\in F$, then $y(f_i) = 0$ for all $i\in I$, hence
%  $$1 =_2 y(1) = y\left(\bigvee_{j\in J} g_j\right).$$
  $y(\bigvee_{j\in J} g_j) = 1 $
 If $x\in G$, we have 
%  $x\left(\bigvee_{j\in J} g_j\right) = 0$. 
  $x(\bigvee_{j\in J} g_j) = 0$. 
  Thus we can define the required $D$ by 
  $D(x) \leftrightarrow x(\bigvee_{j\in J} g_j) = 1$.
  %$$D = \{x:S | x\left(\bigvee_{j\in J} g_j \right) = 0\}$$.
  %, we have $F\subseteq D, G\subseteq \neg D$. 
%  Let $y\in G$. Then $y(\neg g_j) = 0$ for all $j \in J$. 
%  Let $y\in G$. Then $y(g_j) = 0$ for all $j \in J$. 
%  Hence 
%  $$
%  1 
%%  = y(1) 
%  =_2
%%  y(\bigvee_{i\in I} \neg f_i) = y (\neg (\bigwedge_{i\in I} f_i))
%  y\left(\left(\bigvee_{i\in I}  f_i\right) \vee \left(\bigvee_{j\in J}  g_j\right)\right)
%  = 
%%  \left(y\left(\bigvee_{i\in I}  f_i\right)\right) \vee \left(\bigvee_{j\in J} y(g_j)\right)
%%  = 
%  y\left(\bigvee_{i\in I}  f_i\right)
%  $$
%%  Thus $y(\bigwedge_{i\in I} f_i) = 0$. 
%%  Note that if $x\in F$, we have $x(f_i) = 1$ for all $i\in I$, hence 
%%  $x(\bigwedge_{i\in I} f_i) = 1$. 
%%  Thus for $D$ corresponding to $\bigwedge_{i\in I} f_i$, we have that 
%%  $F\subseteq D, G\subseteq \neg D$ as required. 
\end{proof} 

%RedundantByOpenInCHaus%\begin{corollary}\label{StoneOpenSubsets}
%RedundantByOpenInCHaus%  Let $A\subseteq S$ be a subset of a Stone space, then 
%RedundantByOpenInCHaus%  $A$ is open iff there exists some countable family $D_n,~n:\N$ of decidable subsets of $S$ with 
%RedundantByOpenInCHaus%  $A = \bigcup_{n:\N} D_n$. 
%RedundantByOpenInCHaus%\end{corollary}
%RedundantByOpenInCHaus%\begin{proof}
%RedundantByOpenInCHaus%  By \Cref{rmkOpenClosedNegation}, $A$ is open iff $\neg A$ is closed and $A = \neg \neg A$. 
%RedundantByOpenInCHaus%  By \Cref{StoneClosedSubsets}, $\neg A$ is closed iff 
%RedundantByOpenInCHaus%  $\neg A = \bigcap_{n\in \N} E_n$ for some countably family of decidable subsets $E_n,~n:\N$. 
%RedundantByOpenInCHaus%  Thus $\neg \neg A = \neg (\bigcap_{n\in \N} E_n)$. 
%RedundantByOpenInCHaus%  By MP (\Cref{MarkovPrinciple}), we have that 
%RedundantByOpenInCHaus%  $\neg (\bigcap_{n\in \N} E_n)= \bigcup_{n\in \N} \neg E_n$. 
%RedundantByOpenInCHaus%  Thus $D_n := \neg E_n$ is as required. 
%RedundantByOpenInCHaus%\end{proof}
