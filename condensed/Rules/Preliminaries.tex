\subsection{Preliminaries}

In this section, we introduce the type of countably presented Boolean algebras $\Boole$ and of Stone spaces $\Stone$. 
Both of these types carry a natural category structure. 
In later sections, we will axiomatize an anti-equivalence between these categories, 
which is classically valid and called Stone duality. 

%\subsection{Countably Presented Boolean Algebras}
%We will use the type of countably presented (c.p.) boolean algebras $\Boole$,
%for more definitions and notation see \Cref{A-cp-boolean-algebras}.

%\subsection{Countably presented Boolean algebras}
\begin{definition}
  A countably presented Boolean algebra $B$ is a Boolean algebra such that there merely are 
  countable sets $I,J$, 
  a set of generators $g_i,~{i\in I}$ and a set $f_j,~{j\in J}$ of Boolean expressions over these generators 
  such that $B$ is equivalent to the quotient of the free Boolean algebra over the generators by the relations
  $f_j=0$. We denote this algebra by $2[I]/(f_j)_{j:J}$.
\end{definition} 
\begin{remark}
By countable in the previous definition we mean sets that are merely equal to some decidable in $\N$. 
Note that any countably presented algebra is also merely of the form $2[\N]\rangle / (r_n)_{n:\N}$.
%, if we add dummy variables that we equate to $0$, and dummy relations that equate $0$ to itself.
% 0 is not special here right? 
\end{remark}


We will call the family $(f_j)_{j\in J}$ as above a set of relations. 
If $I,J$ are finite, we call $B$ a finitely presented Boolean algebra. 
Once we have postulated the axiom of dependent choice, 
in \Cref{secBooleAsColimits}
we will be able to show that every countably presented algebra 
is actually a colimit of a sequence of finitely presented Boolean algebras.
They are therefore dual to pro-finite objects, which are used 
in the theory of light condensed sets \cite{Scholze,Dagur,TODO}.

\begin{remark}
  We denote the type of countably presented Boolean algebras $\Boole$. 
  Note that this type does not depend on a choice of universe. 
\end{remark}

\begin{example}
  If both the set of generators and relations are empty, we have the Boolean algebra $2$.
  We have $0\neq_2 1$, and the underlying set of $2$ is given by $\{0,1\}$.
\end{example}
Note that any Boolean algebra must contain the elements $0,1$. 
Therefore, $2$ is the initial Boolean algebra. 
We can therefore use it to define points of objects in the category dual to that of countably presented Boolean algebras. 

%\subsection{Stone spaces}
\begin{definition}
  For $B$ a countably presented Boolean algebra, we define $Sp(B)$ as the set of Boolean morphisms from $B$ to $2$. 
\end{definition}
\begin{definition}
  We define the predicate on types $\isSt$ by 
  \begin{equation}
    \isSt(X) := \sum\limits_{B : Boole} X = Sp(B)
  \end{equation} 
  A type $X$ is called \textit{Stone} if $\isSt(X)$ is inhabited.
\end{definition}


%\subsection{Examples}
\begin{example}
  \label{boolean-algebra-examples}
  \begin{enumerate}[(i)]
  \item There is only one Boolean map $2\to 2$, thus $Sp(2)$ is the singleton type $\top$. 
  \item   The trivial Boolean algebra is given by the empty set of generators and the relation $\{1\}$.
    We have $0=1$ in the trivial Boolean algebra. 
    As there cannot be a map from the trivial Boolean algebra into $2$ preserving both $0,1$, 
    the corresponding Stone space is the empty type $\bot$, 
  \item\label{ExampleBAunderCantor}   
    We denote by $C$ the Boolean algebra $2[\N]$ given by $\N$ as a set of generators and no relations. We write $p_n$ for the generator corresponding to $n$.
    A morphism $C\to 2$ corresponds to a function $\mathbb N\to 2$, which is a binary sequence. 
    The Stone space $Sp(C)$ of these binary sequences is denoted $2^{\N}$ and called \notion{Cantor space}.
  \item\label{ExampleBAunderNinfty}
    We denote by $B_\infty$ the quotient of $C$ by the relations $p_m\wedge p_n$ for $m\neq n$. 
    A morphism $B_\infty\to 2$ corresponds to a function $\mathbb N \to 2$ that hits $1$ at most once. 
    The corresponding Stone space is denoted by $\N_\infty$. 
  \end{enumerate}
\end{example}

\begin{remark}\label{BinftyTermsWriting}
  In \Cref{N-co-fin-cp}, we will show that $B_\infty$ is equivalent to the Boolean algebra on 
  subsets of $\N$ which are finite or co-finite. 
  Under this equivalence, the generator $p_n$ is sent to the singleton $\{n\}$. 
  Because of this, we have that any $b:B_\infty$ can be written 
  either as $\bigvee_{i\in I_0} p_i$ or as $\bigwedge_{i\in I_0} \neg p_i$ for some finite $I_0\subseteq \N$. 
\end{remark}


%\begin{remark}
%  As Boolean algebras are rings, any relation of the form $f=g$ with both $f,g$ Boolean expressions 
%  can be written as $h=0$ with $h=f-g$ a Boolean expression. 
%\end{remark} 



