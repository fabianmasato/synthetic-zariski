\subsection{Relating $\ODisc$ and $\Boole$}
\begin{lemma}\label{BooleIsODisc}
  Every countably presented Boolean algebra is merely a sequential colimit of finite Boolean algebras. 
\end{lemma}
\begin{proof}
  Consider a countably presented Boolean algebra of the form $B = 2[\N]/(r_n)_{n:\N}$. 
%  We will show there exists a diagram of shape $\N$ taking values in Boolean algebras 
%  with $B$ as colimit.
%  \paragraph{The diagram}
  For each $n:\N$, let $G_n$ be the union of $\{g_i\ |\ {i\leq n}\}$ and 
  the finite set of generators occurring in $r_i$ for some $i\leq n$. 
  Denote $B_n = 2[G_n]/(r_i)_{i\leq n}$. 
  Each $B_n$ is a finite Boolean algebra, and there are canonical maps $B_n \to B_{n+1}$.
%  The inclusion $G_n \hookrightarrow G_{n+1}$ induces maps $B_n \to B_{n+1}$.
%  Hence $B_n,~n:\N$ is an $(\N,\leq)$-indexed sequence of finite sets. 
  Then $B$ is the colimit of this sequence. 
%
%  \paragraph{The colimit}
%  As $G_n\subseteq G$ and $R_n \subseteq R$, 
%  \Cref{rmkMorphismsOutOfQuotient} also gives us a map $B_n\to \langle G \rangle \langle R \rangle$. 
%  We claim the resulting cocone is a colimit. 
%
%  Suppose we have a cocone $C$ on the diagram $(B_n)_{n\in\N}$. 
%  We need to show that there exists a map $\langle G \rangle / R\to C$ and
%  we need to show this map is unique as map between cocones. 
%  \begin{itemize}
%    \item To show there exists a map $\langle G \rangle / R \to C$, 
%      we use remark \Cref{rmkMorphismsOutOfQuotient} again. 
%      Let $g\in G$ be the $n$'th element of $G$, 
%      note that $g\in G_n$, and consider the image of $g$ under the map $B_n \to C$. 
%      This procedure defines a function from $G$ to the underlying set of $C$. 
%      Let $\phi \in R$ be the $n$'th element of $R$, 
%      note that $\phi \in R_n$, and the map $B_n \to C$ must send $\phi$ to $0$. 
%      Thus the function from $G$ to the underlying set of $C$ also sends $\phi$ to $0$. 
%      This thus defines a map $\langle G \rangle / R \to C$. 
%    \item To show uniqueness, consider that any map of cocones $\langle G \rangle / \langle R \rangle \to C$ 
%      must take the same values on all $g\in G_n$ for all $n\in\N$. 
%      Now all $g\in G$ occur in some $G_n$, so any map of cocones $\langle G \rangle /  \langle R \rangle \to C$ 
%      takes the same values for all $g\in G$. 
%      \Cref{rmkMorphismsOutOfQuotient} now tell us that these values uniquely determine the map. 
%  \end{itemize}
\end{proof}




\begin{corollary}\label{ODiscBAareBoole}
  A Boolean algebra $B$ is overtly discrete if and only if it is countably presented. 
\end{corollary}
\begin{proof}
  Assume $B:\ODisc$. 
  By \Cref{OdiscQuotientCountableByOpen}, we get a surjection $\N\twoheadrightarrow B$ and that $B$ has open equality. 
  Consider the induced surjective morphism $f:2[\N]\twoheadrightarrow B$.
  By countable choice, we get for each $b:2[\N]$
  a sequence $\alpha_{b}:2^\N$ such that 
  $(f(b) = 0)\leftrightarrow \exists_{k:\N} (\alpha_{b}(k) =1)$. 
  Consider 
  $r:2[\N] \to \N \to 2[\N]$ 
  given by 
  \[r(b,k) =\begin{cases}
    b &\text{ if } \alpha_{b}(k) = 1\\
    0   &\text{ if } \alpha_{b}(k) = 0
  \end{cases}
  \] 
  Then $B= 2[\N]/(r(b,k))_{b:2^\N,k:\N}$.
  %By countable choice, we get for each $a,b:2[\N]$
  %a sequence $\alpha_{a,b}:2^\N$ such that 
  %$(f(a) = f(b))\leftrightarrow \exists_{k:\N} (\alpha_{a,b}(k) =1)$. 
  %Consider 
  %$r:2[\N] \times 2[\N] \times \N \to 2[\N]$ 
  %given by 
  %$$r(a,b,k) =\begin{cases}
  %  a-b &\text{ if } \alpha_{(a,b)}(k) = 1\\
  %  0   &\text{ if } \alpha_{(a,b)}(k) = 0
  %\end{cases}
  %$$
  %Then $B= 2[\N]/(r(a,b,k))_{(a,b,n): F \times F \times \N}$. 
  \Cref{BooleIsODisc} gives the converse.
\end{proof}

\begin{remark}\label{BooleEpiMono}
%  In particular equality in overtly discrete types is open. 
  By \Cref{OdiscSigma} and \Cref{ODiscBAareBoole}, 
  it follows that any 
  $g:B\to C$ in $\Boole$ has an overtly discrete kernel.
  As a consequence, the kernel is enumerable and $B/Ker(g)$ is in $\Boole$. 
  By uniqueness of epi-mono factorizations and \Cref{SurjectionsAreFormalSurjections}, 
  the factorization 
  $B\twoheadrightarrow B/Ker(g) \hookrightarrow C$ corresponds to 
  $Sp(C) \twoheadrightarrow Sp(B/Ker(g)) \hookrightarrow Sp(B)$. 
\end{remark}
\begin{remark}\label{decompositionBooleMaps}
  Similarly to \Cref{lemDecompositionOfColimitMorphisms} and 
  \Cref{lemDecompositionOfEpiMonoFactorization}  a map (resp. surjection, injection) 
  in $\Boole$ is a sequential colimit of maps (resp. surjections, injections) between 
  finite Boolean algebras. 
\end{remark}
