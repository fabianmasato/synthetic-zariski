\documentclass{../util/zariski}
\title{Using the topologist's sine curve to separate path-connectedness and $\mathbb I$-contractibility}

\begin{document}
In this note we show that the topologists sine curve $Y$ can be defined in the setting of 
synthetic Stone duality as introduced in \cite{synthetic-stone-duality}. 
We show, using Markov's principle, that $Y$ is not path-connected. 
We will also show that it is $\mathbb I$-contractible. 
Note that anything that is path-connected is $\mathbb I$-contractible 
(Lemma 5.24 of \cite{synthetic-stone-duality}) but the converse thus does not hold. 


\begin{remark}[on notation]
  We will denote $\mathbb I$ and $[0,1]$ for the same object, the interval as defined in \cite{synthetic-stone-duality}. 
  We will use $\mathbb I$ as the domain of a path, and $[0,1]$ for the induced closed subobject of $\mathbb R$. 
  Furthermore, we will use $\pi_1 : [0,1] \times [-1,1] \to [0,1] $ and $\pi_2: [0,1] \times [-1,1] \to [-1,1]$ 
  for the canoncical projection maps. 
\end{remark}

\begin{lemma}
  In the setting of \cite{synthetic-stone-duality}, 
  for each $k:\mathbb N$, we can define the function 
  $t:[\frac1k, 1] \to [-1,1]$ by $t(r) = \sin(\frac1r)$.
\end{lemma}
\begin{proof}
  TODO
\end{proof}
\begin{lemma}
  In the same setting, we can define for each $n:\mathbb N$ 
  the standard metric $d:\mathbb R^n \times \mathbb R^n \to \mathbb R_{\geq 0}$. 
\end{lemma}
\begin{proof}
  TODO
\end{proof}

\begin{lemma}
  For $A\subseteq X$ a closed subset of a compact Hausdorff space and 
  $f: A \to Y$ a function to a compact Hausdorff space, 
  $graph(f) \subseteq X \times Y$ is closed. 
\end{lemma}
\begin{proof}
  $(x,y)$ lies in $graph(f)$ iff $A(x) \wedge f x = y$, 
  equality in $Y$ is closed and $A$ is closed. 
  As closed propositions are conjunction-closed, $graph(f)$ is closed. 
\end{proof}
\begin{definition}[The topologist's sine curve]
  For each $k:\mathbb N$, define $Y_k \subset [0,1] \times [-1,1]$ by 
  $$Y_k = ([0, \frac1k] \times [-1,1]) \cup graph(t)$$
  Then define $Y = \bigcap_{k:\mathbb N} Y_k$. 
  The inclusion maps are called $\iota_k : Y_k \to [0,1] \times [-1,1]$ 
  and $\iota:Y \to [0,1] \times [-1,1]$. 
\end{definition}
\begin{remark}
  $Y_k$ is a closed for each $k:\mathbb N$, 
  as it's the union of two closed subsets. 
  It follows that $Y$ is closed also. 
\end{remark}

\begin{lemma}\label{postiveNumbersCharacter}
  For $x:\mathbb I$ with $x>0$ there exists some $k:\mathbb N$ with $x>\frac1k$. 
\end{lemma}
\begin{proof}
  This follows from Markov's principle.
  \rednote{I think this is nicer if we redefine to use $\frac1{2^k}$ instead of $\frac1k$, 
  this is actually the statement of Markov's principle.}
\end{proof}

\section{The curve is not path-connected}
\begin{lemma}[uniform continuity of maps $\mathbb I \to \mathbb R^n$]\label{uniformContinuity}
  For any map $f:\mathbb I \to [0,1] \times [-1,1]$, and for every $\epsilon:\mathbb R$ with $\epsilon>0$, 
  there exists a $\delta>0$ such that for any $x,y:\mathbb I$ with $d(x,y)<\delta$, we have
  that $d(f(x) , f(y))<\epsilon$.
\end{lemma}
\begin{proof}
  Because $f$ is continuous, for each $x:\mathbb I$ there exists some $n:\mathbb N$ such that 
  if $d(x,y)<\frac1n$, then $d((f(x), f(y)))<\epsilon$. 
  By local choice, we get a Stone space $S$ with a surjection $a:S\twoheadrightarrow \mathbb I$ 
  and a map $b:S\to \mathbb N$ such that if $d(x,a(s))<\frac1{b(s)}$, then $d(f(x), f(a(s)))<\epsilon$. 
  By Lemma 3.1.6 of \cite{synthetic-stone-duality}, $b$ factors over $Fin(k)$. 
  Hence we can take the maximal element $N$ of it's image, and set $\delta = \frac1N$. 
\end{proof}
\begin{lemma}\label{sinCharacter}
  Given $r:[0,1]$ with $r>0$ and 
  $y:[-1,1]$, there is some $s:[0,1]$ with $0<s<r$ such that 
  $\sin(\frac1s) = y$. 
\end{lemma}
\begin{proof}
  By the intermediate value theorem (Theorem 6.5.7 of \cite{synthetic-stone-duality}), 
  it's sufficient to show this for $y \in \{-1,1\}$. 
  We will do this proof for $y = 1$, as the other case is similar. 
  We know that for any natural number $k:\mathbb N$ we have that 
  $\sin( (2 k + \frac12)\cdot \pi) = 1$. 
  It's thus sufficient to provide some $k:\mathbb N$ such that $\frac1{(2k+\frac12)\cdot \pi}<r$. 
  As $(2k+\frac12)\cdot \pi>k$, it's sufficient to provide some $k:\mathbb N$ with $\frac1k<r$, 
  which is given to us by \Cref{postiveNumbersCharacter}.
\end{proof}
\begin{lemma}
  Let $a:[-1,1]$. 
  Suppose $f:\mathbb I \to Y$ is such that $\pi_1(f(0)) > 0$
  If there is some $r:\mathbb I$ with $\pi_1(f(r)) = 0$, 
  there is also some $r':\mathbb I$ with $\pi_2(f(r')) = a$. 
\end{lemma}
\begin{proof}
  By \Cref{sinCharacter}, there is some $s:[0,1]$ with $0<s<\pi_1(f(0))$ with $\sin(\frac1s) = a$. 
  Suppose now an $r:\mathbb I$ with $\pi_1(f(r)) = 0$. 
  By the intermediate value theorem, there must then exist an $r':\mathbb I$ with $\pi_1(f(r')) = s$. 
  As $0<s$, the only such point $f(r')$ must equal $(s, a)$.
  Hence $\pi_2(f(r')) = a$. 
\end{proof}  
\begin{corollary}\label{topologistsSineCurveWillHitInfinitelyManyTimes}
  Let $a:[-1,1]$ and let $f:\mathbb I \to Y$ satisfy $f(1) = (0,0)$, 
  and let $r:\mathbb I$ be such that $\pi_1(f(r))>0$, 
  then there is some $r':\mathbb I$ with $r<r'<1$ with $f(r') = a$. 
\end{corollary}

\begin{theorem}[The topologist's sine curve is not path-connected]
  There is no map $f:\mathbb I \to Y$ such that $f(0) = (1,sin(1))$ and $f(1) = (0,0)$. 
\end{theorem}
\begin{proof}
  By applying countable dependent choice to \Cref{topologistsSineCurveWillHitInfinitelyManyTimes}, 
  we can find increasing sequences $t_i,b_i,~i:\mathbb N$ such that $\pi_2(f(t_i)) = 1, \pi_2(f(b_i)) = -1$ and 
  $t_{i+1}>b_{i+1}>t_i>b_i$ for all $i:\mathbb N$. 
  By \Cref{uniformContinuity} for $\epsilon = 2$, there is some $\delta:\mathbb I$ with $\delta>0$ such that 
  $d(b_i, t_j)\geq \delta$ for any $i,j:\mathbb N$. 
  Therefore, the sequence $t_i$ is increasing with steps of at least size $2\delta>0$. 
  But this means that for $N<\frac1{2\delta}$, we have $t_{N}>1$, while $t_N :\mathbb I$, which is a contradiction. 
\end{proof}

\begin{remark}
  One could define an equivalence relation on any type $X$ such that 
  $x\sim y$ iff there is a path from $x$ to $y$. 
  The equivalence classes of this relation would normally be called the path-connected components. 
  We conjecture that the space of path-connected components of $Y$ is equivalent to the type of open propositions. 
\end{remark}

\section{The shape of the curve}
\begin{lemma}
  $\pi_1\circ \iota : Y \to [0,1]$ is surjective.  
\end{lemma}
\begin{proof}
  We shall show that for each $r:[0,1]$ the fiber $\pi_1\circ \iota^{-1}(r)$ is merely inhabited. 
  As the fiber is closed, it's sufficient to show it's non-empty. 

  Note that $\pi_1\circ \iota^{-1}(r)= \pi_1^{-1}(r) \cap Y$. 
  By Lemma 4.1.5 of \cite{synthetic-stone-duality}, 
  it's sufficient to show that $\pi_1^{-1}(r) \cap Y_k$ 
  is non-empty for any $k:\mathbb N$. 

  Let $k:\mathbb N$. 
  Note that $r\leq \frac1k \vee r \geq \frac 1k$. 
  \begin{itemize}
    \item If $r\leq \frac1k$, we have that $(r,0) \in Y_k$. 
    \item If $r\geq \frac1k$, $\frac1r$ is well-defined and 
      $(r,\sin(\frac1r)) \in Y_k$. 
  \end{itemize}
  Hence in both cases we have that $\pi_1^{-1}(r) \cap Y_k$ is non-empty, as required. 
\end{proof}

\begin{definition}
  A subset $X$ of $\mathbb R^n$ is called convex if 
  for any $\alpha,\beta:\mathbb I$ with $\alpha + \beta = 1$, we have 
  $X(y) \wedge X(y') \to X(\alpha y + \beta y')$
\end{definition}
\begin{lemma}
  For each $k:\mathbb N$ and $r:[0,1]$, $Y_k\cap \pi_1^{-1}(r)$ is convex. 
\end{lemma}
\begin{proof}
  Suppose that $(x,y), (x',y')$ both lie in $Y_k\cap \pi_1^{-1}(r)$. 
  Suppose furthermore that $\alpha,\beta:\mathbb I, \alpha+ \beta = 1$. 
  We will show that $\alpha\cdot (x,y) + \beta\cdot (x',y') $ also lies in 
  $Y_k \cap \pi_1^{-1}(r)$. 

  Recall that 
  $Y_k = [0,\frac1k] \times [-1,1]  \cup graph(t)$. 
  As we will show a proposition, we can make a case distinction 
  for both $(x,y), (x',y')$ on whether they come from $[0,\frac 1k] \times [-1,1]$ or 
  from $graph(t)$.
  \begin{itemize}
    \item 
      If either comes from $[0,\frac1k] \times [-1,1]$, we must have that 
      then $r \leq \frac1k$. 
      Then $\pi_1^{-1}(r) = \{r\} \times [-1,1]\subseteq Y_k$. 
      It follows that $\pi_1^{-1}(r) \cap Y_k = \{r\} \times [-1, 1]$, which is convex. 
    \item 
      If both come from $graph(t)$, 
      then $x = x' = r$ and $y = y' = \sin(\frac1r)$. 
      Hence $(x,y) = (x',y')$ and thus $\alpha(x,y) + \beta(x',y') = (x',y')$. 
  \end{itemize}
  In both cases we conclude that $\alpha (x,y) + \beta(x',y')$ lies in 
  $Y_k\cap \pi_1^{-1}(r)$. 
\end{proof}


\begin{remark}
  Any intersection of convex subspaces of $\mathbb R^n$ is itself convex. 
\end{remark}
\begin{corollary}
  for each $r:\mathbb I$, the fiber 
  $Y_r:=(\pi_1\circ \iota)^{-1}(r) = Y\cap \pi_1^{-1}(r)$ is convex. 
\end{corollary}
\begin{lemma}
  Any convex inhabited subspace of $\mathbb R^n$ is $\mathbb I$-contractible. 
\end{lemma} 
\begin{proof}
  See Lemma 5.24 of \cite{synthetic-stone-duality}. 
  Convex spaces can shown to be path-connected.
\end{proof}
\begin{theorem}
  $Y$ is $\mathbb I$-contractible. 
\end{theorem}
\begin{proof}
  By the above, we can write $Y = \sum\limits_{r:\mathbb I} Y_r$, 
  such that for all $r:\mathbb I$, we have 
  $Y_r$ $\mathbb I$-contractible. 

  Thus both the maps $\sum\limits_{r:\mathbb I} Y_r \to \mathbb I$ and 
  $\mathbb I \to 1$ have $\mathbb I$-contractible fibers. 
  By Lemma 1.33 of \cite{modalities}, so does their composite. 
  It follows that $Y$ is $\mathbb I$-contractible. 
\end{proof}
\printbibliography

\end{document}
