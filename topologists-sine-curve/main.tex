\documentclass{../util/zariski}
\title{Using the topologist's sine curve to separate path-connectedness and $\mathbb I$-contractibility}

\begin{document}
In this note we show that the topologists sine curve can be defined in the setting of 
synthetic Stone duality as introduced in \cite{synthetic-stone-duality}. 
We recall the proof that this space is not path-connected. 
However, we show that it is $\mathbb I$-contractible. 
Note that anything that is path-connected is $\mathbb I$-contractible 
(lemma 5.24 of \cite{synthetic-stone-duality})
but the converse thus does not hold. 

\begin{lemma}
  In SSD, for each $k:\mathbb N$, we can define the function 
  $t:[\frac1k, 1] \to [-1,1]$ by $t(r) = \sin(\frac1r)$.
\end{lemma}
\begin{proof}
  TODO
\end{proof}
\begin{lemma}
  For $A\subseteq X$ a closed subset of a compact Hausdorff space and 
  $f: A \to Y$ a function to a compact Hausdorff space, 
  $graph(f) \subseteq X \times Y$ is closed. 
\end{lemma}
\begin{proof}
  $(x,y)$ lies in $graph(f)$ iff $A(x) \wedge f x = y$, 
  equality in $Y$ is closed and $A$ is closed. 
  As closed propositions are conjunction-closed, $graph(f)$ is closed. 
\end{proof}
\begin{definition}[The topologist's sine curve]
  For each $k:\mathbb N$, define $Y_k \subset [0,1] \times [-1,1]$ by 
  $$Y_k = ([0, \frac1k] \times [-1,1]) \cup graph(t)$$
  Then define $Y = \bigcap_{k:\mathbb N} Y_k$. 
  The inclusion maps are called $\iota_k : Y_k \to [0,1] \times [-1,1]$ 
  and $\iota:Y \to [0,1] \times [-1,1]$. 
\end{definition}
\begin{remark}
  $Y_k$ is a closed for each $k:\mathbb N$, 
  as it's the union of two closed subsets. 
  It follows that $Y$ is closed also. 
\end{remark}
\begin{lemma}[The topologist's since curve is not path-connected]
  There is no continuous function $f:\mathbb I \to Y$ such that $f(0) = (0,0)$ and 
  $f(1) = (1,sin(1))$. 
\end{lemma}
\begin{proof}
  TODO, the classical argument should work here as we're showing a negation.
  However, since we haven't provided a definition for $\sin(\frac1x)$, 
  it might be actually good to show the relevant properties, namely that 
  for any $\epsilon>0$ the block $[0,\epsilon] \times [-\frac12,\frac12]\cap Y$
  is very much not (path)connected.
\end{proof}

\begin{definition}
  Define $e : [0,1]\times [-1,1] \to [0,1]$ by $e(x,y) = x$. 
\end{definition}  

\begin{lemma}
  $e\circ \iota : Y \to [0,1]$ is surjective.  
\end{lemma}
\begin{proof}
  We shall show that for each $r:[0,1]$ the fiber $e\circ \iota^{-1}(r)$ is merely inhabited. 
  As the fiber is closed, it's sufficient to show it's non-empty. 

  Note that $e\circ \iota^{-1}(r)= e^{-1}(r) \cap Y$. 
  By Lemma 4.5, it's sufficient to show that $e^{-1}(r) \cap Y_k$ is non-empty for any $k:\mathbb N$. 

  Let $k:\mathbb N$. 
  Note that $r\leq \frac1k \vee r \geq \frac 1k$. 
  \begin{itemize}
    \item If $r\leq \frac1k$, we have that $(r,0) \in Y_k$. 
    \item If $r\geq \frac1k$, $\frac1r$ is well-defined and 
      $(r,\sin(\frac1r)) \in Y_k$. 
  \end{itemize}
  Hence in both cases we have that $e^{-1}(r) \cap Y_k$ is non-empty, as required. 
\end{proof}

\begin{definition}
  A subset $X$ of $\mathbb R^n$ is called convex if 
  for any $\alpha,\beta:\mathbb I$ with $\alpha + \beta = 1$, we have 
  $X(y) \wedge X(y') \to X(\alpha y + \beta y')$
\end{definition}
\begin{lemma}
  For each $k:\mathbb N$ and $r:[0,1]$, $Y_k\cap e^{-1}(r)$ is convex. 
\end{lemma}
\begin{proof}
  Suppose that $(x,y), (x',y')$ both lie in $Y_k\cap e^{-1}(r)$. 
%  and 
%  $$x = e(x,y) = e(x',y') = x' = r.$$
  Suppose furthermore that $\alpha,\beta:\mathbb I, \alpha+ \beta = 1$. 
  We will show that $\alpha\cdot (x,y) + \beta\cdot (x',y') $ also lies in 
  $Y_k \cap e^{-1}(r)$. 
%  $Y_k$ and that $e(\alpha\cdot (x,y) + \beta\cdot (x',y') ) = r$. 
%  For the latter, note that 
%  $$e(\alpha\cdot (x,y) + \beta\cdot (x',y') ) = 
%   e( \alpha x + \beta x', \alpha y + \beta y') 
%   = \alpha x + \beta x' = \alpha r + \beta r = r.
%  $$
%  For the former, recall that 

  Recall that 
  $Y_k = [0,\frac1k] \times [-1,1]  \cup graph(t)$. 
  As we will show a proposition, we can make a case distinction 
  for both $(x,y), (x',y')$ on whether they come from $[0,\frac 1k] \times [-1,1]$ or 
  from $graph(t)$.
  \begin{itemize}
    \item 
      If either comes from $[0,\frac1k] \times [-1,1]$, we must have that 
      then $r \leq \frac1k$. 
      Then $e^{-1}(r) = \{r\} \times [-1,1]\subseteq Y_k$. 
      It follows that $e^{-1}(r) \cap Y_k = \{r\} \times [-1, 1]$, which is convex. 
    \item 
      If both come from $graph(t)$, 
      then $x = x' = r$ and $y = y' = \sin(\frac1r)$. 
      Hence $(x,y) = (x',y')$ and thus $\alpha(x,y) + \beta(x',y') = (x',y')$. 
  \end{itemize}
  In both cases we conclude that $\alpha (x,y) + \beta(x',y')$ lies in 
  $Y_k\cap e^{-1}(r)$. 
\end{proof}


\begin{remark}
  Any intersection of convex subspaces of $\mathbb R^n$ is itself convex. 
\end{remark}
\begin{corollary}
  for each $r:\mathbb I$, the fiber 
  $Y_r:=(\iota \circ e)^{-1}(r) = Y\cap e^{-1}(r)$ is convex. 
\end{corollary}
\begin{lemma}
  Any convex inhabited subspace of $\mathbb R^n$ is $\mathbb I$-contractible. 
\end{lemma} 
\begin{proof}
  TODO
\end{proof}
\begin{theorem}
  $Y$ is $\mathbb I$-contractible. 
\end{theorem}
\begin{proof}
  By the above, we can write $Y = \sum\limits_{r:\mathbb I} Y_r$, 
  such that for all $r:\mathbb I$, we have 
  $Y_r$ $\mathbb I$-contractible. 

  Thus both the maps $\sum\limits_{r:\mathbb I} Y_r \to \mathbb I$ and 
  $\mathbb I \to 1$ have $\mathbb I$-contractible fibers. 
  By Lemma 1.33 of \cite{modalities}, so does their composite. 
  It follows that $Y$ is $\mathbb I$-contractible. 
\end{proof}

\end{document}
