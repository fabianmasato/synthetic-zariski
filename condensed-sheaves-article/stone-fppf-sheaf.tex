\rednote{TODO rework as Zariski sheaves}

We write $\Boole_{\neg\neg}$ the type of $B:\Boole$ such that $\neg\neg\Spec(B)$. We work in the Zariski topos, so that by \cref{zariski-sheaf-axioms} we can assume the following:

\begin{enumerate}[(i)]
\item For all $B:\Boole$, the map $B\to {2}^{\Spec(B)}$ is an equivalence.
\item For all $B:\Boole$, the type $\Spec(B)$ has choice.
\item Dependent choice.
\end{enumerate}

\begin{definition}
A type $X$ is an fppf sheaf for all $B:\Boole_{\neg\neg}$, we have that the map:
\[X\to X^{\propTrunc{\Spec(B)}}\]
is an equivalence.
\end{definition}

\begin{remark}
We have that $\Boole_{\neg\neg}$ is precisely the type of fppf boolean algebras. This is a straightforward consequence of the fact that every ${2}$-module is flat. This makes the name of fppf sheaf reasonable.
\end{remark}

\begin{lemma}\label{bot-sheaf}
The type $\bot$ is an fppf sheaf.
\end{lemma}

\begin{proof}
To check that the map:
\[\bot\to \bot^{\propTrunc{\Spec(B)}}\]
is an equivalence we just need $\neg\neg\Spec(B)$, which holds.
\end{proof}

\begin{lemma}\label{tensor-boolean-algebras}
Let $B,C$ be boolean alegbras with $a,c:B$ and $b,d:C$. Then $a\otimes b\leq c\otimes d$ if and only if $a=0$ or $b=0$ or $a\leq c$ and $b\leq d$
\end{lemma}

\begin{proof}
Details omitted. The key idea is that for $B,C$ boolean algebras we can define $B\boxtimes C$ as the set of formal sups $\bigvee_ib_i\otimes c_i$. Then we define inductively the preorder relation $a\otimes b\leq \lor_ic_i\otimes d_i$ by asking that it holds whenever one of the following holds:
\begin{itemize}
\item $a\leq c_i$ and $b\leq d_i$ for some $i$.
\item $a=\bigvee_k a_k$ and for all $k$ we have $a_k\otimes b\leq \bigvee_ic_i\otimes d_i$.
\item $b=\bigvee_k b_k$ and for all $k$ we have $a\otimes b_k\leq \bigvee_ic_i\otimes d_i$.
\end{itemize}
We extend this to a preorder on $A\boxtimes B$, and then show that $A\otimes B$ is the obtained from $A\boxtimes B$ by identitying $x$ and $y$ when $x\leq y$ and $y\leq x$.
\end{proof}

\begin{lemma}\label{fppf-subcanonical}
The type ${2}$ is an fppf sheaf. Any $B:\Boole$ is a an fppf sheaf.
\end{lemma}

\begin{proof}
Assume $C:\Boole_{\neg\neg}$ We have a coequalizer in sets:
\[\Spec(C)\times\Spec(C) \rightrightarrows \Spec(C)\to \propTrunc{\Spec(C)} \]
so we have an equaliser in boolean algebras:
\[ {2}^{\propTrunc{\Spec(C)}}\to C\rightrightarrows C\otimes C \]
But if $b\otimes 1 \leq 1\otimes b$, by \cref{tensor-boolean-algebras} we can conclude that $b=0$ or $b=1$, and since $0\not=_C1$ we have that the equaliser of this diagram is indeed ${2}$.

We extend to $\Boole$ by duality.
\end{proof}

\begin{lemma}\label{stone-boundedness}
Assume $B:\Boole$ with a map $f:\Spec(B)\to\N$, then there merely exists $k:\N$ such that $f$ factors through $\mathrm{Fin}(k)$.
\end{lemma}

\begin{proof}
We define $f_n:\Spec(B)\to {2}$ by $f_n(x) = 0$ if and only if $f(x)\not=n$. Then:
\[\Spec(C/(f_1,f_2\cdots)) \simeq \sum_{x:\Spec(b)}\forall (n:\N). f(x)\not=n \simeq \bot\]
so that by duality $(f_1,f_2,\cdots) = 1$ and there exists $k$ such that $(f_1,\cdots,f_k)=1$ allowing to factor $f$ through $\mathrm{Fin}(k)$.
\end{proof}

\begin{lemma}\label{nat-sheaf}
The type $\N$ is an fppf sheaf.
\end{lemma}

\begin{proof}
By \cref{bot-sheaf} we know that identity types in $\N$ are fppf sheaves, so we just need to show that for all $B:\Boole_{\neg\neg}$, the map:
\[\N\to \N^{\propTrunc{\Spec(B)}}\]
is surjective. Since by \cref{stone-boundedness} any map $\Spec(B)\to \N$ merely factors through some $\mathrm{Fin}(k)$. We we just need to show the map:
\[\mathrm{Fin}(k)\to \mathrm{Fin}(k)^{\propTrunc{\Spec(B)}}\]
is surjective, which comes from \cref{fppf-subcanonical}.
\end{proof}

\begin{lemma}\label{proposition-fppf-sheaf}
Let $P$ be a proposition, then $L_{fppf}(P)$ is equivalent to:
\[\propTrunc{\sum_{B:\Boole_{\neg\neg}} \Spec(B)\to P}\]
\end{lemma}

\begin{proof}
First we check that: 
\[\propTrunc{\sum_{B:\Boole_{\neg\neg}}\Spec(B)\to P}\]
is an fppf sheaf. Given $C:\Boole_{\neg\neg}$ such that:
\[\Spec(C)\to \propTrunc{\sum_{B:\Boole_{\neg\neg}}\Spec(B)\to P}\]
by choice for $\Spec(C)$ we merely get $B:\Spec(C)\to \Boole_{\neg\neg}$ such that:
\[\prod_{x:\Spec(C)}\Spec(B(x))\to P\]
but we have that:
\[\sum_{x:\Spec(C)}\Spec(B(x)):\Boole_{\neg\neg}\]
so we can conclude that:
\[\propTrunc{\sum_{B:\Boole_{\neg\neg}}\Spec(B)\to P}\]

Moreover, it is clear that the fibers of the map:
\[P\to \propTrunc{\sum_{B:\Boole_{\neg\neg}}\Spec(B)\to P}\] 
are fppf-contractible as:
\[\propTrunc{\sum_{B:\Boole_{\neg\neg}}\Spec(B)\to P}\to L_{fppf}(P)\]
\end{proof}

\begin{definition}
A map $f:X\to Y$ is called fppf-surjective if for all $y:Y$ we have: 
\[L_{fppf}(\propTrunc{\mathrm{fib}_f(y)})\]
\end{definition}

\begin{lemma}\label{fppf-local-choice}
Assume given $B:\Boole$ and with $i:\Spec(B)\to Y$ and an fppf surjective map $f:X\to Y$. Then there exists $C:\Boole$ with fillers of the following diagram:
\begin{center}
\begin{tikzcd}
\Spec(C)\ar[d]\ar[r] & X\ar[d,"f"]\\
\Spec(B)\ar[r,swap,"i"] & Y
\end{tikzcd}
\end{center}
where the left map has non-empty fibers.
\end{lemma}

\begin{proof}
We have:
\[\prod_{x:\Spec(B)} L_{fppf}(\propTrunc{\mathrm{fib}_f(i(x))})\]
so by \cref{proposition-fppf-sheaf} we have that:
\[\prod_{x:\Spec(B)} \propTrunc{\sum_{D:\Boole_{\neg\neg}} \Spec(D)\to \propTrunc{\mathrm{fib}_f(i(x))}})\]
so by choice for $\Spec(B)$ and $\Spec(D)$ we merely have $D:\Spec(B)\to \Boole_{\neg\neg}$ such that:
\[\prod_{x:\Spec(B)} \Spec(D(x))\to \mathrm{fib}_f(i(x))\]
which gives what we want by defining:
\[\Spec(C) = \sum_{x:\Spec(B)}\Spec(D(x))\]
\end{proof}

\begin{lemma}\label{colimit-fppf}
Assume given a tower:
\[B_0\to B_1\to \cdots\]
in $\Boole_{\neg\neg}$. Then $\mathrm{colim}_i B_i$ is in $\Boole_{\neg\neg}$.
\end{lemma}

\begin{proof}
By duality we have $\neg\neg\Spec(B)$ if only if $0\not=_B1$, and if $0=_{\mathrm{colim}_iB_i}1$ then there is $i$ such that $0=_{B_i}1$
\end{proof}

\begin{lemma}\label{dependent-choice-fppf}
Assume given a tower types:
\[\cdots \to X_2\to X_1 \to X_0 \]
where the maps are fppf surjective. The map:
\[\mathrm{lim}_i X_i\to X_0\]
is fppf surjective.
\end{lemma}

\begin{proof}
By taking $x:X_0$ and consdering the fiber of $\mathrm{lim}_i X_i\to X_0$ over $x$, we can assume $X_0=1$, in which case we just need to prove $L_{fppf}(\propTrunc{\mathrm{lim}_iX_i})$.

By \cref{proposition-fppf-sheaf} since $L_{fppf}(\propTrunc{X_1})$ we get $B_1:\Boole_{\neg\neg}$ with a map:
\[\Spec(B_1)\to X_1\]
Now assume given $B_n:\Boole_{\neg\neg}$ and a map $\Spec(B_n)\to X_n$ then by \cref{fppf-local-choice} we get $B_{n+1}:\Boole$ with:
\begin{center}
\begin{tikzcd}
\Spec(B_{n+1})\ar[d]\ar[r] & X_n\ar[d]\\
\Spec(B_n)\ar[r] & X_{n+1}
\end{tikzcd}
\end{center}
such that the left map has non-empty fibers, so that $B_{n+1}:\Boole_{\neg\neg}$.

Using dependent choice we conclude there is a map:
\[\mathrm{lim}_i\Spec(B_i) \to \mathrm{lim}_iX_i\]
But $\Boole_{\neg\neg}$ is stable under sequential colimit by \cref{colimit-fppf} so we can conclude that $L_{fppf}(\propTrunc{\mathrm{lim}_i\Spec(B_i)})$ so that $L_{fppf}(\propTrunc{\mathrm{lim}_iX_i})$ as desired.
\end{proof}

\begin{theorem}
The interpretation of the following statement holds in the fppf topos:
\begin{enumerate}[(i)]
\item For all $B:\Boole$, if $\neg\neg\Spec(B)$ then $\propTrunc{\Spec(B)}$.
\item For all $B:\Boole$, the map $B\to {2}^{\Spec(B)}$ is an equivalence.
\item For all $B:\Boole$, given $\Spec(B)\to Y$ and a surjective map $X\to Y$, there merely exists $C:\Boole$ with a commutative diagram:
\begin{center}
\begin{tikzcd}
\Spec(C)\ar[d]\ar[r] & X\ar[d]\\
\Spec(B)\ar[r] & Y
\end{tikzcd}
\end{center}
such that the left map is surjective.
\item Dependent choice.
\end{enumerate}
\end{theorem}

\begin{proof}
By \cref{fppf-subcanonical} and \cref{nat-sheaf} we see that the interpretation of $B$ being in $\Boole$ in the fppf topos is just the fppf sheafification of $B$ being in $\Boole$. So we can assume $B:\Boole$.
\begin{enumerate}[(i)]
\item By \cref{bot-sheaf} we can assume $\neg\neg\Spec(B)$ to prove $L_{fppf}(\propTrunc{\Spec(B)})$ which is immediate.
\item By \cref{fppf-subcanonical} we have that $B\to {2}^{\Spec(B)}$ being equivalence is not changed by the interpretation.
\item By \cref{fppf-local-choice}, using that the interpretation of a map being surjective is the map being fppf-surjective.
\item By \cref{dependent-choice-fppf}.
\end{enumerate}
\end{proof}

