Assume given a lex modality $L$ in homotopy type theory. Then we can interpret any statement in homotopy type theory in the universe of modal types. The precise details can be found in \rednote{TODO Kevin Quirin PhD Lawvere-Tierney sheafification in Homotopy Type Theory}. The idea is to give a translation from sequent to sequent. We just give a summary of the translation of any type $A$ as a modal type $[A]_L$.

\begin{eqnarray}
{[\mathrm{Type}]_L} &\equiv& \sum_{X:\mathrm{Type}} X\ \mathrm{is\ modal}\\
{[\prod_{x:A}B(x)]_L} &\equiv& \prod_{x:[A]_L}[B]_L(x)\\
{[\sum_{x:A}B(x)]_L} &\equiv& \sum_{x:[A]_L}[B]_L(x)\\
{[x=_Ay]_L} &\equiv& [x]_L=_{[A]_L}[y]_L\\
{[\propTrunc{A}]_L} &\equiv& L\propTrunc{[A]_L}
\end{eqnarray}

This means we mostly leave things untouched expect we assume type variables to be modal, and use the modal replacement on the propositional truncation. Then for any inductive type $\mathbb{I}$ (without type parameter) we have that:

\begin{eqnarray}
{[\mathbb{I}]_L} &\simeq& L(\mathbb{I})\\
\end{eqnarray}

Note that we do not use equality and just an equivalence, for computational reasons.

\rednote{TODO check with Thierry}.

