We write $\Boole$ for the type of c.p. algebras, and $\Boole_{\neg\neg}$ the type of $B:\Boole$ such that $\neg\neg\Spec(B)$. We work in the Zariski topos, so we assume the following:

\begin{itemize}
\item For all $B:\Boole$, the map $B\to {2}^{\Spec(B)}$ is an equivalence.
\item For all $B:\Boole$, $\Spec(B)$ has choice.
\item Dependent choice.
\end{itemize}

\begin{definition}
A type $X$ is an fppf stack for all $B:\Boole_{\neg\neg}$, we have that the map:
\[X\to X^{\propTrunc{\Spec(B)}}\]
is an equivalence.
\end{definition}

\begin{remark}
Why is it called fppf \rednote{TODO}
\end{remark}

\begin{lemma}\label{bot-stack}
The type $\bot$ is an fppf stack.
\end{lemma}

\begin{proof}
To check that the map:
\[\bot\to \bot^{\propTrunc{\Spec(B)}}\]
is an equivalence we just need $\neg\neg\Spec(B)$, which holds.
\end{proof}

\begin{lemma}\label{fppf-subcanonical}
The type ${2}$ is an fppf stack. Any $B:\Boole$ is a an fppf stack.
\end{lemma}

\begin{proof}
We have a coequalizer in sets:
\[\Spec(B)\times\Spec(B) \rightrightarrows \Spec(B)\to \propTrunc{\Spec(B)} \]
so we have an equaliser in boolean algebras:
\[ {2}^{\propTrunc{\Spec(B)}}\to B\rightrightarrows B\otimes B \]
But since $a\otimes b\leq c\otimes d$ if and only if $a=0$ or $b=0$ or $a\leq c\land b\leq d$ \rednote{(TODO why?)}, the equaliser of this diagram in ${2}$ so we conclude.

We extend to any c.p. boolean algebra by duality.
\end{proof}

\begin{lemma}\label{proposition-fppf-stack}
Let $P$ be a proposition, then $L_{fppf}(P)$ is equivalent to:
\[\propTrunc{\sum_{B:\Boole_{\neg\neg}} \Spec(B)\to P}\]
\end{lemma}

\begin{proof}
First we check that: 
\[\propTrunc{\sum_{B:\Boole_{\neg\neg}}\Spec(B)\to P}\]
is an fppf stack. Given $B:\Boole_{\neg\neg}$ such that:
\[\Spec(B)\to \propTrunc{\sum_{C:\Boole_{\neg\neg}}\Spec(B)\to P}\]
by choice for $\Spec(B)$ we merely get $C:\Spec(B)\to \Boole_{\neg\neg}$ such that:
\[\prod_{x:\Spec(B)}\Spec(C(x))\to P\]
but we have that:
\[\sum_{x:\Spec(B)}\Spec(C(x)):\Boole_{\neg\neg}\]
so we can conclude that:
\[\propTrunc{\sum_{B:\Boole_{\neg\neg}}\Spec(B)\to P}\]

Moreover, it is clear that the fibers of the map:
\[P\to \propTrunc{\sum_{B:\Boole_{\neg\neg}}\Spec(B)\to P}\] 
are fppf-contractible.
\end{proof}

\begin{theorem}
The interpretation of the following statement holds in the fppf topos:
\begin{itemize}
\item For all $B:\Boole$, if $\neg\neg\Spec(B)$ then $\propTrunc{\Spec(B)}$.
\item For all $B:\Boole$, the map $B\to {2}^{\Spec(B)}$ is an equivalence.
\item For all $B:\Boole$, given $P:\Spec(B)\to \mathrm{Type}$ such that $\prod_{x:\Spec(B)}\propTrunc{P(x)}$, then there merely exists a:
\[C:\Spec(B)\to \Boole_{\neg\neg}\] 
such that:
\[\prod_{x:\Spec(B)} \Spec(C(x))\to P(x)\]
\end{itemize}
\end{theorem}

\begin{proof}
By \cref{fppf-subcanonical} we see that the interpretation of $B$ being in $\Boole$ in the fppf topos is just the fppf stackification of $B$ being in $\Boole$. So we can assume $B:\Boole$.

For the first point, by \cref{bot-stack} we can assume $\neg\neg\Spec(B)$ to prove $L_{fppf}(\propTrunc{\Spec(B)})$ which is immediate.

For the second point, $B\to {2}^{\Spec(B)}$ being equivalence is not changed by the interpretation.

For the third point, we assume $P:\Spec(B)\to \mathrm{Type}_{fppf}$ such that:
\[\prod_{x:\Spec(B)}L_{fppf}(\propTrunc{P(x)})\] 
Then by \cref{proposition-fppf-stack} we have that:
\[\prod_{x:\Spec(B)}\propTrunc{\sum_{C:\Boole_{\neg\neg}} \Spec(C)\to \propTrunc{P(x)}}\]
We conclude by choice for $\Spec(B)$ and $\Spec(C)$.
\end{proof}

\rednote{TODO dependent choice}
