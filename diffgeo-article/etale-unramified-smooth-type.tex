\subsection{Definition}

\begin{definition}
A closed proposition is dense if it is merely of the form:
\[r_1=0\land\cdots\land r_n=0\]
for $r_1,\cdots,r_n:R$ nilpotent.
\end{definition}

\begin{definition}
A type $X$ is formally étale (resp. formally unramified, formally smooth) if for all closed dense proposition $P$ the map:
\[X\to X^P\]
is an equivalence (resp. an embedding, surjective).
\end{definition}

\begin{remark}
The map $X\to X^P$ being an equivalence (resp. an embedding, surjective) if and only if for any map $P\to X$ we have a unique (resp. at most one, merely one) dotted lift in:
\begin{center}
\begin{tikzcd}
P\ar[r]\ar[d] & X\\
1\ar[ru,dotted]& \\
\end{tikzcd}
\end{center}
\end{remark}

\begin{definition}
A map is said formally étale (resp. formally unramified, formally smooth) if its fibers are formally étale (resp. formally unramified, formally smooth).
\end{definition}

\begin{remark}
A type (or map) is formally étale if and only if it is formally unramified and formally smooth.
\end{remark}

\begin{lemma}
A type $X$ is formally étale (resp. formally unramified, formally smooth) if and only if for all $\epsilon:R$ such that $\epsilon^2=0$, the map:
\[X\to X^{\epsilon=0}\]
is an equivalence (resp. an embedding, surjective).
\end{lemma}

\begin{proof}
The direct direction is obvious as $\epsilon=0$ is closed dense when $\epsilon^2=0$.

For the converse, assume $P=\Spec(R/N)$ a closed dense proposition. Then the map $R\to R/N$ with $N$ finitely generated nilpotent ideal can be decomposed as:
\[R\to A_1\to \cdots A_n = R/N\]
where $A_k$ is a quotient of $R$ by a finitely generated nilpotent ideal and:
\[A_k\to A_{k+1}\]
is of the form:
\[A\to A/(a)\]
for some $a:A$ with $a^2=0$.

We write $P_k = \Spec(A_k)$ and:
\[i_k:P_{k+1}\to P_k\] 
so that $\mathrm{fib}_{i_k}(x)$ is $a(x)=0$ where $a(x)^2=0$ holds.

Then by hypothesis we have that for all $k$ and $x:P_{k}$ the map:
\[X\to X^{\mathrm{fib}_{i_k}(x)}\]
is an equivalence (resp. an embedding, surjective). So the map:
\[X^{P_{k}} \to \prod_{x:P_{k}}X^{\mathrm{fib}_{i_k}(x)} = X^{P_{k+1}}\]
is an equivalence (resp. an embedding, surjective by $P_{k}$ having choice).
We conclude that the map:
\[X\to X^P\]
is an equivalence (resp. an embedding, surjective).
\end{proof}



\subsection{Stability results}

Being formally étale is a modality given as nullification at all dense closed propostions and therefore lex \cite{modalities}[Corollary 3.12].
This means we have the following results:

\begin{proposition}
\begin{itemize}
\item If $X$ is any type and for all $x:X$ we have a formally étale type $Y_x$, then:
\[\prod_{x:X}Y_x\]
is formally étale. 
\item  If $X$ is formally étale and for all $x:X$ we have a formally étale type $Y_x$, then:
\[\sum_{x:X}Y_x\]
is formally étale. 
\item If $X$ is formally étale then for all $x,y : X$ the type $x=y$ is formally étale.
\item The type of formally étale types is formally étale.
\end{itemize}
\end{proposition}

Formally unramified type are the separated types \cite{localization}[Definition 2.13] associated to formally étale types.
By \cite{localization}[Lemma 2.15], being formally unramified is a nullification modality as well.

\begin{lemma}
A type $X$ is formally unramified if and only if for any $x,y:X$ the type $x=y$ is formally étale.
\end{lemma}

This means we have the following:

\begin{proposition}
\begin{itemize}
\item If $X$ is any type and for all $x:X$ we have a formally unramified type $Y_x$, then:
\[\prod_{x:X}Y_x\]
is formally unramified. 
\item  If $X$ is formally unramified and for all $x:X$ we have a formally unramified type $Y_x$, then:
\[\sum_{x:X}Y_x\]
is formally unramified.
\end{itemize}
\end{proposition}

Being formally smooth is not a modality, indeed we will see it is not stable under identity types. Neverthless we have the following results:

\begin{lemma}\label{smooth-sigma-closed}
\begin{itemize}
\item If $X$ is any type satifying choice and for all $x:X$ we have a formally smooth type $Y_x$, then:
\[\prod_{x:X}Y_x\]
is formally smooth.
\item If $X$ is a formally smooth type and for all $x:X$ we have a formally smooth type $Y_x$, then:
\[\sum_{x:X}Y_x\]
is formally smooth.
\end{itemize}
\end{lemma}


\subsection{Type-theoretic examples}

The next proposition implies that open propositions are formally étale.

\begin{lemma}\label{not-not-stable-prop-etale}
  Any $\neg\neg$-stable proposition is formally étale.
\end{lemma}

\begin{proof}
  Assume $U$ is a $\neg\neg$-stable proposition. For $U$ to be formally étale it is enough to check that $U^P\to U$ for all $P$ closed dense. This holds because for $P$ closed dense we have $\neg\neg P$.
  \end{proof}
  
\begin{lemma}\label{closed-and-etale-decidable}
A closed and formally étale proposition is decidable.
\end{lemma} 

\begin{proof}
Given a formally étale closed proposition $P$, let us prove it is $\neg\neg$-stable. Indeed if $\neg\neg P$ then $P$ is closed dense so that $P\to P$ implies $P$ since $P$ is formally étale. 

Let $I$ be the finitely generated ideal in $R$ such that:
\[P\leftrightarrow I=0\]
We have that $I^2=0$ implies $\neg\neg (I=0)$ which implies $I=0$. But then we have that $I=I^2$, so that by Nakayama (see \cite[Lemma II.4.6]{lombardi-quitte}) there exists $e:R$ such that $eI = 0$ and $1-e\in I$. If $e$ is invertible then $I=0$, if $1-e$ in invertible then $I=R$.
\end{proof}

\begin{proposition}\label{bool-is-etale}
  The type $\Bool$ is formally étale.
\end{proposition}

\begin{proof}
The identity types in $\Bool$ are decidable so $\Bool$ is formally unramified. Consider $\epsilon:R$ such that $\epsilon^2=0$ and a map:
\[\epsilon=0 \to \Bool\]
we want to merely factor it through $1$.

 Since $\Bool\subseteq R$, by duality the map gives $f:R/(\epsilon)$ such that $f^2=f$. Since $R/(\epsilon)$ is local we conclude that $f = 1$ or $f=0$ and so the map has constant value $0:\Bool$ or $1:\Bool$.
\end{proof}

\begin{remark}\label{finite-are-etale}
This means that formally étale (resp. formally unramified, formally smooth) types are stable by finite sums. In particular finite types are formally étale.
\end{remark}

%Not used anywhere, maybe remove TODO?
\begin{proposition}
The type $\N$ is formally étale.
\end{proposition}

\begin{proof}
Identity types in $\N$ are decidable so $\N$ is formally unramified, we want to show it is formally smooth. Assume given a map:
\[P\to \N\]
for $P$ a closed dense proposition, we want to show it merely factors through $1$. By boundedness the map merely factors through a finite type, which is formally étale by \Cref{finite-are-etale} so we conclude.
\end{proof}

\begin{lemma}\label{prop-are-unramified}
Any proposition is formally unramified.
\end{lemma}

This means that any subtype of a formally unramified type is formally unramified.

\begin{remark}
  Given any lex modality, a type is separated if and only if it is a subtype of a modal type,
  so a type is formally unramified if and only if it is a subtype of a formally étale type.
\end{remark}

We also have the following surprising dual result, meaning that any quotient of a formally smooth type is formally smooth:

\begin{proposition}\label{smoothSurjective}
If $X$ is formally smooth and $p:X\to Y$ surjective, then $Y$ is formally smooth.
\end{proposition}

\begin{proof}
For any $P$ closed dense and any diagram:
 \begin{center}
      \begin{tikzcd}
      P \ar[rd,dotted]\ar[d]\ar[r]& Y\\
      1 \ar[r,dotted,swap,"x"]& X\ar[u,swap,"p"]
      \end{tikzcd}
    \end{center} 
    by choice for closed propositions we merely get the dotted diagonal, and since $X$ is formally smooth we get the dotted $x$, and then $p(x)$ gives a lift.
\end{proof}


\subsection{Classical definitions, examples and counter-examples}

We will show in this section that our definition of étale/smooth/unramified maps and types is equivalent to a internal version of the classical definition. It is important to keep in mind, that our schemes are always locally of finite presentation, so the following definition is sensible:

\begin{definition}
  A \notion{étale/unramified/smooth scheme} is a scheme which is formally étale/unramified/smooth.
  A \notion{étale/unramified/smooth map} is a map between schemes which is formally étale/unramified/smooth.
\end{definition}

The following criterion appears as the definition of a formally étale/smooth/unramified map in \cite{EGAIV4}[§17],
with the only difference, that the lifting property is stated in terms of the comparison map into a pullback and also non-finitely presented algebras and general ideals are considered. The latter is superfluous for maps between schemes, where, as stated above, we only consider schemes that are locally of finite presentation. It is however not clear if this internal criterion corresponds to the (external) definitions in \cite{EGAIV4}[§17].

\begin{remark}
  Let $f:X\to Y$ be a map such that every fiber is a scheme.
  Then $f$ is formally étale/smooth/unramified if and only if there is exactly one/at least one/at most one lift in all squares
  \begin{center}
    \begin{tikzcd}
      \Spec(A/N)\ar[r,"t"]\ar[d] & X\ar[d,"f"] \\
      \Spec(A)\ar[r,"b"] & Y
    \end{tikzcd}    
  \end{center}
where $A$ is a finitely presented $R$-algebra, $N$ a finitely generated nilpotent ideal and the map $\Spec(A/N)\to\Spec(A)$ is induced by the quotient map.
\end{remark}

\rednote{TODO: make proof more readable/complete}
\begin{proof}
  Without loss of generality, we can assume $N$ is of the form $(a)$ with $a$ square-zero.
  Let $f:X\to Y$ be a map as be formally smooth.
  Then the existence of a lift can be shown
  by finding a family of lifts for each point $v:\Spec A$ with $\epsilon:=a(v)$ and $\phi$ induced by $t$:
  \begin{center}
    \begin{tikzcd}
      \epsilon=0\ar[r,"\phi"]\ar[d] & \fib_f(b(v))=:Z \\
      1\ar[ur,dotted] &
    \end{tikzcd}
  \end{center}
  For any two lifts $\psi,\xi$, we have $\neg\neg(\psi=\xi)$
  and can assume $Z=\Spec R[X_1,\dots,X_n]/P_1,\dots,P_l$.
  We merely have $y:R^n$ such that the dual to $\phi$ is given by evaluation at $y$.
  Then lifts are given by vectors $x:R^n$ such that $P(x)=0$ and $\epsilon = 0$ implies $x=y$.
  The latter dualizes to an inclusion of ideals $(x_1-y_1,\dots,x_n-y_n)\subseteq (\epsilon)$.
  So we have $x_i-y_i=\alpha_i\epsilon$. Putting everything together, we get that a lift is given by numbers $\alpha_i\epsilon$ such that
  \begin{align*}
     0 = P(y+\alpha \epsilon) = P(y)+dP_y(\epsilon\alpha)
  \end{align*}
  For a different choice of $y$, we get a translated solution, so we get a well-defined module $M_v$ by Krauss-Lemma applied to the 1-type of modules.
  $M_v$ is wqc since for each $y$ it is the image of multiplication by $\epsilon$ of the kernel of the linear map $dP_y$.
  Then by \cite{draft}, $H^1(v:\Spec A, M_v)$ vanishes and we therefore have a global lift. 
\end{proof}

The following lemma is easy to proof -- we conclude this section afterwards with counter examples for smoothness and one example of an étale scheme.

\begin{lemma}\label{An-is-smooth}
For all $k:\N$, we have that $\A^k$ is smooth.
\end{lemma}

\begin{proof}
  Let $P$ be a closed dense proposition and $N$ a nilpotent, finitely generated ideal such that $P=\Spec R/N$.
  Then $\Spec R[X_1,\dots,X_k]=\A^k$ is smooth lifts always exist as indicated below by the universal property of $R[X_1,\dots,X_k]$:
  \begin{center}
    \begin{tikzcd}
      R/N & R[X_1,\dots,X_k]\ar[l]\ar[ld,dotted]  \\
      R\ar[u] & 
    \end{tikzcd}
  \end{center}
\end{proof}

\begin{example}
The affine scheme $\Spec(R[X]/X^2)$ is not smooth.
\end{example}

\begin{proof}
If it were smooth then, for any $\epsilon$ with $\epsilon^3=0$, we would be able to prove $\epsilon^2=0$.
Indeed we would merely have a dotted lift in:
 \begin{center}
      \begin{tikzcd}
        R/(\epsilon^2)& R[X]/(X^2)\ar[l,"\epsilon"]\ar[ld,dashed] \\
        R \ar[u]& 
      \end{tikzcd}
    \end{center}
    that is, an $r:R$ such that $(\epsilon+r\epsilon^2)^2=0$. Then $\epsilon^2=0$.
\end{proof}

\begin{example}
The affine scheme $\Spec(R[X,Y]/XY)$ is not smooth.
\end{example}

\begin{proof}
Again, we assume a lift for any $\epsilon$ with $\epsilon^3=0$:
 \begin{center}
   \begin{tikzcd}
     R/(\epsilon^2) & R[X,Y]/(XY)\ar[l]\ar[ld,dashed] \\
     R\ar[u] & 
   \end{tikzcd}
 \end{center}
 where the top map sends both $X$ and $Y$ to $\epsilon$. Then we have $r,r':R$ such that $(\epsilon+r\epsilon^2)(\epsilon+r'\epsilon^2)=0$ so that $\epsilon^2=0$.
\end{proof}

We will proof a generalization of the following example in \Cref{standard-etale-are-etale}.
The essential step is to improve a zero $g(y)=0$ up to some square-zero $\epsilon$ to an actual zero.

\begin{example}
Let $g$ be a polynomial in $R[X]$ such that for all $x:R$ we have that $g(x)=0$ implies $g'(x)\not=0$. Then:
\[\Spec(R[X]/g)\]
is étale.
\end{example}


\subsection{Being formally étale, unramified or smooth is Zariski local}

\begin{lemma}\label{etale-zariski-local}
Let $X$ with $(U_i)_{i:I}$ be a finite open cover of $X$. Then $X$ is formally étale (resp. formally unramified, formally smooth) if and only if all the $U_i$ are formally étale (resp. formally unramified, formally smooth).
\end{lemma}

\begin{proof}
First, we show this for formally unramified:
\begin{itemize}
\item Any subtype of a formally unramified type is formally unramified by \Cref{prop-are-unramified}.
\item Conversely, assume $X$ with such a cover, for all $x,y:X$ there exists $i:I$ such that $x\in U_i$ and then:
\[x=_Xy \leftrightarrow \sum_{y\in U_i} x=_{U_i}y \]
which is formally étale because open propositions are formally étale by \Cref{not-not-stable-prop-etale}.
\end{itemize}
Now for formally smooth:
\begin{itemize}
\item Open propositions are formally smooth by \Cref{not-not-stable-prop-etale} so that open subtypes of formally smooth types are formally smooth.
\item Conversely if each $U_i$ is formally smooth then $\Sigma_{i:I}U_i$ is formally smooth by \Cref{finite-are-etale}, so we can conclude by applying \Cref{smoothSurjective} to the surjection:
\[\Sigma_{i:I}U_i \to X\]
\end{itemize}
The result for formally étale immediately follows.
\end{proof}

\begin{corollary}
For all $k:\N$, the projective space $\bP^k$ is smooth.
\end{corollary}

\begin{proof}
By \Cref{etale-zariski-local} it is enough to check that $\A^k$ is smooth. This is \Cref{An-is-smooth}
\end{proof}




