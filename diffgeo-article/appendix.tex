\subsection*{Rank of matrices}

\begin{definition}
A matrix is said to have rank $\leq n$ if all its $n+1$-minors are zero. It is said to have rank $n$ if it has rank $\leq n$ and does not have rank $\leq n-1$.
\end{definition}

Having a rank is a property of matrices, as a rank function defined on all matrices would allow to e.g. decide if an $r:R$ is invertible.

\begin{lemma}\label{rank-bloc-matrix}
Assume given a matrix $M$ of rank $n$ decomposed into blocks:
\[M = \begin{pmatrix}
P & Q  \\
R & S \\
\end{pmatrix}\]
Such that $P$ is square of size $n$ and invertible. Then we have:
\[S = RP^{-1}Q\]
\end{lemma}

\begin{proof}
By columns manipulation the matrix is equivalent to:
\[M = \begin{pmatrix}
P & Q  \\
0 & S - RP^{-1}Q \\
\end{pmatrix}\]
but equivalent matrices have the same rank so $S=RP^{-1}Q$.
\end{proof}

\begin{lemma}\label{rank-equivalent-definitions}
If a linear map $R^m \to R^n$ given by multiplication with $M$
has finite free kernel of rank $k$, then $M$ has rank $m-k$.
\end{lemma}

\begin{proof}
  Let $a_1,\dots,a_{k}$ be a basis for the kernel of $M$ in $R^m$, which we complete into a basis of $R^m$ via $b_{k+1},\dots,b_m$.
  By completing $Mb_{k+1},\dots, Mb_m$ to a basis of $R^n$, we get a basis where $M$ is written as:
\[\begin{pmatrix}
I_{m-k} & 0  \\
0 & 0 \\
\end{pmatrix}\]
so that $M$ has rank $m-k$.
\end{proof}

%\begin{definition}
%Two matrices $M,N$ are said equivalent if there are invertible matrices $P,Q$ such that $M = PNQ$.
%\end{definition}

%It is clear that equivalent matrices have the same rank.

%\begin{lemma}\label{rank-equivalent-definitions}
%Assume given a matrix:
%\[M : R^m\to R^k\]
%Then the following are equivalent:
%\begin{enumerate}[(i)]
%\item $M$ has rank $n$.
%\item The kernel of $M$ is equivalent to $R^{m-n}$.
%\item The image of $M$ is equivalent to $R^n$.
%\item $M$ is equivalent to the bloc matrix:
%\[\begin{pmatrix}
%I_n & (0)  \\
%(0) & (0) \\
%\end{pmatrix}\]
%\end{enumerate}
%\end{lemma}

%\begin{proof}
%\end{proof}
