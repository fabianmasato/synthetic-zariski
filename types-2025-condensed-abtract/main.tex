% latexmk -pdf -pvc main.tex
% latexmk -pdf -pvc -interaction=nonstopmode main.tex
\documentclass{../util/zariski}

\RequirePackage[safe]{tipa}

% Authors are joined by \and. 
% Their affiliations are given by \inst, which indexes
% into the list defined using \institute
%
\author{
Felix Cherubini %\inst{1}
% uncomment the following for multiple authors.
\and 
 Thierry Coquand% \inst{2}%
\and 
 Freek Geerligs% \inst{3}%
% \thanks{Speaker.}%
\and
 Hugo Moeneclaey %\inst{4}%
}

% Institutes for affiliations are also joined by \and,
% \institute{
%  University of Gothenburg and Chalmers University of Technology, Gothenburg, Sweden%\\
% }

% \authorrunning{Cherubini, Coquand, Geerligs and Moeneclaey}

\title{Cohomology in Synthetic Stone Duality}

\begin{document}
\maketitle
Peter Scholze and Dustin Clausen \cite{Scholze} introduced light condensed sets, defined as sheaves on the site of profinite sets, which can be used as an alternative to topological spaces. 
An extension of homotopy type theory by axioms which is called synthetic Stone duality was introduced in \cite{synthetic-stone-duality}. In this extension, we can prove some results about the first integral cohomology groups: we have that $H^1(S,\Z) = 0$ for $S$ a Stone space, that $H^1(X,\Z)$ for $X$ compact Hausdorff can be computed using \v{C}ech cohomology and that $H^1(\mathbb{I},\Z) = 0$ where $\mathbb{I}$ is the unit interval \cite{synthetic-stone-duality}. In this talk we will present the extension of these results to higher cohomology groups with non-constant countably presented abelian groups as coefficients.

\subsection*{Synthetic Stone Duality}

In traditional topology, Stone spaces are defined as totally disconnected Hausdorff spaces, or equivalently as profinite sets. In synthetic Stone duality, Stone spaces are precisely (countable) sequential limits of finite sets, which means there is a size restriction on Stone spaces analogous to \emph{light} condensed sets. 

The axioms of synthetic Stone duality postulate Stone duality (Stone spaces are equivalent to countably preented Boolean algebras), completeness (non-empty Stone spaces are merely inhabited), dependent choice and a \emph{local-choice} axiom. The latter is the only axiom mentioning higher types, it says that given a Stone space $S$ and any type family $B$ over $S$ such that $(x:S)\to \| B(x)  \|$, there merely is a Stone space $T$ and a surjection $s:T\to S$ such that $(x:T)\to B(s(x))$. Local choice is crucial when performing the cohomological computations mentioned below, but has consequences beyond that.

Many traditional properties of Stone spaces can be shown synthetically, sometimes phrased in a more type-theoretic way, e.g. Stone spaces are closed under $\Sigma$-types. Open and closed propositions can be defined, inducing a topology on any type such that any map is continuous. This topology is as expected for Stone spaces and compact Hausdorff spaces (i.e. quotient of Stone spaces by closed equivalence relations).

One important example of compact Hausdorff space is the real interval $\I$, from which the real numbers $\R$ are constructed.
As in \cite{shulman-Brouwer-fixed-point} it is important to distinguish topological spaces like $\bS^1:=\{(x,y):\R^2\vert x^2+y^2=1\}$ from homotopical spaces like the higher inductive 1-type $S^1$.

Despite topological spaces being sets, it is possible to use higher types to reason about them. Indeed, for any abelian group $A$, the homotopical Eilenberg-MacLane spaces $K(A,n)$ can be constructed as usual in homotopy type theory. For any type $X$ and dependent abelian group $A:X\to \mathrm{Ab}$, we use the usual synthetic definition of the $n$-th cohomology group $H^n(X,A)$ as $\|(\Pi_{x:X}K(A_x,n)\|_0$. In \cite{synthetic-stone-duality}, it is proven that $H^1(\bS^1,\Z)=\Z$, despite $\bS^1$ being a set.

\subsection*{Barton and Commelin's condensed type theory}

Reid Barton and Johann Commelin proposed a system of axioms called condensed type theory. This system postulates classes of types called compact Hausdorff spaces and overtly discrete types, with certain duality axioms relating them. We prove the axioms of Barton and Commelin in synthetic Stone duality, defining compact Hausdorff spaces as quotients of Stone spaces by closed equivalence relations and overtly discrete types as sequential colimits of finite types.

An abelian group is overtly discrete if and only if it is countably presented, so we will use countably presented abelian groups as coefficients for cohomology. The following are crucial for our cohomology results:
\begin{itemize} 
\item Barton's Commelin dual to Tychonoff's axiom: For any compact Hausdorff space $X$ and any family of overtly discrete $B_x$ for $x:X$, the dependent product $\Pi_{x:X}B(x)$ is overtly discrete) 
\item A generalisation of both Barton and Commelin's Scott continuity and factorisation axioms: Given sequences of compact Hausdorff spaces $(C_k)_{k:\N}$ and overtly discrete types $(I_i)_{i:\N}$, we have that:
\[\mathrm{colim}_{k,i}(C_k\to I_i) \simeq \left(\mathrm{lim}_kC_k \to \mathrm{colim}_iI_i\right)\]
We call it generalised Scott continuity.
\item A dependent version of generalised Scott-continuity, to account for non-constant cohomology coefficients.
\end{itemize}


\subsection*{Vanishing of higher cohomology of Stone spaces}

First we prove that $H^1(S,A) = 0$ with $S$ Stone and  $A : S\to \mathrm{Ab}_{\mathrm{cp}}$. We assume $\alpha:\prod_{x:S}K(A_x,1)$, by local choice we get a surjection $p:T\twoheadrightarrow S$ with $T$ Stone which trivialises $\alpha$. Then we get an approximation of $p$ as a sequential limit of surjective maps $p_k:T_k\to S_k$ between finite sets, we check that the induced trivialisation over $T_k$ gives a trivialisation over $S_k$, and conclude through the dependent version of generalised Scott Continuity that $\alpha$ is trivial over $\mathrm{lim}_kS_k =S$.

We follow an idea of Wärn \cite{cech-draft}[Theorem 3.4] to go from the $H^1(S,A)=0$ to $H^n(S,A)=0$ for $n>1$. The key idea is to proceed by induction on $n$, generalising the induction hypothesis from $H^n(S,A) = 0$ to: 
\begin{enumerate}[(i)]
\item $K(\prod_{x:S}A_x,n) \to \prod_{x:S}K(A_x,n)$ is an equivalence.
\item $K(\prod_{x:S}A_x,n+1) \to \prod_{x:S}K(A_x,n+1)$ is an embedding.
\end{enumerate}
Assume (i) and (ii) for $n-1>0$, let's prove (i) and (ii) for $n$. (ii) follows immediately from induction hypothesis (i). To prove (ii) it is enough to prove $\prod_{x:S}K(A_x,n)$ is connected, i.e. $H^n(S,A)=0$. We assume $\alpha:\prod_{x:S}K(A_x,n)$, by local choice we get a trivialisation $p:T\twoheadrightarrow S$ of $\alpha$ with $T$ Stone. Then writing $T_x$ the fiber of $p$ over $x$, we consider the exact sequence:
\[0\to A_x\to A_x^{T_x}\to L_x\to 0\]
giving an exact sequence:
\[H^{n-1}(S,L)\to H^n(S,A)\to H^n(S,\lambda x. A_x^{T_x})\]
By induction hypothesis (i) we have that $H^{n-1}(S,L) = 0$ so we have an injection:
\[H^n(S,A)\to H^n(S,\lambda x. A_x^{T_x})\]
By induction hypothesis (ii) the map: 
\[H^n(S,\lambda x. A_x^{T_x}) \to H^n(\Sigma_{x:S}T_x,A_x) = H^n(T,A\circ p)\]
is an injection and we get an injection:
\[H^n(S,A)\to H^n(T,A\circ p)\] 
But $p$ trivialise $\alpha$ so it vanishes in $H^n(T,A\circ p)$, therefore $\alpha=0$.

\subsection*{\v{C}ech cohomology for compact Hausdorff spaces}

Given a compact Hausforff type $X$, a \v{C}ech cover for $X$ consists of a Stone space $S$ and a surjective map $p:S\twoheadrightarrow X$. By definition any compact Hausdorff space has a \v{C}ech cover.

Given such a \v{C}ech cover and $A:X\to \mathrm{Ab}_{\mathrm{cp}}$, we define its \v{C}ech complex as:
\[\prod_{x:X}A_x^{S_x} \to \prod_{x:X}A_x^{S_x\times S_x} \to \cdots\]
with the boundary maps defined as expected, and its \v{C}ech cohomology $\check{H}^k(X,A)$ as the $k$-th homology group of its \v{C}ech complex.
%\[\delta(\alpha)(x,u_0,\cdots,u_n) = \sum_{i=0}^n (-1)^i\alpha(x,u_0,\cdots,\hat{u_i},\cdots,u_n)\]

From vanishing, we get an exact sequences:
\[H^{n-1}(X,\lambda x.A_x^{S_x}) \to H^{n-1}(X,L)\to H^n(X,A)\to 0\]
and by the usual long exact sequence argument and direct computations we get exact sequences:
\[\check{H}^{n-1}(X,\lambda x.A_x^{S_x}) \to \check{H}^{n-1}(X,L)\to \check{H}^n(X,A)\to 0\]

We conclude inductively that $H^n(X,A) = \check{H}^n(X,A)$, so that in particular \v{C}ech cohomology do not depend on a choice of \v{C}ech cover. For this induction to go through it is crucial that $A_x^{S_x}$ is overtly discrete, which follows from the dual to Tychonov theorem.

Using finite approximations to the \v{C}ech cover $2^\N\to \mathbb{I}$, we can check that $H^n(\mathbb{I},A) = 0$ for all $A:\mathbb{I}\to \mathrm{Ab}_{\mathrm{cp}}$.

\printbibliography

\end{document}
