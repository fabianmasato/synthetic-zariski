A line bundle on a type $X$ is a map $X\rightarrow \KR$.

%For $BGL_1$ we take $\Sigma_{M:\Mod{R}}\|M = R\|$.

\medskip

 A line bundle $L$ on $\Spec(A)$ will define a f.p. $A$-module $\prod_{x:\Spec(A)}L(x)$ \cite{draft}.
It is presented by a matrix $P$.
Since this f.p. module is locally free, we can find $Q$ such that $PQP = P$ and
$QPQ = Q$ \cite{lombardi-quitte}. We then have $Im(P) = Im(PQ)$ and this is a projective module of rank $1$. We can then assume $P$ square matrix and
$P^2 = P$ and the matrix $I-P$ can  be seen as listing the generators of this module.

If $M$ is a matrix we write $\Delta_l(M)$ for the ideal generated by the $l\times l$ minors of
$M$. We have $\Delta_1(I-P) = 1$ and $\Delta_2(I-P) = 0$, since this projective module is of rank $1$.

The module is free exactly if we can find a column vector $x$ and a line vector $y$ such that
$xy = I-P$. We then have $yx = 1$, since if $r = yx$ we have $I-P = xyxy = rxy = r(I-P)$ and
hence $r = 1$ since $\Delta_1(I-P) = 1$.
%We then
%$YX = 1$ since if we write $r = YX$ we have $Q^2 = XYXY = rXY = rQ$ and so $(1-r)Q = 0$ which implies $1 = r$ since $Q$
%is of rank $1$.

\medskip

The line bundle on $\Spec(A)$ is trivial on $D(f)$ if, and only if, the module $M\otimes A[1/f][X]$ is free, which
is equivalent to the fact that we can find $X$ and $Y$ such that $YX = (f^N)$ and $XY = f^N(I-P)$ for some $N$.

In Appendix 1, we prove the following special case of Horrocks' Theorem.

\begin{lemma}%\label{Horrocks}
  If $A$ is a commutative ring
  then any ideal of $A[X]$ divides a principal ideal $(f)$, with $f$ monic, is itself a principal ideal.
\end{lemma}

We can then apply this result in Synthetic Algebraic Geometry for the ring $R$.

\begin{proposition}
  If we have $L:\A^1\rightarrow \KR$ which is trivial on some $D(f)$ where $f$ in $R[X]$ is monic
  then $L$ is trivial on $\A^1$.
\end{proposition}

\begin{corollary}\label{c1}
  If we have $L:\bP^1\rightarrow \KR$ then we have
  $$\|{\prod_{r:R}L([1:r]) = L([1:0])}\|\,\,\,\,\,\,\,\,\,\,\,\,\,\|{\prod_{r:R}L([r:1]) = L([0:1])}\|$$
\end{corollary}

\begin{proof}
  We have the two open injections $i_0:\A^1\rightarrow \bP^1,~r\mapsto (r:1)$ and
  $i_1:\A^1\rightarrow \bP^1,~r\mapsto (1:r)$.
  By Zariski local choice \cite{draft}, the line bundle $L\circ i_0$ is locally trivial on $\A^1 = \Spec(R[X])$.
  In particular, there exists $g = a_0 + a_1 X + \dots + a_nX^n$ in $R[X]$ such that $0$ is in $D(g)$, i.e. $a_0\neq 0$
  and $L\circ i_0$ is trivial on $D(g)$. Let $f$ be $X(X^n + a_1X^{n-1}+\dots + a_n)$. We have $i_1(D(f))\subseteq D(g)$
  and hence $L\circ i_1$ is trivial on $D(f)$. Since $f$ is monic, $L\circ i_1$ is trivial by the previous Proposition.
  Similarly $L\circ i_0$ is trivial.
  %% On one chart of $\bP^1$, $L$ is trivial on a neighborhood $U$ of $0$, so we get $g:R[X]$ such that $g(0)\neq 0$ and $L$ is trivial on $D(g)$.
  %% Passing to the other chart, there is some $N$ such that $f\colonequiv g(0)^{-1}\cdot g(1/X)\cdot X^N$ is a monic polynomial and $L$ is trivial on $D(f)$,
  %% since $D(f)\subseteq U$.
\end{proof}


