In this section, we present an alternative way to prove that   $\Aut(\bP^n)$ is $\PGL_{n+1}$.

A projective module of rank $1$ over a polynomial
ring $K[X_1,\dots,X_n]$ is free, where $K$ is a discrete field, since this polynomial ring is a gcd domain, see
e.g. \cite{seminormal}. (The result actually holds in general
for rank $n$, this is the famous Serre's Problem \cite{Lam}.) Interpreting this proof dynamically
\cite{lombardi-quitte}, it follows that if $A$ is now an arbitrary commutative ring, and $M$ a
projective module of rank $1$ over $A[X_1,\dots,X_n]$, it is possible to build a binary tree, where
the root is $A$, and each node is an ring extension $i:A\rightarrow B$, and a branch is obtained
by forcing an element $a$ in $A$ to be invertible $B\rightarrow B[1/i(a)]$ or zero $B\rightarrow B/(i(a))$,
and each leaf is such that $M\otimes B[X_1,\dots,X_n]$ is free.

For the ring $R$, we have $a\neq 0\rightarrow \exists_x ax=1$ and hence $\neg \neg (a=0\vee \exists_x a x=1)$.
Hence we have the following result, by a descending induction over finite binrary trees.

\begin{lemma}\label{notnot}
  A projective module of rank $1$ over $R[X_1,\dots,X_n]$ is not not free. Equivalently, 
a line bundle on $\Spec(R[X_1,\dots,X_n]) = R^n$ is not not trivial.
\end{lemma}

This generalizes Corollary 3.3.6 of \cite{draft}.


%% \begin{proof}
%%   Let $L$ be a line bundle $R^n\rightarrow\KR$. As we saw above, the $R[X_1,\dots,X_n]$-module $M = \prod_{u:\Spec(A)}L(u)$,
%%   which is projective of rank $1$,   is presented by an idempotent square matrix $P$, and this module is free
%%   exactly if we can find a column vector $x$ and a line vector $y$ such that $xy = I-P$. Using
%%   \cite{seminormal}, this is satisfied if $A$ is a gcd domain. Since $R$ is not not a discrete
%%   field, in the sense that we have $\forall_{x:R}\neg\neg (x=0 \vee \exists_y (xy = 1))$, we conclude
%%   that we have $\neg\neg \exists_{x~y}xy = I-P$, and hence the $A$-module  $M$
%%   is not not free, in the sense $\neg\neg \exists_{m_0:M}\forall_{m:M}\exists!_{a:A} m = am_0$
%%   (or, equivalently $\neg\neg M = A$, where the equality is equality of $A$-modules).
%% \end{proof}

Similarly, using the fact that maps $\bP^n\rightarrow \bP^m$ are given by $m+1$ homogeneous polynomials in $K[X_0,\dots,X_n]$
over a discrete field $K$, we deduce.

\begin{lemma}\label{weakmap}
  Given $\varphi:\bP^n\rightarrow  \bP^m$, we have not not there exists
  $m+1$ homogeneous polynomials $p = (p_0,\dots,p_m)$ on $R^{n+1}$
  of the same   degree $d$ such that $x\neq 0$ implies $p(x)\neq 0$.
\end{lemma}

%% \begin{proof}
%%   As in the previous Section, let $T$ be the ring of polynomials $u = \Sigma_p u(p)X^p$ with
%% $X^p = X_0^{p_0}\dots X_n^{p_n}$ with $\Sigma p_i = 0$. We write $T_l$ for the subring
%% of $T$ which contains only monomials $X^p$ with $p_i\geqslant 0$ if $i\neq l$
%% and $T_{lm}$ the subring of $T$ 
%% which contains only monomials $X^p$ with $p_i\geqslant 0$ if $i\neq l$ and $i\neq m$.

%%   We view $\varphi(u)$ as a line in $R^{m+1}$. 
%%   If $U_i = Sp(T_i)$ is the affine open subset of
%%   $(x_0:\dots:x_n)$ in $\bP^n$ such that $x_i\neq 0$, we have by the previous Lemma that
%%   $\prod_{u:U_i}\varphi(u)$ is not not a free $T_i$-module and so we not not have
%%   an element $\psi_i:\prod_{u:U_i}R^{m+1}-0$ such that $\psi_i(u)$ generates $\varphi(u)$ for $u:U_i$.
%%   We get then $m+1$ elements $(p_{i0},\dots,p_{im})$ of $T_i$ such that
%%   $\psi_i(u) = (p_{i0}(u),\dots,p_{im}(u))$.

%%   For $u$ in $U_i\cap U_j$ the line $\varphi(u)$ is generated both by $\psi_i(u)$ and $\psi_j(u)$
%%   and we not not have $t_{ij}$ invertible in $T_i[X_i/X_j] = T_j[X_j/X_i] = T_{ij}$ such that
%%   $t_{ij}(u)\psi_i(u) = \psi_j(u)$. By Proposition \ref{units} (which has a much simpler proof in the case
%%   of a discrete field), we have not not $d$ and $s_i$ invertible in $T_i$ such that
%%   $t_{ij} = X_j^d s_j/X_i^d s_i$.
%%   We then not not
%%   get $m+1$ homogeneous polynomials $p_{il}X_i^ds_i = p_{jl}X_j^d s_j = p_l$ for $l = 0,\dots,m$
%%   such that $\varphi(u) = (p_0(u):\dots:p_m(u))$.
%% \end{proof}

 We can then give a proof that   $\Aut(\bP^n)$ is $\PGL_{n+1}$ without relying on Horrocks' Theorem.

We see the element of $\bP^n$ as lines in $R^{n+1}$.
If $p_0,\dots,p_n$ are $n+1$ points of $\bP^n$, we can define the open proposition that
the points $p_0,\dots,p_n$ are independent
by the fact that for any choice of non zero elements $v_0,\dots,v_n$ in respectively $p_0,\dots,p_n$
the determinant of the $(n+1)\times (n+1)$ matrix $v_0\dots v_n$ is $\neq 0$.

\begin{corollary}
    $\Aut(\bP^n)$ is $\PGL_{n+1}$.
\end{corollary}

\begin{proof}
  Let $\varphi$ be an element of $\Aut(\bP^n)$.
  Let $u_0,\dots,u_n$ be the vectors $(1,0,\dots,0), \dots, (0,\dots,0,1)$
  and $u$ be the vector $(1,1,\dots,1)$.

  We can find a $(n+1)\times (n+1)$ matrix $A$ such that $\varphi(R u_i) = R (Au_i)$
  and $\varphi(R u) = R (A u)$. By Lemma \ref{weakmap}, we have not not $\varphi$ is represented by $A$,
  that is $\neg\neg \forall_{v\neq 0}\varphi(R v) = R (Av)$. If follows that $\neg \neg det(A) \neq 0$
  and hence $\det(A)\neq 0$ and $A$ is in $GL_{n+1}$.

  In this way, we reduce the problem of showing that if $\neg\neg \forall_{v\neq 0}\varphi (R v) = R v$
  then $\varphi$ is the identity. But then we have not not $\varphi(U_i)\subseteq U_i$
  and so $\varphi(U_i)\subseteq U_i$ for the standard $n+1$ affine charts $U_i$ of $\bP^n$, and we can
  conclude as in \cite{Demazure}, III, 4.6.
\end{proof}
