Grothendieck advocated for a functor of points approach to schemes early on in his
project of foundation of algebraic geometry (see the introduction of \cite{EGAI}).
In this approach, a scheme is defined as a special kind of (covariant) set valued functor
on the category of commutative rings. This functor should in particular
be a sheaf w.r.t.\ the Zariski topology. As a typical example, the projective space $\bP^n$
is the functor, which to a ring $A$,
associates the set of sub-modules of  rank $1$ of $A^{n+1}$, which are direct factors \cite{Demazure,Eisenbud,Jantzen}.

In the 70s, Anders Kock suggested to use the language of higher-order logic \cite{Church40}
to describe the Zariski topos, the collection of sheaves for the Zariski topology \cite{Kock74,kockreyes}.
This allows for
a more suggestive and geometrical description of schemes, that can now be seen as a special kind
of types satisfying some properties. Morphisms of schemes in this setting are just general maps.
There is in particular a ``generic
local ring'' $R$, which associates to $A$ its underlying set. As described in \cite{kockreyes}
the projective space $\bP^n$ is then the set of lines in $R^{n+1}$.

In \cite{draft}, we presented an axiomatisation of the Zariski {\em higher topos} \cite{lurie-htt},
using instead of the language of higher-order logic the language of dependent type theory
with univalence \cite{hott}. The first axiom is that we have a local ring $R$. We then define
an affine scheme to be a type of the form $\Sp(A) = \Hom_{\Alg{R}}(A,R)$ for some finitely presented
$R$-algebra $A$. The second axiom, inspired from the work of Ingo Blechschmidt \cite{ingo-thesis},
states that the evaluation map $A\rightarrow R^{\Sp(A)}$ is a bijection. The last axiom states
that each $\Sp(A)$ satisfies some form of local choice \cite{draft}. We can then define a notion
of {\em open} proposition, with the corresponding notion of open subset, and define a scheme as a type
covered by a finite number of open subsets that are affine schemes. In particular, we define
$\bP^n$ as in \cite{kockreyes} and show that it is a scheme, by the usual covering by $n+1$
open affine subsets $U_0,\dots,U_n$.

A natural question is if we can show in this setting that the automorphism group of $\bP^n$
is  $\PGL_{n+1}(R)$.
More generally, can we show that any map $\bP^n\rightarrow \bP^m$ is given by $m+1$ homogeneous
polynomials of same degree in $n+1$ variables?
From this, it is possible to deduce the corresponding result about $\bP^n$ defined as
a functor of points (but the maps are now {\em natural transformations}) or about $\bP^n$ defined
as a scheme (but the maps are now {\em maps of schemes}).
While this result is a basic text book in the case of projective space over a field, the general
case is more subtle.
(This general result, though fundamental, is not in \cite{Hartshorne} for instance.)
One goal of this paper is to present such a proof in the setting of \cite{draft},
dependent type theory with univalence extended with these 3 axioms described above.
%we show the above result about maps between $\bP^n$ and $\bP^m$ and the result about automorphisms of $\bP^n$.

Interestingly, though these results are
about the Zariski $1$-topos, the proof makes use of types that
are {\em not} (homotopy) sets (in the sense of \cite{hott}).
The argument proceeds as follows. The first step is to show that any line bundle on $\bP^1$
is trivial on each standard affine chart. The proof here is a reformulation of the standard
argument which follows from Horrocks Theorem \cite{Horrocks,Quillen,lombardi-quitte,Lam}.
The usual way to generalise this result to $\bP^n$ is to use the technique
of Quillen patching \cite{Quillen,lombardi-quitte,Lam}. 
We present an alternative argument which proceeds in characterizing $\bP^n\rightarrow\KR$
as $\ints\times\KR$, where $\KR$ is the delooping
(thus a type which is not a set) of the multiplicative group of units of $R$. Once we have
the result that a line bundle on $\bP^n$ is trivial on each standard affine chart, we
can deduce by a purely algebraic result (Proposition \ref{units}) that
$\Pic(\bP^n) = \ints$ and $\Aut(\bP^n) = \PGL_{n+1}(R)$ and that
any map $\bP^n\rightarrow \bP^m$ is given by $m+1$ homogeneous
polynomials of same degree in $n+1$ variables. We also provide an alternative direct geometric
proof that $\Pic(\bP^n) = \ints$ from our charcterisation of $\bP^n\rightarrow \KR$. 





%% Schemes as special kind of sheaf for Zariski topos.

%% Even nicer in a type theoretic framework

%% Anders Kock property of Zariski topos.

%% Zariski topos higher logic

%% Definition of $\bP^n$ as a set of lines in $R^{n+1}$ coincides with the definition
%% of projective as functor of points (Demazure? Eisenbud?)

%% ``Geometric'' definition

%% Meyers, Blechschmidt use of type theory with univalence

%% Axiomatisation of the Zariski (higher) topos

%% A scheme is defined as a type satisfying some property and a map of schemes is {\em any} function
%% between the corresponding types


