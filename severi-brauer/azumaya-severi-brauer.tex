From now on we assume a lex modality $T$ such that:
\begin{itemize}
\item Schemes are modal.
\item The type of finite free modules is modal.
\end{itemize}
We call $T$-modal types sheaves and we write $\propTrunc{X}_T$ the sheafification of the propositional truncation of $X$.

In \Cref{fppf-sheaves} we constructed such a modality (\Cref{scheme-is-fppf-sheaf} and \Cref{descent-finite-free}).

We fix a natural number $n$ throughout.


\subsection{The type $\AZ_n$ of Azumaya algebras}

\begin{definition}
An Azumaya algebra of rank $n$ is a (non-commutative, unital) $R$-algebra $A$ such that its underlying type is a sheaf and:
\[\propTrunc{A=M_{n+1}(R)}_T\]
\end{definition}

We write $\AZ_n$ for the type of Azumaya algebra of rank $n$.

\begin{lemma}\label{azumayas-are-finite-free}
For all $A:\AZ_n$ we have that $A$ is finite free as a module.
\end{lemma}

\begin{proof}
By hypothesis $A$ being finite free is modal so that $\propTrunc{A=M_{n+1}(R)}_T$ implies $A$ finite free.
\end{proof}

\begin{definition}
Let $V$ be a free $R$-module, we define $\Gr_k(V)$ the $k$-Grassmannian of $V$ as the type of $k$-dimensional subspaces of $V$.
\end{definition}

\begin{lemma}\label{grassmanians-are-schemes}
Let $V$ be a finite free module, then $\Gr_k(V)$ is a scheme.
\end{lemma}

\begin{proof}
We can assume $V=R^n$. The type of $k$-dimensional subspaces of $R^n$ is the type of $n\times k$ matrices of rank $k$ quotiented by the natural action of $\GL_k$. For all $k\times k$ minor, we consider the open proposition stating this minor is non-zero, which well defined as it is invariant under the $\GL_k$-action. This gives a finite open cover of $\Gr_k(R^n)$.

Let us show any piece is affine. For example consider the piece of matrices of the form:
\[\begin{pmatrix}
P & N
\end{pmatrix}\]
where $P$ is invertible of size $k\times k$. Any orbit in this piece has a unique element of the form:
\[\begin{pmatrix}
I_k & N'
\end{pmatrix}\]
where $I_k$ is the identity matrix, so this piece is equivalent to $R^{(n-k)k}$.
\end{proof}

\begin{lemma}\label{being-ideal-in-azumaya-closed}
For all $A:\AZ_n$ and $I:\Gr_{n+1}(A)$, we have that $I$ being a right ideal in $A$ is a closed proposition.
\end{lemma}

\begin{proof}
By \Cref{azumayas-are-finite-free} we have that $A$ is finite free as a module. Consider $a_0,\cdots,a_n$ a basis of $I$ and extend it to a basis of $A$ using $b_1,\cdots,b_l$. 

For any $a:A$, we have that $a\in I$ is a closed as it says that the $b_1,\cdots,b_l$ coordinates of $a$ are zero. 

Then $I$ is an ideal if and only if for any $a$ in the chosen basis of $A$ and any $i$ in the chosen basis of $I$ we have that $ai\in I$, which is a closed proposition.
\end{proof}

\begin{lemma}\label{severi-brauer-are-schemes}
For all $A:\AZ_n$ we define:
\[\RI(A) := \{I:\Gr_{n+1}(A)\ |\ I\ \mathrm{is\ a\ right\ ideal}\}\]
Then $\RI(A)$ is a scheme.
\end{lemma}

\begin{proof}
By \Cref{azumayas-are-finite-free} we have that $A$ is finite free as a module, so that by \Cref{grassmanians-are-schemes} we have that $\Gr_{n+1}(A)$ is a scheme, and then by \Cref{being-ideal-in-azumaya-closed} we have that $\RI(A)$ is closed in a scheme, so it is a scheme.
\end{proof}


\subsection{A remark on Azumaya algebras}

\begin{lemma}\label{MnR-endomorphism-multiplication}
For any $n:\N$, the map $M_{n+1}(R)\otimes M_{n+1}(R)^{op}\to \mathrm{End}_R(M_{n+1}(R))$ sending $M\otimes N$ to $P\mapsto MPN$ is an equivalence.
\end{lemma}

\begin{proof}
Both algebras have the same dimension so we just need to check surjectivity. TODO
\end{proof}

\begin{lemma}
Assume $A:\AZ_n$, then $A$ is finite free as a module and the map $A\otimes A^{op}\to \mathrm{End}_R(A)$ sending $a\otimes b$ to $c\mapsto acb$ is an equivalence.
\end{lemma}

\begin{proof}
The fact that $A$ is finite free is \Cref{azumayas-are-finite-free}. Then both $A\otimes A^{op}$ and $\mathrm{End}_R(A)$ are finite free modules and therefore are $T$-modal, so that the map being an equivalence is $T$-modal and when proving it we can assume $A=M_{n+1}(R)$. Then we conclude by \Cref{MnR-endomorphism-multiplication}.
\end{proof}

\begin{remark}\label{azumaya-independent-modality}
We expect a converse in the sense that if $A$ is an algebra which is finite free as a module, such that the map $A\otimes A^{op} \to \mathrm{End}_R(A)$ is an equivalence, we should get $\propTrunc{A=M_{n+1}(R)}_{\mathrm{Et}}$. This would mean that given any modality $T$ such that:
\begin{itemize}
\item Schemes are $T$-modal.
\item The type of finite free modules is $T$-modal.
\item $T$-modal types are étale sheaves.
\end{itemize}
We have that an algebra $A$ is an Azumaya algebra for $T$ if and only if $A$ finite free as a module and the map $A\otimes A^{op} \to \mathrm{End}_R(A)$ is an equivalence. In particular Azumaya algebras do not depend on the choice of such a $T$.
\end{remark}


\subsection{The type $\SB_n$ of Severi-Brauer varieties}

\begin{definition}
A type $X$ is called a Severi-Brauer variety of dimension $n$ if $X$ is a sheaf and:
\[\propTrunc{X=\bP^n}_T\]
\end{definition}

We write $\SB_n$ the type of Severi-Brauer varieties of dimension $n$.

\begin{lemma}\label{right-ideal-of-matrices-are-projective}
Consider the map:
\[\delta:\bP^n \to \RI(M_{n+1}(R))\]
sending $(x_0:\cdots:x_n):\bP^n$ to:
\[\{M:M_n(R)\ |\ \forall i,j.\ x_i\cdot M_j = x_j\cdot M_i\}\]
where $M_i$ is the $i$-th line of $M$. Then $\delta$ is an equivalence.
\end{lemma}

\begin{proof}
Write $X=(x_0:\cdots:x_n)$. First we check $\delta$ is well defined. It is clear that for all $\lambda\not=0$ we have that:
\[\delta(\lambda X) = \delta(X)\]
and that $\delta(X)$ is a right ideal. To check the dimension assume $x_k\not=0$. Then $M\in\delta(X)$ if and only if for all $i$ we have that $M_i = \frac{x_i}{x_k} M_k$, which means giving $M\in\delta(X)$ is equivalent to giving $M_k$ in $R^{n+1}$, so $\delta(X)$ is free of dimension $n+1$.

Next we check injectivity. Assume given $(x_0:\cdots:x_n)$ and $(y_0:\cdots:y_n)$ in $\bP^n$ such that for all $M:M_n(R)$ we have:
\[(\forall i,j.\ x_i\cdot M_j = x_j\cdot M_i) \leftrightarrow (\forall i,j.\ y_i\cdot M_j = y_j\cdot M_i)\]
In particular considering the matrix $N$ such that $N_j = (y_j,\cdots,y_j)$ we get that:
\[\forall i,j.\ x_iy_j=x_jy_i\] 
so that:
\[(x_0:\cdots:x_n) = (y_0:\cdots:y_n)\]

Finally we check surjectivity. Assume $I:\RI(M_{n+1}(R))$, since $\propTrunc{I=R^{n+1}}$ we have $M\in I$ such that $M\not=0$, for example assume $M_{0,0}\not=0$. Then for all $k$ we have that:
\[ME_{0,k}\in I\]
meaning that we have $N^k\in I$ where:
\[N^k_{i,k} = M_{i,0}\]
and when when $j\not=k$
\[N^k_{i,j} = 0\]
Then since $M_{0,0}\not=0$ the matrices $N^k$ are linearly independent and since $I$ has dimension $n+1$, it is precisely the ideal spanned by the $N^k$. But this ideal is $\delta(M_{0,0}:\cdots:M_{n,0})$.
\end{proof}

\begin{lemma}
If $A$ is an Azumaya algebra, then $\RI(A)$ is a Severi-Brauer variety.
\end{lemma}

\begin{proof}
By \Cref{severi-brauer-are-schemes} and the assumption that schemes are sheaves, we have that $\RI(A)$ is a sheaf. Then to prove:
\[\propTrunc{A=M_{n+1}(R)}_T \to \propTrunc{\RI(A)=\bP^n}_T\]
it is enough to prove:
\[\RI(M_{n+1}(R)) = \bP^n\]
which is \Cref{right-ideal-of-matrices-are-projective}.
\end{proof}