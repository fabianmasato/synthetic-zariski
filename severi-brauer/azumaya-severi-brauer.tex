From now on we assume a lex modality $T$ such that:
\begin{itemize}
\item Schemes are modal.
\item The type of finite free modules is modal.
\end{itemize}
We call $T$-modal types sheaves and we write $\propTrunc{X}_T$ the sheafification of the propositional truncation of $X$.

In \Cref{fppf-sheaves} we constructed such a modality (\Cref{scheme-is-fppf-sheaf} and \Cref{descent-finite-free}).

We fix a natural number $n$ throughout.


\subsection{The type $\AZ_n$ of Azumaya algebras}

\begin{definition}
An Azumaya algebra of rank $n$ is a (non-commutative, unital) $R$-algebra $A$ such that its underlying type is a sheaf and:
\[\propTrunc{A=M_{n+1}(R)}_T\]
\end{definition}

We write $\AZ_n$ for the type of Azumaya algebra of rank $n$.

\begin{lemma}\label{azumayas-are-finite-free}
For all $A:\AZ_n$ we have that $A$ is finite free as a module.
\end{lemma}

\begin{proof}
By hypothesis $A$ being finite free is modal so that $\propTrunc{A=M_{n+1}(R)}_T$ implies $A$ finite free.
\end{proof}

\begin{definition}
Let $V$ be a free $R$-module, we define $\Gr_k(V)$ the $k$-Grassmannian of $V$ as the type of $k$-dimensional subspaces of $V$.
\end{definition}

\begin{lemma}\label{grassmanians-are-schemes}
Let $V$ be a finite free module, then $\Gr_k(V)$ is a scheme.
\end{lemma}

\begin{proof}
We can assume $V=R^n$. The type of $k$-dimensional subspaces of $R^n$ is the type of $n\times k$ matrices of rank $k$ quotiented by the natural action of $\GL_k$. For all $k\times k$ minor, we consider the open proposition stating this minor is non-zero, which well defined as it is invariant under the $\GL_k$-action. This gives a finite open cover of $\Gr_k(R^n)$.

Let us show any piece is affine. For example consider the piece of matrices of the form:
\[\begin{pmatrix}
P & N
\end{pmatrix}\]
where $P$ is invertible of size $k\times k$. Any orbit in this piece has a unique element of the form:
\[\begin{pmatrix}
I_k & N'
\end{pmatrix}\]
where $I_k$ is the identity matrix, so this piece is equivalent to $R^{(n-k)k}$.
\end{proof}

\begin{lemma}\label{being-ideal-in-azumaya-closed}
For all $A:\AZ_n$ and $I:\Gr_{n+1}(A)$, we have that $I$ being a right ideal in $A$ is a closed proposition.
\end{lemma}

\begin{proof}
By \Cref{azumayas-are-finite-free} we have that $A$ is finite free as a module. Consider $a_0,\cdots,a_n$ a basis of $I$ and extend it to a basis of $A$ using $b_1,\cdots,b_l$. 

For any $a:A$, we have that $a\in I$ is a closed as it says that the $b_1,\cdots,b_l$ coordinates of $a$ are zero. 

Then $I$ is an ideal if and only if for any $a$ in the chosen basis of $A$ and any $i$ in the chosen basis of $I$ we have that $ai\in I$, which is a closed proposition.
\end{proof}

\begin{lemma}\label{severi-brauer-are-schemes}
For all $A:\AZ_n$ we define:
\[\RI(A) := \{I:\Gr_{n+1}(A)\ |\ I\ \mathrm{is\ a\ right\ ideal}\}\]
Then $\RI(A)$ is a scheme.
\end{lemma}

\begin{proof}
By \Cref{azumayas-are-finite-free} we have that $A$ is finite free as a module, so that by \Cref{grassmanians-are-schemes} we have that $\Gr_{n+1}(A)$ is a scheme, and then by \Cref{being-ideal-in-azumaya-closed} we have that $\RI(A)$ is closed in a scheme, so it is a scheme.
\end{proof}


\subsection{Quaternion algebras are Azumaya algebras}

We assume $2\not=0$.

\begin{definition}
Given $a,b:R^\times$, we define the quaternion algebra $Q(a,b)$ as the non-commutative algebra:
\[R[i,j]/(i^2=a,j^2=b,ij=-ji)\] 
\end{definition}

\begin{remark}
As a vector space, $Q(a,b)$ is of dimension $4$, generated by $1,i,j,ij$.
\end{remark}

\begin{remark}
By the change of variable $i\mapsto j$ and $j\mapsto i$ we get $Q(a,b) = Q(b,a)$.
\end{remark}

\begin{lemma}\label{quaternion-split}
For all $b:R^\times$, we have that $Q(1,b) = M_2(R)$.
\end{lemma}

\begin{proof}
We send $i$ to:
\[I = \begin{pmatrix}
1 & 0\\
0 & -1\\
\end{pmatrix}\]
and $j$ to:
\[J = \begin{pmatrix}
0 & b\\
1 & 0\\
\end{pmatrix}\]
Then $IJ$ is:
\[K = \begin{pmatrix}
0 & b\\
-1 & 0\\
\end{pmatrix}\]
It is easy to check this define an algebra morphism, and since $1,I,J,K$ form a basis of $M_2(R)$ the map is an isomorphism.
\end{proof}

\begin{lemma}\label{quaternion-change-variable}
For all $a,b,u,v:R^\times$, we have that $Q(a,b) = Q(u^2a,v^2b)$.
\end{lemma}

\begin{proof}
We use the variable change $i\mapsto ui$ and $j\mapsto vj$.
\end{proof}

\begin{lemma}
Given $a,b:R^\times$, we have that $Q(a,b)$ is an Azumaya algebra. 
\end{lemma}

\begin{proof}
We have that $Q(a,b)$ is finite free as a vector space so it is a sheaf. So $Q(a,b)$ being Azumaya is a sheaf and we can assume $\sqrt{a}$. Then by \Cref{quaternion-change-variable} we have $Q(1,b) = Q(\sqrt{a}^2,b) = Q(a,b)$ and we conclude by \Cref{quaternion-split}.
\end{proof}


\subsection{A remark on Azumaya algebras}

\begin{lemma}\label{MnR-endomorphism-multiplication}
For any $n:\N$, the map:
\[M_{n+1}(R)\otimes M_{n+1}(R)^{op}\to \mathrm{End}_R(M_{n+1}(R))\]
\[M\otimes N\mapsto (P\mapsto MPN)\]
is an equivalence.
\end{lemma}

\begin{proof}
Let us denote by $(E_{i,j})_{0\leq i,j\leq n}$ the canonical basis of $M_{n+1}(R)$. We consider the basis: 
\[(E_{i,j}\otimes E_{k,l})_{0\leq i,j,k,l\leq n}\] 
of $M_{n+1}(R)\otimes M_{n+1}(R)^{op}$, as well as the basis:
\[(C_{i,j,k,l})_{0\leq i,j,k,l\leq n}\] 
of $\mathrm{End}_R(M_{n+1}(R))$ where $C_{i,j,k,l}(E_{j,k}) = E_{i,l}$ and $C_{i,j,k,l}$ is null on other element of the basis. It is clear that the morphism sends one basis to the other, and that both algebras have the same multiplication table. 
\end{proof}

\begin{lemma}
Assume $A:\AZ_n$, then $A$ is finite free as a module and the map $A\otimes A^{op}\to \mathrm{End}_R(A)$ sending $a\otimes b$ to $c\mapsto acb$ is an equivalence.
\end{lemma}

\begin{proof}
The fact that $A$ is finite free is \Cref{azumayas-are-finite-free}. Then both $A\otimes A^{op}$ and $\mathrm{End}_R(A)$ are finite free modules and therefore are $T$-modal, so that the map being an equivalence is $T$-modal and when proving it we can assume $A=M_{n+1}(R)$. Then we conclude by \Cref{MnR-endomorphism-multiplication}.
\end{proof}

\begin{remark}\label{azumaya-independent-modality}
We expect a converse in the sense that if $A$ is an algebra which is finite free as a module, such that the map $A\otimes A^{op} \to \mathrm{End}_R(A)$ is an equivalence, we should get $\propTrunc{A=M_{n+1}(R)}_{\mathrm{Et}}$. This would mean that given any modality $T$ such that:
\begin{itemize}
\item Schemes are $T$-modal.
\item The type of finite free modules is $T$-modal.
\item $T$-modal types are étale sheaves.
\end{itemize}
We have that an algebra $A$ is an Azumaya algebra for $T$ if and only if $A$ finite free as a module and the map $A\otimes A^{op} \to \mathrm{End}_R(A)$ is an equivalence. In particular Azumaya algebras do not depend on the choice of such a $T$.
\end{remark}


\subsection{The type $\SB_n$ of Severi-Brauer varieties}

\begin{definition}
A type $X$ is called a Severi-Brauer variety of dimension $n$ if $X$ is a sheaf and:
\[\propTrunc{X=\bP^n}_T\]
\end{definition}

We write $\SB_n$ the type of Severi-Brauer varieties of dimension $n$.

\begin{lemma}\label{right-ideal-of-matrices-are-projective}
Consider the map:
\[\delta:\bP^n \to \RI(M_{n+1}(R))\]
sending $(x_0:\cdots:x_n):\bP^n$ to:
\[\{M:M_n(R)\ |\ \forall i,j.\ x_i\cdot M_j = x_j\cdot M_i\}\]
where $M_i$ is the $i$-th line of $M$. Then $\delta$ is an equivalence.
\end{lemma}

\begin{proof}
Write $X=(x_0:\cdots:x_n)$. First we check $\delta$ is well defined. It is clear that for all $\lambda\not=0$ we have that:
\[\delta(\lambda X) = \delta(X)\]
and that $\delta(X)$ is a right ideal. To check the dimension assume $x_k\not=0$. Then $M\in\delta(X)$ if and only if for all $i$ we have that $M_i = \frac{x_i}{x_k} M_k$, which means giving $M\in\delta(X)$ is equivalent to giving $M_k$ in $R^{n+1}$, so $\delta(X)$ is free of dimension $n+1$.

Next we check injectivity. Assume given $(x_0:\cdots:x_n)$ and $(y_0:\cdots:y_n)$ in $\bP^n$ such that for all $M:M_n(R)$ we have:
\[(\forall i,j.\ x_i\cdot M_j = x_j\cdot M_i) \leftrightarrow (\forall i,j.\ y_i\cdot M_j = y_j\cdot M_i)\]
In particular considering the matrix $N$ such that $N_j = (y_j,\cdots,y_j)$ we get that:
\[\forall i,j.\ x_iy_j=x_jy_i\] 
so that:
\[(x_0:\cdots:x_n) = (y_0:\cdots:y_n)\]

Finally we check surjectivity. Assume $I:\RI(M_{n+1}(R))$, since $\propTrunc{I=R^{n+1}}$ we have $M\in I$ such that $M\not=0$, for example assume $M_{0,0}\not=0$. Then for all $k$ we have that:
\[ME_{0,k}\in I\]
meaning that we have $N^k\in I$ where:
\[N^k_{i,k} = M_{i,0}\]
and when when $j\not=k$
\[N^k_{i,j} = 0\]
Then since $M_{0,0}\not=0$ the matrices $N^k$ are linearly independent and since $I$ has dimension $n+1$, it is precisely the ideal spanned by the $N^k$. But this ideal is $\delta(M_{0,0}:\cdots:M_{n,0})$.
\end{proof}

\begin{lemma}
If $A$ is an Azumaya algebra, then $\RI(A)$ is a Severi-Brauer variety.
\end{lemma}

\begin{proof}
By \Cref{severi-brauer-are-schemes} and the assumption that schemes are sheaves, we have that $\RI(A)$ is a sheaf. Then to prove:
\[\propTrunc{A=M_{n+1}(R)}_T \to \propTrunc{\RI(A)=\bP^n}_T\]
it is enough to prove:
\[\RI(M_{n+1}(R)) = \bP^n\]
which is \Cref{right-ideal-of-matrices-are-projective}.
\end{proof}


\subsection{Conics are Severi-Brauer varieties}

We assume $2\not=0$.

\begin{definition}
Given $a,b:R^\times$, we define the conic$C(a,b)$ as the set of $[x:y:z]:\bP^2$ such that $x^2=ay^2+bz^2$.
\end{definition}

\begin{lemma}\label{conic-one-split}
Assume given $b:R^\times$, then $C(1,b) = \bP^1$.
\end{lemma}

\begin{proof}
Consider the map:
\[\psi:\bP^1\to \bP^2\]
\[ [m:n]\mapsto \left[m^2+bn^2 : m^2-bn^2 : 2mn\right] \]
We now check it is an isomorphism.
\begin{itemize}

\item The ideal $I = (m^2+bn^2, m^2-bn^2, 2mn)$ contains $m^2$ and $n^2$, therefore if $(m,n)=1$ then $I=1$. So the map is well defined.

\item The map $\psi$ takes value in $C(1,b)$. By direct computations, we check that for all $m,n:R$ we have that:
\[(m^2+bn^2)^2 = (m^2-bn^2)^2 + b(2mn)^2\]

\item The map $\psi$ is injective. Indeed assume $[m:n]$ and $[\bar{m}:\bar{n}]$ in $\bP^1$ such that $\psi([m:n])=\psi([\bar{m}:\bar{n}])$, we have that $m\bar{n}=\bar{m}n$. By linear combinations we get that:
\begin{eqnarray}
m^2\bar{m}\bar{n} &=& \bar{m}mn\\
\bar{m}\bar{n}n^2 &=& mn\bar{n}^2\\
\bar{m}^2n^2 &=& m^2\bar{n}^2
\end{eqnarray}
If $m$ and $\bar{m}$ are invertible we conclude by (1). If $n$ and $\bar{n}$ are invertible we conclude by (2). If $\bar{m}$ and $n$ (resp. $m$ and $\bar{n}$) are invertible by (3) we have that $m$ and $\bar{n}$ (resp. $\bar{m}$ and $n$) are invertible, and we conclude as before.

\item The map $\psi$ is surjective. Assume $[x:y:z]:C(a,b)$. We know that $x$, $y$ or $z$ is invertible, and since $x^2=y^2+bz^2$, we have that if $z$ is invertible then $x$ or $y$ is invertible. Since $x$ and $y$ belong to $(x+y, x-y)$, we know one of the two is invertible, say $x+y$. 

Since $\psi$ is an embedding of schemes, its fibers being merely inhabited is an fppf sheaf by \Cref{scheme-is-fppf-sheaf}. So we can define $m=\sqrt{\frac{x+y}{2}}$. Then we can define $n=\frac{z}{2m}$ and check that $n^2 = \frac{x-y}{2b}$ and that $\psi([m:n]) = [x:y:z]$. 

If $x-y$ is invertible, we proceed similarly using $n=\sqrt{\frac{x-y}{2b}}$ and $m=\frac{z}{2n}$.
\end{itemize}
\end{proof}

\begin{lemma}\label{conic-change-variable}
Given $a,b,u,v:R^\times$ we have that $C(a,b) = C(u^2a,v^2b)$.
\end{lemma}

\begin{proof}
Consider the change of variable $y\mapsto uy$ and $z\mapsto vz$.
\end{proof}

\begin{lemma}
Assume given $a,b:R^\times$, then $C(a,b)$ is a Severi-Brauer variety.
\end{lemma}

\begin{proof}
Since a sheaf being a Severi-Brauer variety is an fppf sheaf, we can assume $\sqrt{a}$ . Then by \Cref{conic-change-variable} we have that $C(1,b) = C(\sqrt{a}^2,b)=C(a,b)$ and we conclude by \Cref{conic-one-split}.
\end{proof}

TODO $RI(Q(a,b)) = C(a,b)$

\begin{lemma}
Assume $a,b:R^\times$ such that $\propTrunc{C(a,b)}$, then $\propTrunc{C(a,b)=\bP^1}$.
\end{lemma}

\begin{proof}
TODO
\end{proof}

\begin{remark}
We will see in \Cref{chatelet-theorem} that this holds for any Severi-Brauer variety, i.e. any merely inhabited Severi-Brauer variety is a projective space. 
\end{remark}