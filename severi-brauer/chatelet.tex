\subsection{Generalities on delooping in $T$-sheaves}

\begin{definition}
A type $A$ is $T$-connected if:
\[\forall(x,y:A).\ \propTrunc{x=y}_T\]
\end{definition}

The key intuition for the next lemma is that both $A$ and $B$ are deloopings of the same group in the topos of sheaves.

\begin{lemma}\label{deloopings-equivalence}
Assume $A,B$ pointed $T$-connected sheaves. Let $f:A\to B$ be a pointed map inducing an equivalence:
\[\Omega f : \Omega A \simeq \Omega B\]
Then $f$ is an equivalence.
\end{lemma}

\begin{proof}
First we prove that $f$ is an embedding. We have to prove that for all $x,y:A$ the map:
\[\mathrm{ap}_f : x=y \to f(x)=f(y)\]
is an equivalence. Since $A$ and $B$ are sheaves so are their identity types, so $\mathrm{ap}_f$ being an equivalence is a sheaf, so by $T$-connectedness of $A$ we can assume $x$ and $y$ are the basepoint, in which case it is part of the hypothesis.

Now we prove it is surjective, indeed for any $x:B$ we have that $\fib_f(x)$ is a sheaf and a proposition so when proving it is inhabited we can assume $x$ is the basepoint of $B$ and give the basepoint of $A$ as antecedent.
\end{proof}


\subsection{Both $\Aut(M_{n+1}(R))$ and $\Aut(\bP^n)$ are $\PGL_{n+1}(R)$}

\begin{lemma}\label{finite-projective-free}
Let $M$ be a finite projective module, then $M$ is finite free.
\end{lemma}

\begin{proof}
\cite{TODO}, crucially relies on $R$ being local.
\end{proof}

\begin{lemma}\label{fundamental-system-matrices}
Assume $e_{i,j}:M_{n+1}(R)$ for $0\leq i,j\leq n$ such that:
\[e_{i,j}e_{k,l} = \delta_{j,k}e_{i,l}\]
where $\delta_{j,k} = 1$ if $j=k$ and $0$ otherwise. Moreover assume:
\[e_{0,0}+\cdots+e_{n,n}=1\]
Then there exists $P:GL_{n+1}(R)$ such that:
\[e_{i,j} = PE_{i,j}P^{-1}\]
\end{lemma}

\begin{proof}
We define $e_i = e_{i,i}$, then $e_0+\cdots+e_n = 1$, for all $i$ we have $e_i^2=e_i$ and for all $i\not=j$ we have that $e_ie_j=0$. From this we get:
\[R^{n+1} = V_0\oplus\cdots\oplus V_n\]
where:
\[V_i = \{x\ |\ e_i(x)=x\}\]
and:
\[e_{i,j}:V_j\simeq V_i\]

As a direct summand of a free module we have that $V_0$ is projective, and since $V_0 = e_{0}(R^{n+1})$ we have that $V_0$ is finitely generated, so by \Cref{finite-projective-free} it is finite free. From $V_0^{n+1}=R^{n+1}$ we get that $\propTrunc{V_0=R}$, and therefore that $\propTrunc{V_i=R}$ for all $i$.

Then we choose $v_0$ generating $V_0$ and define $v_i = e_{i,0}(v_0)$ so that $v_i$ generates $V_i$ because $e_{i,0}:V_0\simeq V_i$. We get a basis $v_0,\cdots,v_n$ of $R^{n+1}$.

Let $u_0,\cdots,u_n$ be the canonical basis of $R^{n+1}$ and define $P:GL_{n+1}(R)$ by sending $u_i$ to $v_i$. Then for all $v_k$ we have that:
\[e_{i,j}v_k = PE_{i,j}P^{-1}v_k\]
so we can conclude.
\end{proof}

\begin{proposition}\label{Aut-MnR-PGL}
The map:
\[\alpha:\PGL_{n+1}(R)\to\Aut(M_{n+1}(R))\]
\[P\mapsto (M\mapsto PMP^{-1})\]
is an equivalence.
\end{proposition}

\begin{proof}
It is clearly a group morphism. 

For injectivity we just need to check that if for all $M:M_{n+1}(R)$ we have $PMP^{-1}=M$ then there exists $\lambda\not=0$ such that $P=\lambda I_{n+1}$. This is a well-known linear algebra exercise.

For surjectivity consider $e_{i,j}=\sigma(E_{i,j})$, we can apply \Cref{fundamental-system-matrices} to get $P:GL_{n+1}(R)$ such that:
\[\sigma(E_{i,j}) = PE_{i,j}P^{-1}\]
from which we conclude that for all $M:M_{n+1}(R)$ we have that:
\[\sigma(M) = PMP^{-1}\]
as desired.
\end{proof}

\begin{proposition}\label{Aut-Pn-PGL}
The map:
\[\beta:\PGL_{n+1}(R)\to\Aut(\bP^n)\]
\[X\mapsto PX\]
is an equivalence.
\end{proposition}

\begin{proof}
\cite{TODO}
\end{proof}


\subsection{The Severi-Brauer construction is an equivalence}

\begin{proposition}\label{right-ideal-is-equivalence}
The map:
\[\RI:\AZ_n\to\SB_n\]
is an equivalence.
\end{proposition}

\begin{proof}
By \Cref{deloopings-equivalence} it is enough to prove that the top map in the triangle:
\begin{center}
\begin{tikzcd}
\Aut(M_n(R))\ar[rr,"\Omega\RI"] & & \Aut(\bP^n) \\
& \PGL_{n+1}(R)\ar[ru,swap,"\beta"]\ar[lu,"\alpha"] & \\
\end{tikzcd}
\end{center}
is an equivalence. But since the two other maps in the triangle are equivalences by \Cref{Aut-MnR-PGL} and \Cref{Aut-Pn-PGL}, it is enough to prove that the triangle commutes. To do this we need to check that for all $P:\PGL_{n+1}(R)$ we have that:
\[\delta^{-1}\circ \mathrm{ap}_\RI(\alpha(P))\circ\delta = \beta(P)\]
in $\Aut(\bP^n)$, with $\delta$ defined in \Cref{right-ideal-of-matrices-are-projective}. So we need to prove the following square commutes:
\begin{center}
\begin{tikzcd}
\RI(M_{n+1}(R))\ar[rr,"I\mapsto PIP^{-1}"]&& \RI(M_{n+1}(R)) \\
\bP^n\ar[u,"\delta"]\ar[rr,swap,"X\mapsto PX"]&& \bP^n\ar[u,swap,"\delta"] 
\end{tikzcd}
\end{center}
where $\mathrm{ap}_\RI$ was computed using path induction.

We see that:
\[\delta(Y) = \{M:M_n(R)\ |\ \forall A,B:R^{n+1}.\ A^tX\cdot B^tM = B^tX\cdot A^tM\}\]
To check that:
\[P\delta(X)P^{-1} = \delta(PX)\]
we just need to check an inclusion as both are finite free modules of the same dimension. Assume $M\in\delta(X)$, to check that $PMP^{-1}\in\delta(PX)$ we need to check that for all $A,B:R^{n+1}$ we have that:
\[A^tPX\cdot B^tPMP^{-1} = B^tX\cdot A^tPMP^{-1}\]
but since $M\in\delta(X)$ we have that:
\[(P^tA)^tX\cdot (P^tB)^tM = (P^tB)^tX\cdot (P^tA)^tM\]
which gives us what we want.
\end{proof}

\begin{remark}
By \Cref{severi-brauer-are-schemes} and \Cref{right-ideal-is-equivalence} we can conclude than any Severi-Brauer variety is a scheme. This is not clear because being a scheme is not modal, i.e. we do not have fppf descent for schemes.
\end{remark}

\begin{remark}\label{severi-brauer-independent-modality}
By \Cref{right-ideal-is-equivalence} and \Cref{azumaya-independent-modality} and we can conclude that a type $X$ being a Severi-Brauer variety for any modality $T$ such that:
\begin{itemize}
\item Schemes are $T$-modal.
\item The type of finite free modules is $T$-modal.
\item $T$-modal types are étale sheaves.
\end{itemize}
is equivalent to $X$ being a Severi-Brauer variety for the étale topology. In particular Severi-Brauer varieties do not depend on the choice of such a $T$.
\end{remark}


\subsection{Ch\^atelet's Theorem}

\begin{lemma}\label{azumaya-with-right-ideal}
Assume $A:\AZ_n$ with $I:\RI(A)$, then:
\[A = \mathrm{End}_R(I)^{op}\]
\end{lemma}

\begin{proof}
Since $I$ is an ideal, there is a canonical map of algebra:
\[A \to\mathrm{End}_R(I)^{op}\]
Since both algebras are sheaves (indeed $\propTrunc{I=R^{n+1}}$ implies $I$ is a sheaf), this map being an equivalence is a sheaf so we can assume $A=M_{n+1}(R)$.

By \Cref{right-ideal-of-matrices-are-projective} we can assume $X=(x_0:\cdots:x_n):\bP^n$ such that $I=\delta(X)$. There is a $k$ such that $x_k\not=0$, so we can assume $x_k=1$, then we have an isomorphism:
\[\theta:R^{n+1}\to I\]
sending $Y:R^{n+1}$ to the matrix $M$ with its $i$-th line $M_i=x_iY$. Then for all $M:M_n(R)$ we have a commutative square:
\begin{center}
\begin{tikzcd}
I\ar[rr,"N\mapsto NM"] && I \\
R^{n+1}\ar[u,"\theta"]\ar[rr,swap,"X\mapsto M^tX"] && R^{n+1}\ar[u,swap,"\theta"]\\
\end{tikzcd}
\end{center}
meaning the natural map:
\[ M_{n+1}(R)\to \mathrm{End}_R(I)^{op}\]
sends $M$ to $\delta^{-1}\circ M\circ\delta$, so it is an equivalence.
\end{proof}

\begin{theorem}[Ch\^atelet's Theorem]\label{chatelet-theorem}
Assume $X:\SB_n$, then:
\[\propTrunc{X}\to\propTrunc{X=\bP^n}\]
\end{theorem}

\begin{proof}
By \Cref{right-ideal-is-equivalence} we can assume $X=\RI(A)$ for some $A:\AZ_n$. Then we can assume $I:\RI(A)$, so that by \Cref{azumaya-with-right-ideal} we have that:
\[A=\mathrm{End}_R(I)^{op}\]
Since we merely have that $I=R^{n+1}$, we merely have:
\[A = M_{n+1}(R)^{op} = M_{n+1}(R)\]
Applying \Cref{right-ideal-of-matrices-are-projective} we merely conclude:
\[X=\RI(A)=\RI(M_{n+1}(R)) = \bP^n\]
\end{proof}