% latexmk -pdf -pvc main.tex
% latexmk -pdf -pvc -interaction=nonstopmode main.tex
\documentclass{../util/zariski}
\newcommand{\SB}{\mathrm{SB}}
\newcommand{\RI}{\mathrm{RI}}
\newcommand{\AZ}{\mathrm{AZ}}

\title{Severi-Brauer Varieties}

\begin{document}

\maketitle

{F}ran\c cois {C}h\^atelet introduced the notion of Severi-Brauer variety in his 1944 PhD thesis
\cite{chatelet44} for a generalisation of a result of Poincar\'e about rational curve over a field.
He defines a Severi-Brauer variety to be a variety which becomes isomorphic to some $\bP^n$ after
a separable extension. After recalling the characterisation of a central simple algebra over a field $k$, as
an algebra which becomes isomorphic to a matrix algebra $M_n(k)$ after a separable extension, he notices the fundamental
fact that, $\bP^{n+1}(k)$ and $M_n(k)$ have the same automorphism group $PGL_{n}(k)$. He uses then this
fact to describe a correspondence between Severi-Brauer variety and central simple algebra, and as a corollary
obtains the following generalisation of Poincar\'e's result: a Severi-Brauer variety which has a rational point
is isomorphic to some $\bP^{n+1}(k)$. This result and its proof are described in Serre's book on local fields \cite{serre62}.
(The paper \cite{colliot88} contains also a description of this result.) The notion of central simple algebra over a field
has been generalised to the notion of Azumaya algebra  \cite{azumaya51}, and
Grothendieck \cite{grothendieck68} generalized the notion of Severi-Brauer over over an arbitrary comuutative ring.
The goal of this note is to present a formulation and proof of Ch\^atelet's Theorem over an arbitrary commutative ring
in the setting of synthetic algebraic geometry \cite{draft}, using the results already proved about projective
space \cite{sag-projective} in this context, in particular the fact that any automorphism of the projective space is given
by a homography. We make essential use of basic results about dependent type theory with univalence \cite{hott}
and modalities \cite{modalities}, in particular the fact that, in this context, fppf and \'etale topology can be described
as modalities. The formulation of Ch\^atelet's Theorem becomes then
\[\propTrunc{X=\bP^n}_{T} \to \propTrunc{X} \to \propTrunc{X=\bP^n}\]
where:
\[\propTrunc{X=\bP^n}_{T}\]
is the localisation of $\propTrunc{X=\bP^n}$ for a modality $T$ satisfying some basic properties (valid for the modality
corresponding to \'etale topology).

\tableofcontents

\section{Fppf sheaves}
\label{fppf-sheaves}

\subsection{Affine schemes are fppf sheaves}

\rednote{How to name this topology? Perhaps we should use the étale topology?}

\begin{definition}
A type $X$ is called an fppf sheaf if for all $g:R[X]$ monic we have that $X$ is $\propTrunc{\Spec(R[X]/g)}$-local.
\end{definition}

This means that being an fppf sheaf is a lex modality, as it is localisation at a family of propositions.

\begin{lemma}\label{fppf-subcanonical}
The type $R$ is an fppf sheaf.
\end{lemma}

\begin{proof}
Let $g:R[X]$ be monic and write $S=\Spec(R[X]/g)$. We have a coequaliser in sets:
\[S\times S\rightrightarrows S \to \propTrunc{S}\]
So since $R$ is a set we have an equaliser diagram:
\[R^{\propTrunc{S}} \to R^S\rightrightarrows R^{S\times S}\]
so that it is enough to prove that $R$ is the equaliser of:
\[R[X]/g \rightrightarrows R[X]/g \otimes R[X]/g\]
to conclude. But since $g$ is monic we merely have:
\[R[X]/g \simeq R^n\]
and it is clear that $R$ is the equaliser of:
\[R^n \rightrightarrows R^n\otimes R^n\]
\end{proof}

\begin{remark}\label{R-modal-subcanonical}
If $R$ is modal, then so is $\mathrm{Hom}(A,R)$ for any $R$-algebra $A$ by general reasoning on modalities, so that every affine scheme is modal. By duality this implies that every finitely presented algebra is modal.
\end{remark}


\subsection{Schemes are fppf sheaves}

\rednote{Could prove that if $X$ is a type with a Zariski cover by fppf sheaf, then $X$ is an fppf sheaf, which sounds interesting to state.}

\begin{lemma}\label{scheme-are-sheaf-from-affine}
Assume given a proposition $P$ such that:
\begin{itemize}
\item The type $R$ is $P$-local.
\item Any open proposition is $P$-local.
\item The type of open propositions is $P$-local.
\end{itemize}
Then any scheme is $P$-local.
\end{lemma}

\begin{proof}
Since $R$ is $P$-local, all affine schemes are $P$-local as explained in \Cref{R-modal-subcanonical}.

We check that for all scheme $X$, any map:
\[f:P\to X\]
merely factors through $1$. Take $(U_i)_{i:I}$ a finite cover of $X$ by affine scheme. Then for any $i:I$ we have that $f^{-1}(U_i)$ is an in $P$, so since the type of open is $P$-local, we merely have an open proposition $V_i$ such that for all $x:P$, we have:
\[(x\in f^{-1}(U_i) )\leftrightarrow V_i\]
Since the $f^{-1}(U_i)$ cover $P$, we have that:
\[P\to \lor_{i:I} V_i\]
But open propositions are assumed to be $P$-local, so we have that:
\[ \lor_{i:I} V_i\]
Assume $k:I$ such that $V_k$ holds. Then $f^{-1}(U_k) = P$ and the map $f$ factors through the affine scheme $U_k$. Since affine schemes are $P$-local, we merely have a lift for $f$.

Now we conclude that any scheme is $P$-local by proving that its identity types are $P$-local. Indeed they are schemes, so the previous point implies they are $P$-local.
\end{proof}

\rednote{Next lemma surely has a more direct, algebraic proof.}

\begin{lemma}\label{roots-monic-proper}
For any monic $g:R[X]$, we have that $\Spec(R[X]/g)$ is projective. In particular it is compact, meaning that for any open $U$ in $\Spec(R[X]/g)$ the proposition:
\[\prod_{x:\Spec(R[X]/g)}U(x)\]
is open.
\end{lemma}

\begin{proof}
Assume that:
\[g=X^n+a_{n-1}X^{n-1}+\cdots+a_0\]
Then we consider the homogeneous polynomial:
\[f(X,Y) = X^n + a_{n-1}X^{n-1}Y+\cdots+a_0Y^n\]
We prove that:
\[\sum_{[x,y]:\bP^1}f(x,y) = 0\]
is equivalent to $\Spec(R[X]/g)$. Indeed for any $x,y:R$ such that $f(x,y)=0$, we have that $x\not=0$ implies $y\not=0$, so that $(x,y)\not=0$ implies $y\not=0$. Then:
\[\sum_{[x,y]:\bP^1}f(x,y) = 0\]
is equivalent to:
\[\sum_{x:R} f(x,1)=0\]
which is the type of roots of $g$. 

Now we conclude using the fact that 
\[\sum_{[x,y]:\bP^1}f(x,y) = 0\]
is closed in the compact scheme $\bP^1$, so that it is compact.
\end{proof}

\begin{proposition}\label{scheme-is-fppf-sheaf}
Any scheme is an fppf sheaf.
\end{proposition}

\begin{proof}
Assume given $g:R[X]$ monic, we can apply \Cref{scheme-are-sheaf-from-affine} because:
\begin{itemize}
\item The type $R$ is an fppf sheaf by \Cref{fppf-subcanonical}.
\item Any open proposition $U$ is an fppf sheaf because if:
\[\propTrunc{\Spec(R[X]/g)}\to U\]
then since $\neg\neg\Spec(R[X]/g)$ we have $\neg\neg U$, which implies $U$.
\item Since open propositions are fppf-sheaf it is enough that any map:
\[\propTrunc{\Spec(R[X]/g)}\to \mathrm{Open}\]
merely factors through $1$. But given a constant open $U$ in $\Spec(R[X]/g)$, for any $x:\Spec(R[X]/g)$ we have that:
\[x\in U \leftrightarrow \prod_{y:\Spec(R[X]/g)} y\in U)\]
The right hand side is open by \Cref{roots-monic-proper}, giving the required lift. 
\end{itemize}
\end{proof}


\subsection{Descent for finite free modules}

\begin{lemma}\label{fp-equivalent-pointwise}
If we have $M_x$ a finitely presented $R$-module depending on $x:\Spec(A)$, then $\prod_{x:\Spec(A)}M_x$ is a finitely presented $A$-module.
\end{lemma}

\begin{proof}
\cite{TODO}
\end{proof}

\begin{lemma}\label{descent-sqc-fppf}
Let $M$ be a module that is an fppf sheaf such that $\propTrunc{M\ \mathrm{is\ f.p.}}_{fppf}$, then for any monic $g$ we have that:
\[R[X]/g\otimes M \simeq M^{\Spec(R[X]/g)}\]
\end{lemma}

\begin{proof}
We have that $R[X]/g\otimes M$ is merely equal to $M^n$ where $n$ is the degree of $g$, therefore it is an fppf sheaf. As $M^{\Spec(R[X]/g)}$ is an fppf sheaf as well, the canonical map:
\[R[X]/g\otimes M \to M^{\Spec(R[X]/g)}\]
being an equivalence is an fppf sheaf, and we can assume $M$ finitely presented when proving. In this case it is true because f.p. modules are strongly quasi-coherent \cite{TODO}.
\end{proof}

\begin{lemma}\label{fp-stable-fppf-tensor}
Let $A$ be a fppf algebra and let $M$ be an $R$-module. Then if $A\otimes M$ is f.p. as an $A$-module if and only if $M$ is f.p. as an $R$-module.
\end{lemma}

\begin{proof}
Lombardi-Quitté VIII.6.7.
\end{proof}

\begin{lemma}\label{descent-fp-fcop}
Any finitely presented (resp. finitely copresented) module is an fppf sheaf.

The type of finitely presented (resp. finitely copresented) modules is an fppf sheaf.
\end{lemma}

\begin{proof}
For the first part we just use \Cref{fppf-subcanonical} and the fact that $M=M^{\star\star}$ for $M$ finitely presented of finitely copresented \cite{TODO}.

The types of finitely presented and finitely copresented modules are equivalent so it is enough to prove the result for finitely presented modules. Since any finitely presented module is an fppf sheaf it is enough to prove that being finitely presented is an fppf sheaf. Assume $M$ a module that is an fppf sheaf, and $g$ monic with $A=R[X]/g$ such that:
\[\Spec(A)\to M\ \mathrm{is\ f.p.}\]
Then by \Cref{fp-equivalent-pointwise} we have that $M^{\Spec(A)}$ is an f.p. $A$-module. By \Cref{descent-sqc-fppf} we have that:
\[A\otimes M \simeq M^{\Spec(A)}\]
so that $A\otimes M$ is an f.p. $A$-module and we conclude using \Cref{fp-stable-fppf-tensor}.
\end{proof}

\begin{proposition}\label{descent-finite-free}
The type of finite free modules is an fppf sheaf.
\end{proposition}

\begin{proof}
It is enough to prove that if $M$ is a module that is an fppf sheaf, then $M$ being finite free is an fppf-sheaf because finite free modules are sheaves. Since $R$ is local being finite free is equivalent to being finitely presented and finitely copresented \cite{TODO}. But both are fppf sheaves by \Cref{descent-fp-fcop} so we can conclude.
\end{proof}



\section{Azumaya algebras and their associated Severi-Brauer variety}
From now on we assume a lex modality $T$ such that:
\begin{itemize}
\item Schemes are modal.
\item The type of finite free modules is modal.
\end{itemize}
We call $T$-modal types sheaves and we write $\propTrunc{X}_T$ the sheafification of the propositional truncation of $X$.

In \Cref{fppf-sheaves} we constructed such a modality (\Cref{scheme-is-fppf-sheaf} and \Cref{descent-finite-free}).

We fix a natural number $n$ throughout.


\subsection{The type $\AZ_n$ of Azumaya algebras}

\begin{definition}
An Azumaya algebra of rank $n$ is a (non-commutative, unital) $R$-algebra $A$ such that its underlying type is a sheaf and:
\[\propTrunc{A=M_{n+1}(R)}_T\]
\end{definition}

We write $\AZ_n$ for the type of Azumaya algebra of rank $n$.

\begin{lemma}\label{azumayas-are-finite-free}
For all $A:\AZ_n$ we have that $A$ is finite free as a module.
\end{lemma}

\begin{proof}
By hypothesis $A$ being finite free is modal so that $\propTrunc{A=M_{n+1}(R)}_T$ implies $A$ finite free.
\end{proof}

\begin{definition}
Let $V$ be a free $R$-module, we define $\Gr_k(V)$ the $k$-Grassmannian of $V$ as the type of $k$-dimensional subspaces of $V$.
\end{definition}

\begin{lemma}\label{grassmanians-are-schemes}
Let $V$ be a finite free module, then $\Gr_k(V)$ is a scheme.
\end{lemma}

\begin{proof}
We can assume $V=R^n$. The type of $k$-dimensional subspaces of $R^n$ is the type of $n\times k$ matrices of rank $k$ quotiented by the natural action of $\GL_k$. For all $k\times k$ minor, we consider the open proposition stating this minor is non-zero, which well defined as it is invariant under the $\GL_k$-action. This gives a finite open cover of $\Gr_k(R^n)$.

Let us show any piece is affine. For example consider the piece of matrices of the form:
\[\begin{pmatrix}
P & N
\end{pmatrix}\]
where $P$ is invertible of size $k\times k$. Any orbit in this piece has a unique element of the form:
\[\begin{pmatrix}
I_k & N'
\end{pmatrix}\]
where $I_k$ is the identity matrix, so this piece is equivalent to $R^{(n-k)k}$.
\end{proof}

\begin{lemma}\label{being-ideal-in-azumaya-closed}
For all $A:\AZ_n$ and $I:\Gr_{n+1}(A)$, we have that $I$ being a right ideal in $A$ is a closed proposition.
\end{lemma}

\begin{proof}
By \Cref{azumayas-are-finite-free} we have that $A$ is finite free as a module. Consider $a_0,\cdots,a_n$ a basis of $I$ and extend it to a basis of $A$ using $b_1,\cdots,b_l$. 

For any $a:A$, we have that $a\in I$ is a closed as it says that the $b_1,\cdots,b_l$ coordinates of $a$ are zero. 

Then $I$ is an ideal if and only if for any $a$ in the chosen basis of $A$ and any $i$ in the chosen basis of $I$ we have that $ai\in I$, which is a closed proposition.
\end{proof}

\begin{lemma}\label{severi-brauer-are-schemes}
For all $A:\AZ_n$ we define:
\[\RI(A) := \{I:\Gr_{n+1}(A)\ |\ I\ \mathrm{is\ a\ right\ ideal}\}\]
Then $\RI(A)$ is a scheme.
\end{lemma}

\begin{proof}
By \Cref{azumayas-are-finite-free} we have that $A$ is finite free as a module, so that by \Cref{grassmanians-are-schemes} we have that $\Gr_{n+1}(A)$ is a scheme, and then by \Cref{being-ideal-in-azumaya-closed} we have that $\RI(A)$ is closed in a scheme, so it is a scheme.
\end{proof}


\subsection{Quaternion algebras are Azumaya algebras}

We assume $2\not=0$.

\begin{definition}
Given $a,b:R^\times$, we define the quaternion algebra $Q(a,b)$ as the non-commutative algebra:
\[R[i,j]/(i^2=a,j^2=b,ij=-ji)\] 
\end{definition}

\begin{remark}
As a vector space, $Q(a,b)$ is of dimension $4$, generated by $1,i,j,ij$.
\end{remark}

\begin{remark}
By the change of variable $i\mapsto j$ and $j\mapsto i$ we get $Q(a,b) = Q(b,a)$.
\end{remark}

\begin{lemma}\label{quaternion-split}
For all $b:R^\times$, we have that $Q(1,b) = M_2(R)$.
\end{lemma}

\begin{proof}
We send $i$ to:
\[I = \begin{pmatrix}
1 & 0\\
0 & -1\\
\end{pmatrix}\]
and $j$ to:
\[J = \begin{pmatrix}
0 & b\\
1 & 0\\
\end{pmatrix}\]
Then $IJ$ is:
\[K = \begin{pmatrix}
0 & b\\
-1 & 0\\
\end{pmatrix}\]
It is easy to check this define an algebra morphism, and since $1,I,J,K$ form a basis of $M_2(R)$ the map is an isomorphism.
\end{proof}

\begin{lemma}\label{quaternion-change-variable}
For all $a,b,u,v:R^\times$, we have that $Q(a,b) = Q(u^2a,v^2b)$.
\end{lemma}

\begin{proof}
We use the variable change $i\mapsto ui$ and $j\mapsto vj$.
\end{proof}

\begin{lemma}
Given $a,b:R^\times$, we have that $Q(a,b)$ is an Azumaya algebra. 
\end{lemma}

\begin{proof}
We have that $Q(a,b)$ is finite free as a vector space so it is a sheaf. So $Q(a,b)$ being Azumaya is a sheaf and we can assume $\sqrt{a}$. Then by \Cref{quaternion-change-variable} we have $Q(1,b) = Q(\sqrt{a}^2,b) = Q(a,b)$ and we conclude by \Cref{quaternion-split}.
\end{proof}


\subsection{A remark on Azumaya algebras}

\begin{lemma}\label{MnR-endomorphism-multiplication}
For any $n:\N$, the map:
\[M_{n+1}(R)\otimes M_{n+1}(R)^{op}\to \mathrm{End}_R(M_{n+1}(R))\]
\[M\otimes N\mapsto (P\mapsto MPN)\]
is an equivalence.
\end{lemma}

\begin{proof}
Let us denote by $(E_{i,j})_{0\leq i,j\leq n}$ the canonical basis of $M_{n+1}(R)$. We consider the basis: 
\[(E_{i,j}\otimes E_{k,l})_{0\leq i,j,k,l\leq n}\] 
of $M_{n+1}(R)\otimes M_{n+1}(R)^{op}$, as well as the basis:
\[(C_{i,j,k,l})_{0\leq i,j,k,l\leq n}\] 
of $\mathrm{End}_R(M_{n+1}(R))$ where $C_{i,j,k,l}(E_{j,k}) = E_{i,l}$ and $C_{i,j,k,l}$ is null on other element of the basis. It is clear that the morphism sends one basis to the other, and that both algebras have the same multiplication table. 
\end{proof}

\begin{lemma}
Assume $A:\AZ_n$, then $A$ is finite free as a module and the map $A\otimes A^{op}\to \mathrm{End}_R(A)$ sending $a\otimes b$ to $c\mapsto acb$ is an equivalence.
\end{lemma}

\begin{proof}
The fact that $A$ is finite free is \Cref{azumayas-are-finite-free}. Then both $A\otimes A^{op}$ and $\mathrm{End}_R(A)$ are finite free modules and therefore are $T$-modal, so that the map being an equivalence is $T$-modal and when proving it we can assume $A=M_{n+1}(R)$. Then we conclude by \Cref{MnR-endomorphism-multiplication}.
\end{proof}

\begin{remark}\label{azumaya-independent-modality}
We expect a converse in the sense that if $A$ is an algebra which is finite free as a module, such that the map $A\otimes A^{op} \to \mathrm{End}_R(A)$ is an equivalence, we should get $\propTrunc{A=M_{n+1}(R)}_{\mathrm{Et}}$. This would mean that given any modality $T$ such that:
\begin{itemize}
\item Schemes are $T$-modal.
\item The type of finite free modules is $T$-modal.
\item $T$-modal types are étale sheaves.
\end{itemize}
We have that an algebra $A$ is an Azumaya algebra for $T$ if and only if $A$ finite free as a module and the map $A\otimes A^{op} \to \mathrm{End}_R(A)$ is an equivalence. In particular Azumaya algebras do not depend on the choice of such a $T$.
\end{remark}


\subsection{The type $\SB_n$ of Severi-Brauer varieties}

\begin{definition}
A type $X$ is called a Severi-Brauer variety of dimension $n$ if $X$ is a sheaf and:
\[\propTrunc{X=\bP^n}_T\]
\end{definition}

We write $\SB_n$ the type of Severi-Brauer varieties of dimension $n$.

\begin{lemma}\label{right-ideal-of-matrices-are-projective}
Consider the map:
\[\delta:\bP^n \to \RI(M_{n+1}(R))\]
sending $(x_0:\cdots:x_n):\bP^n$ to:
\[\{M:M_n(R)\ |\ \forall i,j.\ x_i\cdot M_j = x_j\cdot M_i\}\]
where $M_i$ is the $i$-th line of $M$. Then $\delta$ is an equivalence.
\end{lemma}

\begin{proof}
Write $X=(x_0:\cdots:x_n)$. First we check $\delta$ is well defined. It is clear that for all $\lambda\not=0$ we have that:
\[\delta(\lambda X) = \delta(X)\]
and that $\delta(X)$ is a right ideal. To check the dimension assume $x_k\not=0$. Then $M\in\delta(X)$ if and only if for all $i$ we have that $M_i = \frac{x_i}{x_k} M_k$, which means giving $M\in\delta(X)$ is equivalent to giving $M_k$ in $R^{n+1}$, so $\delta(X)$ is free of dimension $n+1$.

Next we check injectivity. Assume given $(x_0:\cdots:x_n)$ and $(y_0:\cdots:y_n)$ in $\bP^n$ such that for all $M:M_n(R)$ we have:
\[(\forall i,j.\ x_i\cdot M_j = x_j\cdot M_i) \leftrightarrow (\forall i,j.\ y_i\cdot M_j = y_j\cdot M_i)\]
In particular considering the matrix $N$ such that $N_j = (y_j,\cdots,y_j)$ we get that:
\[\forall i,j.\ x_iy_j=x_jy_i\] 
so that:
\[(x_0:\cdots:x_n) = (y_0:\cdots:y_n)\]

Finally we check surjectivity. Assume $I:\RI(M_{n+1}(R))$, since $\propTrunc{I=R^{n+1}}$ we have $M\in I$ such that $M\not=0$, for example assume $M_{0,0}\not=0$. Then for all $k$ we have that:
\[ME_{0,k}\in I\]
meaning that we have $N^k\in I$ where:
\[N^k_{i,k} = M_{i,0}\]
and when when $j\not=k$
\[N^k_{i,j} = 0\]
Then since $M_{0,0}\not=0$ the matrices $N^k$ are linearly independent and since $I$ has dimension $n+1$, it is precisely the ideal spanned by the $N^k$. But this ideal is $\delta(M_{0,0}:\cdots:M_{n,0})$.
\end{proof}

\begin{lemma}
If $A$ is an Azumaya algebra, then $\RI(A)$ is a Severi-Brauer variety.
\end{lemma}

\begin{proof}
By \Cref{severi-brauer-are-schemes} and the assumption that schemes are sheaves, we have that $\RI(A)$ is a sheaf. Then to prove:
\[\propTrunc{A=M_{n+1}(R)}_T \to \propTrunc{\RI(A)=\bP^n}_T\]
it is enough to prove:
\[\RI(M_{n+1}(R)) = \bP^n\]
which is \Cref{right-ideal-of-matrices-are-projective}.
\end{proof}


\subsection{Conics are Severi-Brauer varieties}

We assume $2\not=0$.

\begin{definition}
Given $a,b:R^\times$, we define the conic$C(a,b)$ as the set of $[x:y:z]:\bP^2$ such that $x^2=ay^2+bz^2$.
\end{definition}

\begin{lemma}\label{conic-one-split}
Assume given $b:R^\times$, then $C(1,b) = \bP^1$.
\end{lemma}

\begin{proof}
Consider the map:
\[\psi:\bP^1\to \bP^2\]
\[ [m:n]\mapsto \left[m^2+bn^2 : m^2-bn^2 : 2mn\right] \]
We now check it is an isomorphism.
\begin{itemize}

\item The ideal $I = (m^2+bn^2, m^2-bn^2, 2mn)$ contains $m^2$ and $n^2$, therefore if $(m,n)=1$ then $I=1$. So the map is well defined.

\item The map $\psi$ takes value in $C(1,b)$. By direct computations, we check that for all $m,n:R$ we have that:
\[(m^2+bn^2)^2 = (m^2-bn^2)^2 + b(2mn)^2\]

\item The map $\psi$ is injective. Indeed assume $[m:n]$ and $[\bar{m}:\bar{n}]$ in $\bP^1$ such that $\psi([m:n])=\psi([\bar{m}:\bar{n}])$, we have that $m\bar{n}=\bar{m}n$. By linear combinations we get that:
\begin{eqnarray}
m^2\bar{m}\bar{n} &=& \bar{m}^2mn\\
\bar{m}\bar{n}n^2 &=& mn\bar{n}^2\\
\bar{m}^2n^2 &=& m^2\bar{n}^2
\end{eqnarray}
If $m$ and $\bar{m}$ are invertible we conclude by (1). If $n$ and $\bar{n}$ are invertible we conclude by (2). If $\bar{m}$ and $n$ (resp. $m$ and $\bar{n}$) are invertible by (3) we have that $m$ and $\bar{n}$ (resp. $\bar{m}$ and $n$) are invertible, and we conclude as before.

\item The map $\psi$ is surjective. Assume $[x:y:z]:C(a,b)$. We know that $x$, $y$ or $z$ is invertible, and since $x^2=y^2+bz^2$, we have that if $z$ is invertible then $x$ or $y$ is invertible. Since $x$ and $y$ belong to $(x+y, x-y)$, we know one of the two is invertible, say $x+y$. 

Since $\psi$ is an embedding of schemes, its fibers being merely inhabited is an fppf sheaf by \Cref{scheme-is-fppf-sheaf}. So we can define $m=\sqrt{\frac{x+y}{2}}$. Then we can define $n=\frac{z}{2m}$ and check that $n^2 = \frac{x-y}{2b}$ and that $\psi([m:n]) = [x:y:z]$. 

If $x-y$ is invertible, we proceed similarly using $n=\sqrt{\frac{x-y}{2b}}$ and $m=\frac{z}{2n}$.
\end{itemize}
\end{proof}

\begin{lemma}\label{conic-change-variable}
Given $a,b,u,v:R^\times$ we have that $C(a,b) = C(u^2a,v^2b)$.
\end{lemma}

\begin{proof}
Consider the change of variable $y\mapsto uy$ and $z\mapsto vz$.
\end{proof}

\begin{lemma}
Assume given $a,b:R^\times$, then $C(a,b)$ is a Severi-Brauer variety.
\end{lemma}

\begin{proof}
Since a sheaf being a Severi-Brauer variety is an fppf sheaf, we can assume $\sqrt{a}$ . Then by \Cref{conic-change-variable} we have that $C(1,b) = C(\sqrt{a}^2,b)=C(a,b)$ and we conclude by \Cref{conic-one-split}.
\end{proof}

\rednote{Maybe TODO: $RI(Q(a,b)) = C(a,b)$. I am not sure how to do it easily, not sure it is that relevant.}

\begin{lemma}\label{pointed-conics-projective}
Assume $a,b:R^\times$ such that $\propTrunc{C(a,b)}$, then $\propTrunc{C(a,b)=\bP^1}$.
\end{lemma}

\begin{proof}
Let us assume $x_0,y_0,z_0$ such that $x_0^2 = ay_0^2+bz_0^2$. We can assume $x_0\not=0$ by possibly considering $C(a,b) = C(\frac{1}{a},-\frac{b}{a})$. Then we can clearly assume $x_0=1$ without loss of generality, so that $ay_0 + bz_0 = 1$.

 Let us consider the map:
\[\psi:\bP^1\to \bP^2\]
\[[u:v] \mapsto [au^2+bv^2: y_0(au^2-bv^2) + 2buvz_0 : z_0(au^2-bv^2) - 2auvy_0]\]
We will check that it is a bijection.
\begin{itemize}

\item It is well defined. Indeed the ideal generated by the coordinates of $\psi([u,v])$ contains $au^2+bv^2$ and:
\[ay_0(y_0(au^2-bv^2) + 2buvz_0) + bx_0(z_0(au^2-bv^2) - 2auvy_0) = au^2-bv^2\]
therefore it contains $(u^2,v^2)$ which is equal to $1$.

\item It takes value in $C(a,b)$. Indeed we can compute:
\[ a\left( y_0(au^2-bv^2) + 2buvz_0\right)^2 + b\left(z_0(au^2-bv^2) - 2auvy_0\right)^2 = (au^2+bv^2)^2\]

\item It is injective. Assume $[u:v],[\bar{u}:\bar{v}]:\bP^1$ such that $\psi(u,v) = \psi(\bar{u},\bar{v})$. Let us write: 
\[\psi(u,v) = [\alpha_0:\alpha_1:\alpha_2]\]
\[\psi(\bar{u},\bar{v}) = [\bar{\alpha}_0:\bar{\alpha}_1:\bar{\alpha}_2]\]
From:
\[\alpha_0(ay_0\bar{\alpha}_1 + bz_0\bar{\alpha}_2) = \bar{\alpha}_0(ay_0\alpha_1 + bz_0\alpha_2)\]
we get that:
\[(au^2+bv^2)(a\bar{u}^2-b\bar{v}^2)= (a\bar{u}^2+b\bar{v}^2)(au^2-bv^2)\]
and therefore that:
\[u^2\bar{v}^2 = \bar{u}^2v^2\]
From:
\[\alpha_1\bar{\alpha}_2 = \bar{\alpha}_1\alpha_2\]
we get that:
\[uv(a\bar{u}^2-b\bar{v}^2) = \bar{u}\bar{v}(au^2-bv^2)\]
Finally from:
\[\alpha_0(z_0\bar{\alpha}_1-y_0\bar{\alpha}_2) = \bar{\alpha}_0(z_0\alpha_1-y_0\alpha_2)\]
we get that:
\[\bar{u}\bar{v}(au^2+bv^2) = uv(a\bar{u}^2+b\bar{v}^2)\]
from which we get that:
\[\bar{u}\bar{v}u^2 = uv\bar{u}^2\]
\[\bar{u}\bar{v}v^2 = uv\bar{v}^2\]
and then we conclude as in \Cref{conic-one-split}.

\item It is surjective. Assume $[x:y:z]:\bP^2$ such that $x^2=ay^2+bz^2$. We look for $u,v$ such that $\psi(u,v) = [x:y:z]$. As in \Cref{conic-one-split}, we are able to use square roots when doing so. 

Assume $x=0$ and $ay_0y+bz_0z=0$, then we have a contradiction. Indeed $y$ or $z$ is invertible and $ay^2+bz^2=0$, so that $y$ and $z$ are invertible and $b = -a\frac{y^2}{z^2}$ and $y_0z=yz_0$. Moreover $y_0$ or $z_0$ is invertible, so that both $y_0$ and $z_0$ are invertible and $\frac{y}{z} = \frac{y_0}{z_0}$ which means $ay^2+bz^2=0$ implies $ay_0^2+bz_0^2=0$, a contradiction.

Therefore either $x$ or $ay_0y+bz_0z$ is invertible, so either $x+ay_0y+bz_0z$ or $x-ay_0y-bz_0z$ is invertible.

If $x+ay_0y+bz_0z$ is invertible, we take $u=\sqrt{\frac{x+ay_0y+bz_0z}{2a}}$ and $v = \frac{z_0y-y_0z}{2u}$.

If $x-ay_0y-bz_0z$ is invertible, we take $v=\sqrt{\frac{x-ay_0y-bz_0z}{2b}}$ and $u = \frac{z_0y-y_0z}{2v}$.

In both cases we can check that $\psi([u:v]) = [x:y:z]$ using that:
\[u^2 = \frac{x+ay_0y+bz_0z}{2a}\]
\[v^2 = \frac{x-ay_0y-bz_0z}{2b}\]
\[uv = \frac{z_0y-y_0z}{2}\]
\end{itemize}
\end{proof}

\rednote{Proof of \Cref{conic-one-split} is actually \Cref{pointed-conics-projective} when $a=1$ and $[x_0:y_0:z_0] = [1:1:0]$. We should probably factor.}

\begin{remark}
We give an explicit description of the inverse to the map $\psi$ in \Cref{pointed-conics-projective} defined by:
\[\psi(u,v) = [\alpha_0:\alpha_1:\alpha_2]\]
\[[u:v] \mapsto [au^2+bv^2: y_0(au^2-bv^2) + 2buvz_0 : z_0(au^2-bv^2) - 2auvy_0]\]
We use the proven fact that $x+ay_0y+bz_0z$ or $x-ay_0y-bz_0z$ is invertible.

If $x + ay_0y + bz_0z$ is invertible we send $[x,y,z]$ to $[1:\frac{a(z_0y-y_0z)}{x + ay_0y + bz_0z}]$.

If $x - ay_0y - bz_0z$ is invertible we send $[x,y,z]$ to $[\frac{b(z_0y-y_0z)}{x - ay_0y - bz_0z}:1]$.

This is well defined as if both are invertible then:
\[\frac{b(z_0y-y_0z)}{x - ay_0y - bz_0z}\times\frac{a(z_0y-y_0z)}{x + ay_0y + bz_0z} = \frac{ab(z_0y-y_0z)^2}{x^2 - (ay_0y + bz_0z)^2} = 1\]
because:
\[x^2 - (ay_0y + bz_0z)^2 = (ay^2+bz^2)(ay_0^2+bz_0^2) - (ay_0y + bz_0z)^2 = ab(z_0y-y_0z)^2\]
We omit the verification that this is indeed an inverse to $\psi$.
\end{remark}

\begin{remark}
We will see in \Cref{chatelet-theorem} that this holds for any Severi-Brauer variety, i.e. any merely inhabited Severi-Brauer variety is a projective space. 
\end{remark}

\section{$\AZ_n\simeq \SB_n$}
\subsection{Generalities on delooping in $T$-sheaves}

\begin{definition}
A type $A$ is $T$-connected if:
\[\forall(x,y:A).\ \propTrunc{x=y}_T\]
\end{definition}

The key intuition for the next lemma is that both $A$ and $B$ are deloopings of the same group in the topos of sheaves.

\begin{lemma}\label{deloopings-equivalence}
Assume $A,B$ pointed $T$-connected sheaves. Let $f:A\to B$ be a pointed map inducing an equivalence:
\[\Omega f : \Omega A \simeq \Omega B\]
Then $f$ is an equivalence.
\end{lemma}

\begin{proof}
First we prove that $f$ is an embedding. We have to prove that for all $x,y:A$ the map:
\[\mathrm{ap}_f : x=y \to f(x)=f(y)\]
is an equivalence. Since $A$ and $B$ are sheaves so are their identity types, so $\mathrm{ap}_f$ being an equivalence is a sheaf, so by $T$-connectedness of $A$ we can assume $x$ and $y$ are the basepoint, in which case it is part of the hypothesis.

Now we prove it is surjective, indeed for any $x:B$ we have that $\fib_f(x)$ is a sheaf and a proposition so when proving it is inhabited we can assume $x$ is the basepoint of $B$ and give the basepoint of $A$ as antecedent.
\end{proof}


\subsection{$\Aut(M_{n+1}(R)) = \Aut(\bP^n) = \PGL_{n+1}(R)$}

\begin{lemma}\label{finite-projective-free}
Let $M$ be a finite projective module, then $M$ is finite free.
\end{lemma}

\begin{proof}
\cite{TODO}, crucially relies on $R$ being local.
\end{proof}

\begin{lemma}\label{fundamental-system-matrices}
Assume $e_{i,j}:M_{n+1}(R)$ for $0\leq i,j\leq n$ such that:
\[e_{i,j}e_{k,l} = \delta_{j,k}e_{i,l}\]
where $\delta_{j,k} = 1$ if $j=k$ and $0$ otherwise. Moreover assume:
\[e_{0,0}+\cdots+e_{n,n}=1\]
Then there exists $P:GL_{n+1}(R)$ such that:
\[e_{i,j} = PE_{i,j}P^{-1}\]
\end{lemma}

\begin{proof}
We define $e_i = e_{i,i}$, then $e_0+\cdots+e_n = 1$, for all $i$ we have $e_i^2=e_i$ and for all $i\not=j$ we have that $e_ie_j=0$. From this we get:
\[R^{n+1} = V_0\oplus\cdots\oplus V_n\]
where:
\[V_i = \{x\ |\ e_i(x)=x\}\]
and:
\[e_{i,j}:V_j\simeq V_i\]

As a direct summand of a free module we have that $V_0$ is projective, and since $V_0 = e_{0}(R^{n+1})$ we have that $V_0$ is finitely generated, so by \Cref{finite-projective-free} it is finite free. From $V_0^{n+1}=R^{n+1}$ we get that $\propTrunc{V_0=R}$, and therefore that $\propTrunc{V_i=R}$ for all $i$.

Then we choose $v_0$ generating $V_0$ and define $v_i = e_{i,0}(v_0)$ so that $v_i$ generates $V_i$ because $e_{i,0}:V_0\simeq V_i$. We get a basis $v_0,\cdots,v_n$ of $R^{n+1}$.

Let $u_0,\cdots,u_n$ be the canonical basis of $R^{n+1}$ and define $P:GL_{n+1}(R)$ by sending $u_i$ to $v_i$. Then for all $v_k$ we have that:
\[e_{i,j}v_k = PE_{i,j}P^{-1}v_k\]
so we can conclude.
\end{proof}

\begin{proposition}\label{Aut-MnR-PGL}
The map:
\[\alpha:\PGL_{n+1}(R)\to\Aut(M_{n+1}(R))\]
\[P\mapsto (M\mapsto PMP^{-1})\]
is an equivalence.
\end{proposition}

\begin{proof}
It is clearly a group morphism. 

For injectivity we just need to check that if for all $M:M_{n+1}(R)$ we have $PMP^{-1}=M$ then there exists $\lambda\not=0$ such that $P=\lambda I_{n+1}$. This is a well-known linear algebra exercise.

For surjectivity consider $e_{i,j}=\sigma(E_{i,j})$, we can apply \Cref{fundamental-system-matrices} to get $P:GL_{n+1}(R)$ such that:
\[\sigma(E_{i,j}) = PE_{i,j}P^{-1}\]
from which we conclude that for all $M:M_{n+1}(R)$ we have that:
\[\sigma(M) = PMP^{-1}\]
as desired.
\end{proof}

\begin{proposition}\label{Aut-Pn-PGL}
The map:
\[\beta:\PGL_{n+1}(R)\to\Aut(\bP^n)\]
\[X\mapsto PX\]
is an equivalence.
\end{proposition}

\begin{proof}
\cite{TODO}
\end{proof}


\subsection{The Severi-Brauer construction is an equivalence}

\begin{proposition}\label{right-ideal-is-equivalence}
The map:
\[\RI:\AZ_n\to\SB_n\]
is an equivalence.
\end{proposition}

\begin{proof}
By \Cref{deloopings-equivalence} it is enough to prove that the top map in the triangle:
\begin{center}
\begin{tikzcd}
\Aut(M_n(R))\ar[rr,"\Omega\RI"] & & \Aut(\bP^n) \\
& \PGL_{n+1}(R)\ar[ru,swap,"\beta"]\ar[lu,"\alpha"] & \\
\end{tikzcd}
\end{center}
is an equivalence. But since the two other maps in the triangle are equivalences by \Cref{Aut-MnR-PGL} and \Cref{Aut-Pn-PGL}, it is enough to prove that the triangle commutes. To do this we need to check that for all $P:\PGL_{n+1}(R)$ we have that:
\[\delta^{-1}\circ \mathrm{ap}_\RI(\alpha(P))\circ\delta = \beta(P)\]
in $\Aut(\bP^n)$, with $\delta$ defined in \Cref{right-ideal-of-matrices-are-projective}. So we need to prove the following square commutes:
\begin{center}
\begin{tikzcd}
\RI(M_{n+1}(R))\ar[rr,"I\mapsto PIP^{-1}"]&& \RI(M_{n+1}(R)) \\
\bP^n\ar[u,"\delta"]\ar[rr,swap,"X\mapsto PX"]&& \bP^n\ar[u,swap,"\delta"] 
\end{tikzcd}
\end{center}
where $\mathrm{ap}_\RI$ was computed using path induction.

We see that:
\[\delta(Y) = \{M:M_n(R)\ |\ \forall A,B:R^{n+1}.\ A^tX\cdot B^tM = B^tX\cdot A^tM\}\]
To check that:
\[P\delta(X)P^{-1} = \delta(PX)\]
we just need to check an inclusion as both are finite free modules of the same dimension. Assume $M\in\delta(X)$, to check that $PMP^{-1}\in\delta(PX)$ we need to check that for all $A,B:R^{n+1}$ we have that:
\[A^tPX\cdot B^tPMP^{-1} = B^tX\cdot A^tPMP^{-1}\]
but since $M\in\delta(X)$ we have that:
\[(P^tA)^tX\cdot (P^tB)^tM = (P^tB)^tX\cdot (P^tA)^tM\]
which gives us what we want.
\end{proof}

\section{An inhabited Severi-Brauer variety is a projective space}
%\subsection{Azumaya algebras with a $(n+1)$-dimensional right ideal}

\begin{lemma}\label{azumaya-with-right-ideal}
Assume $A:\AZ_n$ with $I:\RI(A)$, then:
\[A = \mathrm{End}_R(I)^{op}\]
\end{lemma}

\begin{proof}
Since $I$ is an ideal, there is a canonical map of algebra:
\[A \to\mathrm{End}_R(I)^{op}\]
Since both algebras are sheaves (indeed $\propTrunc{I=R^{n+1}}$ implies $I$ is a sheaf), this map being an equivalence is a sheaf so we can assume $A=M_{n+1}(R)$.

By \Cref{right-ideal-of-matrices-are-projective} we can assume $X=(x_0:\cdots:x_n):\bP^n$ such that $I=\delta(X)$. There is a $k$ such that $x_k\not=0$, so we can assume $x_k=1$, then we have an isomorphism:
\[\theta:R^{n+1}\to I\]
sending $Y:R^{n+1}$ to the matrix $M$ with its $i$-th line $M_i=x_iY$. Then for all $M:M_n(R)$ we have a commutative square:
\begin{center}
\begin{tikzcd}
I\ar[rr,"N\mapsto NM"] && I \\
R^{n+1}\ar[u,"\theta"]\ar[rr,swap,"X\mapsto M^tX"] && R^{n+1}\ar[u,swap,"\theta"]\\
\end{tikzcd}
\end{center}
meaning the natural map:
\[ M_{n+1}(R)\to \mathrm{End}_R(I)^{op}\]
sends $M$ to $\delta^{-1}\circ M\circ\delta$, so it is an equivalence.
\end{proof}


%\subsection{Inhabited Severi-Brauer variety are projective}

\begin{theorem}
Assume $X:\SB_n$, then:
\[\propTrunc{X}\to\propTrunc{X=\bP^n}\]
\end{theorem}

\begin{proof}
By \Cref{right-ideal-is-equivalence} we can assume $X=\RI(A)$ for some $A:\AZ_n$. Then can assume $I:\RI(A)$, so that by \Cref{azumaya-with-right-ideal} we have that:
\[A=\mathrm{End}_R(I)^{op}\]
Since we merely have that $I=R^{n+1}$, we merely have:
\[A = M_{n+1}(R)^{op} = M_{n+1}(R)\]
Applying \Cref{right-ideal-of-matrices-are-projective} we merely conclude:
\[X=\RI(A)=\RI(M_{n+1}(R)) = \bP^n\]
\end{proof}

\printindex

\printbibliography

\end{document}
