%\subsection{Azumaya algebras with a $(n+1)$-dimensional right ideal}

\begin{lemma}\label{azumaya-with-right-ideal}
Assume $A:\AZ_n$ with $I:\RI(A)$, then:
\[A = \mathrm{End}_R(I)^{op}\]
\end{lemma}

\begin{proof}
Since $I$ is an ideal, there is a canonical map of algebra:
\[A \to\mathrm{End}_R(I)^{op}\]
Since both algebras are sheaves (indeed $\propTrunc{I=R^{n+1}}$ implies $I$ is a sheaf), this map being an equivalence is a sheaf so we can assume $A=M_{n+1}(R)$.

By \Cref{right-ideal-of-matrices-are-projective} we can assume $X=(x_0:\cdots:x_n):\bP^n$ such that $I=\delta(X)$. There is a $k$ such that $x_k\not=0$, so we can assume $x_k=1$, then we have an isomorphism:
\[\theta:R^{n+1}\to I\]
sending $Y:R^{n+1}$ to the matrix $M$ with its $i$-th line $M_i=x_iY$. Then for all $M:M_n(R)$ we have a commutative square:
\begin{center}
\begin{tikzcd}
I\ar[rr,"N\mapsto NM"] && I \\
R^{n+1}\ar[u,"\theta"]\ar[rr,swap,"X\mapsto M^tX"] && R^{n+1}\ar[u,swap,"\theta"]\\
\end{tikzcd}
\end{center}
meaning the natural map:
\[ M_{n+1}(R)\to \mathrm{End}_R(I)^{op}\]
sends $M$ to $\delta^{-1}\circ M\circ\delta$, so it is an equivalence.
\end{proof}


%\subsection{Inhabited Severi-Brauer variety are projective}

\begin{theorem}
Assume $X:\SB_n$, then:
\[\propTrunc{X}\to\propTrunc{X=\bP^n}\]
\end{theorem}

\begin{proof}
By \Cref{right-ideal-is-equivalence} we can assume $X=\RI(A)$ for some $A:\AZ_n$. Then can assume $I:\RI(A)$, so that by \Cref{azumaya-with-right-ideal} we have that:
\[A=\mathrm{End}_R(I)^{op}\]
Since we merely have that $I=R^{n+1}$, we merely have:
\[A = M_{n+1}(R)^{op} = M_{n+1}(R)\]
Applying \Cref{right-ideal-of-matrices-are-projective} we merely conclude:
\[X=\RI(A)=\RI(M_{n+1}(R)) = \bP^n\]
\end{proof}