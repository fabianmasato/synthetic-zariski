% latexmk -pdf -pvc main.tex
% latexmk -pdf -pvc -interaction=nonstopmode main.tex
\documentclass{../util/zariski}

\title{A Foundation for Synthetic Algebraic Geometry}
\author{Felix Cherubini$^1$, Thierry Coquand$^1$ and Matthias Hutzler$^1$}
\date{$^1$ University of Gothenburg and Chalmers University of Technology }

\begin{document}

\maketitle

\begin{abstract}
  This is a foundation for algebraic geometry, developed internal to the Zariski topos, building on the work of Kock and Blechschmidt (\cite{kock-sdg}[I.12], \cite{ingo-thesis}).
  The Zariski topos consists of sheaves on the site opposite to the category of finitely presented algebras over a fixed ring, with the Zariski topology, i.e.\ generating covers are given by localization maps $A\to A_{f_1}$ for finitely many elements $f_1,\dots,f_n$ that generate the ideal $(1)=A\subseteq A$.
  We use homotopy type theory together with three axioms as the internal language of a (higher) Zariski topos.

  One of our main contributions is the use of higher types -- in the homotopical sense -- to define and reason about cohomology.
  Actually computing cohomology groups, seems to need a principle along the lines of our ``Zariski local choice'' axiom,
  which we justify as well as the other axioms using a cubical model of homotopy type theory.
\end{abstract}

\tableofcontents

\section*{Introduction}
TODO

\rednote{Right now I'm throwing texts relating our stuff to the literature here.}

In formal/point-free topology \cite{TODO}, 
we consider that a Boolean algebra $B$ represents a Stone space $Sp(B)$ and a map
$Sp(B') \to Sp(B)$ is represented by a map $B\rightarrow B'$; 
the map $Sp(B')\to Sp(B)$ is then said to be
{\em formally surjective} if the corresponding map $B\to B'$ is injective. 
In the topos of light condensed sets,this becomes a true duality, 
as we shall see in \Cref{FormalSurjectionsAreSurjections}.


\section*{Formalization}
\input{formalization}

\section*{Acknowledgements}
The idea to use the topological characterization of stone spaces as totally disconnected, compact Hausdorff spaces to prove \Cref{stone-sigma-closed} was explained to us by Martín Escardó.
We profited a lot from a discussion with Reid Barton and Johann Commelin. 
David Wärn noticed that Markov's principle (\Cref{MarkovPrinciple}) holds. 
At TYPES 2024, we had an interesting discussion with Bas Spitters on the topic of the article.


\section{Preliminaries}
\subsection{Subtypes and Logic}

We use the notation $\exists_{x:X}P(x)\colonequiv \propTrunc{\sum_{x:X}P(x)}$.
We use $+$ for the coproduct of types and for types $A,B$ we write
\[ A\vee B\colonequiv \propTrunc{ A+B }\rlap{.}\]

We will use subtypes extensively.

\begin{remark}
  We use the word ``\notion{merely}'' throughout this article in the same sense it is used in the HoTT-Book (\cite{hott}).
  We do not use the word ``merely'' in any other sense.
\end{remark}

\begin{definition}
  \index{$\subseteq$}
  Let $X$ be a type.
  A \notion{subtype} of $X$ is a function $U:X\to\Prop$ to the type of propositions.
  We write $U\subseteq X$ to indicate that $U$ is as above.
  If $X$ is a set, a subtype may be called \notion{subset} for emphasis.
  For subtypes $A,B\subseteq X$, we write $A\subseteq B$ as a shorthand for pointwise implication.
\end{definition}

We will freely switch between subtypes $U:X\to\Prop$ and the corresponding embeddings
\[
  \begin{tikzcd}
    \sum_{x:X}U(x) \ar[r,hook] & X
  \end{tikzcd}
  \rlap{.}
\]
In particular, if we write $x:U$ for a subtype $U:X\to\Prop$, we mean that $x:\sum_{x:X}U(x)$ -- but we might silently project $x$ to $X$.

\begin{definition}
  Let $I$ and $X$ be types and $U_i:X\to\Prop$ a subtype for any $i:I$.
  \begin{enumerate}[(a)]
  \item The \notion{union} $\bigcup_{i:I}U_i$\index{$\bigcup_{i:I}U_i$} is the subtype $(x:X)\mapsto \exists_{i:I}U_i(x)$.
  \item The \notion{intersection} $\bigcap_{i:I}U_i$\index{$\bigcap_{i:I}U_i$} is the subtype $(x:X)\mapsto\prod_{i:I}U_i(x)$.
  \end{enumerate}
\end{definition}

We will use common notation for finite unions and intersections.
The following formula hold:

\begin{lemma}
  Let $I$, $X$ be types, $U_i:X\to\Prop$ a subtype for any $i:I$ and $V,W$ subtypes of $X$.
  \begin{enumerate}[(a)]
  \item Any subtype $P:V\to\Prop$ is a subtype of $X$ given by $(x:X)\mapsto\sum_{x:V}P(x)$.
  \item $V\cap \bigcup_{i:I} U_i=\bigcup (V\cap U_i)$.
  \item If $\bigcup_{i:I}U_i=X$ we have $V=\bigcup_{i:I}U_i\cap V$.
  \item If $\bigcup_{i:I}U_i=\emptyset$, then $U_i=\emptyset$ for all $i:I$.
  \end{enumerate}
\end{lemma}

\begin{definition}
  Let $X$ be a type.
  \begin{enumerate}[(a)]
  \item $\emptyset\colonequiv (x:X)\mapsto \emptyset$. \index{$\emptyset$}
  \item For $U\subseteq X$, let $\neg U\colonequiv (x:X)\mapsto \neg U(x)$. \index{$\neg U$}
  \item For $U\subseteq X$, let $\neg\neg U\colonequiv (x:X)\mapsto \neg\neg U(x)$. \index{$\neg\neg U$}
  \end{enumerate}
\end{definition}

\begin{lemma}
  $U=\emptyset$ if and only if $\neg\left(\exists_{x:X}U(x)\right)$.
\end{lemma}

\subsection{Homotopy type theory}

Our truncation levels start at $-2$, so $(-2)$-types are contractible, $(-1)$-types are propositions and $0$-types are sets.

\begin{definition}%
  Let $X$ and $I$ be types.
  A family of propositions $U_i:X\to\Prop$ \notion{covers} $X$,
  if for all $x:X$, there merely is a $i:I$ such that $U_i(x)$.
\end{definition}

\begin{lemma}%
  \label{kraus-glueing}
  Let $X$ and $I$ be types.
  For propositions $(U_i:X \to \Prop)_{i:I}$ that cover $X$ and $P:X\to \nType{0}$, we have the following glueing property: \\
  If for each $i:I$ there is a dependent function $s_i:(x:U_i)\to P(x)$ together with
  proofs of equality on intersections $p_{ij}:(x:U_i\cap U_j)\to (s_i(x)=s_j(x))$,
  then there is a globally defined dependent function $s:(x:X) \to P(x)$,
  such that for all $x:X$ and $i:I$ we have $U_i(x) \to s(x)=s_i(x)$
\end{lemma}

\begin{proof}
  We define $s$ pointwise.
  Let $x:X$.
  Using a Lemma of Kraus\footnote{For example this is the $n=-1$ case of \cite{dagstuhl-kraus}[Theorem 2.1].}
  and the $p_{ij}$, we get a factorization
  \[ \begin{tikzcd}[row sep=0mm]
    \sum_{i:I} U_i(x) \ar[rr, "s_{\pi_1(\_)}(x)"]\ar[rd] & & P(x) \\
    & \propTrunc{\sum_{i:I} U_i(x)}_{-1}\ar[ru,dashed] &
  \end{tikzcd} \]
-- which defines a unique value $s(x):P(x)$.
\end{proof}

Similarly we can prove.

\begin{lemma}%
  \label{kraus-glueing-1-type}
  Let $X$ and $I$ be types.
  For propositions $(U_i:X \to \Prop)_{i:I}$ that cover $X$ 
  and $P:X\to \nType{1}$, we have the following glueing property: \\
  If for each $i:I$ there is a dependent function $s_i:(x:U_i)\to P(x)$ together with
  proofs of equality on intersections $p_{ij}:(x:U_i\cap U_j)\to (s_i(x)=s_j(x))$ satisfying the cocycle
  condition $p_{ij}\cdot p_{jk} = p_{ik}$.
  then there is a globally defined dependent function $s:(x:X) \to P(x)$,
  such that for all $x:X$ and $i:I$ we have $p_i:U_i(x) \to s(x)=s_i(x)$ such that $p_i\cdot p_{ij} = p_j$.
\end{lemma}

This can be generalized to $\nType{k}$ for each {\em external} $k$.

The condition for $\nType{0}$ can be seen as an internal version of the usual patching {\em sheaf} condition.
The condition for $\nType{1}$ is then the internal version of the usual patching {\em $1$-stack} condition.

\subsection{Algebra}

\begin{definition}%
  \label{local-ring}
  A commutative ring $R$ is \notion{local} if $1\neq 0$ in $R$ and
  if for all $x,y:R$ such that $x+y$ is invertible, $x$ is invertible or $y$ is invertible.
\end{definition}

\begin{definition}%
  \label{def-finitely-presented}
  Let $R$ be a commutative ring.
  A \notion{finitely presented} $R$-algebra is an $R$-algebra $A$,
  such that there merely are natural numbers $n,m$ and polynomials $f_1,\dots,f_m:R[X_1,\dots,X_n]$
  and an equivalence of $R$-algebras $A\simeq R[X_1,\dots,X_n]/(f_1,\dots,f_m)$.
\end{definition}

\begin{definition}%
  \label{regular-element}
  Let $A$ be a commutative ring.
  An element $r:A$ is \notion{regular},
  if the multiplication map $r\cdot\_:A\to A$ is injective.
\end{definition}

\begin{lemma}%
  \label{units-products-regular}
  Let $A$ be a commutative ring.
  \begin{enumerate}[(a)]
  \item All units of $A$ are regular.
  \item If $f$ and $g$ are regular, their product $fg$ is regular.
  \end{enumerate}
\end{lemma}

\begin{example}
  The monomials $X^k:A[X]$ are regular.
\end{example}

\begin{lemma}%
  \label{polynomial-with-regular-value-is-regular}
  Let $f : A[X]$ be a polynomial
  and $a : A$ an element
  such that $f(a) : A$ is regular.
  Then $f$ is regular as an element of $A[X]$.
\end{lemma}

\begin{proof}
  After a variable substitution $X \mapsto X + a$
  we can assume that $f(0)$ is regular.
  Now let $g : A[X]$ be given with $fg = 0$.
  Then in particular $f(0) g(0) = 0$,
  so $g(0) = 0$.
  By induction,
  all coefficients of $g$ vanish.
\end{proof}

\begin{definition}
  Let $A$ be a ring and $f:A$.
  Then $A_{f}$ denotes the \notion{localization} of $A$ at $f$,
  i.e. a ring $A_f$ together with a homomorphism $A\to A_f$,
  such that for all homomorphisms $\varphi:A\to B$ such that
  $\varphi(f)$ is invertible, there is a unique homomorphism as indicated in the diagram:
  \begin{center}
    \begin{tikzcd}
      A\ar[r]\ar[rd,"\varphi",swap] & A_f\ar[d,dashed] \\
      & B
    \end{tikzcd}
    \rlap{.}
  \end{center}
  For $a:A$, we denote the image of $a$ in $A_f$ as $\frac{a}{1}$ and the inverse of $f$ as $\frac{1}{f}$.
\end{definition}

\begin{lemma}%
  \label{fg-ideal-local-global}
  Let $A$ be a commutative ring and $f_1,\dots,f_n:A$.
  For finitely generated ideals $I_i\subseteq A_{f_i}$,
  such that $A_{f_if_j}\cdot I_i=A_{f_if_j}\cdot I_j$ for all $i,j$,
  there is a finitely generated ideal $I\subseteq A$,
  such that $A_{f_i}\cdot I=I_i$ for all $i$.
\end{lemma}

\begin{proof}
  Choose generators 
  \[ \frac{g_{i1}}{1},\dots,\frac{g_{ik_i}}{1} \]
  for each $I_i$.
  These generators will still generate $I_i$, if we multiply any of them with any power of the unit $\frac{f_i}{1}$.
  Now
  \[ A_{f_if_j}\cdot I_i\subseteq A_{f_if_j}\cdot I_j \]
  means that for any $g_{ik}$, we have a relation
  \[ (f_if_j)^l g_{ik}=\sum_{l}h_{l}g_{jl}\]
  for some power $l$ and coefficients $h_{l}:A$.
  This means, that $f_i^lg_{ik}$ is contained in $I_j$.
  Multiplying $f_i^lg_{ik}$ with further powers of $f_i$ or multiplying $g_{jl}$ with powers of $f_j$ does not change that.
  So we can repeat this for all $i$ and $k$ to arrive at elements $\tilde{g_{ik}}:A$,
  which generate an ideal $I\subseteq A$ with the desired properties.
\end{proof}

The following definition also appears as \cite{ingo-thesis}[Definition 18.5]
and a version restricted to external finitely presented algebras was already used by Anders Kock in \cite{kock-sdg}[I.12]:

\begin{definition}
  \label{spec}
  The \notion{(synthetic) spectrum}\index{$\Spec A$} of a finitely presented $R$-algebra $A$
  is the set of $R$-algebra homomorphisms from $A$ to $R$:
  \[ \Spec A \colonequiv \Hom_{\Alg{R}}(A, R) \]
\end{definition}

We write $\A^n$ for $\Spec R[X_1, \dots, X_n]$,
which is canonically in bijection with $R^n$
by the universal property of the polynomial ring.
In particular,
$\A^1$ is (in bijection with) the underlying set of $R$.
Our convention is to use the letter $R$
when we think of it as an algebraic object,
and to write $\A^1$ (or $\A^n$) when we think of it as a set or a geometric object.

The $\Spec$ construction is functorial:

\begin{definition}
  \label{spec-on-maps}
  For an algebra homomorphism $f:\Hom_{\Alg{R}}(A,B)$
  between finitely presented $R$-algebras $A$ and $B$,
  we write \notion{$\Spec f$} for the map from $\Spec B$ to $\Spec A$
  given by precomposition with $f$.
\end{definition}

\begin{definition}%
  \label{basic-open-subset}
  Let $A$ be a finitely presented $R$-algebra.
  For $f:A$, the \notion{basic open subset} given by $f$,
  is the subtype 
  \[
    D(f)\colonequiv (x:\Spec A)\mapsto (x(f)\text{ is invertible})
    \rlap{.}
  \]
\end{definition}

later, we will use the following more general and related definitions:

\begin{definition}
  \label{open-closed-affine-subsets}
  Let $A$ be a finitely presented $R$-algebra.
  For $n:\N$ and $f_1,\dots,f_n:A$, there are
  \begin{enumerate}[(i)]
  \item the ``open'' subset
    \[
      D(f_1,\dots,f_n)\colonequiv (x:\Spec A)\mapsto (\text{$\exists_i$ such that $x(f_i)$ is invertible})
    \]  
  \item the ``closed'' subset
    \[
      V(f_1,\dots,f_n)\colonequiv (x:\Spec A)\mapsto (\forall_i\ x(f_i)=0)
    \]  
  \end{enumerate}
  It will be made precise in \Cref{topology-of-schemes}, in which sense these subsets are open or closed.
\end{definition}

We will later also need the notion of a \emph{Zariski-Cover} of a spectrum $\Spec A$,
for some finitely presented $R$-algebra $A$.
Intuitively, this is a collection of basic opens which jointly cover $\Spec A$.
Since it is more practical, we will however stay on the side of algebras.
A finite list of elements $f_1,\dots,f_n:A$ yields a Zariski-Cover,
if and only if they are a \emph{unimodular vector}:

\begin{definition}
  \label{unimodular}
  Let $A$ be a finitely presented $R$-algebra.
  Then a list $f_1,\dots,f_n:A$ of elements of $A$ is called \notion{unimodular}
  if we have an identity of ideals $(f_1,\dots,f_n)=(1)$.
  We use $\Um(A)$\index{$\Um(A)$} to denote the type of unimodular sequences in $A$:
  \[
    \Um(A)\colonequiv \sum_{n:\N}\sum_{f_1,\dots,f_n:A} (f_1,\dots,f_n)=(1)
    \rlap{.}
  \]
  We will sometimes drop the natural number and the equality and just write $(f_1,\dots,f_n):\Um(A)$.
\end{definition}

\begin{definition}
  $\AbGroup$\index{$\AbGroup$} denotes the type of abelian groups.
\end{definition}

\begin{lemma}%
  \label{surjective-abgroup-hom-is-cokernel}
  Let $A,B:\AbGroup$ and $f:A\to B$ be a homomorphism of abelian groups.
  Then $f$ is surjective, if and only if, it is a cokernel.
\end{lemma}

\begin{proof}
  A cokernel is a set-quotient by an effective relation,
  so the projection map is surjective.
  On the other hand, if $f$ is surjective and we are in the situation:
  \begin{center}
    \begin{tikzcd}
      \ker(f)\ar[r,hook]\ar[dr] & A\ar[r,"f",->>]\ar[dr,"g"] & B \\
      & 0\ar[r] & C
    \end{tikzcd}
  \end{center}
  then we can construct a map $\varphi:B\to C$ as follows.
  For $x:B$, we define the type of possible values $\varphi(x)$ in $C$ as
  \[
    \sum_{z:C}\exists_{y:A}(f(y)=x) \wedge g(y)=z
  \]
  which is a proposition by algebraic calculation.
  By surjectivity of $f$, this type is inhabited and therefore contractible.
  So we can define $\varphi(x)$ as its center of contraction.
\end{proof}

\ignore{
    - injective/embedding/-1-truncated map
  pushouts:
    - inclusions are jointly surjective,
    - pushouts of embeddings between sets are sets
  subtypes:
    - embeddings (composition, multiple definitions, relation to injection)  
    - we freely switch between predicates and types
    - subtypes of subtypes are subtypes
  pullbacks:
    - pasting (reference)
    - pullback of subtype = composition
  algebra:
    - free comm algebras, quotients
    - other definitions of polynomials
    - fp closed under: quotients, adjoining variables, tensor products
}


\section{Axioms}
%In this section, we will introduce the basic rules we'll use in this paper. 
%We will first state our axioms, and then draw out some first consequences.
%Most notably, 
%we will see that Markov's principle (\Cref{MarkovPrinciple}) and the 
%lesser limited principle of omniscience (\Cref{LLPO}) can be shown. 
%%There are equivalent axiom systems we could have stated instead, which we discuss in \Cref{NotesOnAxioms}.
\subsection{Axioms}\label{Axioms}
%In this section, we will state our axioms. 
%In \Cref{NotesOnAxioms}, we will discuss alternative versions of our axiom system. 
\begin{axiom}[Stone duality]\label{AxStoneDuality}
  For any $B:\Boole$, 
  the evaluation map $B\rightarrow  2^{Sp(B)}$ is an isomorphism.
\end{axiom} 

%\begin{axiom}[Propositional completeness]
%  For $S:\Stone$, we have that $\neg \neg S \leftrightarrow || S ||$
%\end{axiom}

\begin{axiom}[Surjections are formal surjections]\label{SurjectionsAreFormalSurjections}
  For $g:B\to C$ a map in $\Boole$, $g$ is injective if and only if
  $(-)\circ g: Sp(C) \to Sp(B)$ is surjective. 
%  A map $f:Sp(B')\to Sp(B)$ is surjective iff the corresponding map $B \to B'$ is injective.
\end{axiom} 
%
%\begin{lemma}\label{LemSurjectionsFormalToCompleteness}
% For $S:\Stone$, we have that $\neg \neg S \to || S ||$
%\end{lemma}
%\begin{proof}
%  First, assume that surjections are formal surjections. 
%  Let $B:\Boole$ and suppose $\neg \neg Sp(B)$. 
%  %Note that if $0=1$ in $B$, then $Sp(B) =\emptyset$, meaning $\neg Sp(B)$. 
%  %Therefore, we have $0\neq 1$ in $B$. 
%  We will show that the map $f:2\to B$ is injective. 
%  Let $f:2 \to B$, note that if $f(0) = f(1)$ then $0=1$ in $B$, 
%  If $0=1$ in $B$, there are no maps $B\to 2$ preserving $0$ and $1$, thus $\neg Sp(B)$. 
%  This is a contradiction with $\neg \neg Sp(B)$. Thus we may conclude that $f(0)\neq f(1)$. 
%  Hence by case distinction on $2$ we can show $f$ we have that $f x = f y$ implies $ x= y$. Thus 
%  $f$ is injective thus the map $Sp(B) \to Sp(2) = 1$ is surjective, thus $Sp(B)$ is merely inhabited. 
%\end{proof} 
%Actually, we will see in \Cref{CorDoubleNegToAx2} that the converse is also true. 

%\begin{axiom}[Local choice]
%  Whenever $S$ Stone and $E\twoheadrightarrow S$ surjective, then there is some $T$ Stone,
%    a surjection $T \twoheadrightarrow S$ and a map $T\to E$ 
%    such that the following diagram commutes:
%    \begin{equation}\begin{tikzcd}
%      E \arrow[d,""',two heads]\\
%      S & \arrow[l, "", two heads, dashed] T\arrow[lu, ""',dashed ]
%    \end{tikzcd}\end{equation}  
%\end{axiom} 
%\begin{axiom}[Local choice]\label{AxLocalChoice}
%  Whenever we have $S:\Stone$, $E,F$ arbitrary types, a map $f:S \to F$ and a 
%  surjection $e:E \twoheadrightarrow F$, 
%  there exists a Stone space $T$, a surjective map 
%  $T\twoheadrightarrow S$ and an arrow $T\to E$ making the following diagram commute:
%    \begin{equation}\begin{tikzcd}
%      T \arrow[d,dashed, two heads ] \arrow[r,dashed]&  E \arrow[d,""',two heads, "e"]\\
%      S  \arrow[r, "f"] & F
%    \end{tikzcd}\end{equation}  
%\end{axiom}

%\begin{axiom}[Local choice]\label{AxLocalChoice}
%  Whenever we have $S:\Stone$, $X$ an arbitrary type and a predicate $P:S \times X \to \Prop$, 
%  such that $\forall_{s:S} \exists_{x:X} P(s,x)$, then there merely exists some $T:\Stone$ with surjection 
%  $q:T\twoheadrightarrow S$ and a function $\Pi_{t:T} \Sigma_{x:X} P(q(t), x)$. 
%\end{axiom}
%\begin{axiom}[Local choice]\label{AxLocalChoice}
%  Whenever we have $S:\Stone$, and some type family $P:S\to\Type$ such that 
%  $\Pi_{s:S} ||P s||$, then there 
%  merely exists some $T:\Stone$ and surjection $q:T\to S$ with 
%$  \Pi_{t:T} P(q(t))$.
%\end{axiom}
\begin{axiom}[Local choice]\label{AxLocalChoice}
  Whenever we have $B:\Boole$, and some type family $P$ over $Sp(B)$ with 
  $\Pi_{s:Sp(B)} \propTrunc{P(s)}$, then there 
  merely exists some $C:\Boole$ and surjection $q:Sp(C)\to Sp(B)$ with 
$  \Pi_{t:Sp(C)} P(q(t))$.
\end{axiom}

\begin{axiom}[Dependent choice]\label{axDependentChoice}
Given types $(E_n)_{n:\N}$ with for all $n:\N$ a surjection $E_{n+1}\twoheadrightarrow E_n$, the projection from the sequential limit $\lim_kE_k$ to $E_0$ is surjective.
\end{axiom}
%\begin{remark}
%  Local choice can also be formulated as follows:
%  whenever we have $S:\Stone$, $E,F$ arbitrary types, a map $f:S \to F$ and a 
%  surjection $e:E \twoheadrightarrow F$, 
%  there exists a Stone space $T$, a surjective map 
%  $T\twoheadrightarrow S$ and an arrow $T\to E$ making the following diagram commute:
%    \begin{equation}\begin{tikzcd}
%      T \arrow[d,dashed, two heads ] \arrow[r,dashed]&  E \arrow[d,""',two heads, "e"]\\
%      S  \arrow[r, "f"] & F
%    \end{tikzcd}\end{equation}  
%\end{remark}


\section{Affine schemes}
\input{affine}

\section{Topology of schemes}%
\label{topology-of-schemes}

\subsection{Closed subtypes}

\begin{definition}%
  \label{closed-proposition}\label{closed-subtype}
  \begin{enumerate}[(a)]
  \item
    A \notion{closed proposition} is a proposition
    which is merely of the form $x_1 = 0 \land \dots \land x_n = 0$
    for some elements $x_1, \dots, x_n \in R$.
  \item
    Let $X$ be a type.
    A subtype $U : X \to \Prop$ is \notion{closed}
    if for all $x : X$, the proposition $U(x)$ is closed.
  \item
    For $A$ a finitely presented $R$-algebra
    and $f_1, \dots, f_n : A$,
    we set
    $V(f_1, \dots, f_n) \colonequiv
    \{\, x : \Spec A \mid f_1(x) = \dots = f_n(x) = 0 \,\}$.
  \end{enumerate}
\end{definition}

Note that $V(f_1, \dots, f_n) \subseteq \Spec A$ is a closed subtype
and we have $V(f_1, \dots, f_n) = \Spec (A/(f_1, \dots, f_n))$.

\begin{proposition}[using \axiomref{sqc}]%
  There is an order-reversing isomorphism of partial orders
  \begin{align*}
    \text{f.g.-ideals}(R) &\xrightarrow{{\sim}} \Omega_{cl} \\
    I &\mapsto (I = (0))
  \end{align*}
  between the partial order of finitely generated ideals of $R$
  and the partial order of closed propositions.
\end{proposition}

\begin{proof}
  For a finitely generated ideal $I = (x_1, \dots, x_n)$,
  the proposition $I = (0)$ is indeed a closed proposition,
  since it is equivalent to $x_1 = 0 \land \dots \land x_n = 0$.
  It is also evident that we get all closed propositions in this way.
  What remains to show is that
  \[ I = (0) \Rightarrow J = (0)
     \qquad\text{iff}\qquad
     J \subseteq I
     \rlap{\text{.}}
  \]
  For this we use synthetic quasicoherence.
  Note that the set $\Spec R/I = \Hom_{\Alg{R}}(R/I, R)$ is a proposition
  (has at most one element),
  namely it is equivalent to the proposition $I = (0)$.
  Similarly, $\Hom_{\Alg{R}}(R/J, R/I)$ is a proposition
  and equivalent to $J \subseteq I$.
  But then our claim is just the equation
  \[ \Hom(\Spec R/I, \Spec R/J) = \Hom_{\Alg{R}}(R/J, R/I) \]
  which holds by \Cref{spec-embedding},
  since $R/I$ and $R/J$ are finitely presented $R$-algebras
  if $I$ and $J$ are finitely generated ideals.
\end{proof}

\begin{lemma}[using \axiomref{sqc}]%
  \label{ideals-embed-into-closed-subsets}
  We have $V(f_1, \dots, f_n) \subseteq V(g_1, \dots, g_m)$
  as subsets of $\Spec A$
  if and only if
  $(g_1, \dots, g_m) \subseteq (f_1, \dots, f_n)$
  as ideals of $A$.
\end{lemma}

\begin{proof}
  The inclusion $V(f_1, \dots, f_n) \subseteq V(g_1, \dots, g_m)$
  means a map $\Spec (A/(f_1, \dots, f_n)) \to \Spec (A/(g_1, \dots, g_m))$
  over $\Spec A$.
  By \Cref{spec-embedding}, this is equivalent to
  a homomorphism $A/(g_1, \dots, g_m) \to A/(f_1, \dots, f_n)$,
  which in turn means the stated inclusion of ideals.
\end{proof}

\begin{lemma}[using \axiomref{loc}, \axiomref{sqc}, \axiomref{Z-choice}]%
  \label{closed-subtype-affine}
  A closed subtype $C$ of an affine scheme $X=\Spec A$ is an affine scheme
  with $C=\Spec (A/I)$ for a finitely generated ideal $I\subseteq A$.
\end{lemma}

\begin{proof}
  By \axiomref{Z-choice} and boundedness,
  there is a cover $D(f_1),\dots,D(f_l)$, such that
  on each $D(f_i)$, $C$ is the vanishing set of functions
  \[ g_1,\dots,g_n:D(f_i)\to R\rlap{.} \]
  By \Cref{ideals-embed-into-closed-subsets},
  the ideals generated by these functions
  agree in $A_{f_i f_j}$,
  so by \Cref{fg-ideal-local-global},
  there is a finitely generated ideal $I\subseteq A$,
  such that $A_{f_i}\cdot I$ is $(g_1,\dots,g_n)$
  and $C=\Spec A/I$.
\end{proof}

\subsection{Open subtypes}

While we usually drop the prefix ``qc'' in the definition below,
one should keep in mind, that we only use a definition of quasi compact open subsets.
The difference to general opens does not play a role so far,
since we also only consider quasi compact schemes later.

\begin{definition}%
  \label{qc-open}
  \begin{enumerate}[(a)]
  \item A proposition $P$ is \notion{(qc-)open}, if there merely are $f_1,\dots,f_n:R$,
    such that $P$ is equivalent to one of the $f_i$ being invertible.
  \item Let $X$ be a type.
    A subtype $U:X\to\Prop$ is \notion{(qc-)open}, if $U(x)$ is an open proposition for all $x:X$.
  \end{enumerate}
\end{definition}

\begin{proposition}[using \axiomref{loc}, \axiomref{sqc}]%
  \label{open-iff-negation-of-closed}
  A proposition $P$ is open
  if and only if
  it is the negation of some closed proposition
  (\Cref{closed-proposition}).
\end{proposition}

\begin{proof}
  Indeed, by \Cref{generalized-field-property},
  the proposition $\inv(f_1) \lor \dots \lor \inv(f_n)$
  is the negation of ${f_1 = 0} \land \dots \land {f_n = 0}$.
\end{proof}

\begin{proposition}[using \axiomref{loc}, \axiomref{sqc}]%
  \label{open-union-intersection}
  Let $X$ be a type.
  \begin{enumerate}[(a)]
  \item The empty subtype is open in $X$.
  \item $X$ is open in $X$.
  \item Finite intersections of open subtypes of $X$ are open subtypes of $X$.
  \item Finite unions of open subtypes of $X$ are open subtypes of $X$.
  \item Open subtypes are invariant under pointwise double-negation.
  \end{enumerate}
  Axioms are only needed for the last statement.
\end{proposition}

In \Cref{open-subscheme} we will see that open subtypes of open subtypes of a scheme are open in that scheme.
Which is equivalent to open propositions being closed under dependent sums.

\begin{proof}[of \Cref{open-union-intersection}]
  For unions, we can just append lists.
  For intersections, we note that invertibility of a product
  is equivalent to invertibility of both factors.
  Double-negation stability
  follows from \Cref{open-iff-negation-of-closed}.
\end{proof}

\begin{lemma}%
  \label{preimage-open}
  Let $f:X\to Y$ and $U:Y\to\Prop$ open,
  then the \notion{preimage} $U\circ f:X\to\Prop$ is open.
\end{lemma}

\begin{proof}
  If $U(y)$ is an open proposition for all $y : Y$,
  then $U(f(x))$ is an open proposition for all $x : X$.
\end{proof}

\begin{lemma}[using \axiomref{loc}, \axiomref{sqc}]%
  \label{open-inequality-subtype}
  Let $X$ be affine and $x:X$, then the proposition
  \[ x\neq y \]
  is open for all $y:X$.
\end{lemma}

\begin{proof}
  We show a proposition, so we can assume $\iota: X\to \A^n$ is a subtype.
  Then for $x,y:X$, $x\neq y$ is equivalent to $\iota(x)\neq\iota(y)$.
  But for $x,y:\A^n$, $x\neq y$ is the open proposition that $x-y\neq 0$.
\end{proof}

The intersection of all open neighborhoods of a point in an affine scheme,
is the formal neighborhood of the point.
We will see in \Cref{intersection-of-all-opens}, that this also holds for schemes.

\begin{lemma}[using \axiomref{loc}, \axiomref{sqc}]%
  \label{affine-intersection-of-all-opens}
  Let $X$ be affine and $x:X$, then the proposition
  \[ \prod_{U:X\to \Open}U(x)\to U(y) \]
  is equivalent to $\neg\neg (x=y)$.
\end{lemma}

\begin{proof}
  By \Cref{open-union-intersection}, $\neg\neg (x=y)$ implies $\prod_{U:X\to \Open}U(x)\to U(y)$.
  For the other implication,
  $\neg (x=y)$ is open by \Cref{open-inequality-subtype}, so we get a contradiction.
\end{proof}

We now show that our two definitions (\Cref{affine-open}, \Cref{qc-open})
of open subtypes of an affine scheme are equivalent.

\begin{theorem}[using \axiomref{loc}, \axiomref{sqc}, \axiomref{Z-choice}]%
  \label{qc-open-affine-open}
  Let $X=\Spec A$ and $U:X\to\Prop$ be an open subtype,
  then $U$ is affine open, i.e. there merely are $h_1,\dots,h_n:X\to R$ such that
  $U=D(h_1,\dots,h_n)$.
\end{theorem}

\begin{proof}
  Let $L(x)$ be the type of finite lists of elements of $R$,
  such that one of them being invertible is equivalent to $U(x)$.
  By assumption, we know
  \[\prod_{x:X}\propTrunc{L(x)}\rlap{.}\]
  So by \axiomref{Z-choice}, we have $s_i:\prod_{x:D(f_i)}L(x)$.
  We compose with the length function for lists to get functions $l_i:D(f_i)\to\N$.
  By \Cref{boundedness}, the $l_i$ are bounded.
  Since we are proving a proposition, we can assume we have actual bounds $b_i:\N$.
  So we get functions $\tilde{s_i}:D(f_i)\to R^{b_i}$,
  by append zeros to lists which are too short,
  i.e. $\widetilde{s}_i(x)$ is $s_i(x)$ with $b_i-l_i(x)$ zeros appended.

  Then one of the entries of $\widetilde{s}_i(x)$ being invertible,
  is still equivalent to $U(x)$.
  So if we define $g_{ij}(x)\colonequiv \pi_j(\widetilde{s}_i(x))$,
  we have functions on $D(f_i)$, such that
  \[
    D(g_{i1},\dots,g_{ib_i})=U\cap D(f_i)
    \rlap{.}
  \]
  By \Cref{affine-open-trans}, this is enough to solve the problem on all of $X$.
\end{proof}

This allows us to transfer one important lemma from affine-opens to qc-opens.
The subtlety of the following is that while it is clear that the intersection of two
qc-opens on a type, which are \emph{globally} defined is open again, it is not clear,
that the same holds, if one qc-open is only defined on the other.

\begin{lemma}[using \axiomref{loc}, \axiomref{sqc}, \axiomref{Z-choice}]%
  \label{qc-open-trans}
  Let $X$ be a scheme, $U\subseteq X$ qc-open in $X$ and $V\subseteq U$ qc-open in $U$,
  then $V$ is qc-open in $X$.
\end{lemma}

\begin{proof}
  Let $X_i=\Spec A_i$ be a finite affine cover of $X$.
  It is enough to show, that the restriction $V_i$ of $V$ to $X_i$ is qc-open.
  $U_i\colonequiv X_i\cap U$ is qc-open in $X_i$, since $X_i$ is qc-open.
  By \Cref{qc-open-affine-open}, $U_i$ is affine-open in $X_i$,
  so $U_i=D(f_1,\dots,f_n)$.
  $V_i\cap D(f_j)$ is affine-open in $D(f_j)$, so by \Cref{affine-open-trans},
  $V_i\cap D(f_j)$ is affine-open in $X_i$.
  This implies $V_i\cap D(f_j)$ is qc-open in $X_i$ and so is $V_i=\bigcup_{j}V_i\cap D(f_j)$.
\end{proof}

\begin{lemma}[using \axiomref{loc}, \axiomref{sqc}, \axiomref{Z-choice}]%
  \label{qc-open-sigma-closed}
  \begin{enumerate}[(a)]
  \item qc-open propositions are closed under dependent sums:
    if $P : \Open$ and $U : P \to \Open$,
    then the proposition $\sum_{x : P} U(x)$ is also open.
  \item Let $X$ be a type. Any open subtype of an open subtype of $X$ is an open subtype of $X$.
  \end{enumerate}
\end{lemma}

\begin{proof}
  \begin{enumerate}[(a)]
  \item Apply \Cref{qc-open-trans} to the point $\Spec R$.
  \item Apply the above pointwise.
  \end{enumerate}
\end{proof}

\begin{remark}
  \Cref{qc-open-sigma-closed} means that
  the (qc-) open propositions constitute a \notion{dominance}
  in the sense of~\cite{rosolini-phd-thesis}.
\end{remark}

The following fact about the interaction of closed and open propositions
is due to David Wärn.

\begin{lemma}%
  \label{implication-from-closed-to-open}
  Let $P$ and $Q$ be propositions
  with $P$ closed and $Q$ open.
  Then $P \to Q$ is equivalent to $\lnot P \lor Q$.
\end{lemma}

\begin{proof}
  For any propositions $P$, $Q$, it is the case that $\lnot P \lor Q$ implies $P \to Q$ and that $P \to Q$ implies $\lnot \lnot (\lnot P \lor Q)$.
  When $P$ is closed and $Q$ is open, we have that $\lnot P$ is open by \Cref{open-iff-negation-of-closed}, so $\lnot P \lor Q$ is also an open proposition and thus double-negation stable by \Cref{open-union-intersection}.
\end{proof}



\section{Schemes}
\input{schemes}

\section{Projective space}
We follow the notations and setting for Synthetic Algebraic Geometry \cite{draft}.
In particular, $R$ denotes the generic local ring and $R^\times$ is the multiplicative group of units of $R$.

In Synthetic Algebraic Geometry, a scheme is defined as a set satisfying some property \cite{draft}. In particular
the projective space $\bP^n$ can be defined to be the quotient of $R^{n+1}\setminus\{0\}$ by the
equivalence relation $a\sim b$ which expresses that $a$ and $b$ are proportional, %i.e. $a_ib_j=a_jb_i$,
which is equal to $\Sigma_{r:R^\times}ar = b$. We can then prove \cite{draft}
that this set is a scheme. This definition goes back to \cite{Kock74}.

 In this setting, a map of schemes is simply an arbitrary set theoretic map. An application of this work is to show
 that the maps $\bP^n\rightarrow \bP^m$ are given by $m+1$ homogeneous polynomials of the same degree in $n+1$ variables.

\medskip


There is another definition of $\bP^n$ which uses ``higher'' notions. Let $\KR$ be the delooping
of $R^\times$. It can be defined as the type of lines $\Sigma_{M:\Mod{R}}\|{M=R^1}\|$. Over $\KR$ we have the
family of sets
$$E_n(l) = l^{n+1}\setminus\{0\}$$
Note that we use the same notation for an element $l : \KR$,
its underlying $R$-module and its underlying set.
An equivalent definition of $\bP^n$ is then
$$
\bP^n = \sum_{l:\KR}E_n(l)
$$
That is, we replaced the quotient, here a set of orbits for a free group action, by a sum type over the delooping of this group
\cite{Sym}.
More explicitly, we will use the following identifications:
\begin{remark}\label{identification-Pn}
  Projective $n$-space $\bP^n$ is given by the following equivalent constructions of which we prefer
  the first in this article:
  \begin{center}
  
    \begin{enumerate}[(i)]
    \item $\sum_{l:\KR}E_n(l)$
    \item The set-quotient $R^{n+1}\setminus\{0\}/R^\times$, where $R^\times$ acts on non-zero vectors in $R^{n+1}$ by multiplication.
    \item For any $k$ and $R$-module $V$ we define the \emph{Grassmannian}
      \[ \Gr(k,V)\colonequiv \{ U\subseteq V \mid \text{$U$ is an $R$-submodule and $\|U=R^k\|$ }\} \rlap{.}\]
      Projective $n$-space is then $\Gr(1,R^{n+1})$.
    \end{enumerate}
    
  \end{center}
  We use the following, well-defined identifications:
  \begin{center}
  
  \begin{enumerate}
  \item[] (i)$\to$(iii): Map $(l,s)$ to $R\cdot (u s_0,\dots, u s_n)$ where $u:l=R^1$
  \item[] (iii)$\to $(i): Map $L\subseteq R^{n+1}$ to $(L, x)$ for a non-zero $x\in L$
  \item[] (ii)$\leftrightarrow$ (iii): A line through a non-zero $x:R^{n+1}$
          is identified with $[x]:R^{n+1}\setminus\{0\}/R^\times$
  \end{enumerate}
    
  \end{center}
\end{remark}

We construct the standard line bundles $\OO(d)$ for all $d\in\Z$,
which are classically known as \emph{Serre's twisting sheaves} on $\bP^n$ as follows:

\begin{definition}
  For $d:\Z$, the line bundle $\OO(d):\bP^n\to \KR$ is given by $\OO(d)(l,s) = l^{\otimes d}$
  and the following definition of $l^{\otimes d}$ by cases:
  \begin{enumerate}[(i)]
  \item $d \geqslant 0$: $l^{\otimes d}$ using the tensor product of $R$-modules
  \item $d < 0$: $(l^{\vee})^{-d}$, where $l^{\vee}\colonequiv\Hom_{\Mod{R}}(l,R^1)$ is the dual of $l$.
  \end{enumerate}
\end{definition}

This definition of $\OO(d)$ agrees with \cite{draft}[Definition 6.3.2] where $\OO(-1)$
is given on $\Gr(1,R^{n+1})$ by mapping submodules of $R^{n+1}$ to $\KR$.
Using the identification of $\bP^n$ from \Cref{identification-Pn} we can give the following explicit equality:

\begin{remark}
  We have a commutative triangle:
  \begin{center}
    \begin{tikzcd}
        \sum_{l:\KR} E_n(l)\ar[rr]\ar[dr,swap,"\OO(1)"] && R^{n+1}\setminus\{0\}/R^\times\ar[ld,"\OO(1)"] \\
                  & \KR &
    \end{tikzcd}
  \end{center}

by the isomorphism given for $(l,s)$ by mapping $x:l$ to $r(u s_0,\dots, u s_n)\mapsto r(u x)$ for some isomorphism $u:l\cong R^1$.
\end{remark}

\medskip

 Connected to this definition of $\bP^n$, we will prove some equalities in the following.
 To prove these equalities, we will make use of the following lemma, which holds in synthetic algebraic geometry:
 
\begin{lemma}\label{invariant-implies-homogenous}
  Let $n,d:\N$ and $\alpha:R^n\to R$ be a map such that
  \[\alpha(\lambda x)=\lambda^d\alpha(x)\]
  then $\alpha$ is a homogenous polynomial of degree $d$.
\end{lemma}

\begin{proof}
  By duality, any map $\alpha:R^n\to R$ is a polynomial.
  To see it is homogenous of degree $d$, let us first note that any $P:R[\lambda]$ with $P(\lambda)=\lambda^d P(1)$
  for all $\lambda:R^\times$ also satisfies this equation for all $\lambda : R$ and is therefore homogenous of degree $d$.
  Then for $\alpha'_x:R[\lambda]$ given by $\alpha'_x(\lambda)\colonequiv \alpha(\lambda\cdot x)$
  we have $\alpha'_x(\lambda)=\lambda^d \alpha'_x(1)$. This means any coeffiecent of $\alpha'_x$
  of degree different from $d$ is 0. Since this means every monomial appearing in $\alpha$,
  which is not of degree $d$, is zero for all $x$ and therefore 0.   
\end{proof}

\begin{proposition}\label{end}
  $$\prod_{l:\KR}l^n\rightarrow l \;\;\;=\;\;\; \Hom(R^n,R)$$
\end{proposition}

\begin{proof}
We rewrite $\Hom(R^n,R)$, the set of $R$-module morphism, as
$$
\sum_{\alpha:R^n\rightarrow R}\prod_{\lambda:R^\times}\prod_{x:R^n}\alpha(\lambda x) = \lambda \alpha(x)
$$
using \Cref{invariant-implies-homogenous} with $d=1$.

\medskip

It is then a general fact that if we have a pointed connected groupoid $(A,a)$ and a family of
sets $T(x)$ for $x:A$, then $\prod_{x:A}T(x)$ is the set of fixedpoints of $T(a)$ for the $(a=a)$ action
\cite{Sym}.
\end{proof}

We will use the following remark, proved in \cite{draft}[Remark 6.2.5].

\begin{lemma}\label{ext}
  Any map $R^{n+1}\setminus\{0\}\rightarrow R$ can be uniquely extended to a map $R^{n+1}\rightarrow R$ for $n>0$.
\end{lemma}

We will also use the following proposition, already noticed in \cite{draft}.

\begin{proposition}\label{const}
  Any map from $\bP^n$ to $R$ is constant.
\end{proposition}

\begin{proof}
  Since $\bP^n$ is a quotient of $R^{n+1}\setminus\{0\}$, the set $\bP^n\rightarrow R$ is
  the set of maps $\alpha:R^{n+1}\setminus\{0\}\rightarrow R$
  such that $\alpha(\lambda x) = \alpha(x)$ for all $\lambda$ in $R^\times$.
  These are exactly the constant maps
  using \Cref{ext} and \Cref{invariant-implies-homogenous} with $d=0$.
\end{proof}

\begin{proposition}\label{aut}
  For all $n:\N$ we have:
$$\prod_{l:\KR}E_n(l)\rightarrow E_n(l) \;\;=\;\; \GL_{n+1}$$
\end{proposition}

\begin{proof}
  For $n=0$, this is the direct computation that a Laurent-polynomial $\alpha:(R[X,1/X])^\times$ which satisfies
  $\alpha(\lambda x)=\lambda \alpha(x)$ is $\lambda\alpha(1)$ where $\alpha(1):R^\times=\GL_1$.
  
  \medskip
  
  For $n>0$, the proposition follows from two remarks.

  The first remark is that maps $E_n(R)\to E_n(R)$, which are invariant under the induced $\KR$ action, are linear.
  To prove this remark, we first map from $E_n(l)\to E_n(l)$ to $E_n(l)\to l^{n+1}$ by composing with the inclusion.
  Maps of the latter kind can be uniquely extended to maps $l^{n+1}\to l^{n+1}$, since by 
  \Cref{ext} the restriction map
$$
(l^{n+1}\rightarrow l)\rightarrow ((l^{n+1}\setminus\{0\})\rightarrow l)
$$
is a bijection for $n>0$ and all $l:\KR$.

\medskip

The second remark is that a linear map $u:R^{m}\rightarrow R^{m}$ such that
$$
x\neq 0~\rightarrow~u(x)\neq 0
$$
is exactly an element of $\GL_{m}$.

We show this by induction on $m$. For $m=1$ we have $u(1)\neq 0$ iff $u(1)$ invertible.

For $m>1$, we look at $u(e_1) = \Sigma \alpha_ie_i$ with $e_1,\dots,e_m$ basis of $R^m$.
We have that some $\alpha_j$ is invertible.
By composing $u$ with an element in $\GL_m$, we can then
assume that $u(e_1) = e_1+v_1$ and $u(e_i) = v_i$, for $i>1$, with $v_1,\dots,v_m$ in $Re_2+\dots+Re_m$.
We can then conclude by induction.
\end{proof}

We can generalize \Cref{end}
and get a result related to \Cref{aut} as follows.
 
\begin{lemma}\label{hom}
  \begin{enumerate}[(i)]
    \item
      \[  \prod_{l:\KR}l^n\rightarrow l^{\otimes d} \;\;=\;\; (R[X_1, \dots, X_n])_d \]
      That is,
      every element of the left-hand side is given by
      a unique homogeneous polynomial of degree $d$ in $n$ variables.
    \item
      An element in
      $$\prod_{l:\KR}E_n(l)\rightarrow E_m(l^{\otimes d})$$
      is given by $m+1$ homogeneous polynomials $p = (p_0,\dots,p_m)$ of degree $d$ such that
      $x\neq 0$ implies $p(x)\neq 0$.
  \end{enumerate}
\end{lemma}

\begin{proof}
We show the first item. Following \cite{Sym} again, this product is the set of maps $\alpha:R^n\rightarrow R^{\otimes d}$
which are invariant by the $R^\times$-action which in this case acts by mapping $\alpha$ to $r^d\alpha(r^{-1} x)$ for each $r:R^\times$.
So by \Cref{invariant-implies-homogenous} these are exactly the maps given by homogeneous polynomials of degree $d$.
\end{proof}


\section{Bundles and cohomology}
In this section we compute $H^1(S,\Z) = 0$ for all $S$ Stone, and show that $H^1(X,\Z)$ for $X$ compact Hausdorff can be computed using \v{C}ech cohomology. We use this to compute $H^1(\I,\Z)=0$. 

\begin{remark}
We only work with the first cohomology group with coefficients in $\Z$ as it is sufficient for the proof of Brouwer's fixed-point theorem, but the results could be extended to $H^n(X,A)$ for $A$ any family of countably presented abelian groups indexed by $X$.
\end{remark}

\begin{remark}
We write $\mathrm{Ab}$ for the type of abelian groups and if $G:\mathrm{Ab}$ we write $\B G$ for the delooping of $G$ \cite{hott,davidw23}. This means that $H^1(X,G)$ is the set truncation of $X \to \B G$. 
\end{remark}

\subsection{\v{C}ech cohomology}

\begin{definition}
Given a type $S$, types $T_x$ for $x:S$ and $A:S\to\mathrm{Ab}$, we define $\check{C}(S,T,A)$ as the chain complex
\[
\begin{tikzcd}
     \prod_{x:S}A_x^{T_x} \ar[r,"d_0"] & \prod_{x:S}A_x^{T_x^2}\ar[r,"d_1"] &  \prod_{x:S}A_x^{T_x^3}
\end{tikzcd}
\]
where the boundary maps are defined as
\begin{align*}
d_0(\alpha)_x(u,v) =&\ \alpha_x(v)-\alpha_x(u)\\
d_1(\beta)_x(u,v,w) =&\ \beta_x(v,w) - \beta_x(u,w) + \beta_x(u,v)
\end{align*}
\end{definition}

\begin{definition}
Given a type $S$, types $T_x$ for $x:S$ and $A:S\to\mathrm{Ab}$, we define its \v{C}ech cohomology groups by
\[
  \check{H}^0(S,T,A) = \mathrm{ker}(d_0)\quad \quad \quad \check{H}^1(S,T,A) = \mathrm{ker}(d_1)/\mathrm{im}(d_0)
\]
We call elements of $\mathrm{ker}(d_1)$ cocycles and elements of $\mathrm{im}(d_0)$ coboundaries.
\end{definition}

This means that $\check{H}^1(S,T,A) = 0$ if and only if $\check{C}(S,T,A)$ is exact at the middle term. Now we give three general lemmas about \v{C}ech complexes.

\begin{lemma}\label{section-exact-cech-complex}
Assume a type $S$, types $T_x$ for $x:S$ and $A:S\to\mathrm{Ab}$ with $t:\prod_{x:S}T_x$. Then $\check{H}^1(S,T,A)=0$.
\end{lemma}

\begin{proof}
Assume given a cocycle, i.e. $\beta:\prod_{x:S}A_x^{T_x^2}$ such that for all $x:S$ and $u,v,w:T_x$ we have that $\beta_x(u,v)+\beta_x(v,w) = \beta_x(u,w)$. We define $\alpha:\prod_{x:S}A_x^{T_x}$ by $\alpha_x(u) = \beta_x(t_x,u)$. Then for all $x:S$ and $u,v:T_x$ we have that $d_0(\alpha)_x(u,v) =  \beta_x(t_x,v) - \beta_x(t_x,u) = \beta_x(u,v)$ so that $\beta$ is a coboundary.
\end{proof}

\begin{lemma}\label{canonical-exact-cech-complex}
Given a type $S$, types $T_x$ for $x:S$ and $A:S\to\mathrm{Ab}$, we have that $\check{H}^1(S,T,\lambda x.A_x^{T_x})=0$.
\end{lemma}

\begin{proof}
Assume given a cocycle, i.e. $\beta:\prod_{x:S}A_x^{T_x^3}$ such that for all $x:S$ and $u,v,w,t:T_x$ we have that $\beta_x(u,v,t)+\beta_x(v,w,t) = \beta_x(u,w,t)$. We define $\alpha:\prod_{x:S}A_x^{T_x^2}$ by $\alpha_x(u,t) = \beta_x(t,u,t)$. Then for all $x:S$ and $u,v,t:T_x$ we have that $d_0(\alpha)_x(u,v,t) = \beta_x(t,v,t) - \beta_x(t,u,t) = \beta_x(u,v,t)$ so that $\beta$ is a coboundary.
\end{proof}

\begin{lemma}\label{exact-cech-complex-vanishing-cohomology}
Assume a type $S$ and types $T_x$ for $x:S$ such that $\prod_{x:S}\propTrunc{T_x}$ and $A:S\to\mathrm{Ab}$ such that $\check{H}^1(S,T,A) = 0$.
Then given $\alpha:\prod_{x:S}\B A_x$ with $\beta:\prod_{x:S} (\alpha(x) = *)^{T_x}$, we can conclude $\alpha = *$.
\end{lemma}

\begin{proof}
We define $g : \prod_{x:S} A_x^{T_x^2}$ by $g_x(u,v) = \beta_x(v) - \beta_x(u)$.
It is a cocycle in the \v{C}ech complex, so that by exactness there is $f:\prod_{x:S}A_x^{T_x}$ such that for all $x:S$ and $u,v:T_x$ we have that $g_x(u,v)= f_x(v) - f_x(u)$.
Then we define $\beta' : \prod_{x:S}(\alpha(x)=*)^{T_x}$ by $\beta'_x(u) = \beta_x(u) - f_x(u)$
so that for all $x:S$ and $u,v:T_x$ we have that $\beta'_x(u) = \beta'_x(v)$ is equivalent to $f_x(v) - f_x(u) = \beta_x(v) - \beta_x(u)$, which holds by definition. So $\beta'$ is constant on each $T_x$ and therefore gives $\prod_{x:S} (\alpha(x)=*)^{\propTrunc{T_x}}$. By $\prod_{x:S}\propTrunc{T_x}$ we conclude $\alpha = *$.
\end{proof}


\subsection{Cohomology of Stone spaces}

%
%%\subsection{Needed results}
%
%%\rednote{Should probably be moved elsewhere}
%
\begin{lemma}\label{finite-approximation-surjection-stone}
Assume given $S:\Stone$ and $T:S\to\Stone$ such that $\prod_{x:S}\propTrunc{T(x)}$.
Then there exists a sequence of finite types $(S_k)_{k:\N}$ with limit $S$ 
%\rednote{Should the maps in the sequence be mentioned? (Maps $p_k$ are mentioned below)}
%\begin{equation}
%\begin{tikzcd}
%S_0 & S_1 \ar[l,"p_0"]& S_2\ar[l,"p_1"] & \cdots\ar[l]\\
%\end{tikzcd}
%\end{equation}
%such that: 
%\[\mathrm{lim}_kS_k = S\]
and a compatible sequence of families of finite types $T_k$ over $S_k$
with $\prod_{x:S_k}\propTrunc{T_k(x)}$ and 
$\mathrm{lim}_k\left(\sum_{x:S_k}T_k(x)\right) = \sum_{x:S}T(x)$. 
%
%Given $S:\Stone$ and $T:S\to\Stone$ such that $\prod_{x:S}\propTrunc{T(x)}$, there exists a sequence of finite types $(S_k)_{k:\N}$
%\rednote{Should the maps in the sequence be mentioned? (Maps $p_k$ are mentioned below)}
%%\begin{equation}
%%\begin{tikzcd}
%%S_0 & S_1 \ar[l,"p_0"]& S_2\ar[l,"p_1"] & \cdots\ar[l]\\
%%\end{tikzcd}
%%\end{equation}
%such that: 
%\[\mathrm{lim}_kS_k = S\]
%and for each $k:\N$ we have a family of finite types $T_k(x)$ for $x:S_k$ such that $\prod_{x:S_k}\propTrunc{T_k(x)}$ with maps $T_{k+1}(x) \to T_k(p_k(x))$ such that:
%\[\mathrm{lim}_k\left(\sum_{x:S_k}T_k(x)\right) = \sum_{x:S}T(x)\]
\end{lemma}

\begin{proof}
By theorem \Cref{stone-sigma-closed} and the usual correspondence between surjections and families of inhabited types, a family of inhabited Stone spaces over $S$ correspond to a Stone space $T$ with a surjection $T\to S$. Then we conclude using \Cref{ProFiniteMapsFactorization}.
%\rednote{ \@ Hugo This follows from \Cref{ProFiniteMapsFactorization} and \Cref{stone-sigma-closed} 
%  and considering the surjection $(\Sigma_{x:S} T(x)) \to S$, but we discussed whether it might be easier to 
%  refactor the proof where you use the above or make a remark after \Cref{stone-sigma-closed}}
\end{proof}

\begin{lemma}\label{cech-complex-vanishing-stone}
Assume given $S:\Stone$ with $T:S\to\Stone$ such that $\prod_{x:S}\propTrunc{T_x}$. Then we have that $\check{H}^1(S,T,\Z) = 0$.
\end{lemma}


\begin{proof}
We apply \cref{finite-approximation-surjection-stone} to get $S_k$ and $T_k$ finite. Then by \cref{scott-continuity} we have that $\check{C}(S,T,\Z)$ is the sequential colimit of the $\check{C}(S_k,T_k,\Z)$. By \cref{section-exact-cech-complex} we have that each of the $\check{C}(S_k,T_k,\Z)$ is exact, and a sequential colimit of exact sequences is exact.
\end{proof}

\begin{lemma}\label{eilenberg-stone-vanish}
Given $S:\Stone$, we have that $H^1(S,\Z) = 0$. 
\end{lemma}

\begin{proof}
Assume given a map $\alpha:S\to \B\Z$. We use local choice to get $T:S\to\Stone$ such that $\prod_{x:S}\propTrunc{T_x}$ with $\beta:\prod_{x:S}(\alpha(x)=*)^{T_x}$. Then we conclude by \cref{cech-complex-vanishing-stone} and \cref{exact-cech-complex-vanishing-cohomology}.
\end{proof}

\begin{corollary}\label{stone-commute-delooping}
For any $S:\Stone$ the canonical map $\B(\Z^S) \to (\B\Z)^S$ is an equivalence.
\end{corollary}

\begin{proof}
This map is always an embedding. To show it is surjective it is enough to prove that $(\B\Z)^S$ is connected, which is precisely \Cref{eilenberg-stone-vanish}.
\end{proof}


\subsection{\v{C}ech cohomology of compact Hausdorff spaces}

\begin{definition}
A \v{C}ech cover consists of $X:\CHaus$ and $S:X\to\Stone$ such that $\prod_{x:X}\propTrunc{S_x}$ and $\sum_{x:X}S_x:\Stone$.
\end{definition}

By definition any compact Hausdorff space $X$ is part of a \v{C}ech cover $(X,S)$.

\begin{lemma}\label{cech-eilenberg-0-agree}
Given a \v{C}ech cover $(X,S)$ and $A:X\to\mathrm{Ab}$, we have an isomorphism $H^0(X,A) = \check{H}^0(X,S,A)$ natural in $A$.
\end{lemma}

\begin{proof}
By definition an element in $\check{H}^0(X,S,A)$ is a map $f:\prod_{x:X}A_x^{S_x}$
such that for all $u,v:S_x$ we have $f(u)=f(v)$. Since $A_x$ is a set and the $S_x$ are merely inhabited, this is equivalent to $\prod_{x:X}A_x$. Naturality in $A$ is immediate.
\end{proof}

\begin{lemma}\label{eilenberg-exact}
Given a \v{C}ech cover $(X,S)$ we have an exact sequence
\[H^0(X,\lambda x.\Z^{S_x}) \to H^0(X,\lambda x.\Z^{S_x}/\Z) \to H^1(X,\Z)\to 0\]
\end{lemma}

\begin{proof}
We use the long exact cohomology sequence associated to
\[0 \to \Z \to \Z^{S_x} \to \Z^{S_x}/\Z\to 0\]
We just need $H^1(X,\lambda x.\Z^{S_x}) = 0$ to conclude. But by \cref{stone-commute-delooping} we have that $H^1(X,\lambda x.\Z^{S_x}) = H^1\left(\sum_{x:X}S_x,\Z\right)$ which vanishes by \cref{eilenberg-stone-vanish}.
\end{proof}

\begin{lemma}\label{cech-exact}
Given a \v{C}ech cover $(X,S)$ we have an exact sequence
\[\check{H}^0(X,S,\lambda x.\Z^{S_x}) \to \check{H}^0(X,S,\lambda x.\Z^{S_x}/\Z) \to \check{H}^1(X,S,\Z)\to 0\]
\end{lemma}

\begin{proof}
For $n=1,2,3$, we have that $\Sigma_{x:X}S_x^n$ is Stone so that  $H^1(\Sigma_{x:X}S_x^n, \Z) = 0$ by \cref{eilenberg-stone-vanish}, giving short exact sequences
\[0\to \Pi_{x:X}\Z^{S_x^n} \to \Pi_{x:X}(\Z^{S_x})^{S_x^n}\to \Pi_{x:X}(\Z^{S_x}/\Z)^{S_x^n}\to 0\]
They fit together in a short exact sequence of complexes
\[0 \to \check{C}(X,S,\Z) \to \check{C}(X,S,\lambda x.\Z^{S_x}) \to \check{C}(X,S,\lambda x.\Z^{S_x}/\Z)\to 0\]
But since $\check{H}^1(X,\lambda x.\Z^{S_x}) = 0$ by \cref{canonical-exact-cech-complex}, we conclude using the associated long exact sequence.
\end{proof}

\begin{theorem}\label{cech-eilenberg-1-agree}
Given a \v{C}ech cover $(X,S)$, we have that $H^1(X,\Z) = \check{H}^1(X,S,\Z)$
\end{theorem}

\begin{proof}
By applying \cref{cech-eilenberg-0-agree}, \cref{eilenberg-exact} and \cref{cech-exact} we get that $H^1(X,\Z)$ and $\check{H}^1(X,S,\Z)$ are cokernels of isomorphic maps, so they are isomorphic.
\end{proof}

This means that \v{C}ech cohomology does not depend on $S$.

\subsection{Cohomology of the interval}
%
%Recall that we denote $C_n=2^n$ with a binary relation $\sim_n$ on $C_n$ such that for all $x,y:2^\N$ we have that:
%\[\left(\forall(n:\N).\ x|_n\sim_n y|_n\right) \leftrightarrow x=_\I y\]
%
%\begin{lemma}\label{description-Cn-simn}
%We have that $(C_n,\sim_n)$ is equivalent to $(\Fin(2^n),\lambda x,y.\ |x-y|\leq 1)$.
%\end{lemma}
\begin{remark}\label{description-Cn-simn}
  Recall from \Cref{def-cs-Interval} that 
  there is a binary relation $\sim_n$ on $2^n=:\I_n$ such that 
  $(2^n,\sim_n)$ is equivalent to  $(\Fin(2^n),\lambda x,y.\ |x-y|\leq 1)$
  and for $\alpha,\beta:2^\N$ we have $(cs(\alpha) = cs(\beta)) \leftrightarrow 
  \left(\forall_{n:\N}\alpha|_n \sim_n \beta|_n\right)$. 
\end{remark}

We define $\I_n^{\sim2} = \Sigma_{x,y:\I_n}x\sim_n y$ and $\I_n^{\sim3} = \Sigma_{x,y,z:\I_n}x\sim_n y \land y\sim_n z\land x\sim_n z$.

\begin{lemma}\label{Cn-exact-sequence}
For any $n:\N$ we have an exact sequence
\[0\to \Z\overset{d_0}{\longrightarrow} \Z^{\I_n} \overset{d_1}{\longrightarrow} \Z^{\I_n^{\sim2}} \overset{d_2}{\longrightarrow} \Z^{\I_n^{\sim3}}\]
where $d_0(k) = (\_\mapsto k)$ and
\begin{eqnarray}
 d_1(\alpha)(u,v) &=& \alpha(v)-\alpha(u)\nonumber\\
 d_2(\beta)(u,v,w) &=& \beta(v,w)-\beta(u,w)+\beta(u,v).\nonumber
\end{eqnarray}
\end{lemma}

\begin{proof}
It is clear that the map $\Z\to \Z^{\I_n}$ is injective as $\I_n$ is inhabited, so the sequence is exact at $\Z$. Assume a cocycle $\alpha:\Z^{\I_n}$, meaning that for all $u,v:\I_n$, if $u\sim_nv$ then $\alpha(u)=\alpha(v)$. Then by \cref{description-Cn-simn} we see that $\alpha$ is constant, so the sequence is exact at $\Z^{\I_n}$.

Assume a cocycle $\beta:\Z^{\I_n^{\sim2}}$, meaning that for all $u,v,w:\I_n$ such that $u\sim_nv$, $v\sim_nw$ and $u\sim_nw$ we have that $\beta(u,v)+\beta(v,w) = \beta(u,w)$. %This is equivalent to asking $\beta(u,u)=0$ and $\beta(u,v) = -\beta(v,u)$.
Using \cref{description-Cn-simn} to pass along the equivalence between $2^n$ and $\Fin(2^n)$, we define $\alpha(k) = \beta(0,1)+\cdots+\beta(k-1,k)$.
We can check that $\beta(k,l) = \alpha(l)-\alpha(k)$, so that $\beta$ is indeed a coboundary and the sequence is exact at $\Z^{\I_n^{\sim2}}$.
\end{proof}

\begin{proposition}\label{cohomology-I}
We have that $H^0(\I,\Z) = \Z$ and $H^1(\I,\Z) = 0$.
\end{proposition}

\begin{proof}
Consider $cs:2^\N\to\I$ and the associated \v{C}ech cover $T$ of $\I$ defined by: 
\[T_x = \Sigma_{y:2^\N} (x=_\I cs(y))\]
Then for $l=2,3$ we have that $\mathrm{lim}_n\I_n^{\sim l} = \sum_{x:\I} T_x^l$. By \cref{Cn-exact-sequence} and stability of exactness under sequential colimit, we have an exact sequence
\[ 0\to \Z\to \mathrm{colim}_n \Z^{\I_n} \to \mathrm{colim}_n \Z^{\I_n^{\sim2}}\to \mathrm{colim}_n \Z^{\I_n^{\sim3}}\]
By \cref{scott-continuity} this sequence is equivalent to
\[ 0\to \Z\to \Pi_{x:\I}\Z^{T_x} \to  \Pi_{x:\mathbb{I}}\Z^{T_x^2} \to  \Pi_{x:\mathbb{I}}\Z^{T_x^3}\]
So it being exact implies that $\check{H}^0(\I,T,\Z) = \Z$ and $\check{H}^1(\I,T,\Z) = 0$.
We conclude by \cref{cech-eilenberg-0-agree} and \cref{cech-eilenberg-1-agree}.
\end{proof}

\begin{remark}
We could carry a similar computation for $\mathbb{S}^1$, by approximating it with $2^n$ with $0^n\sim_n1^n$ added. We would find $H^1(\mathbb{S}^1,\Z)=\Z$. We will give an alternative, more conceptual proof in the next section.
\end{remark}


\subsection{Brouwer's fixed-point theorem}

Here we consider the modality defined by localising at $\I$ as explained in \cite{modalities}. It is denoted by $L_\I$. We say that $X$ is $\I$-local if $L_\I(X) = X$ and that it is $\I$-contractible if $L_\I(X)=1$.

\begin{lemma}\label{Z-I-local}
$\Z$ and $2$ are $\I$-local.
\end{lemma}

\begin{proof}
By \cref{cohomology-I}, from $H^0(\I,\Z)=\Z$ we get that the map $\Z\to \Z^\I$ is an equivalence, so $\Z$ is $\I$-local. We see that $2$ is $\I$-local as it is a retract of $\Z$.
\end{proof}

\begin{remark}
Since $2$ is $\I$-local, we have that any Stone space is $\I$-local.
\end{remark}

\begin{lemma}\label{BZ-I-local}
$\B\Z$ is $\I$-local.
\end{lemma}

\begin{proof}
Any identity type in $\B\Z$ is a $\Z$-torsor, so it is $\I$-local by \cref{Z-I-local}. So the map $\B\Z\to \B\Z^{\I}$ is an embedding. From $H^1(\I,\Z)=0$ we get that it is surjective, hence an equivalence.
\end{proof}

\begin{lemma}\label{continuously-path-connected-contractible}
Assume $X$ a type with $x:X$ such that for all $y:X$ we have $f:\I\to X$ such that $f(0)=x$ and $f(1)=y$. Then $X$ is $\I$-contractible.
\end{lemma}

\begin{proof}
%First we prove that the map:
%\[\eta_X:X\to L_\I(X)\] 
%is surjective. Indeed its fiber are $\I$-contractible, but for any type $F$ we have a map:
%\[L_\I(F) \to L_\mathbb{F}(\propTrunc{F}) = \propTrunc{F}\] 
For all $y:X$ we get a map $g:\I\to L_\I(X)$ such that $g(0) = [x]$ and $g(1)=[y]$. Since $L_\I(X)$ is $\I$-local this means that $\prod_{y:X}[x]=[y]$. We conclude $\prod_{y:L_\I(X)}[x]=y$ by applying the elimination principle for the modality.
\end{proof}

\begin{corollary}\label{R-I-contractible}
We have that $\R$ and $\mathbb{D}^2=\{(x,y):\mathbb R^2\ \vert\ x^2+y^2\leq 1\}$ are $\I$-contractible.
\end{corollary}

\begin{proposition}\label{shape-S1-is-BZ}
$L_\I(\R/\Z) = \B\Z$.
\end{proposition}

\begin{proof}
As for any group quotient, the fibers of the map $\R\to\R/\Z$ are $\Z$-torsors, so we have an induced pullback square
\[
\begin{tikzcd}
\R\ar[r]\ar[d] & 1\ar[d] \\
\R/\Z\ar[r] & \B\Z
\end{tikzcd}
\]
Now we check that the bottom map is an $\I$-localisation. Since $\B\Z$ is $\I$-local by \cref{BZ-I-local}, it is enough to check that its fibers are $\I$-contractible. Since $\B\Z$ is connected it is enough to check that $\R$ is $\I$-contractible. This is \cref{R-I-contractible}.
\end{proof}

\begin{remark}
By \cref{BZ-I-local}, for any $X$ we have that $H^1(X,\Z) = H^1(L_{\I}(X),\Z)$, so that by \cref{shape-S1-is-BZ} we have that $H^1(\R/\Z,\Z) = H^1(\B\Z,\Z) = \Z$.
\end{remark}

We omit the proof that $\mathbb{S}^1=\{(x,y):\R^2\ \vert\ x^2+y^2=1\}$ is equivalent to $\R/\Z$.
The equivalence can be constructed using trigonometric functions, which exist by Proposition 4.12 in \cite{Bishop}.

\begin{proposition}
\label{no-retraction}
The map $\mathbb{S}^1\to \mathbb{D}^2$ has no retraction.
\end{proposition}

\begin{proof}
By \cref{R-I-contractible} and \cref{shape-S1-is-BZ} we would get a retraction of $\B\Z\to 1$, so $\B\Z$ would be contractible.
\end{proof}

\begin{theorem}[Intermediate value theorem]
  \label{ivt}
  For any $f: \I\to \I$ and $y:\I$ such that $f(0)\leq y$ and $y\leq f(1)$,
  there exists $x:\I$ such that $f(x)=y$.
\end{theorem}

\begin{proof}
  By \Cref{InhabitedClosedSubSpaceClosedCHaus}, the proposition $\exists_{x:\I}\, f(x)=y$ is closed and therefore $\neg\neg$-stable, so we can proceed with a proof by contradiction.
  If there is no such $x:\I$, we have $f(x)\neq y$ for all $x:\I$.
  By \cref{LesserOpenPropAndApartness} we have that $a<b$ or $b<a$ for all distinct numbers $a,b:\I$. So the following two sets cover $\I$
  \[
    U_0:= \{x:\I\mid f(x)<y\} \quad\quad
    U_1:= \{x:\I\mid y<f(x)\}
    \]
  Since $U_0$ and $U_1$ are disjoint, we have $\I=U_0+U_1$ which allows us to define a non-constant function $\I\to 2$, which contradicts \Cref{Z-I-local}.
\end{proof}

\begin{theorem}[Brouwer's fixed-point theorem]
  For all $f:\mathbb{D}^2\to \mathbb{D}^2$ there exists $x:\mathbb{D}^2$ such that $f(x)=x$.
\end{theorem}

\begin{proof}
  As above, by \Cref{InhabitedClosedSubSpaceClosedCHaus}, we can proceed with a proof by contradiction,
  so we assume $f(x)\neq x$ for all $x:\mathbb{D}^2$.
  For any $x:\mathbb{D}^2$ we set $d_x= x-f(x)$, so we have that one of the coordinates of $d_x$ is invertible.
  Let $H_x(t) = f(x) + t\cdot d_x $ be the line through $x$ and $f(x)$.
  The intersections of $H_x$ and $\partial\mathbb{D}^2=\mathbb{S}^1$ are given by the solutions of an equation quadratic in $t$. By invertibility of one of the coordinates of $d_x$, there is exactly one solution with $t> 0$.
  We denote this intersection by $r(x)$ and the resulting map $r:\mathbb D^2\to\mathbb S^1$ has the property that it preserves $\mathbb{S}^1$.
  Then $r$ is a retraction from $\mathbb{D}^2$ onto its boundary $\mathbb{S}^1$, which is a contradiction by \Cref{no-retraction}.
\end{proof}

\begin{remark}
In constructive reverse mathematics \cite{HannesDiener}, it is known that both the intermediate value theorem and Brouwer's fixed-point theorem are equivalent to LLPO. But LLPO does not hold in real cohesive homotopy type theory, so \cite{shulman-Brouwer-fixed-point} prove a variant of the statement involving a double negation.
\end{remark}


\section{Type Theoretic justification of axioms}
\newcommand{\inc}{\mathsf{inc}}
\newcommand{\inl}{\mathsf{inl}}
\newcommand{\inr}{\mathsf{inr}}
\newcommand{\idd}{\mathsf{id}}
%\newcommand{\UU}{\mathcal{U}}
\newcommand{\II}{\mathbf{I}}
\newcommand{\nats}{\mathbb{N}}


\newcommand{\ext}{\mathsf{ext}}
\newcommand{\patch}{\mathsf{patch}}
\newcommand{\cov}{\mathsf{cov}}
\newcommand{\isSheaf}{\mathsf{isSheaf}}
\newcommand{\isIso}{\mathsf{isIso}}
\newcommand{\Fib}{\mathsf{Fib}}

\newcommand{\Typp}{\mathsf{Type}}
\newcommand{\Elem}{\mathsf{Elem}}
\newcommand{\Cont}{\mathsf{Cont}}

\newcommand{\BB}{\square}
\newcommand{\CC}{\mathcal{C}}
\newcommand{\UU}{\mathcal{U}}
\newcommand{\WW}{\mathcal{W}}
\newcommand{\VV}{\mathcal{V}}

In this section, we present a model of the 3 axioms stated in \Cref{statement-of-axioms}.
This model is best described as an \emph{internal} model
of a presheaf model. The first part can then be described purely syntactically, starting from any model
of 4 other axioms that are valid in a suitable \emph{presheaf} model. We obtain then the sheaf model by defining
a family of open left exact modalities, and the new model is the model of types that are modal for all these modalities.
This method works both in a $1$-topos framework and for models of univalent type theory.
Throughout this section, we use the words \emph{internal} and \emph{external} relative to the model satisfying the 4 axioms below or state explicitly to which model they refer.

\subsection{Internal sheaf model}

\subsubsection{Axioms for the presheaf model}

We start from 4 axioms. The 3 first axioms can be seen as variation of our 3 axioms for synthetic algebraic geometric.

\begin{enumerate}[(1)]
\item $R$ is a ring,
\item for any f.p.\ $R$-algebra $A$, the canonical map $A\rightarrow R^{\Spec(A)}$ is an equivalence
\item for any f.p.\ $R$-algebra $A$, the set $\Spec(A)$ satisfies choice, which can be formulated as
  the fact that for any family of types $P(x)$ for $x:\Spec(A)$ there is a map
  $(\Pi_{x:\Spec(A)}\norm{P(x)})\rightarrow \norm{\Pi_{x:\Spec(A)}P(x)}$.
\item for any f.p.\ $R$-algebra $A$, the diagonal map $\nats\rightarrow\nats^{\Spec(A)}$ is an equivalence.
\end{enumerate}

As before, $\Spec(A)$ denotes the type of $R$-algebra maps from $A$ to $R$, and
if $r$ is in $R$, we write $D(r)$ for the proposition $\Spec(R_r)$.

Note that the first axiom does not require
$R$ to be local, and the third axiom states that $\Spec(A)$ satisfies \emph{choice} and not only Zariski local choice,
for any f.p. $R$-algebra $A$.


\subsubsection{Justification of the axioms for the presheaf model}

\newcommand{\FP}{\mathsf{FP}}

We justify briefly the second axiom (synthetic quasi-coherence). This justification will be done
in a $1$-topos setting, but exactly the same argument holds in the setting of presheaf models of
univalent type theory, since it only involves strict presheaves. A similar direct verification holds
for the other axioms.

We work with presheaves on the opposite of the category of finitely presented $k$-algebras. We write
$L,M,N,\dots$ for such objects, and $f,g,h,\dots$ for the morphisms. A presheaf $F$ on this category is given
by a collection of sets $F(L)$ with restriction maps $F(L)\rightarrow F(M),~u\mapsto f u$ for
$f:L\rightarrow M$ satisfying the usual uniformity conditions.
The ring $R$ is interpreted as the presheaf given by $R(L)\colonequiv L$.

We first introduce the presheaf $\FP$ of {\em finite presentations}. This is internally the type
$$
\Sigma_{n:\nats}\Sigma_{m:\nats}R[X_1,\dots,X_n]^m
$$
which is interpreted by $\FP(L) = \Sigma_{n:\nats}\Sigma_{m:\nats}L[X_1,\dots,X_n]^m$.
If $\xi = (n,m,q_1,\dots,q_m)\in\FP(L)$ is such a presentation, we build a natural extension
$\iota:L\rightarrow L_{\xi} = L[X_1,\dots,X_n]/(q_1,\dots,q_m)$ where the system $q_1 = \dots = q_m = 0$
has a solution $s_{\xi}$. Furthermore, if we have another extension $f:L\rightarrow M$
and a solution $s\in M^n$ of this system in $M$, there exists a unique map $i(f,s):L_{\xi}\rightarrow M$
such that $i(f,s) s_{\xi} = s$ and $i(f,s)\circ \iota = f$.
Note that $i(\iota,s_{\xi}) = \id$.

\medskip

Internally, we have a map $A:\FP\rightarrow R\mathsf{-alg}(\UU_0)$, which to any presentation
$\xi = (n,m,q_1,\dots,q_m)$ associates the $R$-algebra $A(\xi) = L[X_1,\dots,X_n]/(q_1,\dots,q_m)$.
This corresponds externally to the presheaf on the category of elements of $\FP$ defined
by $A(L,\xi) = L_{\xi}$.

Internally, we have a map $\Spec(A):\FP\rightarrow \UU_0$, defined by $\Spec(A)(\xi) = Hom(A(\xi),R)$.
We can replace it by the isomorphic map which to $\xi = (n,m,q_1,\dots,q_m)$ associates the set
$S(\xi)$ of solutions of the system $q_1=\dots=q_m= 0$ in $R^n$.
Externally, this corresponds to the presheaf on the category of elements of $\FP$ so that
$\Spec(A)(L,n,m,q_1,\dots,q_m)$ is the set of solutions of the system $q_1=\dots=q_m=0$ in $L^n$.

\medskip

We now define externally two inverse maps $\varphi:A(\xi)\rightarrow R^{\Spec(A(\xi))}$ and
$\psi:R^{\Spec(A(\xi))}\rightarrow A(\xi)$.

\medskip

Notice first that $R^{\Spec(A)}(L,\xi)$, for $\xi = (n,m,q_1,\dots,q_m)$,
is the set of families of elements $l_{f,s}:M$ indexed by $f:L\rightarrow M$
and $s:M^n$ a solution of $fq_1 = \dots = fq_m=0$, satisfying the uniformity condition
$g(l_{f,s}) = l_{(g\circ f),gs}$ for $g:M\rightarrow N$.

\medskip

For $u$ in $A(L,\xi) = L_{\xi}$ we define $\varphi~u$ in $R^{\Spec(A)}(L,\xi)$ by
$$
(\varphi~u)_{f,s} = i(f,s)~u
$$
and for $l$ in $R^{\Spec(A)}(L,\xi)$ we define $\psi~l$ in $A(L,\xi) = L_{\xi}$ by
$$
\psi~ l = l_{\iota,s_{\xi}}
$$
These maps are natural, and one can check
$$
\psi~(\varphi~u) = (\varphi~u)_{\iota,s_{\xi}} = i(\iota,s_{\xi})~u = u
$$
and
$$
(\varphi~(\psi~l))_{f,s} = i(f,s)~(\psi~l) = i(f,s)~l_{\iota,s_{\xi}} = l_{(i(f,s)\circ \iota),(i(f,s)~s_{\xi})} = l_{f,s}
$$
which shows that $\varphi$ and $\xi$ are inverse natural transformations.

Furthermore, the map $\varphi$ is the external version of the canonical map $A(\xi)\rightarrow R^{\Spec(A(\xi))}$.
The fact that this map is an isomorphism is an (internally) equivalent statement of the second axiom.



\subsubsection{Sheaf model obtained by localisation from the presheaf model}

We define now a family of propositions. As before, if $A$ is a ring, we let $\Um(A)$ be the type of unimodular sequences
(\Cref{unimodular})
$f_1,\dots,f_n$ in $A$, i.e.\ such that $(1) = (f_1,\dots,f_n)$. To any element $\vec{r} = r_1,\dots,r_n$
in $\Um(R)$ we associate
the proposition $D(\vec{r}) = D(r_1)\vee\dots\vee D(r_n)$. If $\vec{r}$ is the empty sequence then
$D(\vec{r})$ is the proposition $1 =_R 0$. %For $n=0$, we get the proposition $1=_R 0$.

  Starting from any model of dependent type theory with univalence satisfying the 4 axioms above, we build a new
  model of univalent type theory by considering the types $T$ that are modal for all modalities defined by the propositions
  $D(\vec{r})$, i.e.\ such that all diagonal maps $T\rightarrow T^{D(\vec{r})}$ are equivalences.
  This new model is called the \emph{sheaf model}.

    This way of building a new sheaf model can be described purely syntactically, as in \cite{Quirin16}. In \cite{CRS21}, we extend
    this interpretation to cover inductive data types. In particular, we describe there the sheafification $\nats_S$ of the type
    of natural numbers with the unit map $\eta:\nats\rightarrow\nats_S$. 

    A similar description can be done starting with the $1$-presheaf model. In this case, we use for the propositional truncation of a
    presheaf $A$ the image of the canonical map $A\rightarrow 1$. We however get a model of type theory {\em without} universes when we
    consider modal types.

    \begin{proposition}\label{modal}
      The ring $R$ is modal. It follows that any f.p.\ $R$-algebra is modal.
    \end{proposition}

    \begin{proof}
      If $r_1,\dots,r_n$ is in $\Um(R)$, we build a patch function $R^{D(r_1,\dots,r_n)}\rightarrow R$.
      Any element $u:R^{D(r_1,\dots,r_n)}$ gives a compatible family of elements $u_i:R^{D(r_i)}$, hence
      a compatible family of elements in $R_{r_i}$ by quasi-coherence. But then it follows from local-global
      principle \cite{lombardi-quitte}, that we can patch this family to a unique element of $R$.
      
      If $A$ is a f.p.\ $R$-algebra, then $A$ is isomorphic to $R^{\Spec(A)}$ and hence is modal.
    \end{proof}

    \begin{proposition}
      In this new sheaf model, $\perp_S$ is $1 =_R 0$.
    \end{proposition}

    \begin{proof}
      The proposition $1=_R0$ is modal by the previous proposition.
      If $T$ is modal, all diagonal maps $T\rightarrow T^{D(\vec{r})}$ are equivalences. For the empty sequence $\vec{r}$
      we have that $D(\vec{r})$ is $\perp$, and the empty sequence is unimodular exactly when $1 =_R 0$. So $1=_R0$
      implies that $T$ and $T^{\perp}$ are equivalent, and so implies that $T$ is contractible. By extensionality,
      we get that $(1=_R0)\rightarrow T$ is contractible when $T$ is modal.
    \end{proof}
    
    \begin{lemma}\label{Um}
      For any f.p.\ $R$-algebra $A$, we have $\Um(R)^{\Spec(A)} = \Um(A)$.
    \end{lemma}

    \begin{proof}
      Note that the fact that $r_1,\dots,r_n$ is unimodular is expressed by
      $$\norm{\Sigma_{s_1,\dots,s_n:R}r_1s_1+\dots+r_ns_n = 1}$$
      and we can use these axioms 2 and 3 to get
      $$\norm{\Sigma_{s_1,\dots,s_n:R}r_1s_1+\dots+r_ns_n = 1}^{\Spec(A)} = \norm{\Sigma_{v_1,\dots,v_n:A}\Pi_{x:\Spec(A)}r_1v_1(x)+\dots+r_nv_n(x) = 1}$$
      The result follows then from this and axiom 4.
    \end{proof}      
      %, that $A$ is quasi-coherent.
%      \rednote{I think it follows from axioms 4,2 and 3.}


%%     If $A$ is a ring, a fundamental system of orthogonal idempotents $e_1,\dots,e_p$ of $A$ is a sequence of 
%%     idempotent elements satisying $e_1+\dots+e_p = 1$ and $e_ie_j = 0$ if $i\neq j$. We then have a partition
%%     of $\Spec(A)$ into open subsets $\Spec(A_{e_i})$.

%%     \begin{lemma}\label{nats}
%%       For any function $u:\nats^{\Spec(A)}$ there exists a fundamental system of orthogonal idempotents $e_1,\dots,e_p$, and corresponding
%%       numbers $n_1,\dots,n_p$ such that $u$ is constant and equal to $n_i$ on $\Spec(A_{e_i})$.
%%     \end{lemma}

%%     We write $Um(A)$ for $\Sigma_{n:\nats}\Um(A)$.

%%     \begin{corollary}
%%       If $A$ is a f.p. ring, then any function in $Um(R)^{\Spec(A)}$ is given by
%%       a fundamental system of orthogonal idempotents $e_1,\dots,e_p$, and corresponding elements in $Um(A_{e_1}),\dots,Um(A_{e_p})$.
%%     \end{corollary}
    
    For an f.p.\ $R$-algebra $A$, we can define the type of presentations $Pr_{n,m}(A)$ as the type $A[X_1,\dots,X_n]^m$.
    Each element in $Pr_{n,m}(A)$ defines an
    f.p.\ $A$-algebra. Since $Pr_{n,m}(A)$ is a modal type since $A$ is f.p., the type of presentations $Pr_{n,m}(A)_S$ in the sheaf model
    defined for $n$ and $m$ in $\nats_S$ will be such that $Pr_{\eta p,\eta q}(A)_S = Pr_{p,q}(A)$ \cite{CRS21}.
%    We have $Pr(R)^{\Spec(A)} = Pr(A)$. \rednote{Def of Pr missing}
    
    \begin{lemma}\label{propsheaf}
      If $P$ is a proposition, then the sheafification of $P$ is
      $$\norm{\Sigma_{(r_1,\dots,r_n):\Um(R)}P^{D(r_1,\dots,r_n)}}$$
    \end{lemma}
    
    \begin{proof}
      If $Q$ is a modal proposition and $P\rightarrow Q$ we have
      $$\norm{\Sigma_{(r_1,\dots,r_n):\Um(R)}P^{D(r_1,\dots,r_n)}}\rightarrow Q$$
      since
      $P^{D(r_1,\dots,r_n)}\rightarrow Q^{D(r_1,\dots,r_n)}$ and $Q^{D(r_1,\dots,r_n)}\rightarrow Q$.
      It is thus enough to show that
      $$P_0 = \norm{\Sigma_{(r_1,\dots,r_n):\Um(R)}P^{D(r_1,\dots,r_n)}}$$
      is modal.
      If $s_1,\dots,s_m$ is in $\Um(R)$ we show $P_0^{D(s_1,\dots,s_m)}\rightarrow P_0$. This follows
      from $\Um(R)^{D(r)} = \Um(R_r)$, Lemma \ref{Um}.
    \end{proof}
    

    \begin{proposition}\label{norm}
      For any modal type $T$, the proposition $\norm{T}_S$ is
      $$\norm{\Sigma_{(r_1,\dots,r_n):\Um(R)}T^{D(r_1)}\times\dots\times T^{D(r_n)}}$$
    \end{proposition}
    
    \begin{proof}
      It follows from Lemma \ref{propsheaf} that the proposition $\norm{T}_S$ is
      $$\norm{\Sigma_{(r_1,\dots,r_n):\Um(R)}\norm{T}^{D(r_1,\dots,r_n)}} = \norm{\Sigma_{(r_1,\dots,r_n):\Um(R)}\norm{T}^{D(r_1)}\times\dots\times\norm{T}^{D(r_n)}}$$
      and we get the result using the fact that choice holds for each $D(r_i)$, so that
      \[\norm{T}^{D(r_1)}\times\dots\times\norm{T}^{D(r_n)} = \norm{T^{D(r_1)}}\times\dots\times\norm{T^{D(r_n)}} =
        \norm{T^{D(r_1)}\times\dots\times T^{D(r_n)}}\]
    \end{proof}
    
    \begin{proposition}
      In the sheaf model, $R$ is a local ring.
    \end{proposition}

    \begin{proof}
      This follows from \Cref{norm} and Lemma \ref{Um}.
    \end{proof}

    \begin{lemma}\label{localfp}
      If $A$ is a $R$-algebra which is modal and there exists $r_1,\dots,r_n$ in $\Um(R)$ such that each
      $A^{D(r_i)}$ is a f.p.\ $R_{r_i}$-algebra, then $A$ is a f.p.\ $R$-algebra.
    \end{lemma}
    
    \begin{proof}
      Using the local-global principles presented in \cite{lombardi-quitte}, we can patch together the f.p.\ $R_{r_i}$-algebra
      to a global f.p.\ $R$-algebra. This f.p.\ $R$-algebra is modal by Proposition \ref{modal}, and is locally equal to $A$
      and hence equal to $A$ since $A$ is modal.
    \end{proof}

    \begin{corollary}
      The type of f.p.\ $R$-algebras is modal and is the type of f.p.\ $R$-algebras in the sheaf model.
    \end{corollary}

    \begin{proof}
          For any $R$-algebra $A$, we can form a type $\Phi(n,m,A)$ expressing that $A$ has a presentation for some $v:Pr_{n,m}(R)$,
    as the type stating that there is some map $\alpha:R[X_1,\dots,X_n]\rightarrow A$ and that $(A,\alpha)$ is universal such that
    $\alpha$ is $0$ on all elements of $v$. We can also look at this type $\Phi(n,m,A)_S$ in the sheaf model. Using the translation
    from \cite{Quirin16,CRS21}, we see that the type $\Phi(\eta n,\eta m,A)_S$ is exactly the type stating that $A$ is presented by
    some $v:Pr_{n,m}(A)$ among the modal $R$-algebras. This is actually equivalent to $\Phi(n,m,A)$ since any f.p. $R$-algebra is modal.

     If $A$ is a modal $R$-algebra which is f.p. in the sense of the sheaf model, this means that we have
     $$\norm{\Sigma_{n:\nats_S}\Sigma_{m:\nats_S}\Phi(n,m,A)_S}_S$$
     This is equivalent to
     $$\norm{\Sigma_{n:\nats}\Sigma_{m:\nats}\Phi(\eta n,\eta m,A)_S}_S$$
     which in turn is equivalent to
     $$\norm{\Sigma_{n:\nats}\Sigma_{m:\nats}\Phi(n,m,A)}_S$$
     Using Lemma \ref{localfp} and Proposition \ref{norm}, this is equivalent to $\norm{\Sigma_{n:\nats}\Sigma_{m:\nats}\Phi(n,m,A)}$.
    \end{proof}

     Note that the type of f.p. $R$-algebra is universe independent.

    \begin{proposition}
      For any f.p.\ $R$-algebra $A$, the type $\Spec(A)$ is modal and satisfies the axiom of Zariski local choice in
      the sheaf model.
    \end{proposition}
    
    \begin{proof}
      Let $P(x)$ be a family of types over $x:\Spec(A)$ and assume $\Pi_{x:\Spec(A)}\norm{P(x)}_S$. By Proposition \ref{norm},
      this means $\Pi_{x:\Spec(A)}\norm{\Sigma_{(r_1,\dots,r_n):Um}P(x)^{D(r_1)}\times\dots\times P(x)^{D(r_n)}}$. The result follows
      then from choice over $\Spec(A)$ and Lemma \ref{Um}.
    \end{proof}      

%It is then natural to ask how the global section operation behaves for this model, and we show that
%it satisfies a property similar to Zariski local choice
%\rednote{We do not understand what the global section operation has to do with Zariski choice}. 

    \subsection{Presheaf models of univalence}

    We recall first how to build presheaf models of univalence \cite{CCHM,survey},
    and presheaf models satisfying the 3 axioms of the previous section.

The constructive models of univalence are presheaf models parametrised by an interval object $\II$
(presheaf with two global distinct elements $0$ and $1$ and which is tiny) and a classifier object
$\Phi$ for cofibrations. The model is then obtained as an internal model of type theory inside the
presheaf model. For this, we define $C:U\rightarrow U$, uniform in the universe $U$, operation
closed by dependent products, sums and such that $C(\Sigma_{X:U}X)$ holds. It further satisfies, for $A:U^{\II}$, the transport principle
$$
(\Pi_{i:\II}C(Ai))\rightarrow (A0\rightarrow A1)
$$
We get then a model of univalence by interpreting a type as a presheaf $A$ together with an element
of $C(A)$.

 This is over a base category $\BB$.
 
 If we have another category $\CC$, we automatically get a new model of univalent type theory by
 changing $\BB$ to $\BB\times\CC$.

 A particular case is if $\CC$ is the opposite of the category of f.p.\@ $k$-algebras, where $k$ is a
 fixed commutative ring.

 We have the presheaf $R$ defined by $R(J,A) = Hom(k[X],A)$ where $J$ is an object of $\BB$ and $A$ is an object of $\CC$.

  The presheaf $\Gm$ is defined by $\Gm(J,A) = Hom(k[X,1/X],A) = A^{\times}$, the set of invertible elements of $A$.

\subsection{Propositional truncation}

    We start by giving a simpler interpretation of propositional truncation. This will simplify
    the proof of the validity of choice in the presheaf model.

    We work in the presheaf model over a base category $\BB$ which interprets univalent type theory,
    with a presheaf $\Phi$ of cofibrations. The interpretation of the propositional
    truncation $\norm{T}$ {\em does not} require the use of the interval $\II$.

    We recall that in the models, to be contractible can be formulated as having an operation
    $\ext(\psi,v)$ which extends any partial element $v$ of extent $\psi$ to a total element.

    The (new) remark is then that to be a (h)proposition can be formulated as having instead
    an operation $\ext(u,\psi,v)$ which, now {\em given}
    an element $u$, extends any partial element $v$ of extent $\psi$ to a total element.

\medskip    

Propositional truncation is defined as follows. An element of $\norm{T}$ is either of the form
$\inc(a)$ with $a$ in $T$, or of the form $\ext(u,\psi,v)$ where $u$ is in $\norm{T}$ and $\psi$
in $\Phi$ and $v$ a partial element of extent $\psi$.

In this definition, the special constructor $\ext$ is a ``constructor with restrictions'' which
satisfies $\ext(u,\psi,v) = v$ on the extent $\psi$ \cite{CoquandHM18}.

\subsection{Choice}

We prove choice in the presheaf model: if $A$ is a f.p.\@ algebra over $R$ then we have a map
$$
l:(\Pi_{x:\Spec(A)}\norm{P})\rightarrow \norm{\Pi_{x:\Spec(A)}P}
$$

For defining the map $l$, we define $l(v)$ by induction on $v$.
The element $v$ is in $(\Pi_{x:\Spec(A)}\norm{P})(B)$, which can be seen as
an element of $\norm{P}(A)$. If it is $\inc(u)$ we associate $\inc(u)$ and 
if it is $\ext(u,\psi,v)$ the image is $\ext(l(u),\psi,l(v))$.

\subsection{$1$-topos model}

For any small category $\CC$ we can form the presheaf model of type theory over the base category $\CC$ \cite{hofmann,huber-phd-thesis}.
%\rednote{Reference to Hoffmann/Simon's thesis?}.

\medskip

We look at the special case where $\CC$ is the opposite of the category of finitely presented $k$-algebras for a fixed
ring $k$.

    In this model we have a presheaf $R(A) = Hom(k[X],A)$ which has a ring structure.

    In the {\em presheaf} model, we can check that we have $\neg\neg (0=_R 1)$. Indeed, at any stage $A$ we have
    a map $\alpha:A\rightarrow 0$ to the trivial f.p. algebra $0$, and $0 =_R 1$ is valid at the stage $0$.

    The previous internal description of the sheaf model applies as well in the $1$-topos setting.

    \medskip

    However the type of modal types in a given universe is not modal in this $1$-topos setting. This problem can actually be seen as a
    motivation for introducing the notion of stacks, and is solved when we start from a constructive model of univalence.

    \subsection{Some properties of the sheaf model}

    \subsubsection{Quasi-coherence}

A module $M$ in the sheaf model defined at stage $A$, where $A$ is a f.p.\@ $k$-algebra, is given by a sheaf over the category
of elements of $A$. It is thus given by a family of modules $M(B,\alpha)$, for $\alpha:A\rightarrow B$, and restriction maps
$M(B,\alpha)\rightarrow M(C,\gamma\alpha)$ for $\gamma:B\rightarrow C$. In general this family is not determined by
its value $M_A = M(A,\idd_A)$ at $A,\idd_A$.
The next proposition expresses internally in the sheaf model, when a module has this property.
This characterisation is due to Blechschmidt \cite{ingo-thesis}.

\begin{proposition}
  $M$ is internally quasi-coherent\footnote{In the sense that the canonical map $M\otimes A\rightarrow M^{\Spec(A)}$ is an isomorphism for any
  f.p. $R$-algebra $A$.} iff we have $M(B,\alpha) = M_A\otimes_A B$ and the restriction map for
  $\gamma:B\rightarrow C$ is $M_A\otimes_A\gamma$.
\end{proposition}

    \subsubsection{Projective space}

We have defined $\bP^n$ to be the set of lines in $V = R^{n+1}$, so we have
$$
\bP^n ~=~ \Sigma_{L:V\rightarrow \Omega}[\exists_{v:V}\neg (v = 0)\wedge L = R v]
$$
The following was noticed in \cite{kockreyes}.

\begin{proposition}
  $\bP^n(A)$ is the set of submodules of $A^{n+1}$ factor direct in $A^{n+1}$ and of rank $1$.
\end{proposition}

\begin{proof}
  $\bP^n$ is the set of pairs $L,0$ where $L:\Omega^V(A)$ satisfies the proposition $\exists_{v:V}\neg (v = 0)\wedge L = Rv$ at stage
  $A$. This condition implies that $L$ is a quasicoherent submodule of $R^{n+1}$ defined at stage $A$.
  It is thus determined by its value $L(A,\idd_A) = L_A$.

  Furthermore, the condition also implies that $L_A$ is locally free of rank $1$. By local-global principle \cite{lombardi-quitte},
  $L_A$ is finitely generated. We can then apply Theorem 5.14 of
  \cite{lombardi-quitte} to deduce that $L_A$ is factor direct in $A^{n+1}$ and of rank $1$.
\end{proof}

One point in this argument was to notice that the condition
$$
\exists_{v:V}\neg (v = 0)\wedge L = R v
$$
implies that $L$ is quasi-coherent. This would be direct in presence of univalence, since we would have then $L = R$ as a $R$-module
and $R$ is quasi-coherent. But it can also be proved without univalence by transport along isomorphism: a $R$-module which is
isomorphic to a quasi-coherent module is itself quasi-coherent.


\subsection{Global sections and Zariski global choice}

We let $\Box T$ the type of global sections of a globally defined sheaf $T$.
If $c = r_1,\dots,r_n$ is in $\Um(R)$ we let $\Box_c T$ be the type $\Box T^{D(r_1)}\times\dots\times\Box T^{D(r_n)}$.

Using these notations, we can state the principle of Zariski global choice
$$
(\Box \norm{T})\leftrightarrow \norm{\Sigma_{c:\Um(k)}\Box_c T}
$$

This principle is valid in the present model.

Using this principle, we can show that $\Box K(\Gm,1)$ is equal to the type of projective modules of rank $1$ over $k$
and that each $\Box K(R,n)$ for $n>0$ is contractible.
                                                                                  
%This should work over $\bP^n$ as well.

 


\appendix

\section{Negative results}
\input{negative-results}

\printindex

\printbibliography

\end{document}
